%% Copyright (C) 2009-2011, Gostai S.A.S.
%%
%% This software is provided "as is" without warranty of any kind,
%% either expressed or implied, including but not limited to the
%% implied warranties of fitness for a particular purpose.
%%
%% See the LICENSE file for more information.

\section{FormatInfo}

A \dfn{format info} is used when formatting a la \code{printf}. It
store the formatting pattern itself and all the format information it
can extract from the pattern.

\subsection{Prototypes}

\begin{refObjects}
\item[Object]
\end{refObjects}

\subsection{Construction}

The constructor expects a string as argument, whose syntax is similar
to \code{printf}'s.  It is detailed below.

\begin{urbiscript}[firstnumber=1]
FormatInfo.new("%+2.3d");
[00000001] %+2.3d
\end{urbiscript}

A formatting pattern must one of the following (brackets denote
optional arguments):
\begin{itemize}
\item \lstinline&%\var{rank}%&
\item \lstinline&%[\var{rank}$]\var{options} \var{spec}&
\item \lstinline&%|[\var{rank}$]\var{options}[\var{spec}]|&
\end{itemize}

\noindent
where:
\begin{itemize}
\item \var{rank} is an non-null integer which denotes a positional argument:
  a means to output arguments in a different order.

\item \var{options} is a sequence of 0 or several of the following
  characters:

\begin{center}
  \begin{tabular}{|c|l|}
    \hline
    \samp{-} & Left alignment.\\
    \samp{=} & Centered alignment.\\
    \samp{+} & Show sign even for positive number.\\
    \samp{ } & If the string does not begin with \samp{+} or \samp{-}, insert
    a space before the converted string.\\
    \samp{0} & Pad with 0's (inserted after sign or base indicator).\\
    \samp{\#} & Show numerical base, and decimal point.\\
    % \samp{'} & Split thousands (\samp{1 000}).\\
    \hline
  \end{tabular}
\end{center}


\item \var{spec} is the conversion character and must be one of the
  following:

\begin{center}
  \begin{tabular}{|c|l|}
    \hline
    \samp{s} & Default character, prints normally\\
    \samp{d} & Case modifier: lowercase \\
    \samp{D} & Case modifier: uppercase \\
    \samp{x} & Prints in hexadecimal lowercase \\
    \samp{X} & Prints in hexadecimal uppercase \\
    \samp{o} & Prints in octal\\
    % \samp{b} & Prints in binary\\
    \samp{e} & Prints floats in scientific format\\
    \samp{E} & Prints floats in scientific format uppercase\\
    \samp{f} & Prints floats in fixed format\\
    \hline
  \end{tabular}
\end{center}
\end{itemize}

When accepted, the format string is decoded, and its features are made
available as separate slots of the FormatInfo object.

\begin{urbiscript}
do (FormatInfo.new("%5$+'=#06.12X"))
{
  assert
  {
    rank      == 5;    // 5$
    prefix    == "+";  // +
    group     == " ";  // '
    alignment == 0;    // =
    alt       == true; // #
    pad       == "0";  // 0
    width     == 6;    // 6
    precision == 12;   // .12
    uppercase == 1;    // X
    spec      == "x";  // X
  };
}|;
\end{urbiscript}

Formats that do not conform raise errors.

\begin{urbiscript}
FormatInfo.new("foo");
[00000001:error] !!! new: format: pattern does not begin with %: foo

FormatInfo.new("%20m");
[00000002:error] !!! new: format: invalid conversion type character: m

FormatInfo.new("%");
[00000003:error] !!! new: format: trailing `%'

FormatInfo.new("%ss");
[00062475:error] !!! new: format: spurious characters after format: s

FormatInfo.new("%.ss");
[00071153:error] !!! new: format: invalid width after `.': s

FormatInfo.new("%|-8.2f|%%");
[00034983:error] !!! new: format: spurious characters after format: %%
\end{urbiscript}



\subsection{Slots}
\begin{urbiscriptapi}
\item[alignment]
  Requested alignment: \lstinline|-1| for left, \lstinline|0| for
  centered, \lstinline|1| for right (default).
\begin{urbiassert}
FormatInfo.new("%s") .alignment == 1;
FormatInfo.new("%=s").alignment == 0;
FormatInfo.new("%-s").alignment == -1;

 "%5s" % 1 == "    1";
"%=5s" % 2 == "  2  ";
"%-5s" % 3 == "3    ";
\end{urbiassert}


\item[alt]
  Whether the ``alternative'' display is requested (\samp{\#}).
\begin{urbiassert}
FormatInfo.new("%s") .alt == false;
FormatInfo.new("%#s").alt == true;

 "%s" % 12.3 == "12.3";
"%#s" % 12.3 == "12.3000";
 "%x" % 12 == "c";
"%#x" % 12 == "0xc";
\end{urbiassert}


\item[group]%
  Separator to use for thousands as a \refObject{String}.  Corresponds to
  the \samp{'} \var{option}.  Currently produces no effect at all.
\begin{urbiassert}
FormatInfo.new("%s") .group == "";
FormatInfo.new("%'s").group == " ";

"%d" % 123456 == "%'d" % 123456 == "123456";
\end{urbiassert}


\item[pad]
  The padding character to use for alignment requests.  Defaults to space.
\begin{urbiassert}
FormatInfo.new("%s") .pad == " ";
FormatInfo.new("%0s").pad == "0";

 "%5s" % 1 == "    1";
"%05s" % 1 == "00001";
\end{urbiassert}


\item[pattern]
  The pattern given to the constructor.
\begin{urbiassert}
FormatInfo.new("%#'12.8s").pattern == "%#'12.8s";
\end{urbiassert}


\item[precision]
  When formatting a \refObject{Float}, the maximum number of digits
  after decimal point when in fixed or scientific mode, and in total
  when in default mode.  When formatting other objects with spec-char
  \samp{s}, the conversion string is truncated to the precision first
  chars. The eventual padding to \lstinline|width| is done after
  truncation.
\begin{urbiassert}
FormatInfo.new("%s")    .precision == 6;
FormatInfo.new("%23.3s").precision == 3;

  "%f" % 12.3 == "12.300000";
"%.0f" % 12.3 == "12";
"%.2f" % 12.3 == "12.30";
"%.8f" % 12.3 == "12.30000000";
\end{urbiassert}


\item[prefix]
  The string to display before positive numbers.  Defaults to empty.
\begin{urbiassert}
FormatInfo.new("%s") .prefix == "";
FormatInfo.new("% s").prefix == " ";
FormatInfo.new("%+s").prefix == "+";
\end{urbiassert}


\item[rank]%
  In the case of a positional argument, its number, otherwise 0.
\begin{urbiassert}
FormatInfo.new("%s")   .rank == 0;
FormatInfo.new("%2$s") .rank == 2;
FormatInfo.new("%03$s").rank == 3;
FormatInfo.new("%4%")  .rank == 4;

 "%3$s%2$s%2$s%1$s" % ["bar", "o", "f"]
  == "%3%%2%%2%%1%" % ["bar", "o", "f"] == "foobar";
\end{urbiassert}
Cannot be null.
\begin{urbiscript}
FormatInfo.new("%00$s").rank;
[00001243:error] !!! new: format: invalid positional argument: 00
\end{urbiscript}

\item[spec]
  The specification character, regardless of the case conversion
  requests.
\begin{urbiassert}
FormatInfo.new("%s")    .spec == "s";
FormatInfo.new("%23.3s").spec == "s";
FormatInfo.new("%'X")   .spec == "x";
\end{urbiassert}


\item[uppercase]
  Case conversion: \lstinline|-1| for lower case, \lstinline|0| for no
  conversion (default), \lstinline|1| for conversion to uppercase.
  The value depends on the case of specification character, except for
  \samp{\%s} which corresponds to \lstinline|0|.
\begin{urbiassert}
FormatInfo.new("%s")  .uppercase ==  0;
FormatInfo.new("%d")  .uppercase == -1;
FormatInfo.new("%D")  .uppercase ==  1;
FormatInfo.new("%x")  .uppercase == -1;
FormatInfo.new("%X")  .uppercase ==  1;
FormatInfo.new("%|D|").uppercase ==  1;
FormatInfo.new("%|d|").uppercase == -1;
\end{urbiassert}


\item[width]
  Width requested for alignment.
\begin{urbiassert}
FormatInfo.new("%s")   .width == 0;
FormatInfo.new("%10s") .width == 10;
FormatInfo.new("%-10s").width == 10;
FormatInfo.new("%8.2f").width == 8;
\end{urbiassert}
\end{urbiscriptapi}

%%% Local Variables:
%%% coding: utf-8
%%% mode: latex
%%% TeX-master: "../urbi-sdk"
%%% ispell-dictionary: "american"
%%% ispell-personal-dictionary: "../urbi.dict"
%%% fill-column: 76
%%% End:
