%% Copyright (C) 2009-2010, Gostai S.A.S.
%%
%% This software is provided "as is" without warranty of any kind,
%% either expressed or implied, including but not limited to the
%% implied warranties of fitness for a particular purpose.
%%
%% See the LICENSE file for more information.

\section{FormatInfo}

A \dfn{format info} is used when formatting a la \code{printf}. It
store the formatting pattern itself and all the format information it
can extract from the pattern.

\subsection{Prototypes}

\begin{refObjects}
\item[Object]
\end{refObjects}

\subsection{Construction}

The constructor expects a string as argument, whose syntax is similar
to \code{printf}'s.  It is detailed below.

\begin{urbiscript}[firstnumber=1]
var f = FormatInfo.new("%+2.3d");
[00000001] %+2.3d
\end{urbiscript}

A formatting pattern must one of the following (brackets denote
optional arguments):
\begin{itemize}
\item \verb&%&\var{options} \var{spec}
\item \verb&%|&\var{options}[\var{spec}]\verb&|&
\end{itemize}

\noindent
\var{options} is a sequence of 0 or several of the following
characters:

\begin{center}
  \begin{tabular}{|c|l|}
    \hline
    \samp{-} & Left alignment.\\
    \samp{=} & Centered alignment.\\
    \samp{+} & Show sign even for positive number.\\
    \samp{ } & If the string does not begin with \samp{+} or \samp{-}, insert
    a space before the converted string.\\
    \samp{0} & Pad with 0's (inserted after sign or base indicator).\\
    \samp{\#} & Show numerical base, and decimal point.\\
    % \samp{'} & Split thousands (\samp{1 000}).\\
    \hline
  \end{tabular}
\end{center}

\noindent
\var{spec} is the conversion character and must be one of the
following:

\begin{center}
  \begin{tabular}{|c|l|}
    \hline
    \samp{s} & Default character, prints normally\\
    \samp{d} & Case modifier: lowercase \\
    \samp{D} & Case modifier: uppercase \\
    \samp{x} & Prints in hexadecimal lowercase \\
    \samp{X} & Prints in hexadecimal uppercase \\
    \samp{o} & Prints in octal\\
    % \samp{b} & Prints in binary\\
    \samp{e} & Prints floats in scientific format\\
    \samp{E} & Prints floats in scientific format uppercase\\
    \samp{f} & Prints floats in fixed format\\
    \hline
  \end{tabular}
\end{center}

\subsection{Slots}
\begin{urbiscriptapi}
\item[alignment]
  Requested alignment: \lstinline|-1| for left, \lstinline|0| for
  centered, \lstinline|1| for right (default).
\begin{urbiassert}
FormatInfo.new("%s").alignment == 1;
FormatInfo.new("%=s").alignment == 0;
FormatInfo.new("%-s").alignment == -1;
\end{urbiassert}

\item[alt]
  Whether the ``alternative'' display is requested (\samp{\#}).
\begin{urbiassert}
FormatInfo.new("%s").alt == false;
FormatInfo.new("%#s").alt == true;
\end{urbiassert}

\item[group]
  Separator to use for thousands.  Corresponds to the \samp{'}
  \var{option}.
\begin{urbiassert}
FormatInfo.new("%s").group == "";
FormatInfo.new("%'s").group == " ";
\end{urbiassert}

\item[pad]
  The padding character to use for alignment requests.  Defaults to space.
\begin{urbiassert}
FormatInfo.new("%s").pad == " ";
FormatInfo.new("%0s").pad == "0";
\end{urbiassert}

\item[pattern]
  The pattern given to the constructor.
\begin{urbiassert}
FormatInfo.new("%#'12.8s").pattern == "%#'12.8s";
\end{urbiassert}

\item[precision]
  When formatting a \refObject{Float}, the maximum number of digits
  after decimal point when in fixed or scientific mode, and in total
  when in default mode.  When formatting other objects with spec-char
  \samp{s}, the conversion string is truncated to the precision first
  chars. The eventual padding to \lstinline|width| is done after
  truncation.
\begin{urbiassert}
FormatInfo.new("%s").precision == 6;
FormatInfo.new("%23.3s").precision == 3;
\end{urbiassert}

\item[prefix]
  The string to display before positive numbers.  Defaults to empty.
\begin{urbiassert}
FormatInfo.new("%s").prefix == "";
FormatInfo.new("% s").prefix == " ";
FormatInfo.new("%+s").prefix == "+";
\end{urbiassert}

\item[spec]
  The specification character, regardless of the case conversion
  requests.
\begin{urbiassert}
FormatInfo.new("%s").spec == "s";
FormatInfo.new("%23.3s").spec == "s";
FormatInfo.new("%'X").spec == "x";
\end{urbiassert}

\item[uppercase]
  Case conversion: \lstinline|-1| for lower case, \lstinline|0| for no
  conversion (default), \lstinline|1| for conversion to uppercase.
  The value depends on the case of specification character, except for
  \samp{\%s} which corresponds to \lstinline|0|.
\begin{urbiassert}
FormatInfo.new("%s").uppercase == 0;
FormatInfo.new("%d").uppercase == -1;
FormatInfo.new("%D").uppercase == 1;
FormatInfo.new("%x").uppercase == -1;
FormatInfo.new("%X").uppercase == 1;
\end{urbiassert}

\item[width]
  Width requested for alignment.
\begin{urbiassert}
FormatInfo.new("%s").width == 0;
FormatInfo.new("%10s").width == 10;
\end{urbiassert}
\end{urbiscriptapi}

%%% Local Variables:
%%% coding: utf-8
%%% mode: latex
%%% TeX-master: "../urbi-sdk"
%%% ispell-dictionary: "american"
%%% ispell-personal-dictionary: "../urbi.dict"
%%% fill-column: 76
%%% End:
