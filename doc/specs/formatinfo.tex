\section{FormatInfo}

A \dfn{format info} is a prototype that corresponds to a formatting
pattern like those used in printf. It store the formatting pattern
itself and all the format informations it can extract from the
pattern.

\subsection{Prototypes}

\begin{itemize}
\item \refObject{Object}
\end{itemize}

\subsection{Construction}

Format info are created for the formatting pattern as a string.

\begin{urbiscript}
var f = FormatInfo.new("%+2.3d");
[00000001] %+2.3d
\end{urbiscript}

\subsection{Syntax}

A formatting pattern must one of the following (characters that are
between brackets are optional):

\begin{itemize}
\item \verb&%&\emph{options} \emph{spec}
\item \verb&%|&\emph{options}[\emph{spec}]\verb&|&
\end{itemize}

\emph{options} is flags and may be zero of more of the following:
\begin{itemize}
\item \verb&"-"&: Left alignment.
\item \verb&"="&: Centered alignment.
\item \verb&"+"&: Show sign even for positive number.
\item \verb&" "&: If the string does not begin with "+" or "-", insert a space before the converted string.
\item \verb&"0"&: Pad with 0's (inserted after sign or base indicator).
\item \verb&"#"&: Show numerical base, and decimal point.
\item \verb&"'"&: Splits Thousands ($1000 \rightarrow 1 000$).
\end{itemize}

\emph{spec} is the convertion character and may be one of the following:
\begin{itemize}
\item \verb&"s"&: Default character, prints normally.
\item \verb&"d", "D"&: Case modifier, \verb&"d"& for lowercase and \verb&"D"& for uppercase.
\item \verb&"x", "X"&: Prints in hexadecimal, \verb&"x"& for lowercase and \verb&"X"& for uppercase.
\item \verb&"o"&: Prints in octal.
\item \verb&"b"&: Prints in binary.
\end{itemize}
