%% Copyright (C) 2009-2010, Gostai S.A.S.
%%
%% This software is provided "as is" without warranty of any kind,
%% either expressed or implied, including but not limited to the
%% implied warranties of fitness for a particular purpose.
%%
%% See the LICENSE file for more information.

\section{Mutex}

\dfn{Mutex} allow to define critical sections.

\subsection{Prototypes}
\begin{itemize}
\item \refObject{Tag}
\end{itemize}

\subsection{Construction}
A Mutex can be constructed like any other Tag but without name.

\begin{urbiscript}[firstnumber=1]
var m = Mutex.new;
[00000000] Mutex_0x964ed40
\end{urbiscript}

You can define critical sections by tagging your code using the Mutex.

\begin{urbiscript}[firstnumber=1]
var m = Mutex.new|;
m: echo("this is critical section");
[00000001] *** this is critical section
\end{urbiscript}

As a critical section, two pieces of code tagged by the same ``Mutex''
will never be executed at the same time.

\subsection{Slots}

\begin{urbiscriptapi}
\item[asMutex]  Return \this.
\begin{urbiscript}
var m1 = Mutex.new|;
assert
{
  m1.asMutex === m1;
};
\end{urbiscript}
\end{urbiscriptapi}


%%% Local Variables:
%%% coding: utf-8
%%% mode: latex
%%% TeX-master: "../urbi-sdk"
%%% ispell-dictionary: "american"
%%% ispell-personal-dictionary: "../urbi.dict"
%%% fill-column: 76
%%% End:
