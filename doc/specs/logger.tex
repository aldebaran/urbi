%% Copyright (C) 2011, Gostai S.A.S.
%%
%% This software is provided "as is" without warranty of any kind,
%% either expressed or implied, including but not limited to the
%% implied warranties of fitness for a particular purpose.
%%
%% See the LICENSE file for more information.

\section{Logger}

\lstinline|Logger| is used to report information to the final user or to the
developer. May it be a warning, an error, a debug information or a system
log, \lstinline|Logger| will pretty print it for you using appropriate
colors. \lstinline|Logger| can also be used as a \lstinline|Tag| for it to
handle nested calls indentation. A log message is assigned a category which
is shown between brackets at beginning of lines. Log level definition and
categories filtering can be changed using environment variables defined in
\autoref{sec:tools:env}.

\begin{urbiscript}
Logger.log("message", "Category") |;
[       Category        ] message

Logger.log("message", "Category") :
{
  Logger.log("indented message", "SubCategory")
}|;
[       Category        ] message
[      SubCategory      ]   indented message
\end{urbiscript}

\subsection{Prototypes}
\begin{refObjects}
\item[Tag]
\end{refObjects}

\subsection{Construction}

\lstinline|Logger| can be used as is, without being cloned. If one wants to
set the category name it is possible to clone \lstinline|Logger| with this
name as an argument.

\begin{urbiscript}
Logger.log("foo");
[00027928:error] !!! log: no category defined

Logger.log("foo", "Category") |;
[       Category        ] foo

var l = Logger.new("Category2");
[00044703] Logger<Category2>
l.log("foo") |;
[       Category2       ] foo
l.log("foo", "ForcedCategory") |;
[    ForcedCategory     ] foo

var l2 = Logger.new;
[00090939] Logger_0x86900c8
l2.log("foo");
[00097912:error] !!! log: no category defined
l2.log("foo", "Category") |;
[       Category        ] foo
\end{urbiscript}

\subsection{Slots}

\begin{urbiscriptapi}

\item[debug](<message>, <category> = category)%
  Report a debug \var{message} of \var{category} to the user. It will be
  shown if the debug level is DEBUG or DUMP. Return \this to allow chained
  operations.

\item[dump](<message>, <category> = category)%
  Report a debug \var{message} of \var{category} to the user. It will be
  shown if the debug level is DUMP. Return \this to allow chained
  operations.

\item[err](<message>, <category> = category)%
  Report an error \var{message} of \var{category} to the user. Return \this
  to allow chained operations.

\item[init](<category>)%
  Define the \var{category} of the new \lstinline|Logger| object. If no
  category is given the new \lstinline|Logger| will inherit the category of
  its prototype.

\item[log](<message>, <category> = category)%
  Report a debug \var{message} of \var{category} to the user. It will be
  shown if debug is not disabled. Return \this to allow chained operations.

\item[onEnter]%
  The primitive called when \lstinline|Logger| is used as a \lstinline|Tag|
  and is entered. This primitive only increments the indentation level.

\item[onLeave]%
  The primitive called when \lstinline|Logger| is used as a \lstinline|Tag|
  and is left. This primitive only decrements the indentation level.

\item[trace](<message>, <category> = category)%
  Report a debug \var{message} of \var{category} to the user. It will be
  shown if the debug level is TRACE, DEBUG or DUMP. Return \this to allow
  chained operations.

\item[warn](<message>, <category> = category)%
  Report a warning \var{message} of \var{category} to the user. Return \this
  to allow chained operations.
\end{urbiscriptapi}

%%% Local Variables:
%%% mode: latex
%%% TeX-master: "../urbi-sdk"
%%% ispell-dictionary: "american"
%%% ispell-personal-dictionary: "../urbi.dict"
%%% fill-column: 76
%%% End:
