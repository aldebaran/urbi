\section{Math}

This object is actually meant to play the role of a name-space in
which the mathematical functions are defined with a more conventional
notation.  Indeed, in an object-oriented language, writing
\lstinline|pi.cos| makes perfect sense, yet \lstinline|cos(pi)| is
more usual.

\subsection{Prototypes}
\begin{itemize}
\item \refObject{Singleton}
\end{itemize}

\subsection{Construction}

Since it is a \refObject{Singleton}, you are not expected to build
other instances.

\subsection{Slots}

\begin{itemize}
\item \lstinline|abs(\var{float})|\\
  Bounce to \lstinline|\var{float}.abs|.

\item \lstinline|acos(\var{float})|\\
  Bounce to \lstinline|\var{float}.acos|.

\item \lstinline|asin(\var{float})|\\
  Bounce to \lstinline|\var{float}.asin|.

\item \lstinline|atan(\var{float})|\\
  Bounce to \lstinline|\var{float}.atan|.

\item \lstinline|atan2(\var{x}, \var{y})|\\
  Bounce to \lstinline|\var{x}.atan2(\var{y})|.

\item \lstinline|cos(\var{float})|\\
  Bounce to \lstinline|\var{float}.cos|.

\item \lstinline|exp(\var{float})|\\
  Bounce to \lstinline|\var{float}.exp|.

\item \lstinline|inf|\\
  Bounce to \lstinline|Float.inf|.

\item \lstinline|log(\var{float})|\\
  Bounce to \lstinline|\var{float}.log|.

\item \lstinline|max(\var{arg1}, ...)|\\
  Bounce to \lstinline|[\var{arg1}, ...].max|, see \refObject{List}.
\begin{urbiscript}
assert(max( 100,   20,   3 ) == 100);
assert(max("100", "20", "3") == "3");
\end{urbiscript}

\item \lstinline|min(\var{arg1}, ...)|\\
  Bounce to \lstinline|[\var{arg1}, ...].min|, see \refObject{List}.
\begin{urbiscript}[firstnumber=last]
assert(min( 100,   20,   3 ) ==     3);
assert(min("100", "20", "3") == "100");
\end{urbiscript}

\item \lstinline|nan|\\
  Bounce to \lstinline|Float.nan|.

\item \lstinline|pi|\\
  Bounce to \lstinline|Float.pi|.

\item \lstinline|random(\var{float})|\\
  Bounce to \lstinline|\var{float}.random|.

\item \lstinline|round(\var{float})|\\
  Bounce to \lstinline|\var{float}.round|.

\item \lstinline|sgn(\var{float})|\\
  Bounce to \lstinline|\var{float}.sgn|.

\item \lstinline|sin(\var{float})|\\
  Bounce to \lstinline|\var{float}.sin|.

\item \lstinline|sqr(\var{float})|\\
  Bounce to \lstinline|\var{float}.sqr|.

\item \lstinline|sqrt(\var{float})|\\
  Bounce to \lstinline|\var{float}.sqrt|.

\item \lstinline|tan(\var{float})|\\
  Bounce to \lstinline|\var{float}.tan|.

\item \lstinline|trunc(\var{float})|\\
  Bounce to \lstinline|\var{float}.trunc|.
\end{itemize}


%%% Local Variables:
%%% mode: latex
%%% TeX-master: "../urbi-sdk"
%%% End:
