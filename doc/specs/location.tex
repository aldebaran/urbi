%% Copyright (C) 2010, 2011, Gostai S.A.S.
%%
%% This software is provided "as is" without warranty of any kind,
%% either expressed or implied, including but not limited to the
%% implied warranties of fitness for a particular purpose.
%%
%% See the LICENSE file for more information.

\section{Location}

This class aggregates two Positions and provides a way to print them as done
in error messages.

\subsection{Prototypes}
\begin{refObjects}
\item[Object]
\end{refObjects}

\subsection{Construction}

Without argument, a newly constructed Location has its Positions initialized
to the first line and the first column.

\begin{urbiscript}[firstnumber=1]
Location.new;
[00000001] 1.1
\end{urbiscript}

With a Position argument \var{p}, the Location will clone the Position into
the begin and end Positions.

\begin{urbiscript}[firstnumber=1]
Location.new(Position.new("file.u",14,25));
[00000001] file.u:14.25
\end{urbiscript}

With two Positions arguments \var{begin} and \var{end}, the Location will
clone both Positions into its own fields.

\begin{urbiscript}[firstnumber=1]
Location.new(Position.new("file.u",14,25), Position.new("file.u",14,35));
[00000001] file.u:14.25-34
\end{urbiscript}

\subsection{Slots}

\begin{urbiscriptapi}

\item['=='](<other>)%
  Compare the begin and end \lstinline|Position|.
\begin{urbiscript}
{
  var p1 = Position.new("file.u",14,25);
  var p2 = Position.new("file.u",16,35);
  var p3 = Position.new("file.u",18,45);
  assert
  {
    Location.new(p1, p3) != Location.new(p1, p2);
    Location.new(p1, p3) == Location.new(p1, p3);
    Location.new(p1, p3) != Location.new(p2, p3);
  };
};
\end{urbiscript}

\item[asString]%
  Present Locations with less variability as possible as either:
  \begin{itemize}
  \item \samp{\var{file}:\var{ll}.\var{cc}}
  \item \samp{\var{file}:\var{ll}.\var{cc}-\var{cc}}
  \item \samp{\var{file}:\var{ll}.\var{cc}-\var{ll}.\var{cc}}
  \end{itemize}
  or the same without file name when the file name is not defined.
\begin{urbiassert}
Location.new(Position.new("file.u",14,25)).asString == "file.u:14.25";
Location.new(Position.new(14,25)).asString == "14.25";

Location.new(
  Position.new("file.u", 14, 25),
  Position.new("file.u", 14, 35)
).asString == "file.u:14.25-34";

Location.new(
  Position.new(14, 25),
  Position.new(14, 35)
).asString == "14.25-34";

Location.new(
  Position.new("file.u", 14, 25),
  Position.new("file.u", 15, 35)
).asString == "file.u:14.25-15.34";

Location.new(
  Position.new(14, 25),
  Position.new(15, 35)
).asString == "14.25-15.34";
\end{urbiassert}

\item[begin]%
  The begin Position used by the Location.  Modifying a copy of this field
  does not modify the Location.
\begin{urbiassert}
Location.new(
  Position.new("file.u", 14, 25),
  Position.new("file.u", 16, 35)
).begin == Position.new("file.u", 14, 25);
\end{urbiassert}

\item[end]%
  The end Position used by the Location.  Modifying a copy of this field
  does not modify the Location.
\begin{urbiassert}
Location.new(
  Position.new("file.u",14,25),
  Position.new("file.u",16,35)
).end == Position.new("file.u",16,35);
\end{urbiassert}
\end{urbiscriptapi}

%%% Local Variables:
%%% coding: utf-8
%%% mode: latex
%%% TeX-master: "../urbi-sdk"
%%% ispell-dictionary: "american"
%%% ispell-personal-dictionary: "../urbi.dict"
%%% fill-column: 76
%%% End:
