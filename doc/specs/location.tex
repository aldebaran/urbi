\section{Location}

This class aggregate two Positions and provide a way to print them as done
in error messages.

\subsection{Prototypes}
\begin{itemize}
\item \refObject{Object}
\end{itemize}

\subsection{Construction}

Without argument, newly constructed locations has its positions initialized
to the first line and the first column.

\begin{urbiscript}
Location.new;
[00000001] 1.1
\end{urbiscript}

With a position argument \var{p}, the location will clone the position into
the begin and end positions.

\begin{urbiscript}
Location.new(Position.new("file.u",14,25));
[00000001] file.u:14.25
\end{urbiscript}

With two positions arguments \var{begin} and \var{end}, the location will
clone both positions into its own fields.

\begin{urbiscript}
Location.new(Position.new("file.u",14,25), Position.new("file.u",14,35));
[00000001] file.u:14.25-34
\end{urbiscript}

\subsection{Slots}

\begin{itemize}

\item \lstinline|asString|\\
  Present locations with less variability as possible as either:
  \begin{itemize}
  \item \samp{\var{file}:\var{ll}.\var{cc}}
  \item \samp{\var{file}:\var{ll}.\var{cc}-\var{cc}}
  \item \samp{\var{file}:\var{ll}.\var{cc}-\var{ll}.\var{cc}}
  \end{itemize}
  or the same without file name when the file name is not defined.
\begin{urbiassert}[firstnumber=last]
Location.new(Position.new("file.u",14,25)).asString == "file.u:14.25";
Location.new(Position.new(14,25)).asString == "14.25";

Location.new(
  Position.new("file.u",14,25),
  Position.new("file.u",14,35)
).asString == "file.u:14.25-34";

Location.new(
  Position.new(14,25),
  Position.new(14,35)
).asString == "14.25-34";

Location.new(
  Position.new("file.u",14,25),
  Position.new("file.u",15,35)
).asString == "file.u:14.25-15.34";

Location.new(
  Position.new(14,25),
  Position.new(15,35)
).asString == "14.25-15.34";
\end{urbiassert}

\item \lstinline|begin|\\
  The begin position used by the location.  Modifying a copy of this field
  does not modify the location.
\begin{urbiassert}[firstnumber=last]
Location.new(
  Position.new("file.u",14,25),
  Position.new("file.u",16,35)
).begin == Position.new("file.u",14,25);
\end{urbiassert}

\item \lstinline|end|\\
  The end position used by the location.  Modifying a copy of this field
  does not modify the location.
\begin{urbiassert}[firstnumber=last]
Location.new(
  Position.new("file.u",14,25),
  Position.new("file.u",16,35)
).end == Position.new("file.u",16,35);
\end{urbiassert}


\end{itemize}

%%% Local Variables:
%%% mode: latex
%%% TeX-master: "../urbi-sdk"
%%% End:
