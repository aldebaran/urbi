\section{Tuple}

A \dfn{tuple} is a container storing a fixed number of objects.
Examples include \refObject{Pair} and \refObject{Triplet}.

\subsection{Prototype}
\begin{itemize}
\item \refObject{Object}
\end{itemize}

\subsection{Construction}

The \lstinline|Tuple| object is not meant to be instantiated, its main
purpose is to share code for its descendants, such as \refObject{Pair}.
Yet it accepts its members as a list.

\begin{urbiscript}[firstnumber=1]
var t = Tuple.new([1, 2, 3]);
[00000000] (1, 2, 3)
\end{urbiscript}

The output generated for a \lstinline|Tuple| can also be used to create a
\lstinline|Tuple|.  Expressions are put inside parenthesis and separated by
commas.  One extra comma is allowed after the last element.  To avoid
confusion between a 1 member \lstinline|Tuple| and a parenthesized
expression, the extra comma must be added.  \lstinline|Tuple| with no
expressions are also accepted.

\begin{urbiscript}
// not a Tuple
(1);
[00000000] 1

// Tuples
();
[00000000] ()
(1,);
[00000000] (1,)
(1, 2);
[00000000] (1, 2)
(1, 2, 3, 4,);
[00000000] (1, 2, 3, 4)
\end{urbiscript}


\subsection{Slots}
\begin{urbiscriptapi}
\item[asString]
  Generate the string \samp{(\var{first}, \var{second}, ...)} using
  \code{asPrintable} to convert members to strings.

\item \lstinline|'[]'(\var{index})|\\
  Return the \var{index}-th element.  \var{index} must be in bounds.
\begin{urbiassert}
(1, 2, 3)[0] == 1;
(1, 2, 3)[1] == 2;
\end{urbiassert}

\item \lstinline|'[]='(\var{index}, \var{value})|\\
  Set (and return) the \var{index}-th element to \var{value}.
  \var{index} must be in bounds.
\begin{urbiscript}
{
  var t = (1, 2, 3);
  assert
  {
    (t[0] = 2) == 2;
    t == (2, 2, 3);
  };
};
\end{urbiscript}

\item \lstinline|'<'(\var{other})|\\
  Lexicographic comparison between two tuples.
\begin{urbiassert}
(0, 0) < (0, 1);
(0, 0) < (1, 0);
(0, 1) < (1, 0);
\end{urbiassert}

\item \lstinline|'=='(\var{other})|\\
  Whether \lstinline|this| and \lstinline|other| have the same
  contents (equality-wise).
\begin{urbiassert}
  (1, 2) == (1, 2);
!((1, 1) == (2, 2));
\end{urbiassert}
\end{urbiscriptapi}



%%% Local Variables:
%%% mode: latex
%%% TeX-master: "../urbi-sdk"
%%% ispell-dictionary: "american"
%%% ispell-personal-dictionary: "../urbi.dict"
%%% End:
