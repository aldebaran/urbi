\chapter*{Introduction}

\section*{Status of this document}

This document defines the specifications of the \us language version
2.0. It defines the expected behavior from the \us interpreter, the
standard library, and the \sdk. It can be used to check whether some code
is valid, or browse \us or \Cxx \api for a desired feature. Random reading
can also provide you with
advanced knowledge or subtleties about some \us aspects.

This document is not an \us tutorial; it is not structured in a
progressive manner and is too
detailed. Think of this document as a dictionary: one does not
learn a foreign language by reading a dictionary. The \us Tutorial, or
the live \us tutorial built in the interpreter are good introductions
to \us.

This document does not aim at giving advanced programming
techniques. Its only goal is to define the language and its
libraries.

\section*{\us}

\dfn[urbiscript@\us]{\us} is an interpreted language designed for highly
interactive and parallel behavior programming. %FILLME

\section*{Structure of this document}

This document structured as follows:

\begin{description}
  \newcommand{\xitem}[2]{\item[\autoref{#1} --- #2]~\\}
\xitem{sec:tools}{Tools specifications}%
  Presentation and usage of the different tools available with the
  \urbi framework related to \us, such as the \urbi server, the
  command line client, \umake, \ldots

\xitem{sec:lang}{\us language specifications}%
  Core constructs of the language and their behavior.

\xitem{sec:stdlib}{\us standard library specifications}%
  Listing of all classes and methods provided in the standard library.

\xitem{sec:sdk}{\urbi \sdk specifications}%
  The \urbi software development kit that enable to
  interact with \urbi from \Cxx.
\end{description}

\chapter{Programs}
\label{sec:tools}

\section{\command{urbi} --- Running an Urbi Server}
\label{sec:tools:urbi}

The \command{urbi} program launches an \urbi server, for either batch,
interactive, or network-based executions.  It is subsumed by, but
simpler to use than, \command{urbi-launch}
(\autoref{sec:tools:urbi-launch}).

\subsection{Options}

\paragraph{Tuning}
\begin{options}
\item[-s, --stack-size=\var{size}] set the coroutine \dfn{stack size}.
  The unit of \var{size} is KB; it defaults to 128.

  This option should not be needed unless you have ``stack exhausted''
  messages from \command{urbi} in which case you should try
  \option{--stack-size=512} or more.

  Alternatively you can define the environment variable
  \code{URBI\_STACK\_SIZE}.  The option \option{--stack-size} has
  precedence over the \code{URBI\_STACK\_SIZE}.
\end{options}

\paragraph{Networking}
\begin{options}
\item[-H, --host=\var{address}] Set the \var{address} on which network
  connections are listened to.  Typical values of \var{address}
  include:
  \begin{description}
  \item[localhost] only local connections are allowed (no other
    computer can reach this server).
  \item[127.0.0.1] same as \code{localhost}.
  \item[0.0.0.0] any IP v4 connection is allowed, including from
    remote computers.
  \end{description}
  Defaults to \code{0.0.0.0}.
\item[-P, --port=\var{port}] Set the port to listen incoming
  connections to.  If \var{port} is \code{-1}, no networking.  If
  \var{port} is \code{0}, then the system will chose any available
  port (see \option{--port-file}).  Defaults to \code{-1}.
\item[-w, --port-file=\var{file}] When the system is up and running,
  and when it is ready for network connections, create the file named
  \var{file} which contains the number of the port the server listens
  to.
\end{options}


\paragraph{Execution}
\begin{options}
\item[-e, --expression=\var{exp}] send the \us expression \var{exp}.
  No separator is added, you have to pass yours.
\item[-f, --file=\var{file}] send the contents of the file \var{file}.
  No separator is added, you have to pass yours.
\item[-i, --interactive] start an interactive session.
\end{options}

The options \option{-e}, \option{-f} accumulate, and are run in the
same \refObject{Lobby} as \option{-i} if used.  In other words, the
following session is valid:

\begin{shell}
# Create a file "two.u".
$ echo "var two = 2;" >two.u
# urbi -e 'var one = 1;' -f two.u -i
[00000000] 1
[00000000] 2
one + two;
[00000000] 3
\end{shell}%$

\section{\command{urbi-launch} --- Running a UObject}
\label{sec:tools:urbi-launch}

The \command{urbi-launch} program launches an \urbi system.  It is
more general than \command{urbi} (\autoref{sec:tools:urbi}):
everything \command{urbi} can do, \command{urbi-launch} can do it too.

\command{urbi-launch} launches UObjects, either in plugged-in mode, or
in remote mode.  Since UObjects can also accept options, the command
line features two parts, separated by \samp{--}:

\begin{shell}
urbi-launch [\var{urbi-launch-option}...] \var{module}... [-- \var{module-option}...]
\end{shell}

\paragraph{Urbi-launch options}
\begin{options}
\item[-h, --help] display the help message and exit successfully.
\item[--version] display version information and exit successfully.
\item[-c, --customize=\var{file}] start the \urbi server in
  \var{file}.  This option is mostly for developpers.
\item[-d, --debug=\var{level}] Set the verbosity level of traces.
  This option is mostly for developpers, but it is very useful when
  tracking problems such as a UObject that fails to load properly.
  Valid values for \var{level} are, in increasing verbosity order:
  \begin{enumerate}
  \item \code{NONE}, no log messages at all.
  \item \code{LOG}, the default value.
  \item \code{TRACE}
  \item \code{DEBUG}
  \item \code{DUMP}, maximum verbosity.
  \end{enumerate}
\end{options}

\paragraph{Mode selection}
\begin{options}
\item[-p, --plugin] attach the \var{module} onto a currently running
  \urbi server (identified by \var{host} and \var{port}).  This is
  equivalent to running \lstinline|loadModule("\var{module}")| on the
  corresponding server.

\item[-r, --remote] run the \var{modules} as separated processes,
  connected to a running Urbi server (identified by \var{host} and
  \var{port}) via network connection.

\item[-s, --start] Start an Urbi server with plugged-in \var{modules}.
\end{options}

\paragraph{Networking}
\command{urbi-launch} supports the same networking options
(\option{--host}, \option{--port}, \option{--port-file}) as
\command{urbi}, see \autoref{sec:tools:urbi}.



%%% Local Variables:
%%% mode: latex
%%% TeX-master: "urbi-specs"
%%% End:

\FloatBarrier
\chapter{\us Language Specifications}
\label{sec:lang}

\section{Syntax}
\subsection{Comments}

\dfn{Comments} are used to document the code, they are ignored by the
\us interpreter. Both \Cxx comment types are supported.

\begin{itemize}
\item A \lstinline|//| introduces a comment that lasts until the end
  of the line.
\item A \lstinline|/*| introduces a comment that lasts until
  \lstinline|*/| is encountered. Comments nest, contrary to \C/\Cxx:
  if two \lstinline|/*| are encountered, the
  comment will end after two \lstinline|*/|, not one.
\end{itemize}

\begin{urbiscript}
// C++ style comment
/* C style comment */
/* These comments /* do */ nest */
\end{urbiscript}

\subsection{Identifiers}
\label{sec:us-syn-id}

\dfn{Identifiers} in \us are composed of one or more alphanumeric or
underscore (\lstinline|_|) characters, not starting by a digit, i.e.,
identifiers match the \lstinline|[a-zA-Z_][a-zA-Z0-9_]*| regular
expression.  Additionally, identifiers must not match any of the \us
reserved words\footnote{
%%
  The only exception to this rule is \lstinline|new|, which can be
  used as the method identifier in a method call.
%%
} documented in \autoref{sec:syn-key}. Identifiers can also be written
between simple quotes (\lstinline|'|), in which case they may contain
any character.

\begin{urbiscript}[firstnumber=last]
var x;
var foobar51;
var this.a_name_with_underscores;
// Invalid.
// var 3x;
// obj.3x();

// Invalid because "if" is a keyword.
// var if;
// obj.if();
// However, keywords can be escaped with simple quotes.
var 'if';
var this.'else';

// Identifiers can be escaped with simple quotes
var '%x';
var '1 2 3';
var this.'[]';
\end{urbiscript}

\subsection{Keywords}
\label{sec:syn-key}

\dfn{Keywords} are reserved words that cannot be used as identifiers,
for instance.  They are listed in \autoref{tab:keywords}.

\renewcommand{\baselinestretch}{.85}
\begin{table}[\floatpos]
  \centering
  \begin{tabular}{|c|c||c|c|}
    \hline
    Keyword                       & Remark                                  &
    Keyword                       & Remark                                  \\
    \hline
    \lstinline"and"               & Synonym for  operator                   &
    \lstinline"long"              & Reserved                                \\
    \lstinline"and_eq"            & Synonym for  operator                   &
    \lstinline"loop"              & \lstinline|loop&| and
                                    \lstinline-loop|- flavors               \\
    \lstinline"asm"               & Reserved                                &
    \lstinline"loopn"             & Deprecated, use \lstinline|for|         \\
    \lstinline"at"                &                                         &
    \lstinline"mutable"           & Reserved                                \\
    \lstinline"auto"              & Reserved                                &
    \lstinline"namespace"         & Reserved                                \\
    \lstinline"bitand"            & Synonym for \lstinline|&| operator      &
    \lstinline"new"               &                                         \\
    \lstinline"bitor"             & Synonym for \lstinline-|- operator      &
    \lstinline"not"               & Synonym for \lstinline|!| operator      \\
    \lstinline"bool"              & Reserved                                &
    \lstinline"not_eq"            & Synonym for \lstinline|!=| operator     \\
    \lstinline"break"             &                                         &
    \lstinline"object"            &                                         \\
    \lstinline"call"              &                                         &
    \lstinline"onleave"           &                                         \\
    \lstinline"case"              &                                         &
    \lstinline"or"                & Synonym for \lstinline-||- operator     \\
    \lstinline"catch"             & Reserved                                &
    \lstinline"or_eq"             & Synonym for \lstinline-|=- operator     \\
    \lstinline"char"              & Reserved                                &
    \lstinline"private"           & Ignored                                 \\
    \lstinline"class"             &                                         &
    \lstinline"protected"         & Ignored                                 \\
    \lstinline"closure"           &                                         &
    \lstinline"public"            & Ignored                                 \\
    \lstinline"compl"             & Synonym for \lstinline|~|               &
    \lstinline"register"          & Reserved                                \\
    \lstinline"const"             & Reserved                                &
    \lstinline"reinterpret_cast"  & Reserved                               \\
    \lstinline"const_cast"        & Reserved                                &
    \lstinline"return"            &                                         \\
    \lstinline"continue"          &                                         &
    \lstinline"short"             & Reserved                                \\
    \lstinline"default"           & Reserved                                &
    \lstinline"signed"            & Reserved                                \\
    \lstinline"delete"            &                                         &
    \lstinline"sizeof"            & Reserved                                \\
    \lstinline"do"                &                                         &
    \lstinline"static"            & Deprecated                              \\
    \lstinline"double"            & Reserved                                &
    \lstinline"static_cast"       & Reserved                                \\
    \lstinline"dynamic_cast"      & Reserved                                &
    \lstinline"stopif"            &                                         \\
    \lstinline"else"              &                                         &
    \lstinline"struct"            & Reserved                                \\
    \lstinline"emit"              & Deprecated                              &
    \lstinline"switch"            &                                         \\
    \lstinline"enum"              & Reserved                                &
    \lstinline"template"          & Reserved                                \\
    \lstinline"event"             &                                         &
    \lstinline"this"              &                                         \\
    \lstinline"every"             &                                         &
    \lstinline"throw"             & Reserved                                \\
    \lstinline"explicit"          & Reserved                                &
    \lstinline"timeout"           &                                         \\
    \lstinline"export"            & Reserved                                &
    \lstinline"try"               & Reserved                                \\
    \lstinline"extern"            & Reserved                                &
    \lstinline"typedef"           & Reserved                                \\
    \lstinline"external"          &                                         &
    \lstinline"typeid"            & Reserved                                \\
    \lstinline"float"             & Reserved                                &
    \lstinline"typename"          & Reserved                                \\
    \lstinline"for"               & \lstinline|for&| and \lstinline-for|- flavors&
    \lstinline"union"             & Reserved                                \\
    \lstinline"foreach"           & Deprecated, use \lstinline|for|    &
    \lstinline"unsigned"          & Reserved                                \\
    \lstinline"freezeif"          &                                         &
    \lstinline"using"             & Reserved                                \\
    \lstinline"friend"            & Reserved                                &
    \lstinline"var"               &                                         \\
    \lstinline"from"              &                                         &
    \lstinline"virtual"           & Reserved                                \\
    \lstinline"function"          &                                         &
    \lstinline"volatile"          & Reserved                                \\
    \lstinline"goto"              & Reserved                                &
    \lstinline"waituntil"         &                                         \\
    \lstinline"if"                &                                         &
    \lstinline"wchar_t"           & Reserved                                \\
    \lstinline"in"                &                                         &
    \lstinline"whenever"          &                                         \\
    \lstinline"inline"            & Reserved                                &
    \lstinline"while"             & \lstinline|while&| and
                                    \lstinline-while|- flavors              \\
    \lstinline"int"               & Reserved                                &
    \lstinline"xor"               & Synonym for \lstinline|^| operator      \\
    \lstinline"internal"          & Deprecated                              &
    \lstinline"xor_eq"            & Synonym \lstinline|^=| operator         \\
    \hline
  \end{tabular}
  \caption{Keywords}
  \label{tab:keywords}
\end{table}
\renewcommand{\baselinestretch}{1}

\subsection{Literals}
\subsubsection{Durations}

\dfn{Durations} are floats (see \autoref{sec:us-syn-lit-float})
followed by a time unit. They are simply equivalent to the same float,
expressed in seconds. For instance, \lstinline|1s 1ms|, which stands
for ``one second and 1 millisecond'', is strictly equivalent to
\lstinline|1.0001|. Available units and their equivalent are shown in
\autoref{tab:durations}.

\begin{table}[\floatpos]
  \centering
  \begin{tabular}{|c|c|c|}
    \hline
    unit        & abbreviation & equivalence for $n$  \\
    \hline
    millisecond & ms           & $n / 1000$         \\
    second      & s            & $n$                \\
    minute      & mn           & $n \times 60$           \\
    hour        & h            & $n \times 60 \times 60$      \\
    day         & d            & $n \times 60 \times 60 \times 24$ \\
    \hline
  \end{tabular}
  \caption{Duration units}
  \label{tab:durations}
\end{table}

\subsubsection{Floats}
\label{sec:us-syn-lit-float}

Literal \dfn{floats} consist in a succession of digit, representing
the integral part of the number in decimal base, possibly followed by
a dot (\lstinline|.|) and another succession of digits representing
the decimal part. Briefly, float literals match the
\lstinline|[0-9]+(\.[0-9]+)?| regular
expression. The listing below is an example of literal
floats.  See \refObject{Float} for more details.

\begin{urbiscript}[firstnumber=last]
1;
[00000000] 1
1.1;
[00000000] 1.1
\end{urbiscript}

\subsubsection{Lists}
\label{sec:us-syn-lit-list}

Literal \dfn{lists} are represented with a comma-separated, potentially
empty list of arbitrary expressions enclosed in square brackets
(\lstinline|[]|), as shown in the listing below.  See
\refObject{List} for more details.

\begin{urbiscript}[firstnumber=last]
[]; // The empty list
[00000000] []
[1, 2, 3];
[00000000] [1, 2, 3]
\end{urbiscript}

\subsubsection{Strings}
\label{sec:us-syn-lit-string}

\dfn{String} literals are enclosed in double quotes (\lstinline|"|)
and can contain arbitrary characters, which stand for themselves, with
the exception of the escape character, backslash (\lstinline|\|), see
below.  Backslash introduces the following escapes:

\begin{tabular}{|c|p{.8\linewidth}|}
  \hline
  \lstinline|\\| & backslash             \\
  \lstinline|\a| & bell ring             \\
  \lstinline|\b| & backspace             \\
  \lstinline|\f| & form feed             \\
  \lstinline|\n| & line feed             \\
  \lstinline|\r| & carriage return       \\
  \lstinline|\t| & tabulation            \\
  \lstinline|\v| & vertical tabulation   \\

  \lstinline|\[0-7]{3}|
  & eight-bit character corresponding to a three-digit octal number.
  For instance, \lstinline|\000| and \lstinline|177|. \\

  \lstinline|\x[0-9a-fA-F]{2}|
  & eight-bit character corresponding to a two-digit hexadecimal
  number.  For instance, \lstinline|0xfF|. \\

  \lstinline|\B(\var{length})(\var{content})|
  & binary blob.  A \var{length}-long sequence of verbatim
  \var{content}.  \var{length} is expressed in decimal.  \var{content}
  is not interpreted in any way.  The parentheses below to the syntax,
  and are required.  For instance \lstinline|\B(2)(\B)|\\
  \hline
\end{tabular}

Consecutive string literals are glued together into a unique string.
This is useful to split large strings into chunks that fit usual
programming widths.

\begin{urbiscript}[firstnumber=last]
assert("foo" "bar" "baz" == "foobarbaz");
assert("\B(3)("\")" == "\"\\\"");
\end{urbiscript}

The interpreter prints the strings escaped; for instance, line feed
will be printed out as \lstinline|\n| when a string result is dumped
and so forth. An actual line feed will of course be output if a string
content is printed with echo for instance.

\begin{urbiscript}[firstnumber=last]
"";
[00000000] ""
"foo";
[00000000] "foo"
"a\nb"; // urbiScript escapes string when dumping them
[00000000] "a\nb"
echo("a\nb"); // We can see there is an actual line feed
[00000000] *** a
[:]b
echo("a\\nb");
[00000000] *** a\nb
\end{urbiscript}

See \refObject{String} for more details.

\subsection{Operators}

\us supports many \dfn{operators}, most of which are inspired from
\Cxx. Their syntax is presented here, and they are sorted and
described with their original semantics --- that is, \lstinline|+| is
an arithmetic operator that sums two numeric values. However, as in
\Cxx, these operators might be use for any other purpose --- that is,
\lstinline|+| can also be used as the concatenation operator on lists
and strings. Their semantics is thus not limited to what is presented
here.

Tables in this section sort operators top-down, by precedence order.
Group of rows (not separated by horizontal lines) describe operators
that have the same precedence. Many operators are syntactic sugar that
bounce on a method. In this case, the equivalent desugared expression
is shown in the ``Equivalence'' column. This can help you determine
what method to override to define an operator for an object (see
\autoref{sec:tut:operators}).

This section defines the syntax, precedence and associativity of the
operators. Their semantics is described in \autoref{sec:stdlib} in the
documentation of the classes that provide them.

% Operator generators
\newcommand{\operatorhead}{Operator & Use & Associativity & Original semantic
  & Equivalence\\}


\newcommand{\operator}[6][ ]{\lstinline@#2@&\lstinline@#3@&#4&#5&\lstinline@#6@#1\\}
\newcommand{\boperator}[3]{\operator{#1}{a #1 b}{#2}{#3}{a.'#1'(b)}}
\newcommand{\poperator}[3]{\operator{#1}{#1a}{#2}{#3}{a.'#1'()}}

\newcommand{\operatordot}    {\operator  {.}    {a.b}              {-}     {Message sending}          {Not redefinable}       }
\newcommand{\operatordota}   {\operator  {.}    {a.b(args)}        {-}     {Message sending}          {Not redefinable}       }
\newcommand{\operatorsub}    {\operator  {[]}   {a[args]}          {-}     {Subscript}                {a.'[]'(args)}          }
\newcommand{\operatorsubass} {\operator  {[] =} {a[args] = v}      {-}     {Subscript assignment}     {a.'[]='(args, v)}      }
\newcommand{\operatorass}[2][ ]    {\operator[#1]
                                         {=}    {a = b}            {Right} {Assignment}               {updateSlot("a", b)}    }

\newcommand{\operatoriass}[1]{\operator  {#1=}  {a #1= b}          {Right} {In place assignment}      {a = a #1 b}            }
\newcommand{\operatorsiass}  {
    \operatoriass{+}
    \operatoriass{-}
    \operatoriass{*}
    \operatoriass{/}
    \operatoriass{\%}
    \operatoriass{\^}
%    \operatoriass{\~}
}
\newcommand{\operatorinc}    {\operator  {++}   {a++}              {-}     {Incrementation}           {(a = a + 1) - 1}       }
\newcommand{\operatordec}    {\operator  {--}   {a--}              {-}     {Incrementation}           {(a = a - 1) + 1}       }

\newcommand{\operatoruplus}  {\poperator {+}    {-}                {Identity}               }
\newcommand{\operatorumin}   {\poperator {-}    {-}                {Opposite}               }
\newcommand{\operatorexp}    {\boperator {**}   {Right}            {Exponentiation}         }
\newcommand{\operatormult}   {\boperator {*}    {Left}             {Multiplication}         }
\newcommand{\operatordiv}    {\boperator {/}    {Left}             {Division}               }
\newcommand{\operatormod}    {\boperator {\%}   {Left}             {Modulo}                 }
\newcommand{\operatorplus}   {\boperator {+}    {Left}             {Sum}                    }
\newcommand{\operatorminus}  {\boperator {-}    {Left}             {Difference}             }
\newcommand{\operatorlshift} {\boperator {<<}   {Left}             {Left bit shift}         }
\newcommand{\operatorrshift} {\boperator {>>}   {Left}             {Right bit shift}        }
\newcommand{\operatoreq}     {\boperator {==}   {Non Associative}  {Equality}               }
\newcommand{\operatorneq}    {\boperator {!=}   {Non Associative}  {Inequality}             }
\newcommand{\operatorpeq}    {\boperator {===}  {Non Associative}  {Physical equality}      }
\newcommand{\operatorpneq}   {\boperator {!==}  {Non Associative}  {Physical Inequality}    }
\newcommand{\operatoraeq}    {\boperator {\~=}  {Non Associative}  {Relative Approximative equality} }
\newcommand{\operatoreqaeq}  {\boperator {=~=}  {Non Associative}  {Absolute Approximative equality} }
\newcommand{\operatorinf}    {\boperator {<}    {Non Associative}  {Less than}              }
\newcommand{\operatorinfeq}  {\boperator {<=}   {Non Associative}  {Less than or equal to}  }
\newcommand{\operatorsup}    {\boperator {>}    {Non Associative}  {Greater than}           }
\newcommand{\operatorsupeq}  {\boperator {>=}   {Non Associative}  {Greater than or equal to}}
\newcommand{\operatorbxor}   {\boperator {^}    {Left}             {Bitwise exclusive or}   }
\newcommand{\operatorneg}    {\poperator {!}    {Left}             {Logical negation}       }
\newcommand{\operatorand}    {\boperator {\&\&} {Left}             {Logical and}            }
\newcommand{\operatoror}     {\boperator {||}   {Left}             {Logical or}             }

\subsubsection{Arithmetic operators}

\us supports classic \dfn{arithmetic operators}, with the classic
semantics on numeric values. See the table and the listing below.

\begin{table}[\floatposh]
  \centering
  \begin{tabular}{|c|c|c|c|c|c|}
    \hline
    \operatorhead
    \hline
    \operatoruplus
    \operatorumin
    \hline
    \operatorexp
    \hline
    \operatormult
    \operatordiv
    \operatormod
    \hline
    \operatorplus
    \operatorminus
    \hline
  \end{tabular}
  \caption{Arithmetic operators}
\end{table}

\begin{urbiscript}[firstnumber=last]
1 + 1;
[00000000] 2
1 - 2;
[00000000] -1
2 * 3;
[00000000] 6
10 / 2;
[00000000] 5
2 ** 10;
[00000000] 1024
-(1 + 2);
[00000000] -3
\end{urbiscript}

\subsubsection{Assignment operators}

\dfn{Assignment} in \us can be performed with the \lstinline|=|
operator.  Shorthands such as \lstinline|+=| exist; they are not
redefinable since they are equivalent to a regular assignment combined
with another operator. See the table and the listing below.


\begin{table}[\floatposh]
  \centering
  \begin{tabular}{|c|c|c|c|c|c|}
    \hline
    \operatorhead
    \hline
    \operatorass[\footnotemark]{}
    \operatorsiass
    \hline
  \end{tabular}
  \caption{Assignment operators}
\end{table}
\footnotetext{For object fields only. Assignment to local variables
  cannot be redefined. }

% FIXME: this in place modulo example was removed
%         because %= is a lame Urbi operator.
%  x %= 3;


\begin{urbiscript}[firstnumber=last]
var y = 0;
[00000000] 0
y = 10;
[00000000] 10
y += 10;
[00000000] 20
y /= 5;
[00000000] 4
y++;
[00000000] 4
y;
[00000000] 5
\end{urbiscript}

\subsubsection{Bitwise operators}

\us features \dfn{bitwise operators}.  They are also used for other
purpose than bit-related operations. See the table and the listing below.

\begin{table}[\floatposh]
  \centering
  \begin{tabular}{|c|c|c|c|c|c|}
    \hline
    \operatorhead
    \hline
    \operatorlshift
    \operatorrshift
    \hline
    \operatorbxor
    \hline
  \end{tabular}
  \caption{Bitwise operators}
\end{table}

\begin{urbiscript}[firstnumber=last]
4 << 2;
[00000000] 16
4 >> 2;
[00000000] 1
\end{urbiscript}

\subsubsection{Logical operators}

\us supports the usual \dfn{Boolean operators}. See the table and the
listing below. The operators \lstinline|&&| and \lstinline-||- are
short-circuiting: their right-hand side is evaluated only if needed.

\begin{table}[\floatposh]
  \centering
  \begin{tabular}{|c|c|c|c|c|c|}
    \hline
    \operatorhead
    \hline
    \operatorneg
    \hline
    \operatorand
    \hline
    \operatoror
    \hline
  \end{tabular}
  \caption{Boolean operators}
\end{table}

\begin{urbiscript}[firstnumber=last]
true && true;
[00000000] true
true || false;
[00000000] true
!true;
[00000000] false
true || (1 / 0);
[00000000] true
false && (1 / 0);
[00000000] false
\end{urbiscript}

\subsubsection{Comparison operators}

\us supports classical \dfn{comparison operators}. See the table and the listing below.

\begin{table}[\floatposh]
  \centering
  \begin{tabular}{|c|c|c|c|c|c|}
    \hline
    \operatorhead
    \hline
    \operatoreq
    \operatorneq
    \operatorpeq
    \operatorpneq
    \operatoraeq
    \operatoreqaeq
    \operatorinf
    \operatorinfeq
    \operatorsup
    \operatorsupeq
    \hline
  \end{tabular}
  \caption{Comparison operators}
\end{table}

\begin{urbiscript}[firstnumber=last]
0 < 0;
[00000000] false
0 <= 0;
[00000000] true
0 == 0;
[00000000] true
0 === 0;
[00000000] false
var z = 0;
[00000000] 0
z === z;
[00000000] true
z !== z;
[00000000] false
\end{urbiscript}

\subsubsection{Miscellaneous operators}

These operators do not fit the previous categories. See the table and
the listing below. Note that the \dfn[operator!subscript]{subscript}
(square bracket) operator is \dfn{variadic}: it takes any number of
arguments that will be passed to the \lstinline|'[]'| method of the
targeted object.

\begin{table}[\floatposh]
  \caption{Miscellaneous operators}
  \centering
  \begin{tabular}{|c|c|c|c|c|c|}
    \hline
    \operatorhead
    \hline
    \operatordot
    \operatordota
    \hline
    \operatorsub
    \operatorsubass
    \hline
  \end{tabular}
\end{table}

\begin{urbiscript}[firstnumber=last]
// On lists.
var l = [1, 2, 3, 4, 5];
[00000000] [1, 2, 3, 4, 5]
l[0];
[00000001] 1
l[-1];
[00000002] 5
l[0] = 10;
[00000003] 10
l;
[00000004] [10, 2, 3, 4, 5]

// On strings.
var s = "abcdef";
[00000005] "abcdef"
s[0];
[00000006] "a"
s[1,3];
[00000007] "bc"
s[1,3] = "foo";
[00000008] "foo"
s;
[00000009] "afoodef"
\end{urbiscript}

\clearpage
\subsubsection{All operators summary}

\autoref{tab:operators-summary} is a summary of all operators, to
highlight the overall precedences. Operators are sorted by decreasing
precedence. Groups of rows represent operators with the same
precedence.

\begin{table}[\floatposh]
  \caption{Operators summary}
  \label{tab:operators-summary}
  \centering
  \begin{tabular}{|c|c|c|c|c|c|}
    \hline
    Operator               & Use                                    & Associativity
    & Original semantic    & Equivalence                            \\
    \hline
    \operatordot
    \operatordota
    \hline
    \operatorsub
    \operatorsubass
    \hline
    \operatoruplus
    \operatorumin
    \hline
    \operatorexp
    \hline
    \operatormult
    \operatordiv
    \operatormod
    \hline
    \operatorplus
    \operatorminus
    \hline
    \operatorlshift
    \operatorrshift
    \hline
    \operatoreq
    \operatorneq
    \operatorpeq
    \operatorpneq
    \operatoreqaeq
    \operatoraeq
    \operatorinf
    \operatorinfeq
    \operatorsup
    \operatorsupeq
    \hline
    \operatorbxor
    \hline
    \operatorneg
    \hline
    \operatorand
    \hline
    \operatoror
    \hline
    \operatorass
    \operatorsiass
    \hline
    \operatorinc
    \operatordec
    \hline
  \end{tabular}

\end{table}


\section{Scopes and local variables}

\subsection{Scopes}

% FIXME: This is wrong: the last separator isn't ignored; a pipe voids
% the result, for instance.
\dfn{Scopes} are sequences of statements, enclosed in curly brackets
(\lstinline|{}|). Statements are separated with the four statements
separators (see \fixme{autoref{sec:fixme}}). The last separator is
optional and ignored if present. Scopes are themselves expressions,
and can thus be used in composite expressions, nested, and so forth.

\begin{urbiscript}[firstnumber=last]
// Scopes evaluate to their last expression
{
  1;
  2;
  3; // This last separator is optional.
};
[00000000] 3
// Scopes can be used as expressions
{1; 2; 3} + 1;
[00000000] 4
\end{urbiscript}

\subsection{Local variables}

\dfn{Local variables} are introduced with the \lstinline|var| keyword,
followed by an identifier (see \autoref{sec:us-syn-id}) and an optional
initialization value assignment. If the initial value is omitted, it
defaults to \refObject{void}. Variable
declarations evaluate to
the initialization value. They can later be referred to by their
name. Their value can be changed with the assignment operator; such an
assignment expression returns the new value. The use of local
variables is illustrated below.

\begin{urbiscript}[firstnumber=last]
// This declare variable x with value 42, and evaluates to 42.
var t = 42;
[00000000] 42
// x equals 42
t;
[00000000] 42
// We can assign it a new value
t = 51;
[00000000] 51
t;
[00000000] 51
// Initialization defaults to void
var u;
u.isVoid;
[00000000] true
\end{urbiscript}

The lifespan of local variables is the same as their enclosing scope. They
are thus only accessible from their scope and its
subscopes\footnote{Local variables can actually escape their scope
  with lexical closures, see \autoref{sec:us-fun-closures}.}. Two
variables with the same name cannot be defined in the same scope. A
variable with the same name can be defined in an inner scope, in which
case references refer to the innermost variable, as shown below.

\begin{urbiscript}[firstnumber=last]
{
  var x = "x";
  var y = "outer y";
  {
    var y = "inner y";
    var z = "z";
    // We can access variables of parent scopes.
    echo(x);
    // This refers to the inner y.
    echo(y);
    echo(z);
  };
  // This refers to the outer y.
  echo(y);
  // This would be invalid: z does not exist anymore.
  // echo(z);
  // This would be invalid: x is already declared in this scope.
  // var x;
};
[00000000] *** x
[00000000] *** inner y
[00000000] *** z
[00000000] *** outer y
\end{urbiscript}


\section{Functions}

\subsection{Function Definition}

\dfn{Functions} in \us are first class citizens: a function is a
value, like floats and strings, and can be handled as such.  This is
different from most \C-like languages.  One can create a functional
value thanks to the \lstinline|function| keyword, followed by the list
of formal arguments and a compound statement representing the body of
the function. Formal arguments are a possibly-empty comma-separated
list of identifiers.  Non-empty lists of formal arguments may
optionaly end with a trailing comma. The listing below illustrates
this.

\begin{urbiscript}[firstnumber=last]
function () { echo(0) };
[00000000] function () {
[:]  echo(0)
[:]}

function (arg1, arg2) { echo(0) };
[00000000] function (arg1, arg2) {
[:]  echo(0)
[:]}

function (
           arg1, // Ignored argument.
           arg2, // Also ignored.
          )
{
  echo(0)
};
[00000000] function (arg1, arg2) {
[:]  echo(0)
[:]}
\end{urbiscript}

Usually functions are bound to an identifier to be invoked later.
The listing below shows a short-hand to define a named
function.

\begin{urbiscript}[firstnumber=last]
// Functions are often stored in variables to call them later.
var f1 = function () {
  echo("hello")
}|
f1();
[00000000] *** hello

// This form is strictly equivalent, yet simpler.
function f2()
{
  echo("hello")
}|
f2();
[00000000] *** hello
\end{urbiscript}


\subsection{Arguments}

The list of formal arguments defines the number of argument the
function requires. They are accessible by their name from within the
body. If the list of formal arguments is omitted, the number of
effective arguments is not checked, and arguments themselves are not
evaluated. Arguments can then be manipulated with the call message,
explained below.

\begin{urbiscript}[firstnumber=last]
var f = function(a, b) {
  echo(b + a);
}|
f(1, 0);
[00000000] *** 1
// Calling a function with the wrong number of argument is an error.
f(0);
[00000000:error] !!! f: Expected 2 arguments, given 1
f(0, 1, 2);
[00000000:error] !!! f: Expected 2 arguments, given 3
\end{urbiscript}

Non-empty lists of effective arguments may end with an optional comma.
\begin{urbiscript}[firstnumber=last]
f(
  "bar",
  "foo",
 );
[00000000] *** foobar
\end{urbiscript}


\subsection{Return value}

The \dfn[function!return value]{return value} of the function is the
evaluation of its body --- that is, since the body is a scope, the
last evaluated expression in the scope.  Values can be returned
manually with the \lstinline|return| keyword followed by the value, in
which case the evaluation of the function is stopped. If
\lstinline|return| is used with no value, the function returns
\lstinline|void|.

\begin{urbiscript}[firstnumber=last]
function g1(a, b)
{
  echo(a);
  echo(b);
  a // Return value is a
}|
g1(1, 2);
[00000000] *** 1
[00000000] *** 2
[00000000] 1

function g2(a, b)
{
  echo(a);
  return a; // Stop execution at this point and return a
  echo(b); // This is not executed
}|
g2(1, 2);
[00000000] *** 1
[00000000] 1

function g3()
{
  return; // Stop execution at this point and return void
  echo(0); // This is not executed
}|
g3(); // Returns void, so nothing is printed.
\end{urbiscript}

\subsection{Call messages}
\label{sec:us-fun-callmsg}

Functions can access meta-information about how they were called,
through a \lstinline|CallMessage| object. The \dfn{call message}
associated with a function can be accessed with the \lstinline|call|
keyword. It contains several information such as not-yet evaluated
arguments, the name of the function, the target \ldots

\subsection{Strictness}

\us features two different function calls:
\dfn[function!strict]{strict} function calls, effective arguments are
evaluated before invoking the function, and \dfn[function!lazy]{lazy}
function calls, arguments are passed as-is to the function.  As a
matter of fact, the difference is rather that there are strict
functions and lazy function.

Functions defined with a (possibly empty) list of formal arguments in
parentheses are strict: the effective arguments are first evaluated,
and then their value is given to the called function.

\begin{urbiscript}[firstnumber=last]
function first1(a, b) {
  echo(a); echo(b)
}|
first1({echo("Arg1"); 1},
       {echo("Arg2"); 2});
[00000000] *** Arg1
[00000000] *** Arg2
[00000000] *** 1
[00000000] *** 2
\end{urbiscript}

A function declared with no formal argument list is lazy.  Use its
call message to manipulate its \emph{unevaluated} arguments.
The listing below gives an example.  More information about
this can be found in the \refObject{CallMessage} class documentation.

\begin{urbiscript}[firstnumber=last]
function first2
{
  echo(call.evalArgAt(0));
  echo(call.evalArgAt(1));
}|
first2({echo("Arg1"); 1},
       {echo("Arg2"); 2});
[00000000] *** Arg1
[00000000] *** 1
[00000000] *** Arg2
[00000000] *** 2
\end{urbiscript}

A lazy function may implement a strict interface by evaluating its
arguments and storing them as local variables, see below.  This is
less efficient than defining a strict function.

\begin{urbiscript}[firstnumber=last]
function first3
{
  var a = call.evalArgAt(0);
  var b = call.evalArgAt(1);
  echo(a); echo(b);
}|
first3({echo("Arg1"); 1},
       {echo("Arg2"); 2});
[00000000] *** Arg1
[00000000] *** Arg2
[00000000] *** 1
[00000000] *** 2
\end{urbiscript}

\subsection{Lexical closures}
\label{sec:us-fun-closures}

\dfn{Lexical closures} are an additional scoping rule, with which a function
can refer to a local variable located outside the function --- but still
in the current context. \us supports read/write lexical closures,
meaning that the variable is shared between the function and the outer
environment, as shown below.

\begin{urbiscript}[firstnumber=last]
var n = 0|
function cl()
{
  // x refers to a variable outside the function
  n++;
  echo(n);
}|
cl();
[00000000] *** 1
n;
[00000000] 1
n++;
[00000000] 1
cl();
[00000000] *** 3
\end{urbiscript}

The following listing illustrate that local variables can even
escape their declaration scope by lexical closure.

\begin{urbiscript}[firstnumber=last]
function wrapper()
{
  // Normally, x is local to 'wrapper', and is limited to this scope.
  var x = 0;
  at (x > 1)
    echo("ping");
  // Here we make it escape the scope by returning a closure on it.
  return function() { x++ };
} |
[00000001:warning] !!! expensive feature: at (<expression>), prefer at (<event>)

var w = wrapper()|
w();
[00000000] 0
w();
[00000000] 1
[00000000] *** ping
\end{urbiscript}

See \autoref{sec:faq:atexp} for more details about the warning.


\section{Objects}

Any value in \us is an object. Objects contain:

\begin{itemize}
\item A list of prototypes, which are also objects.
\item A list of slots, which to a name associate an object.
\end{itemize}

\subsection{Slots}

\subsubsection{Manipulation}

\dfn{Objects} can contain any number of \dfn{slots}, every slot has a
name and a value. Slots are often called ``fields'', ``attributes'' or
``members'' in other object-oriented languages.

The \lstinline|createSlot| function adds a slot to an object with the
void (\autoref{sec:std-void}) value. The \lstinline|updateSlot|
function changes the value of a slot; \lstinline|getSlot| reads
it. The \lstinline|setSlot| method creates a slot with a given
value. Finally, the \lstinline|localSlotNames| method returns the list of
the object slot's name. The listing below shows how to manipulate
slots. More documentation about these methods can be found in
\autorefObject{Object}.

\begin{urbiscript}[firstnumber=last]
var o = Object.new|
o.localSlotNames;
[00000000] []
o.createSlot("test");
o.localSlotNames;
[00000000] ["test"]
o.getSlot("test").asString;
[00000000] "void"
o.updateSlot("test", 42);
[00000000] 42
o.getSlot("test");
[00000000] 42
\end{urbiscript}

\subsubsection{Syntactic Sugar}

There is some syntactic sugar for slot methods:
\begin{itemize}
\item \lstinline|var o.name| is equivalent to
  \lstinline|o.createSlot("name")|.
\item \lstinline|var o.name = value| is equivalent to
  \lstinline|o.setSlot("name", value)|.
\item \lstinline|o.name = value| is equivalent to
  \lstinline|o.updateSlot("name", value)|.
\end{itemize}


\subsection{Prototypes}

\subsubsection{Manipulation}

\us is a prototype-based language, unlike most classical object
oriented language, which are class-based. In prototype-based
languages, objects have no type, only \dfn{prototypes}, from which they
inherit behavior.

\us objects can have several prototypes. The list of prototypes is
given by the \lstinline|protos| method; they can be added or removed
with \lstinline|addProto| and \lstinline|removeProto|.  See
\autorefObject{Object} for more documentation.

\begin{urbiscript}[firstnumber=last]
var ob = Object.new|
ob.protos;
[00000000] [Object]
ob.addProto(Pair);
[00000000] (nil, nil)
ob.protos;
[00000000] [(nil, nil), Object]
ob.removeProto(Object);
[00000000] (nil, nil)
ob.protos;
[00000000] [(nil, nil)]
\end{urbiscript}

\subsubsection{Inheritance}

Objects inherit their prototypes' slots: \lstinline|getSlot| will also
look in an object prototypes' slots. \lstinline|getSlot| performs a
depth-first traversal of the prototypes hierarchy to find slots. That
is, when looking for a slot in an object:

\begin{itemize}
\item \lstinline|getSlot| checks first if the object itself has the
  requested slot. If so, it returns its value.
\item Otherwise, it applies the same research on every prototype, in
  the order of the prototype list (since addProto inserts in the front
  of the prototype list, the last prototype added has priority). This
  search is recursive: \lstinline|getSlot| will also look in the first
  prototype's prototype, etc before looking in the second
  prototype. If the slot is found in a prototype, it is returned.
\item Finally, if no prototype had the slot, an error is raised.
\end{itemize}

This listing shows how slots are inherited.

\begin{urbiscript}[firstnumber=last]
var a = Object.new|
var b = Object.new|
var c = Object.new|
a.setSlot("x", "slot in a")|
b.setSlot("x", "slot in b")|
// c has no "x" slot
c.getSlot("x");
[00000000:error] !!! lookup failed: x
// c can inherit the "x" slot from a.
c.addProto(a)|
c.getSlot("x");
[00000000] "slot in a"
// b is prepended to the prototype list, and has thus priority
c.addProto(b)|
c.getSlot("x");
[00000000] "slot in b"
// a local slot in c has priority over prototypes
c.setSlot("x", "slot in c")|
c.getSlot("x");
[00000000] "slot in c"
\end{urbiscript}

\subsubsection{Copy on write}

The \lstinline|updateSlot| method has a particular behavior with
respect to prototypes. Although performing an \lstinline|updateSlot|
on a non existent slot is an error, it is valid if the slot is
inherited from a prototype. In this case, the slot is however not
updated in the prototype, but rather created in the object itself,
with the new value. This process is called \dfn{copy on write}; thanks
to it, prototypes are not altered when the slot is overridden in a
child object.

\begin{urbiscript}[firstnumber=last]
var p = Object.new|
var p.slot = 0|
var d = Object.new|
d.addProto(p)|
d.slot;
[00000000] 0
d.slot = 1;
[00000000] 1
// p's slot was not altered
p.slot;
[00000000] 0
// It was copied in d
d.slot;
[00000000] 1
\end{urbiscript}

\subsection{Sending messages}

A \dfn{message} in \us consists in a message name and arguments. One can
send a message to an object with the dot (\lstinline|.|) operator,
followed by the message name (which can be any valid identifier) and
the arguments, as shown below. When there are no
arguments, the parentheses can be omitted. As you might see,
sending messages is very similar to calling methods in classical
languages.

\begin{urbifixme}
// Send the message msg to object obj, with arguments arg1 and arg2.
obj.msg(arg1, arg2);
// Send the message msg to object obj, with no arguments.
obj.msg();
// This is strictly equivalent.
obj.msg;
\end{urbifixme}

When a message ``msg'' is sent to object \lstinline|obj|:

\begin{itemize}
\item The ``msg'' slot of \lstinline|obj| is retrieved (i.e.,
  \lstinline|obj.getSlot("msg")|). If the slot is not found, the
  classic lookup error is raised.
\item If the object is not a \lstinline|Routine|
  (\fixme{autoref{sec:fixme}}), it's
  the result of the message sending. In this case, there must be no
  argument, otherwise an error is raised.
\item If the object is a \lstinline|Routine|, it is invoked with the
  message sending arguments, and the returned value is the result. As
  a consequence, the number of arguments in the message sending must
  match the one required by the \lstinline|Routine|.
\end{itemize}

Such message sending is illustrated below.

\begin{urbiscript}[firstnumber=last]
var obj = Object.new|
var obj.a = 42|
var obj.b = function (x) { x + 1 }|
obj.a;
[00000000] 42
obj.a();
[00000000] 42
obj.a(50);
[00000000:error] !!! a: Expected 0 argument, given 1
obj.b;
[00000000:error] !!! b: Expected 1 argument, given 0
obj.b();
[00000000:error] !!! b: Expected 1 argument, given 0
obj.b(50);
[00000000] 51
\end{urbiscript}

\section{Imperative flow control}

\subsection{break}

When encountered within a \lstinline|for| or a \lstinline|while| loop,
\lstinline|break| makes the execution jump after the loop.

\begin{urbiscript}[firstnumber=last]
var i = 5|
for (; true; echo(i))
{
  if (i > 8)
    break;
  i++;
};
[00000000] *** 6
[00000000] *** 7
[00000000] *** 8
[00000000] *** 9
\end{urbiscript}

\subsection{continue}

When encountered within a \lstinline|for| or a \lstinline|while| loop,
\lstinline|continue| short-circuits the rest of the loop-body, and
runs the next iteration (if there remains one).

\begin{urbiscript}[firstnumber=last]
for (var i = 0; i < 8; i++)
{
  if (i % 2 != 0)
    continue;
  echo(i);
};
[00000000] *** 0
[00000000] *** 2
[00000000] *** 4
[00000000] *** 6
\end{urbiscript}

\subsection{do}

The \lstinline|do| construct changes the target (\lstinline|this|)
when evaluating an expression.  It is a convenient means to avoid
repeating the same target several times.

\begin{urbifixme}[frame=, backgroundcolor=, ]
do (\var{target})
{
  \var{body}
};
\end{urbifixme}

It evaluates \var{body}, with \lstinline|this| being \var{target}, as
shown below.  The whole construct evaluates to the value
of \var{body}.

\begin{urbiscript}[firstnumber=last]
do (1024)
{
  assert(this == 1024);
  assert(sqrt == 32);
  setSlot("y", 23);
}.y;
[00000000] 23
\end{urbiscript}


\subsection{if}
\label{sec:lang:if}
As in most programming languages, conditionals are expressed with
\lstinline|if|.

\begin{urbifixme}[frame=, backgroundcolor=, ]
if (\var{condition}) \var{then-clause}
if (\var{condition}) \var{then-clause} else \var{else-clause}
\end{urbifixme}

First \var{condition} is evaluated; if if evaluates to a value which
is true (\fixme{point to a definition of truth values}), evaluate
\var{then-clause}, otherwise, if applicable, evaluate
\var{else-clause}.

\begin{urbiscript}[firstnumber=last]
if (true) assert(true) else assert(false);
if (false) assert(false) else assert(true);
if (true) assert(true);
\end{urbiscript}

Beware that \emph{there must not be a terminator after the
  \var{then-clause}}:

\begin{urbiscript}[firstnumber=last]
if (true)
  assert(true);
else
  assert(false);
[00000002:error] !!! syntax error, unexpected else
\end{urbiscript}

Contrary to \C/\Cxx, it has value: it also implements the
\lstinline|\var{condition} ? \var{then-clause} : \var{else-clause}|
construct.  Unfortunately, due to syntactic constraints inherited from
\C, it is a \emph{statement}: it cannot be used directly as an
expression.  But as everywhere else in \us, to use a statement where
an expression is expected, use braces:

\begin{urbiscript}[firstnumber=last]
assert(1 + if (true) 3 else 4 == 4);
[00000003:error] !!! syntax error, unexpected if
assert(1 + { if (true) 3 else 4 } == 4);
\end{urbiscript}

The \var{condition} can be any statement list.  Variables which it
declares are visible in both the \var{then-clause} and the
\var{else-clause}, but do not escape the \lstinline|if| construct.

\begin{urbiscript}[firstnumber=last]
assert({if (false) 10 else 20} == 20);
assert({if (true)  10 else 20} == 10);

assert({if (true) 10         } == 10);

assert({if (var x = 10) x + 2 else x - 2} == 12);
assert({if (var x = 0)  x + 2 else x - 2} == -2);

if (var xx = 123) xx | xx;
[00000005:error] !!! lookup failed: xx
\end{urbiscript}

\subsection{for}
\label{sec:lang:for}
\lstinline|for| comes in several flavors.

\subsubsection{C-like for}

\us support the classical \C-like \lstinline|for| construct.

\begin{urbifixme}[frame=, backgroundcolor=, ]
for (\var{initialization}; \var{condition}; \var{increment})
  \var{body}
\end{urbifixme}

It has the exact same behavior as \C's \lstinline|for|:

\begin{enumerate}
\item The \var{initialization} is evaluated.
\item \var{condition} is evaluated. If the result is false, executions
  jump after \lstinline|for|.
\item \var{body} is evaluated. If \lstinline|continue| is encountered,
  execution jumps to point 4. If \lstinline|break| is encountered,
  executions jumps after the \lstinline|for|.
\item The \var{increment} is evaluated.
\item Execution jumps to point 2.
\item The loop evaluates to \lstinline|void|.
\end{enumerate}

\subsubsection{for in}
\label{sec:lang:for:each}

\us supports iteration over a collection with another form of the
\lstinline|for| loop.

\begin{urbifixme}[frame=, backgroundcolor=, ]
for (var \var{name} in \var{collection})
   \var{body};
\end{urbifixme}

It evaluates \var{body} for each element in \var{collection}. The loop
evaluates to \lstinline|void|.  Inside \var{body}, the current element
is accessible via the \var{name} local variable. The listing below
illustrates this.

\begin{urbiscript}[firstnumber=last]
for (var x in [0, 1, 2, 3, 4])
  echo(x.sqr);
[00000000] *** 0
[00000000] *** 1
[00000000] *** 4
[00000000] *** 9
[00000000] *** 16
\end{urbiscript}

This form of \lstinline|for| simply sends the \lstinline|each| message
to \var{collection} with one argument: the function that takes the
current element and performs \lstinline|action| over it. Thus, you can
make any object acceptable in a \lstinline|for| by defining an
adequate \lstinline|each| method.

\begin{urbiscript}[firstnumber=last]
var Hobbits = Object.new|
function Hobbits.each (action)
{
  action("Frodo");
  action("Merry");
  action("Pippin");
  action("Sam");
}|
for (var name in Hobbits)
  echo("%s is a hobbit." % [name]);
[00000000] *** Frodo is a hobbit.
[00000000] *** Merry is a hobbit.
[00000000] *** Pippin is a hobbit.
[00000000] *** Sam is a hobbit.
// This for statement is equivalent to:
Hobbits.each(function (name) { echo("%s is a hobbit." % [name]) });
[00000000] *** Frodo is a hobbit.
[00000000] *** Merry is a hobbit.
[00000000] *** Pippin is a hobbit.
[00000000] *** Sam is a hobbit.
\end{urbiscript}

\subsubsection{for $n$-times}
\label{sec:lang:for:n}

\us provides some support for simple replication of computations: it
allow to repeat a loop body $n$-times.  With the exception that the
loop index is not available within the body, \lstinline|for (n)| is
equivalent to \lstinline|for (var i: n)|.  It supports the same
flavors: \lstinline|for;|, \lstinline{for|}, and \lstinline|for&|. The
loop evaluates to \lstinline|void|.

\begin{urbiscript}[firstnumber=last]
assert({ var res = []; for (3) { res << 1; res << 2 } ; res }
        == [1, 2, 1, 2, 1, 2]);

assert({ var res = []; for|(3) { res << 1; res << 2 } ; res }
        == [1, 2, 1, 2, 1, 2]);

assert({ var res = []; for&(3) { res << 1; res << 2 } ; res }
        == [1, 1, 1, 2, 2, 2]);
\end{urbiscript}

Note that since these \lstinline|for| loops are merely anynomous
foreach-style loops, the argument needs not being an integer, any
iterable value can be used.

\begin{urbiscript}[firstnumber=last]
assert(3 == { var r = 0; for ([1, 2, 3]) r += 1; r});
assert(3 == { var r = 0; for ("123")     r += 1; r});
\end{urbiscript}


\subsection{if}

\us supports the usual \lstinline|if| constructs.

\begin{urbifixme}[frame=, backgroundcolor=, ]
if (\var{condition})
  \var{action};

if (\var{condition})
  \var{action}
else
  \var{otherwise};
\end{urbifixme}

If the \var{condition} evaluation is true, \var{action} is
evaluated. Otherwise, in the latter version, \var{otherwise} is
executed.  Contrary to \C/\Cxx, there \emph{must not} be a semicolon
after the \var{action}; it would end the
\lstinline|if|/\lstinline|else| construct prematurely.

\subsection{loop}

Endless loops can be created with \lstinline|loop|, which is
equivalent to \lstinline|while (true)|.  The loop evaluates to
\lstinline|void|.

\begin{urbiscript}[firstnumber=last]
var v = 1 |
loop
{
  v *= 2;
  if (1024 <= v)
    break;
};
v;
[00000000] 1024
\end{urbiscript}

\subsection{switch}

The \lstinline|switch| statement in \us is similiar to \C's one.

\begin{urbifixme}[frame=, backgroundcolor=, ]
switch (\var{value})
{
  case \var{value_one}:
    \var{action_one};
  case \var{value_two}:
    \var{action_two};
//case ...:
//  ...
  default:
    \var{default_action};
};
\end{urbifixme}

It might contain an arbitrary number of cases, and optionally a
default case. The \var{value} is evaluated first, and then the
result is compared sequentially with the evaluation of all cases
values, with the \lstinline|==| operator, until one comparison is
true. If such a match is found, the corresponding action is executed,
and execution jumps after the \lstinline|switch|. Otherwise, the
default case --- if any --- is executed, and execution jumps after the
switch. The switch itself evaluates to case that was evaluated, or to
void if no match was found and there's no default case. The listing below
illustrates \lstinline|switch| usage.

Unlike \C, there are no \lstinline|break| to end \lstinline|case|
clauses: execution will never span over several cases.  Since the
comparisons are performed with the generic \lstinline|==| operator,
\lstinline|switch| can be performed on any comparable data type.

\begin{urbiscript}[firstnumber=last]
function sw(v)
{
  switch (v)
  {
    case "":
      echo("Empty string");
    case "foo":
      "bar";
    default:
      v[0];
  }
} | {};
sw("");
[00000000] *** Empty string
sw("foo");
[00000000] "bar"
sw("foobar");
[00000000] "f"
\end{urbiscript}
% $ Pacify emacs math mode.

\subsection{while}

The \lstinline|while| loop is similar to \C's one.

\begin{urbifixme}[frame=, backgroundcolor=, ]
while (\var{condition})
  \var{body};
\end{urbifixme}

If \var{condition} evaluation, is true, \var{body} is evaluated and
execution jumps before the \lstinline|while|, otherwise execution
jumps after the \lstinline|while|.

\begin{urbiscript}[firstnumber=last]
var j = 3|
while (0 < j)
{
  echo(j);
  j--;
};
[00000000] *** 3
[00000000] *** 2
[00000000] *** 1
\end{urbiscript}


\section{Exceptions}

\subsection{Throwing exceptions}

Use the \lstinline|throw| keyword to \dfn[exception!throwing]{throw
  exceptions}, as shown below. Thrown exceptions will
break the execution upward until they are caught, or until they reach
the toplevel --- as in \Cxx.  Contrary to \Cxx, exceptions reaching
the toplevel are printed, and won't abort the kernel --- other and new
connections will continue to execute normally.

\begin{urbiscript}[firstnumber=last]
throw 42;
[00000000:error] !!! 42
function inner() { throw "exn" } |
function outer() { inner() }|
// Exceptions propagate to parent call up to the toplevel
outer();
[00000000:error] !!! exn
[00000000:error] !!!    called from: 3.20-26: inner
[00000000:error] !!!    called from: 4.1-7: outer
\end{urbiscript}

\subsection{Catching exceptions}

Exceptions are \dfn[exception!catching]{caught} with the
\lstinline|try|/\lstinline|catch| construct. It consists of a first
block (the \dfn{try-block}), from which we want to catch exceptions,
and one or more catch clauses to stop the exception
(\dfn{catch-blocks}). Each catch clause defines a pattern against
which the thrown exception is matched. If no pattern is specified, the
catch clause matches systematically (equivalent to
\lstinline|catch (...)| in \Cxx).

Exceptions thrown from the \texttt{try} block are matched sequentially
against all catch clauses. The first matching clause is executed, and
control jumps after the whole try/catch block. If no catch clause
matches, the exceptions isn't stopped and continues
upward.

\begin{urbiscript}[firstnumber=last]
function test(e)
{
  try
  { throw e;  }
  catch (0)
  { echo("zero") }
  catch ([var x, var y])
  { echo(x + y) }
} | {};
test(0);
[00002126] *** zero
test([22, 20]);
[00002131] *** 42
test(51);
[00002143:error] !!! 51
[00002143:error] !!!    called from: 12.1-8: test

\end{urbiscript}

\subsection{Inspecting exceptions}

\subsection{Exceptions and parallelism}




\section{Parallel and event-based flow control}

\subsection{at}

\subsection{every}

The \lstindex{every} statement enables to execute a block of code
repeatedly, with the given period.

\begin{urbiscript}[firstnumber=last]
// Print out a message every second.
every (1s)
  echo("Are you still there?");
/*(*/sleep(2.1s);/*)*/
[00000000] *** Are you still there?
[00001000] *** Are you still there?
[00002000] *** Are you still there?
\end{urbiscript}

It exists in two flavors. The default flavor launches the execution of
the block in the background every given period. Iterations may
overlap.

% Cut the previous every, no [firstnumber=last]
\begin{urbiscript}
// If an iteration is longer than the given period, it will overlap
// with the next one.
every (1s)
{
  echo("In");
  sleep(1.5s);
  echo("Out");
};
/*(*/sleep(2.6s);/*)*/
[00000000] *** In
[00001000] *** In
[00001500] *** Out
[00002000] *** In
[00002500] *** Out
\end{urbiscript}

On the other hand, the \lstindex{every|} flavor will not let
iterations overlap. If an iteration takes too long, the following
iterations are delayed. That is, the next iterations will start
immediately after the end of the current one, and next iterations will
occur normally from this point.

% Cut the previous every, no [firstnumber=last]
\begin{urbiscript}
var too_long = true|;

// Every other iteration exceeds the period, and will delay the
// following one.
every| (1s)
{
  if (too_long)
  {
    too_long = false;
    echo("Long in");
    sleep(1.5s);
    echo("Long out");
  }
  else
  {
    too_long = true;
    echo("Short");
  }
};
/*(*/sleep(4.1s);/*)*/
[00000000] *** Long in
[00001500] *** Long out
[00001500] *** Short
[00002500] *** Long in
[00004000] *** Long out
[00004000] *** Short
\end{urbiscript}

The flow-control constructs \lstinline|break| and \lstinline|continue|
are supported.

% Cut the previous every, no [firstnumber=last]
\begin{urbiscript}
{
  var count = 0;
  every| (250ms)
  {
    count += 1;
    if (count == 2)
      continue;
    if (count == 4)
      break;
    echo(count);
  }
};
/*(*/sleep(2s);/*)*/
[00000000] *** 1
[00001500] *** 3
\end{urbiscript}

\subsection{for\& (:)}
\subsection{for\& (n)}
\subsection{waituntil}
\subsection{whenever}


\section{Pattern matching}

\section{Trajectories}

% Local Variables:
%%% mode: latex
%%% TeX-master: "urbi-specs"
%%% End:

\FloatBarrier
%% Copyright (C) 2008-2010, Gostai S.A.S.
%%
%% This software is provided "as is" without warranty of any kind,
%% either expressed or implied, including but not limited to the
%% implied warranties of fitness for a particular purpose.
%%
%% See the LICENSE file for more information.

\chapter{\us Standard Library}
\label{sec:stdlib}

%% Redefine \section is this chapter so that we don't have to
%% call \labelObject each time.  See the bottom of this file for the
%% restoring of \section.
\let\sectionOrig\section
\let\section\sectionObject

\section{Barrier}

\lstinline|Barrier| is used to wait until another job raise a signal.
This can be used to implements blocking calls which are waiting until
a resource is made available.

\subsection{Prototypes}

\begin{refObjects}
\item[Object]
\end{refObjects}

\subsection{Construction}

A \lstinline|Barrier| can be created with no argument.  Signals and wait
calls done on this instance are restricted to this instance.

\begin{urbiscript}[firstnumber=1]
Barrier.new;
[00000000] Barrier_0x25d2280
\end{urbiscript}

\subsection{Slots}

\begin{urbiscriptapi}

\item \lstinline|signal(\var{payload})|
  Wake up one of the job waiting for a signal.  The \var{payload} is sent to
  the \var{wait} method.  This method returns the number of job woken up.

\begin{urbiscript}
do (Barrier.new)
{
  echo(wait) &
  echo(wait) &
  assert
  {
    signal(1) == 1;
    signal(2) == 1
  }
}|;
[00000000] *** 1
[00000000] *** 2
\end{urbiscript}


\item \lstinline|signalAll(\var{payload})|
  Wake up all of the job waiting for a signal.  The \var{payload} is sent to
  all \var{wait} methods.  This method returns the number of job woken up.

\begin{urbiscript}
do (Barrier.new)
{
  echo(wait) &
  echo(wait) &
  assert
  {
    signalAll(1) == 2;
    signalAll(2) == 0
  }
}|;
[00000000] *** 1
[00000000] *** 1
\end{urbiscript}


\item[wait]
  Block until a signal is received.  The \var{payload} sent with the signal
  function is returned by the \lstinline|wait| method.

\begin{urbiscript}
do (Barrier.new)
{
  echo(wait) &
  signal(1)
}|;
[00000000] *** 1
\end{urbiscript}

\end{urbiscriptapi}

%%% Local Variables:
%%% mode: latex
%%% TeX-master: "../urbi-sdk"
%%% ispell-dictionary: "american"
%%% ispell-personal-dictionary: "../urbi.dict"
%%% End:

%% Copyright (C) 2009-2010, Gostai S.A.S.
%%
%% This software is provided "as is" without warranty of any kind,
%% either expressed or implied, including but not limited to the
%% implied warranties of fitness for a particular purpose.
%%
%% See the LICENSE file for more information.

\section{Binary}

A Binary object, sometimes called a \dfn{blob}, is raw memory,
decorated with a user defined header.

\subsection{Prototypes}
\begin{refObjects}
\item[Object]
\end{refObjects}

\subsection{Construction}

Binaries are usually not made by users, but they are heavily used by
the internal machinery when exchanging Binary UValues.  A binary
features some \lstinline|content| and some \lstinline|keywords|, both
simple \refObject[String]{Strings}.

\begin{urbiscript}[firstnumber=1]
Binary.new("my header", "my content");
[00000001] BIN 10 my header
[:]my content
\end{urbiscript}

Beware that the third line above (\samp{my content}), was output by
the system, although not preceded by a timestamp.

\subsection{Slots}

\begin{urbiscriptapi}
\item['+'](<that>)%
  Return a new Binary whose keywords are those of \this if
  not empty, otherwise those of \var{that}, and whose data is the
  concatenation of both.
\begin{urbiassert}
Binary.new("0", "0") + Binary.new("1", "1")
       == Binary.new("0", "01");
Binary.new("", "0") + Binary.new("1", "1")
       == Binary.new("1", "01");
\end{urbiassert}

\item['=='](<other>)%
  Whether \lstinline|keywords| and \lstinline|data| are equal.
\begin{urbiassert}
Binary.new("0", "0") == Binary.new("0", "0");
Binary.new("0", "0") != Binary.new("0", "1");
Binary.new("0", "0") != Binary.new("1", "0");
\end{urbiassert}

\item[asString]
  Display using the syntactic rules of the UObject/UValue protocol.
  Incoming binaries must use a semicolon to separate the header part
  from the content, while outgoing binaries use a carriage-return.
\begin{urbiscript}
assert(Binary.new("head", "content").asString
       == "BIN 7 head\ncontent");
var b = BIN 7 header;content;
[00000002] BIN 7 header
[:]content
assert(b == Binary.new("header", "content"));
\end{urbiscript}

This syntax (\lstinline|BIN \var{size} \var{header}; \var{content}|)
is \emph{partially} supported in \us, but it is strongly discouraged.
Rather, use the \lstinline|\B(\var{size})(\var{data})| special escape
(see \autoref{sec:us-syn-lit-string}):

\begin{urbiassert}
Binary.new("head", "\B(7)(content)").asString
       == "BIN 7 head\ncontent";
\end{urbiassert}


\item[data]
  The data carried by the Binary.
\begin{urbiassert}
Binary.new("head", "content").data == "content";
\end{urbiassert}

\item[empty]
  Whether the data is empty.
\begin{urbiassert}
Binary.new("head", "").empty;
!Binary.new("head", "content").empty;
\end{urbiassert}

\item[keywords]
  The headers carried by the Binary.
\begin{urbiassert}
Binary.new("head", "content").keywords == "head";
\end{urbiassert}
\end{urbiscriptapi}


%%% Local Variables:
%%% coding: utf-8
%%% mode: latex
%%% TeX-master: "../urbi-sdk"
%%% ispell-dictionary: "american"
%%% ispell-personal-dictionary: "../urbi.dict"
%%% fill-column: 76
%%% End:

\section{Boolean}

There is no object \lstinline|Boolean| in \us, but two specific
objects \lstinline|true| and \lstinline|false|.  They are the result
of all comparison statement.


\subsection{Prototypes}

The objects \lstinline|true| and \lstinline|false| have the following
prototype.

\begin{itemize}
\item \refObject{Singleton}
\end{itemize}

\subsection{Construction}

There are no constructors, use \lstinline|true| and \lstinline|false|.
Since they are singletons, \lstinline|clone| will return themselves,
not new copies.

\begin{urbiscript}
assert(true);
assert(!false);
assert(2 < 6 === true);
assert(true.new === true);
assert(6 < 2 === false);
\end{urbiscript}

\subsection{Methods}

\begin{itemize}
\item \lstinline|'&&'(\var{that})|\\
  Short-circuiting logical and. If \lstinline|this| is
  \lstinline|true| evaluate and return \var{that}.  If
  \lstinline|this| is \lstinline|false|, return itself without
  evaluating \var{that}.
\begin{urbiscript}[firstnumber=last]
assert_eq(true && 2, 2);
assert_eq(false && 1 / 0, false);
\end{urbiscript}

\item \lstinline|'||'(\var{that})|\\
  Short-circuiting logical or. If \lstinline|this| is
  \lstinline|false| evaluate and return \var{that}.  If
  \lstinline|this| is \lstinline|true|, return itself without
  evaluating \var{that}.
\begin{urbiscript}[firstnumber=last]
assert_eq(true || 1/0, true);
assert_eq(false || 2, 2);
\end{urbiscript}

\item \lstinline|'!'|\\
  Logical negation. If \lstinline|this| is \lstinline|false| return
  \lstinline|true| and vice-versa.
\begin{urbiscript}[firstnumber=last]
assert_eq(!true, false);
assert_eq(!false, true);
\end{urbiscript}
\end{itemize}

%%% Local Variables:
%%% mode: latex
%%% TeX-master: "../urbi-sdk"
%%% End:

%% Copyright (C) 2009-2010, Gostai S.A.S.
%%
%% This software is provided "as is" without warranty of any kind,
%% either expressed or implied, including but not limited to the
%% implied warranties of fitness for a particular purpose.
%%
%% See the LICENSE file for more information.

\section{CallMessage}
Capturing a method invocation: its target and arguments.

\subsection{Examples}
\subsubsection{Evaluating an argument several times}
\label{sec:std-callmsg-examples-several}

The following example implements a lazy function which takes an integer
\var{n}, then arguments.  The \var{n}-th argument is evaluated twice using
\refSlot{evalArgAt}.

\begin{urbiscript}[firstnumber=1]
function callTwice
{
  var n = call.evalArgAt(0);
  call.evalArgAt(n);
  call.evalArgAt(n)
} |;

// Call twice echo("foo").
callTwice(1, echo("foo"), echo("bar"));
[00000001] *** foo
[00000002] *** foo

// Call twice echo("bar").
callTwice(2, echo("foo"), echo("bar"));
[00000003] *** bar
[00000004] *** bar
\end{urbiscript}


\subsubsection{Strict Functions}

Strict functions do support \lstinline|call|.

\begin{urbiscript}
function strict(x)
{
  echo("Entering");
  echo("Strict: " + x);
  echo("Lazy:   " + call.evalArgAt(0));
} |;

strict({echo("1"); 1});
[00000011] *** 1
[00000013] *** Entering
[00000012] *** Strict: 1
[00000013] *** 1
[00000014] *** Lazy:   1
\end{urbiscript}


\subsection{Slots}

\begin{urbiscriptapi}
\item[args]
  The list of unevaluated arguments.
\begin{urbiscript}
function args { call.args }|
assert
{
  args == [];
  args() == [];
  args({echo(111); 1}) == [Lazy.new(closure() {echo(111); 1})];
  args(1, 2) == [Lazy.new(closure () {1}),
                 Lazy.new(closure () {2})];
};
\end{urbiscript}


\item[argsCount]
  Return the number of arguments.  Do not evaluate them.
\begin{urbiscript}
function argsCount { call.argsCount }|;
assert
{
  argsCount == 0;
  argsCount() == 0;
  argsCount({echo(1); 1}) == 1;
  argsCount({echo(1); 1}, {echo(2); 2}) == 2;
};
\end{urbiscript}

\item[code]
  The body of the called function as a \refObject{Code}.
\begin{urbiscript}
function code { call.getSlot("code") }|
assert (code == getSlot("code"));
\end{urbiscript}

\item[evalArgAt](<n>)%
  Evaluate the \var{n}-th argument, and return its value.  \var{n}
  must evaluate to an non-negative integer.  Repeated invocations
  repeat the evaluation, see
  \autoref{sec:std-callmsg-examples-several}.
\begin{urbiscript}
function sumTwice
{
  var n = call.evalArgAt(0);
  call.evalArgAt(n) + call.evalArgAt(n)
}|;

function one () { echo("one"); 1 }|;

sumTwice(1, one, one + one);
[00000008] *** one
[00000009] *** one
[00000010] 2
sumTwice(2, one, one + one);
[00000011] *** one
[00000012] *** one
[00000011] *** one
[00000012] *** one
[00000013] 4

sumTwice(3, one, one);
[00000014:error] !!! evalArgAt: invalid index: 3
sumTwice(3.14, one, one);
[00000015:error] !!! evalArgAt: invalid index: 3.14
\end{urbiscript}

\item[evalArgs]
  Call \lstinline|evalArgAt| for each argument, return the list of
  values.
\begin{urbiscript}
function twice
{
  call.evalArgs + call.evalArgs
}|;
twice({echo(1); 1}, {echo(2); 2});
[00000011] *** 1
[00000012] *** 2
[00000011] *** 1
[00000012] *** 2
[00000013] [1, 2, 1, 2]
\end{urbiscript}

\item[message]
  The name under which the function was called.
\begin{urbiscript}
function myself { call.message }|
assert(myself == "myself");
\end{urbiscript}

\end{urbiscriptapi}


%%% Local Variables:
%%% mode: latex
%%% TeX-master: "../urbi-sdk"
%%% ispell-dictionary: "american"
%%% ispell-personal-dictionary: "../urbi.dict"
%%% fill-column: 76
%%% End:

%% Copyright (C) 2009-2010, Gostai S.A.S.
%%
%% This software is provided "as is" without warranty of any kind,
%% either expressed or implied, including but not limited to the
%% implied warranties of fitness for a particular purpose.
%%
%% See the LICENSE file for more information.

\section{Channel}
Returning data, typically asynchronous, with a label so that the
``caller'' can find it in the flow.

\subsection{Prototypes}

\begin{itemize}
\item \refObject{Object}
\end{itemize}

\subsection{Construction}

Channels are created like any other object. The constructor must be
called with a string which will be the label.

\begin{urbiscript}[firstnumber=1]
var ch1 = Channel.new("my_label");
[00000201] Channel_0x7985810

ch1 << 1;
[00000201:my_label] 1

var ch2 = ch1;
[00000201] Channel_0x7985810

ch2 << 1/2;
[00000201:my_label] 0.5
\end{urbiscript}

\subsection{Slots}

\begin{urbiscriptapi}
\item['<<'](<value>)%
  Send \var{value} to \this tagged by its label if non-empty.

\begin{urbiscript}
Channel.new("label") << 42;
[00000000:label] 42

Channel.new("") << 51;
[00000000] 51
\end{urbiscript}

\item[echo](<value>)%
  Same as \lstinline|lobby.echo(\var{value}, name)|, see
  \refSlot[Lobby]{echo}.

\begin{urbiscript}
Channel.new("label").echo(42);
[00000000:label] *** 42

Channel.new("").echo("Foo");
[00000000] *** Foo
\end{urbiscript}

\item[enabled] Whether the Channel is enabled.  Disabled Channels
  produce no output.
\begin{urbiscript}
var c = Channel.new("")|;

c << "enabled";
[00000000] "enabled"

c.enabled = false|;
c << "disabled";

c.enabled = true|;
c << "enabled";
[00000000] "enabled"
\end{urbiscript}

\item[Filter] Filtering channel.

The Filter channel outputs text that can be parsed without error by the liburbi.
It does this by filtering types not handled by the liburbi, and displaying
them using \refSlot{echo}.

\begin{urbiscript}
// Use a filtering channel on our lobby output.
topLevel = Channel.Filter.new("")|;
// liburbi knows about List, Dictionary, String and Float, so standard display.
[1, "foo", ["test" => 5]];
[00000001] [1, "foo", ["test" => 5]]
// liburbi does not know 'lobby', so it is escaped with echo:
lobby;
[00000002] *** Lobby_0xADDR
// The following list contains a function which is not handled by liburbi, so
// it gets escaped too.
[1, function () {}];
[00000003] *** [1, function () {}]
// Restore default display to see the difference.
topLevel = Channel.topLevel|;
// The echo is now gone.
[1, function () {}];
[00001758] [1, function () {}]
\end{urbiscript}

\item[quote] Whether the strings are output escaped (the default)
  instead of raw strings.
\begin{urbiscript}
var d = Channel.new("")|;

assert(d.enabled);
d << "A \"String\"";
[00000000] "A \"String\""

d.quote = false|;
d << "A \"String\"";
[00000000] A "String"
\end{urbiscript}

\item[name] The name of the Channel, used to label the output.
\begin{urbiscript}
assert
{
  Channel.new("").name == "";
  Channel.new("foo").name == "foo";
};
\end{urbiscript}

\item[null] A predefined stream whose \lstinline|enabled| is
  \lstinline|false|.
\begin{urbiscript}
Channel.null << "Message";
\end{urbiscript}


\item[topLevel] A predefined stream for regular output.  Strings are
  output escaped.
\begin{urbiscript}
Channel.topLevel << "Message";
[00015895] "Message"
Channel.topLevel << "\"quote\"";
[00015895] "\"quote\""
\end{urbiscript}

\item[warning] A predefined stream for warning messages.  Strings sent
  to it are not escaped.
\begin{urbiscript}
Channel.warning << "Message";
[00015895:warning] Message
Channel.warning << "\"quote\"";
[00015895:warning] "quote"
\end{urbiscript}
\end{urbiscriptapi}

%%% Local Variables:
%%% coding: utf-8
%%% mode: latex
%%% TeX-master: "../urbi-sdk"
%%% ispell-dictionary: "american"
%%% ispell-personal-dictionary: "../urbi.dict"
%%% fill-column: 76
%%% End:

\section{Code}

Objects that can be ``invoked'', comparable to \Cxx's Function Objects.

\subsection{Prototypes}

\begin{refObjects}
\item[Comparable]
\item[Executable]
\item[Object]
\end{refObjects}

\subsection{Slots}

\begin{itemize}
\item \lstinline|==(\var{that})|\\
  Whether \lstinline|this| and \var{that} are the same source code.
  It actually checks that both have the same \lstinline|asString|.
\begin{urbiassert}
function () { 1 } == function () { 1 };
function () { 1 } != closure  () { 1 };
closure  () { 1 } != function () { 1 };

function () { 1 + 1 } == function () { 1 + 1 };
function () { 1 + 2 } != function () { 2 + 1 };

function () { 1 } != function { 1 };
function () { 1 } != function (ignored) { 1 };
\end{urbiassert}

\item \lstinline|apply(\var{args})|\\
  Invoke the routine, with all the arguments.  The target,
  \lstinline|this|, will be set to \lstinline|\var{args}[0]| and the
  remaining arguments with be given as arguments.
\begin{urbiassert}
function (x, y) { x+y }.apply([nil, 10, 20]) == 30;
function () { this }.apply([123]) == 123;

// There is Object.apply.
1.apply([this]) == 1;
\end{urbiassert}
\begin{urbiscript}
function () {}.apply([]);
[00000001:error] !!! apply: list of arguments must begin with `this'

function () {}.apply([1, 2]);
[00000002:error] !!! apply: expected 0 argument, given 1
\end{urbiscript}

\item \lstinline|asString|\\
  Conversion to \refObject{String}.
\begin{urbiassert}
closure  () { 1 }.asString == "closure () {\n  1\n}";
function () { 1 }.asString == "function () {\n  1\n}";
\end{urbiassert}

\item \lstinline|bodyString|\\
  Conversion to \refObject{String} of the routine body.
\begin{urbiassert}
closure  () { 1 }.bodyString == "1";
function () { 1 }.bodyString == "1";
\end{urbiassert}

\end{itemize}

%%% Local Variables:
%%% mode: latex
%%% TeX-master: "../urbi-sdk"
%%% ispell-personal-dictionary: "../urbi.dict"
%%% End:

\section{Comparable}
Objects that can be compared for equality and inequality.

%%% Local Variables:
%%% mode: latex
%%% TeX-master: "../urbi-sdk"
%%% End:

%% Copyright (C) 2009-2010, Gostai S.A.S.
%%
%% This software is provided "as is" without warranty of any kind,
%% either expressed or implied, including but not limited to the
%% implied warranties of fitness for a particular purpose.
%%
%% See the LICENSE file for more information.

\section{Container}

This object is meant to be used as a prototype for objects that
support \lstinline|has| and \lstinline|hasNot| methods.
Any class using this prototype must redefine either has, hasNot or both.

\subsection{Prototypes}

\begin{itemize}
\item \refObject{Object}
\end{itemize}

\subsection{Slots}

\begin{itemize}
\item \lstinline|has(\var{e})|\\
  Return \lstinline|true| when the Container has a key or item matching
  \var{e}.
\item \lstinline|hasNot(\var{e})|\\
  Return \lstinline|true| when the Container has no key or item matching
  \var{e}.
\end{itemize}

%%% Local Variables:
%%% mode: latex
%%% TeX-master: "../urbi-sdk"
%%% ispell-dictionary: "american"
%%% ispell-personal-dictionary: "../urbi.dict"
%%% fill-column: 76
%%% End:

%% Copyright (C) 2010, Gostai S.A.S.
%%
%% This software is provided "as is" without warranty of any kind,
%% either expressed or implied, including but not limited to the
%% implied warranties of fitness for a particular purpose.
%%
%% See the LICENSE file for more information.

\section{Control}

\lstinline|Control| is designed as a namespace for control sequences.
Some of these entities are used by the \urbi engine to execute some
\us features; in other words, users are not expected to you use it,
much less change it.

\subsection{Prototypes}

\begin{refObjects}
\item[Object]
\end{refObjects}

\subsection{Slots}

\begin{urbiscriptapi}
\item[detach](<exp>)%
  Detach the evaluation of the expression \var{exp} from the current
  evaluation.  The \var{exp} is evaluated in parallel to the current code
  and keep the current tag which are attached to it.  Return the spawned
  \refObject{Job}.  Same as calling \refSlot[System]{spawn}:
  \lstinline|System.spawn(closure () { \var{exp} }, true|.

\begin{urbiscript}
{
  var jobs = [];
  var res = [];
  for (var i : [0, 1, 2])
  {
    jobs << detach({ res << i; res << i }) |
    if (i == 2)
      break
  };
  assert (res == [0, 1, 0]);
  jobs
};
[00009120] [Job<shell_11>, Job<shell_12>, Job<shell_13>]
\end{urbiscript}


\item[disown](<exp>)%%
  Same as \refSlot{detach} except that tags used to tag the
  \lstinline|disown| call are not inherited inside the expression.  Return
  the spawned \refObject{Job}.  Same as calling \refSlot[System]{spawn}:
  \lstinline|System.spawn(closure () { \var{exp} }, true|.

\begin{urbiscript}
{
  var jobs = [];
  var res = [];
  for (var i : [0, 1, 2])
  {
    jobs << disown({ res << i; res << i }) |
    if (i == 2)
      break
  };
  jobs.each (function (var j) { j.waitForTermination });
  assert (res == [0, 1, 0, 2, 1, 2]);
  jobs
};
[00009120] [Job<shell_14>, Job<shell_15>, Job<shell_16>]
\end{urbiscript}


\item[persist](<expression>, <delay>)%
  Return an object whose \var{val} slot evaluates to true if the
  \var{expression} has been continuously true for this \var{delay} and false
  otherwise.

  This function is used to implement

\begin{urbiunchecked}
at (condition ~ delay)
  action
\end{urbiunchecked}

  \noindent
  as

\begin{urbiunchecked}
var u = persist (condition, delay);
at (u.val)
  action
\end{urbiunchecked}

  The \lstinline|persist| action will be controlled by the same tags
  as the initial \lstinline|at| block.



\end{urbiscriptapi}


%%% Local Variables:
%%% coding: utf-8
%%% mode: latex
%%% TeX-master: "../urbi-sdk"
%%% ispell-dictionary: "american"
%%% ispell-personal-dictionary: "../urbi.dict"
%%% fill-column: 76
%%% End:

%% Copyright (C) 2009-2010, Gostai S.A.S.
%%
%% This software is provided "as is" without warranty of any kind,
%% either expressed or implied, including but not limited to the
%% implied warranties of fitness for a particular purpose.
%%
%% See the LICENSE file for more information.

\section{Date}

This class is meant to record dates in time, with microsecond resolution.
\experimental{}

\subsection{Prototypes}
\begin{refObjects}
\item[Orderable]
\item[Comparable]
\end{refObjects}

\subsection{Construction}

Without argument, newly constructed Dates refer to the current date.

\begin{urbiunchecked}[firstnumber=1]
Date.new;
[00000001] 2010-08-17 14:40:52.549726
\end{urbiunchecked}

With a string argument \var{d}, refers to the date contained in \var{d}.
The string should be formatted as \samp{\var{yyyy}-\var{mm}-\var{dd}
    \var{hh}:\var{mn}:\var{ss}} (see \refSlot{asString}). \var{mn}
and \var{ss} are optional. If the block \samp{\var{hh}:\var{mn}:\var{ss}}
is absent, the behavior is undefined.


\begin{urbiscript}
Date.new("2003-10-10 20:10:50");
[00000001] 2003-10-10 20:10:50

Date.new("2003-Oct-10 20:10");
[00000002] 2003-10-10 20:10:00

Date.new("2003-10-10 20");
[00000003] 2003-10-10 20:00:00
\end{urbiscript}

\subsection{Slots}

\begin{urbiscriptapi}
\item['+'](<that>)%
  The date which corresponds to waiting \refObject{Duration} \var{that}
  after \this.
\begin{urbiassert}
Date.new("2010-08-17 12:00") + 60s == Date.new("2010-08-17 12:01");
\end{urbiassert}

\item['-'](<that>)%
  If \var{that} is a Date, the difference between \this and \var{that} as a
  \refObject{Duration}.
\begin{urbiassert}
Date.new("2010-08-17 12:01") - Date.new("2010-08-17 12:00") ==  60s;
Date.new("2010-08-17 12:00") - Date.new("2010-08-17 12:01") == -60s;
\end{urbiassert}

If \var{that} is a Duration or a Float, the corresponding Date.

\begin{urbiassert}
Date.new("2010-08-17 12:01") - 60s == Date.new("2010-08-17 12:00");
Date.new("2010-08-17 12:01") - 60s
  == Date.new("2010-08-17 12:01") - Duration.new(60s);
\end{urbiassert}

\item['=='](<that>)%
  Equality test.
\begin{urbiassert}
Date.new("2010-08-17 12:00") == Date.new("2010-08-17 12:00");
Date.new("2010-08-17 12:00") != Date.new("2010-08-17 12:01");
\end{urbiassert}

\item['<'](<that>)%
  Order comparison.
\begin{urbiassert}
   Date.new("2010-08-17 12:00") < Date.new("2010-08-17 12:01");
! (Date.new("2010-08-17 12:01") < Date.new("2010-08-17 12:00"));
\end{urbiassert}

\item[asFloat] The duration since the \refSlot{epoch}, as a Float.
\begin{urbiscript}
var d = Date.new("2002-01-20 23:59:59")|;
assert
{
  d.asFloat == d - d.epoch;
  d.asFloat.isA(Float);
};
\end{urbiscript}

\item[asString] Present as \samp{\var{yyyy}-\var{mm}-\var{dd}
    \var{hh}:\var{mn}:\var{ss}.\var{us}} where \var{yyyy} is the four-digit
  year, \var{mm} the three letters name of the month (Jan, Feb, ...),
  \var{dd} the two-digit day in the month (from 1 to 31), \var{hh} the
  two-digit hour (from 0 to 23), \var{mn} the two-digit number of minutes in
  the hour (from 0 to 59), and \var{ss} the two-digit number of seconds in
  the minute (from 0 to 59).
\begin{urbiassert}
Date.new("2009-02-14 00:31:30").asString == "2009-02-14 00:31:30";
\end{urbiassert}

\item[epoch]
  A fixed value, the ``origin of times'': January 1st 1970, at
  midnight.
\begin{urbiunchecked}
Date.epoch == Date.new("1970-01-01 00:00");
\end{urbiunchecked}

\item[now] The current date. Equivalent to Date.new.
\begin{urbiunchecked}
Date.now;
[00000000] 2012-03-02 15:31:42
\end{urbiunchecked}

\item[timestamp] Synonym for \refSlot{asFloat}.
\end{urbiscriptapi}


%%% Local Variables:
%%% mode: latex
%%% TeX-master: "../urbi-sdk"
%%% ispell-dictionary: "american"
%%% ispell-personal-dictionary: "../urbi.dict"
%%% fill-column: 76
%%% End:

\section{Dictionary}

A \dfn{dictionary} is an \dfn{associative array}, also known as a
\dfn{hash} in some programming languages.  They are arrays whose
indexes are strings.

In a way objects are dictionaries: one can use \lstinline|setSlot|,
\lstinline|updateSlot|, and \lstinline|getSlot|.  This is unsafe since
slots also contains value and methods that object depend upon to run
properly.

\subsection{Example}

The following session demonstrates the features of the Dictionary
objects.

\begin{urbiscript}
var d = Dictionary.new("one", 1, "two", 2);
[00000001] Dictionary {"one" => 1, "two" => 2}
for (var p : d)
  echo (p.first + " -> " + p.second);
[00000003] *** one -> 1
[00000002] *** two -> 2
"three" in d;
[00000004] false
d["three"] = d["one"] + d["two"] | {};
"three" in d;
[00000005] true
d.getWithDefault("four", 4);
[00000006] 4
\end{urbiscript}


\subsection{Prototypes}

\begin{itemize}
\item \refObject{Object}
\end{itemize}

\subsection{Construction}

Dictionnary are created like any other object. The constructor can
take couples of argument to store them in the dictionary.

\begin{urbiscript}[firstnumber=last]
Dictionary.new("one", 1, "two", 2);
[00000000] Dictionary {"one" => 1, "two" => 2}
\end{urbiscript}

\subsection{Methods}

\begin{itemize}
\item \lstinline|asBool|\\
  Negation of \lstinline|Dictionary.empty|.
\begin{urbiscript}[firstnumber=last]
assert(Dictionary.new.asBool == false);
assert(Dictionary.new.set("key", "value").asBool == true);
\end{urbiscript}

\item \lstinline|asList|\\
  Return the contents of the dictionary as a \refObject{Pair} list
  (\var{key}, \var{value}).  This makes it easier to iterate over a
  Dictionary using \lstinline|for|.  No particular order is ensured.

\begin{urbiscript}[firstnumber=last]
Dictionary.new("one", 1, "two", 2).asList;
[00000000] [("one", 1), ("two", 2)]
\end{urbiscript}

\item \lstinline|clear|\\
  Empty the dictionary.

\begin{urbiscript}[firstnumber=last]
assert(Dictionary.new("one", 1).clear.empty());
\end{urbiscript}

\item \lstinline|empty|\\
  Whether the dictionary is empty.

\begin{urbiscript}[firstnumber=last]
assert(Dictionary.new.empty);
assert(!Dictionary.new.set("key", "value").empty);
\end{urbiscript}

\item \lstinline|erase(\var{key})|\\
  Remove the mapping for \lstinline|\var{key}|.

\begin{urbiscript}[firstnumber=last]
Dictionary.new("one", 1, "two", 2).erase("two");
[00386250] Dictionary {"one" => 1}
\end{urbiscript}

\item \lstinline|get(\var{key})|\\
  Return the value associated to  \lstinline|\var{key}| if it exists,
  \lstinline|void| otherwise.
\begin{urbiscript}[firstnumber=last]
assert(Dictionary.new("one", 1, "two", 2).get("one") == 1);
assert(Dictionary.new("one", 1, "two", 2).get("three").isVoid);
\end{urbiscript}


\item \lstinline|getWithDefault(\var{key}, \var{default-value})|\\
  Return the value associated to  \lstinline|\var{key}| if it exists,
  \lstinline|\var{default-value}| otherwise.

\begin{urbiscript}[firstnumber=last]
do (Dictionary.new("one", 1, "two", 2))
{
  assert(getWithDefault("one",  -1) == 1);
  assert(getWithDefault("three", 3) == 3);
}|;
\end{urbiscript}


\item \lstinline|has(\var{key})|\\
  Whether the dictionary has a mapping for \lstinline|\var{key}|.

\begin{urbiscript}[firstnumber=last]
do (Dictionary.new("one", 1))
{
  assert(has("one"));
  assert(!has("zero"));
}|;
\end{urbiscript}

\item \lstinline|init(\var{key1}, \var{value1}, ...)|~\\
  Insert the mapping from \lstinline|\var{key1}| to
  \lstinline|\var{value1}| and so forth.

\begin{urbiscript}[firstnumber=last]
Dictionary.clone.init("one", 1, "two", 2);
[00000000] Dictionary {"one" => 1, "two" => 2}
\end{urbiscript}

\item \lstinline|keys|\\
  The list of all the keys.  No particular order is ensured.
\begin{urbiscript}[firstnumber=last]
assert(Dictionary.new("one", 1, "two", 2).keys == ["one", "two"]);
\end{urbiscript}

\item \lstinline|set(\var{key}, \var{value})|\\
  Map \lstinline|\var{key}| to \lstinline|\var{value}| and return
  \lstinline|this| so that invocations to \lstinline|set| can be
  chained.  The possibly existing previous mapping is overriden.

\begin{urbiscript}[firstnumber=last]
Dictionary.new.set("one", 2).set("one", 1);
[00000000] Dictionary {"one" => 1}
\end{urbiscript}

\item \lstinline|[]=(\var{key}, \var{value})|\\
  Syntactic sugar for \lstinline|set(\var{key}, \var{value})|, but
  returns \var{value}.

\begin{urbiscript}[firstnumber=last]
{
  var d = Dictionary.new("one", "2");
  assert((d["one"] = 1) == 1);
  assert(d["one"] == 1);
};
\end{urbiscript}

\item \lstinline|[](\var{key})|\\
  Syntactic sugar for \lstinline|get(\var{key})|.

\begin{urbiscript}[firstnumber=last]
assert(Dictionary.new("one", 1)["one"] == 1);
\end{urbiscript}

\end{itemize}


%%% Local Variables:
%%% mode: latex
%%% TeX-master: "../urbi-sdk"
%%% End:

%% Copyright (C) 2009-2010, Gostai S.A.S.
%%
%% This software is provided "as is" without warranty of any kind,
%% either expressed or implied, including but not limited to the
%% implied warranties of fitness for a particular purpose.
%%
%% See the LICENSE file for more information.

\section{Directory}

A \dfn{Directory} represents a directory of the file system.

\subsection{Prototypes}
\begin{refObjects}
\item[Object]
\end{refObjects}

\subsection{Construction}

A \dfn{Directory} can be constructed with one argument: the path of
the directory using a \refObject{String} or a \refObject{Path}. It can
also be constructed by the method open of \refObject{Path}.

\begin{urbiscript}[firstnumber=1]
Directory.new(".");
[00000001] Directory(".")
Directory.new(Path.new("."));
[00000002] Directory(".")
\end{urbiscript}

\subsection{Slots}
\begin{urbiscriptapi}
\item[asList]
  The contents of the directory as a \refObject{Path} list.  The
  various paths include the name of the directory \lstinline|this|.

\item[content]
  The contents of the directory as a \refObject{String} list.  The
  strings include only the last component name; they do not contain
  the directory name of \lstinline|this|.

\item \lstinline|fileCreated(\var{name)|\\
  Event launched when a file is created inside the directory.  Be careful,
  this slot may not exists if the \us interpreter does not support for it.

\begin{urbiscript}
// watch the fileCreated event.
at (Directory.new(".").fileCreated?(var name))
  echo("File '%s' created." % name);

// Create a file to launch the fileCreated event.
File.create("./dummy-file.txt")|;

// wait until the event is reported.
sleep(1s);
[00000001] *** File 'dummy-file.txt' created.
\end{urbiscript}

\item \lstinline|fileDeleted(\var{name)|\\
  Event launched when a file is deleted from the directory.  Be careful,
  this slot may not exists if the \us interpreter does not support for it.

\begin{urbiscript}
// watch the fileDeleted event.
at (Directory.new(".").fileDeleted?(var name))
  echo("File '%s' deleted." % name);

// Remove a file to launch the fileDeleted event.
File.new("./dummy-file.txt").remove|;

// wait until the event is reported.
sleep(1s);
[00000001] *** File 'dummy-file.txt' deleted.
\end{urbiscript}


\end{urbiscriptapi}


%%% Local Variables:
%%% mode: latex
%%% TeX-master: "../urbi-sdk"
%%% ispell-dictionary: "american"
%%% ispell-personal-dictionary: "../urbi.dict"
%%% fill-column: 76
%%% End:

\section{Duration}

This class records differences between \refObject[s]{Date}.  It is
experimental, and very likely to be changed in the future.

\subsection{Prototypes}
\begin{itemize}
\item \refObject{Float}
\end{itemize}

\subsection{Construction}

Without argument, a null duration.

\begin{urbiscript}[firstnumber=1]
Duration.new;
[00000001] Duration(0s)
Duration.new(1h);
[00023593] Duration(3600s)
\end{urbiscript}

Durations can be negative.

\begin{urbiscript}
Duration.new(-1);
[00000001] Duration(-1s)
\end{urbiscript}


\subsection{Slots}

\begin{itemize}
\item \lstinline|asFloat|\\
  Return the duration as a \refObject{Float}.
\begin{urbiassert}
Duration.new(1000).asFloat == 1000;
\end{urbiassert}

\item \lstinline|asString|\\
  Return the duration as a \refObject{String}.
\begin{urbiassert}
Duration.new(1000).asString == "1000s";
\end{urbiassert}

\item \lstinline|seconds|\\
  Return the duration as a \refObject{Float}.
\begin{urbiassert}
Duration.new(1000).seconds == 1000;
\end{urbiassert}
\end{itemize}


%%% Local Variables:
%%% mode: latex
%%% TeX-master: "../urbi-sdk"
%%% ispell-dictionary: "american"
%%% ispell-personal-dictionary: "../urbi.dict"
%%% End:

\section{Event}
Entities that can be ``emited'' and ``caught'', or ``sent'' and
``received''.  See also \autoref{sec:tut:events}.

\subsection{Prototypes}
\begin{itemize}
\item \refObject{Object}
\end{itemize}

\subsection{Construction}

An \lstinline{Event} is created like any other object, without arguments.

\begin{urbiscript}
var e = Event.new;
[00000001] Event_0x9ad8118
\end{urbiscript}

\subsection{Methods}
\begin{itemize}
\item \lstinline|asEvent|\\
  Return \lstinline|this|.

\item \lstinline|'emit'|\\
  Throw an \lstinline|Event|. The operator bang can also be used. This
  function can take zero or more argument of same or different type.
  These argument are passed when the throwed event are caugth. An
  event can also be throwed for a certain time using \lstinline|~|.

\item \lstinline|syncEmit|\\
  Throw a synchronized event. This call waits that all functions that
  have to react to this event have returned. This function can have
  the same argument as \lstinline|emit|.

\item \lstinline|trigger|\\
  This function is used to launch an event during a not already known
  time. Calling this function launchs and keeps the event trigerred
  and returns an object that can be stopped with a method named stop
  to stop launching the event.
\end{itemize}

%%% Local Variables:
%%% mode: latex
%%% TeX-master: "../urbi-sdk"
%%% End:

%% Copyright (C) 2009-2010, Gostai S.A.S.
%%
%% This software is provided "as is" without warranty of any kind,
%% either expressed or implied, including but not limited to the
%% implied warranties of fitness for a particular purpose.
%%
%% See the LICENSE file for more information.

\section{Exception}

Exceptions are used to handle errors.  More generally, they are a
means to escape from the normal control-flow to handle exceptional
situations.

The language support for throwing and catching exceptions (using
\lstinline|try|/\lstinline|catch| and \lstinline|throw|, see
\autoref{sec:lang:except}) work perfectly well with any kind of
object, yet it is a good idea to throw only objects that derive from
\lstinline|Exception|.

\subsection{Prototypes}

\begin{refObjects}
\item[Object]
\item[Traceable]
\end{refObjects}

\subsection{Construction}

There are several types of exceptions, each of which corresponding to
a particular kind of error.  The top-level object,
\lstinline|Exception|, takes a single argument: an error message.

\begin{urbiscript}[firstnumber=1]
Exception.new("something bad has happened!");
[00000001] Exception `something bad has happened!'
Exception.Arity.new("myRoutine", 1, 10, 23);
[00000002] Exception.Arity `myRoutine: expected between 10 and 23 arguments, given 1'
\end{urbiscript}


\subsection{Slots}

Exception has many slots which are specific exceptions.  See
\autoref{sec:specs:except:sub} for their documentation.

\begin{urbiscriptapi}
\item[backtrace] The call stack at the moment the exception was thrown (not
  created), as a \refObject{List} of \refObject[s]{StackFrame}, from the
  innermost to the outermost call.  Uses \refSlot[Traceable]{backtrace}.
\begin{urbiscript}
//#push 1 "file.u"
try
{
  function innermost () { throw Exception.new("Ouch") };
  function inner     () { innermost() };
  function outer     () { inner() };
  function outermost () { outer() };

  outermost();
}
catch (var e)
{
  assert
  {
    e.backtrace[0].location.asString == "file.u:4.27-37";
    e.backtrace[0].name == "innermost";

    e.backtrace[1].location.asString == "file.u:5.27-33";
    e.backtrace[1].name == "inner";

    e.backtrace[2].location.asString == "file.u:6.27-33";
    e.backtrace[2].name == "outer";

    e.backtrace[3].location.asString == "file.u:8.3-13";
    e.backtrace[3].name == "outermost";
  };
};
//#pop
\end{urbiscript}

\item[location] The location from which the exception was thrown (not
  created).
\begin{urbiscript}
eval("1/0");
[00090441:error] !!! 1.1-3: /: division by 0
[00090441:error] !!!    called from: eval
try
{
  eval("1/0");
}
catch (var e)
{
  assert (e.location.asString == "1.1-3");
};
\end{urbiscript}

\item[message] The error message provided at construction.
\begin{urbiassert}
Exception.new("Ouch").message == "Ouch";
\end{urbiassert}
\end{urbiscriptapi}

\subsection{Specific Exceptions}
\label{sec:specs:except:sub}

In the following, since these slots are actually Objects, what is presented
as arguments to the slots are actually arguments to pass to the constructor
of the corresponding exception type.
\begin{urbiscriptapi}
\item[ArgumentType](<routine>, <index>, <effective>, <expected>)
  Derives from \refSlot{Type}.  The \var{routine} was
  called with a \var{index}-nth argument of type \var{effective}
  instead of \var{expected}.
\begin{urbiscript}
Exception.ArgumentType.new("myRoutine", 1, "hisResult", "Expectation");
[00000003] Exception.ArgumentType `myRoutine: unexpected "hisResult" for argument 1, expected a String'
\end{urbiscript}

\item[Arity](<routine>, <effective>, <min>, <max> = void)
  The \var{routine} was called with an incorrect number of arguments
  (\var{effective}).  It requires at least \var{min} arguments, and,
  if specified, at most \var{max}.
\begin{urbiscript}
Exception.Arity.new("myRoutine", 1, 10, 23);
[00000004] Exception.Arity `myRoutine: expected between 10 and 23 arguments, given 1'
\end{urbiscript}
%% try
%% {
%%   Math.cos(1, 2);
%% }
%% catch (var e)
%% {
%%   assert(e == Exception.Arity.new("cos", 2, 1));
%% };
\item[BadInteger](<routine>, <fmt>, <effective>)
  The \var{routine} was called with an inappropriate integer
  (\var{effective}).  Use the format \var{fmt} to create an error
  message from \var{effective}.  Derives from
  \refSlot{BadNumber}.
\begin{urbiscript}
Exception.BadInteger.new("myRoutine", "bad integer: %s", 12);
[00000005] Exception.BadInteger `myRoutine: bad integer: 12'
\end{urbiscript}

\item[BadNumber](<routine>, <fmt>, <effective>)
  The \var{routine} was called with an inappropriate number
  (\var{effective}).  Use the format \var{fmt} to create an error
  message from \var{effective}.
\begin{urbiscript}
Exception.BadNumber.new("myRoutine", "bad number: %s", 12.34);
[00000005] Exception.BadNumber `myRoutine: bad number: 12.34'
\end{urbiscript}

\item[Constness](<msg>)
  An attempt was made to change a constant value.
\begin{urbiscript}
Exception.Constness.new;
[00000006] Exception.Constness `cannot modify const slot'
\end{urbiscript}

\item[FileNotFound](<name>)
  The file named \var{name} cannot be found.
\begin{urbiscript}
Exception.FileNotFound.new("foo");
[00000007] Exception.FileNotFound `file not found: foo'
\end{urbiscript}

\item[ImplicitTagComponent](<msg>)
  An attempt was made to create an implicit tag, a component of which
  being undefined.
\begin{urbiscript}
Exception.ImplicitTagComponent.new;
[00000008] Exception.ImplicitTagComponent `invalid component in implicit tag'
\end{urbiscript}

\item[Lookup](<object>, <name>)
  A failed name lookup was performed om \var{object} to find a slot
  named \var{name}.  If \lstinline|Exception.Lookup.fixSpelling| is
  true (which is the default), suggest what the user might have meant
  to use.
\begin{urbiscript}
Exception.Lookup.new(Object, "GetSlot");
[00000009] Exception.Lookup `lookup failed: Object'
\end{urbiscript}

\item[MatchFailure]
  A pattern matching failed.
\begin{urbiscript}
Exception.MatchFailure.new;
[00000010] Exception.MatchFailure `pattern did not match'
\end{urbiscript}

\item[NegativeNumber](<routine>, <effective>)
  The \var{routine} was called with a negative number
  (\var{effective}).  Derives from \refSlot{BadNumber}.
\begin{urbiscript}
Exception.NegativeNumber.new("myRoutine", -12);
[00000005] Exception.NegativeNumber `myRoutine: expected non-negative number, got -12'
\end{urbiscript}

\item[NonPositiveNumber](<routine>, <effective>)
  The \var{routine} was called with a non-positive number
  (\var{effective}).  Derives from \refSlot{BadNumber}.
\begin{urbiscript}
Exception.NonPositiveNumber.new("myRoutine", -12);
[00000005] Exception.NonPositiveNumber `myRoutine: expected positive number, got -12'
\end{urbiscript}

\item[Primitive](<routine>, <msg>)
  The built-in \var{routine} encountered an error described by
  \var{msg}.
\begin{urbiscript}
Exception.Primitive.new("myRoutine", "cannot do that");
[00000011] Exception.Primitive `myRoutine: cannot do that'
\end{urbiscript}

\item[Redefinition](<name>)
  An attempt was made to refine a slot named \var{name}.
\begin{urbiscript}
Exception.Redefinition.new("foo");
[00000012] Exception.Redefinition `slot redefinition: foo'
\end{urbiscript}

\item[Scheduling](<msg>)
  Something really bad has happened with the \urbi task scheduler.
\begin{urbiscript}
Exception.Scheduling.new("cannot schedule");
[00000013] Exception.Scheduling `cannot schedule'
\end{urbiscript}

\item[Syntax](<loc>, <message>, <input>)
  Declare a syntax error in \var{input}, at location \var{loc},
  described by \var{message}.  \var{loc} is the location of the syntax
  error, \var{location} is the place the error was thrown.  They are
  usually equal, except when the errors are caught while using
  \refSlot[System]{eval} or \refSlot[System]{load}.  In that case
  \var{loc} is really the position of the syntax error, while
  \var{location} refers to the location of the \refSlot[System]{eval}
  or \refSlot[System]{load} invocation.
\begin{urbiscript}
Exception.Syntax.new(Location.new(Position.new("file.u", 14, 25)),
                     "unexpected pouCharque", "file.u");
[00000013] Exception.Syntax `file.u:14.25: syntax error: unexpected pouCharque'

try
{
  eval("1 / / 0");
}
catch (var e)
{
  assert
  {
    e.isA(Exception.Syntax);
    e.loc.asString == "1.5";
    e.input == "1 / / 0";
    e.message == "unexpected /";
  }
};
\end{urbiscript}


\item[Type](<effective>, <expected>)
  A value of type \var{effective} was received, while a value of type
  \var{expected} was expected.
\begin{urbiscript}
Exception.Type.new("hisResult", "Expectation");
[00000014] Exception.Type `unexpected "hisResult", expected a String'
\end{urbiscript}

\item[UnexpectedVoid] An attempt was made to read the value of
  \lstinline|void|.
\begin{urbiscript}
Exception.UnexpectedVoid.new;
[00000015] Exception.UnexpectedVoid `unexpected void'
var a = void;
a;
[00000016:error] !!! unexpected void
[00000017:error] !!! lookup failed: a
\end{urbiscript}

\end{urbiscriptapi}


%%% Local Variables:
%%% mode: latex
%%% TeX-master: "../urbi-sdk"
%%% ispell-dictionary: "american"
%%% ispell-personal-dictionary: "../urbi.dict"
%%% fill-column: 76
%%% End:

\section{Executable}

This class is used only as a common ancestor to \refObject{Primitive}
and \refObject{Code}.

\subsection{Prototypes}
\begin{itemize}
\item \refObject{Object}
\end{itemize}

\subsection{Construction}

There is no point in constructing an Executable.

\subsection{Slots}

\begin{urbiscriptapi}
\item[asExecutable] Return \lstinline|this|.
\end{urbiscriptapi}


%%% Local Variables:
%%% mode: latex
%%% TeX-master: "../urbi-sdk"
%%% ispell-dictionary: "american"
%%% ispell-personal-dictionary: "../urbi.dict"
%%% fill-column: 76
%%% End:

%% Copyright (C) 2009-2010, Gostai S.A.S.
%%
%% This software is provided "as is" without warranty of any kind,
%% either expressed or implied, including but not limited to the
%% implied warranties of fitness for a particular purpose.
%%
%% See the LICENSE file for more information.

\section{File}

\subsection{Prototypes}
\begin{refObjects}
\item[Object]
\end{refObjects}

\subsection{Construction}

Files may be created from a \refObject{String}, or from a \refObject{Path}.
Using \lstinline|new|, the file must exist on the file system, and must be a
file.  You may use \refSlot{create} to create a file that does not exist (or
to override an existing one).

\begin{urbiscript}[firstnumber=1]
File.create("file.txt");
[00000001] File("file.txt")

File.new(Path.new("file.txt"));
[00000001] File("file.txt")
\end{urbiscript}

You may use \refObject{InputStream} and \refObject{OutputStream} to
read or write to Files.

\subsection{Slots}

\begin{urbiscriptapi}
\item[asList]
  Read the file, and return its content as a list of its lines.
\begin{urbiscript}
File.save("file.txt", "1\n2\n");
assert(File.new("file.txt").asList == ["1", "2"]);
\end{urbiscript}

\item[asPrintable]
\begin{urbiscript}
File.save("file.txt", "1\n2\n");
assert(File.new("file.txt").asPrintable == "File(\"file.txt\")");
\end{urbiscript}

\item[asString]
  The name of the opened file.
\begin{urbiscript}
File.save("file.txt", "1\n2\n");
assert(File.new("file.txt").asString == "file.txt");
\end{urbiscript}

\item[content]
  The content of the file as a \refObject{Binary} object.
\begin{urbiscript}
File.save("file.txt", "1\n2\n");
assert
{
  File.new("file.txt").content == Binary.new("", "1\n2\n");
};
\end{urbiscript}

\item[create](<name>)%
  If the file \var{name} does not exist, create it and a return a File to
  it.  Otherwise, first empty it.  See \refObject{OutputStream} for methods
  to add content to the file.
\begin{urbiscript}
var p = Path.new("create.txt") |
assert (!p.exists);

// Create the file, and put something in it.
var f = File.create(p)|;
var o = OutputStream.new(f)|;
o << "Hello, World!"|;
o.close;

assert
{
  // The file exists, with the expect contents.
  p.exists;
  f.content.data == "Hello, World!";

  // If we create is again, it is empty.
  File.create(p).isA(File);
  f.content.data == "";
};
\end{urbiscript}

\item[remove]
  Remove the current file.  Returns void.
\begin{urbiscript}[firstnumber=1]
var p = Path.new("foo.txt") |
p.exists;
[00000002] false

var f = File.create(p);
[00000003] File("foo.txt")
p.exists;
[00000004] true

f.remove;
p.exists;
[00000006] false
\end{urbiscript}

\item[rename](<name>)%
  Rename the file to \var{name}.  If the target exists, it is replaced by
  the opened file.
\begin{urbiscript}
File.save("file.txt", "1\n2\n");
File.new("file.txt").rename("bar.txt");
assert
{
  !Path.new("file.txt").exists;
  File.new("bar.txt").content.data == "1\n2\n";
};
\end{urbiscript}

\item[save](<name>, <content>)
  Use \refSlot{create} to create the File named \var{name}, store the
  \var{content} in it, and close the file.  Return void.
\begin{urbiassert}
File.save("file.txt", "1\n2\n").isVoid;
File.new("file.txt").content.data == "1\n2\n";
\end{urbiassert}

\end{urbiscriptapi}


%%% Local Variables:
%%% mode: latex
%%% TeX-master: "../urbi-sdk"
%%% ispell-dictionary: "american"
%%% ispell-personal-dictionary: "../urbi.dict"
%%% fill-column: 76
%%% End:

\section{Finalizable}

Objects that derive from this object will execute their
\lstinline|finalize| routine right before being destroyed (reclaimed)
by the system.  It is comparable to the a \dfn{destructor}.

\subsection{Example}

The following object is set up to die verbosely.

\begin{urbiscript}
var obj =
  do (Finalizable.new)
  {
    function finalize ()
    {
      echo ("Ouch");
    }
  }|;
\end{urbiscript}

\noindent
It is reclaimed by the system when it is no longer referenced by any
other object.

\begin{urbiscript}
var alias = obj|;
obj = nil|;
\end{urbiscript}

\noindent
Here, the object is still alive, since \lstinline|alias| references
it.   Once it no longer does, the object dies.

\begin{urbiscript}
alias = nil|;
[00000004] *** Ouch
\end{urbiscript}

\subsection{Prototypes}

\begin{refObjects}
\item[Object]
\end{refObjects}

\subsection{Construction}

The constructor takes no argument.

\begin{urbiscript}
Finalizable.new;
[00000527] Finalizable_0x135360
\end{urbiscript}

\subsection{Slots}

\begin{urbiscriptapi}
\item[finalize] a simple function that takes no argument that will be
  evaluated when the object is reclaimed.  Its return value is
  ignored.
\begin{urbiscript}
Finalizable.new.setSlot("finalize", function() { echo("Ouch") })|;
[00033240] *** Ouch
\end{urbiscript}
\end{urbiscriptapi}

%%% Local Variables:
%%% mode: latex
%%% TeX-master: "../urbi-sdk"
%%% ispell-dictionary: "american"
%%% ispell-personal-dictionary: "../urbi.dict"
%%% End:

\section{Float}

Float is an \us primitive to represent floating point number.

\subsection{Prototypes}

\begin{itemize}
\item \refObject{Comparable}
\item \refObject{Orderable}
\end{itemize}

\subsection{Construction}

The most common way to create fresh floats is by using the literal
syntax presented in \autoref{sec:us-syn-lit-float}. A null float can also be
obtained with \lstinline|Float|'s \lstinline|new| method.

\begin{urbiscript}
Float.new;
[00000000] 0
\end{urbiscript}

\subsection{Methods}

\subsubsection{Float.abs}

Return the absolute value of the target.

\begin{urbiscript}
(-5).abs;
[00000000] 5
(5).abs;
[00000000] 5
\end{urbiscript}

\subsubsection{Float.acos}

Return the arccosine of the target.

\begin{urbiscript}
1.acos;
[00000000] 0
\end{urbiscript}

\subsubsection{Float.asin}

Return the arcsine of the target.

\begin{urbiscript}
0.asin;
[00000000] 0
\end{urbiscript}

\subsubsection{Float.asString}

Return a string representing the target.

\begin{urbiscript}
42.asString;
[00000000] "42"
\end{urbiscript}

\subsubsection{Float.atan}

Return the arctangent of the target.

\begin{urbiscript}
0.atan;
[00000000] 0
\end{urbiscript}

\subsubsection{Float.bitand}

Return the bitwise and between the target and the argument.

\begin{urbiscript}
3 bitand 6;
[00000000] 2
\end{urbiscript}

\subsubsection{Float.bitor}

Return the bitwise or between the target and the argument.

\begin{urbiscript}
3 bitor 6;
[00000000] 7
\end{urbiscript}

\subsubsection{Float.clone}

Return a fresh Float with the same value as the target.

\begin{urbiscript}
var x = 0;
[00000000] 0
var y = x.clone;
[00000000] 0
x === y;
[00000000] false
\end{urbiscript}

\subsubsection{Float.compl}

Return the 1-complement of the target.

\begin{urbiscript}
compl 0;
[00000000] 4294967295
compl 4294967295;
[00000000] 0
\end{urbiscript}

\subsubsection{Float.cos}

Return the cosine of the target.

\begin{urbiscript}
0.cos;
[00000000] 1
\end{urbiscript}

\subsubsection{Float.exp}

Return the exponential of the target.

\begin{urbiscript}
1.exp;
[00000000] 2.71828
\end{urbiscript}

\subsubsection{Float.inf}

Return the infinity.

\begin{urbiscript}
Float.inf;
[00000000] inf
\end{urbiscript}

\subsubsection{Float.log}

Return the logarithm of the target.

\begin{urbiscript}
2.71828.log;
[00000000] 0.999999
\end{urbiscript}

\subsubsection{Float.nan}

Return the ``not a number'' special float value.

\begin{urbiscript}
Float.nan;
[00000000] nan
\end{urbiscript}

\subsubsection{Float.random}

Return a random number between 0 and the target.

\begin{urbiscript}
5.random;
[00000000] 3
5.random;
[00000000] 1
5.random;
[00000000] 2
\end{urbiscript}

\subsubsection{Float.round}

Return the target, rounded.

\begin{urbiscript}
1.6.round;
[00000000] 2
1.4.round;
[00000000] 1
\end{urbiscript}

\subsubsection{Float.seq}

Return the sequence of integer from 0 to target - 1 as a list.

\begin{urbiscript}
3.seq;
[00000000] [0, 1, 2]
\end{urbiscript}

\subsubsection{Float.sin}

Return the sinus of the target.

\begin{urbiscript}
0.sin;
[00000000] 0
\end{urbiscript}

\subsubsection{Float.sqrt}

Return the square root of the target.

\begin{urbiscript}
1024.sqrt;
[00000000] 32
\end{urbiscript}

\subsubsection{Float.tan}

Return the tangent of the target.

\begin{urbiscript}[caption=Float.tan, label=lst:float-tan]
0.tan;
[00000000] 0
\end{urbiscript}

\subsubsection{Float.trunc}

Return the target truncated.

\begin{urbiscript}
1.9.trunc;
[00000000] 1
\end{urbiscript}

\subsubsection{Float.asFloat}

Return the target.

\begin{urbiscript}
51.asFloat;
[00000000] 51
\end{urbiscript}

\subsubsection{Float.sqr}

Return the square of the target.

\begin{urbiscript}
32.sqr;
[00000000] 1024
\end{urbiscript}

\subsubsection{Float.sgn}

Return 1 if the target is positive, -1 otherwise.

\begin{urbiscript}
(-1164).sgn;
[00000000] -1
(1164).sgn;
[00000000] 1
\end{urbiscript}

\subsubsection{Float.times}

Take one functional argument and call it target times.

\begin{urbiscript}
5.times(function () { echo("ping")});
[00000000] *** ping
[00000000] *** ping
[00000000] *** ping
[00000000] *** ping
[00000000] *** ping
\end{urbiscript}

\subsubsection{Float.each}

Take one functional argument and call it on every integer from 0 to
target - 1.

\begin{urbiscript}
5.each(function (i) { echo(i * 2)});
[00000000] *** 0
[00000000] *** 2
[00000000] *** 4
[00000000] *** 6
[00000000] *** 8
\end{urbiscript}

\subsubsection{Float.\^{}}

Return the bitwise exclusive or between the target and the first argument.

\begin{urbiscript}
3 ^ 6;
[00000000] 5
\end{urbiscript}

\subsubsection{Float.\textgreater\textgreater}

\lstinline|a >> b| return the \lstinline|a| shifted by \lstinline|b|
bit towards the right.

\begin{urbiscript}
4 >> 2;
[00000000] 1
\end{urbiscript}

\subsubsection{Float.\textless}

\lstinline|a < b| returns whether \lstinline|a| is inferior to
\lstinline|b|. Note that other comparison operators
(\lstinline|<=|, \lstinline|>|, \ldots) can thus also be applied on
floats since Float inherits \refObject{Orderable}.

\begin{urbiscript}
0 < 1;
[00000000] true
1 < 0;
[00000000] false
\end{urbiscript}

\subsubsection{Float.\textless\textless}

\lstinline|a << b| return the \lstinline|a| shifted by \lstinline|b|
bit towards the left.

\begin{urbiscript}
4 << 2;
[00000000] 16
\end{urbiscript}

\subsubsection{Float.$-$}

\lstinline|a - b| returns \lstinline|a| minus \lstinline|b|.

\begin{urbiscript}
6 - 3;
[00000000] 3
\end{urbiscript}

\subsubsection{Float.+}

\lstinline|a + b| returns \lstinline|a| plus \lstinline|b|.

\begin{urbiscript}
1 + 1;
[00000000] 2
\end{urbiscript}

\subsubsection{Float./}

\lstinline|a / b| returns the quotient of \lstinline|a| divided by
\lstinline|b|.

\begin{urbiscript}
50 / 10;
[00000000] 5
\end{urbiscript}

\subsubsection{Float.\%}

\lstinline|a % b|
returns \lstinline|a| modulo \lstinline|b|.

\begin{urbiscript}
50 % 11;
[00000000] 6
\end{urbiscript}

\subsubsection{Float.*}

\lstinline|a * b| returns the product of \lstinline|a| by
\lstinline|b|.

\begin{urbiscript}
2 * 3;
[00000000] 6
\end{urbiscript}

\subsubsection{Float.**}

\lstinline|a ** b| returns \lstinline|a| to the \lstinline|b| power.

\begin{urbiscript}
2 ** 10;
[00000000] 1024
\end{urbiscript}

\subsubsection{Float.==}

\lstinline|a == b| returns whether \lstinline|a| equals \lstinline|b|.

\begin{urbiscript}
1 == 1;
[00000000] true
1 == 2;
[00000000] false
\end{urbiscript}

%%% Local Variables:
%%% mode: latex
%%% TeX-master: "../urbi-sdk"
%%% End:

% LocalWords:  Orderable lst acos arccosine asin arcsine asString atan bitand
% LocalWords:  arctangent bitwise bitor compl sqrt trunc asFloat sqr sgn rshift
% LocalWords:  lshift eq

\input{specs/float-limits}
%% Copyright (C) 2009-2011, Gostai S.A.S.
%%
%% This software is provided "as is" without warranty of any kind,
%% either expressed or implied, including but not limited to the
%% implied warranties of fitness for a particular purpose.
%%
%% See the LICENSE file for more information.

\section{FormatInfo}

A \dfn{format info} is used when formatting a la \code{printf}. It
store the formatting pattern itself and all the format information it
can extract from the pattern.

\subsection{Prototypes}

\begin{refObjects}
\item[Object]
\end{refObjects}

\subsection{Construction}

The constructor expects a string as argument, whose syntax is similar
to \code{printf}'s.  It is detailed below.

\begin{urbiscript}[firstnumber=1]
FormatInfo.new("%+2.3d");
[00000001] %+2.3d
\end{urbiscript}

A formatting pattern must one of the following (brackets denote
optional arguments):
\begin{itemize}
\item \lstinline&%\var{rank}%&
\item \lstinline&%[\var{rank}$]\var{options} \var{spec}&
\item \lstinline&%|[\var{rank}$]\var{options}[\var{spec}]|&
\end{itemize}

\noindent
where:
\begin{itemize}
\item \var{rank} is an non-null integer which denotes a positional argument:
  a means to output arguments in a different order.

\item \var{options} is a sequence of 0 or several of the following
  characters:

\begin{center}
  \begin{tabular}{|c|l|}
    \hline
    \samp{-} & Left alignment.\\
    \samp{=} & Centered alignment.\\
    \samp{+} & Show sign even for positive number.\\
    \samp{ } & If the string does not begin with \samp{+} or \samp{-}, insert
    a space before the converted string.\\
    \samp{0} & Pad with 0's (inserted after sign or base indicator).\\
    \samp{\#} & Show numerical base, and decimal point.\\
    % \samp{'} & Split thousands (\samp{1 000}).\\
    \hline
  \end{tabular}
\end{center}


\item \var{spec} is the conversion character and must be one of the
  following:

\begin{center}
  \begin{tabular}{|c|l|}
    \hline
    \samp{s} & Default character, prints normally\\
    \samp{d} & Case modifier: lowercase \\
    \samp{D} & Case modifier: uppercase \\
    \samp{x} & Prints in hexadecimal lowercase \\
    \samp{X} & Prints in hexadecimal uppercase \\
    \samp{o} & Prints in octal\\
    % \samp{b} & Prints in binary\\
    \samp{e} & Prints floats in scientific format\\
    \samp{E} & Prints floats in scientific format uppercase\\
    \samp{f} & Prints floats in fixed format\\
    \hline
  \end{tabular}
\end{center}
\end{itemize}

When accepted, the format string is decoded, and its features are made
available as separate slots of the FormatInfo object.

\begin{urbiscript}
do (FormatInfo.new("%5$+'=#06.12X"))
{
  assert
  {
    rank      == 5;    // 5$
    prefix    == "+";  // +
    group     == " ";  // '
    alignment == 0;    // =
    alt       == true; // #
    pad       == "0";  // 0
    width     == 6;    // 6
    precision == 12;   // .12
    uppercase == 1;    // X
    spec      == "x";  // X
  };
}|;
\end{urbiscript}

Formats that do not conform raise errors.

\begin{urbiscript}
FormatInfo.new("foo");
[00000001:error] !!! new: format: pattern does not begin with %: foo

FormatInfo.new("%20m");
[00000002:error] !!! new: format: invalid conversion type character: m

FormatInfo.new("%");
[00000003:error] !!! new: format: trailing `%'

FormatInfo.new("%ss");
[00062475:error] !!! new: format: spurious characters after format: s

FormatInfo.new("%.ss");
[00071153:error] !!! new: format: invalid width after `.': s

FormatInfo.new("%|-8.2f|%%");
[00034983:error] !!! new: format: spurious characters after format: %%
\end{urbiscript}



\subsection{Slots}
\begin{urbiscriptapi}
\item[alignment]
  Requested alignment: \lstinline|-1| for left, \lstinline|0| for
  centered, \lstinline|1| for right (default).
\begin{urbiassert}
FormatInfo.new("%s") .alignment == 1;
FormatInfo.new("%=s").alignment == 0;
FormatInfo.new("%-s").alignment == -1;

 "%5s" % 1 == "    1";
"%=5s" % 2 == "  2  ";
"%-5s" % 3 == "3    ";
\end{urbiassert}


\item[alt]
  Whether the ``alternative'' display is requested (\samp{\#}).
\begin{urbiassert}
FormatInfo.new("%s") .alt == false;
FormatInfo.new("%#s").alt == true;

 "%s" % 12.3 == "12.3";
"%#s" % 12.3 == "12.3000";
 "%x" % 12 == "c";
"%#x" % 12 == "0xc";
\end{urbiassert}


\item[group]%
  Separator to use for thousands as a \refObject{String}.  Corresponds to
  the \samp{'} \var{option}.  Currently produces no effect at all.
\begin{urbiassert}
FormatInfo.new("%s") .group == "";
FormatInfo.new("%'s").group == " ";

"%d" % 123456 == "%'d" % 123456 == "123456";
\end{urbiassert}


\item[pad]
  The padding character to use for alignment requests.  Defaults to space.
\begin{urbiassert}
FormatInfo.new("%s") .pad == " ";
FormatInfo.new("%0s").pad == "0";

 "%5s" % 1 == "    1";
"%05s" % 1 == "00001";
\end{urbiassert}


\item[pattern]
  The pattern given to the constructor.
\begin{urbiassert}
FormatInfo.new("%#'12.8s").pattern == "%#'12.8s";
\end{urbiassert}


\item[precision]
  When formatting a \refObject{Float}, the maximum number of digits
  after decimal point when in fixed or scientific mode, and in total
  when in default mode.  When formatting other objects with spec-char
  \samp{s}, the conversion string is truncated to the precision first
  chars. The eventual padding to \lstinline|width| is done after
  truncation.
\begin{urbiassert}
FormatInfo.new("%s")    .precision == 6;
FormatInfo.new("%23.3s").precision == 3;

  "%f" % 12.3 == "12.300000";
"%.0f" % 12.3 == "12";
"%.2f" % 12.3 == "12.30";
"%.8f" % 12.3 == "12.30000000";
\end{urbiassert}


\item[prefix]
  The string to display before positive numbers.  Defaults to empty.
\begin{urbiassert}
FormatInfo.new("%s") .prefix == "";
FormatInfo.new("% s").prefix == " ";
FormatInfo.new("%+s").prefix == "+";
\end{urbiassert}


\item[rank]%
  In the case of a positional argument, its number, otherwise 0.
\begin{urbiassert}
FormatInfo.new("%s")   .rank == 0;
FormatInfo.new("%2$s") .rank == 2;
FormatInfo.new("%03$s").rank == 3;
FormatInfo.new("%4%")  .rank == 4;

 "%3$s%2$s%2$s%1$s" % ["bar", "o", "f"]
  == "%3%%2%%2%%1%" % ["bar", "o", "f"] == "foobar";
\end{urbiassert}
Cannot be null.
\begin{urbiscript}
FormatInfo.new("%00$s").rank;
[00001243:error] !!! new: format: invalid positional argument: 00
\end{urbiscript}

\item[spec]
  The specification character, regardless of the case conversion
  requests.
\begin{urbiassert}
FormatInfo.new("%s")    .spec == "s";
FormatInfo.new("%23.3s").spec == "s";
FormatInfo.new("%'X")   .spec == "x";
\end{urbiassert}


\item[uppercase]
  Case conversion: \lstinline|-1| for lower case, \lstinline|0| for no
  conversion (default), \lstinline|1| for conversion to uppercase.
  The value depends on the case of specification character, except for
  \samp{\%s} which corresponds to \lstinline|0|.
\begin{urbiassert}
FormatInfo.new("%s")  .uppercase ==  0;
FormatInfo.new("%d")  .uppercase == -1;
FormatInfo.new("%D")  .uppercase ==  1;
FormatInfo.new("%x")  .uppercase == -1;
FormatInfo.new("%X")  .uppercase ==  1;
FormatInfo.new("%|D|").uppercase ==  1;
FormatInfo.new("%|d|").uppercase == -1;
\end{urbiassert}


\item[width]
  Width requested for alignment.
\begin{urbiassert}
FormatInfo.new("%s")   .width == 0;
FormatInfo.new("%10s") .width == 10;
FormatInfo.new("%-10s").width == 10;
FormatInfo.new("%8.2f").width == 8;
\end{urbiassert}
\end{urbiscriptapi}

%%% Local Variables:
%%% coding: utf-8
%%% mode: latex
%%% TeX-master: "../urbi-sdk"
%%% ispell-dictionary: "american"
%%% ispell-personal-dictionary: "../urbi.dict"
%%% fill-column: 76
%%% End:

%% Copyright (C) 2009-2011, Gostai S.A.S.
%%
%% This software is provided "as is" without warranty of any kind,
%% either expressed or implied, including but not limited to the
%% implied warranties of fitness for a particular purpose.
%%
%% See the LICENSE file for more information.

\section{Formatter}

A \dfn{formatter} stores format information of a format string like
used in \code{printf} in the C library or in \code{boost::format}.

\subsection{Prototypes}

\begin{refObjects}
\item[Object]
\end{refObjects}

\subsection{Construction}

Formatters are created from format strings: they are split into regular
strings and formatting patterns (\refObject{FormatInfo}), and stores them.

\begin{urbiscript}[firstnumber=1]
Formatter.new("Name:%s, Surname:%s;");
[00000001] Formatter ["Name:", %s, ", Surname:", %s, ";"]
\end{urbiscript}

All the patterns are introduced with the percent character (\lstinline|%|),
and they must conform to a specific syntax, detailed in the section on the
construction of the \refObject{FormatInfo}.  To denote the percent character
instead of introducing a formatting-specification, use two percent
characters.

\begin{urbiscript}
var f = Formatter.new("%10s level: %-4.1f%%");
[00039525] Formatter [%10s, " level: ", %-4.1f, "%"]

for (var d: ["Battery" => 83.3, "Sound" => 60])
  echo (f % d.asList());
[00041133] ***    Battery level: 83.3%
[00041138] ***      Sound level: 60  %
\end{urbiscript}
\begin{urbicomment}
removeSlots("f");
\end{urbicomment}

Patterns can either all be non-positional (e.g., \lstinline|%s%s|), or all
positional (e.g., \lstinline|%1$s%2$s|).

\begin{urbiscript}
Formatter.new("%s%s");
[00371506] Formatter [%s, %s]
Formatter.new("%1$s%2$s");
[00385602] Formatter [%1$s, %2$s]

Formatter.new("%1$s%s");
[00409657:error] !!! new: format: cannot mix positional and non-positional\
[:] arguments: %1$s vs. %s
\end{urbiscript}

\subsection{Slots}

\begin{urbiscriptapi}
\item[asList]
  Return the content of the \dfn{formatter} as a list of strings and
  \refObject{FormatInfo}.
\begin{urbiassert}
Formatter.new("Name:%s, Surname:%s;").asList().asString()
       == "[\"Name:\", %s, \", Surname:\", %s, \";\"]";
\end{urbiassert}

\item \lstinline|'%'(\var{args})|\\
  Use \this as format string and \var{args} as the list of arguments, and
  return the result (a \refObject{String}).

  This operator concatenates regular strings and the strings that are result
  of \lstinline|asString| called on members of \var{args} with the
  appropriate \refObject{FormatInfo}.
\begin{urbiassert}
Formatter.new("=>") % [] == "=>";
Formatter.new("=> %s") % [1] == "=> 1";
Formatter.new("Name:%s, Surname:%s;") % ["Foo", "Bar"]
       == "Name:Foo, Surname:Bar;";
\end{urbiassert}

The arity of the Formatter (i.e., the number of expected arguments) and the
size of \var{args} must match exactly.

\begin{urbiscript}
var f = Formatter.new("%s")|;
f % [];
[00000002:error] !!! %: format: too few arguments

f% ["foo", "bar"];
[00000004:error] !!! %: format: too many arguments
\end{urbiscript}
\begin{urbicomment}
removeSlots("f");
\end{urbicomment}

  If \var{args} is not a \refObject{List}, then the call is equivalent
  to calling \lstinline|'%'([\var{args}])|.
\begin{urbiassert}
Formatter.new("%06.3f") % Math.pi
       == "03.142";
\end{urbiassert}

  Note that \lstinline|String.'%'| provides a nicer interface to this
  operator:
\begin{urbiassert}
"%06.3f" % Math.pi == "03.142";
\end{urbiassert}

  It is nevertheless interesting to use the Formatter for performance
  reasons if the format is reused many times.
\begin{urbiscript}
// Some large database of people.
var people =
  [["Foo", "Bar" ],
   ["One", "Two" ],
   ["Un",  "Deux"],]|;

var f = Formatter.new("Name:%7s, Surname:%7s;")|;
for (var p: people)
  echo (f % p);
[00031939] *** Name:    Foo, Surname:    Bar;
[00031940] *** Name:    One, Surname:    Two;
[00031941] *** Name:     Un, Surname:   Deux;
\end{urbiscript}
\end{urbiscriptapi}

%%% Local Variables:
%%% coding: utf-8
%%% mode: latex
%%% TeX-master: "../urbi-sdk"
%%% ispell-dictionary: "american"
%%% ispell-personal-dictionary: "../urbi.dict"
%%% fill-column: 76
%%% End:

\section{Global}

\fixme{Lots of work to do here.}

\subsection{Prototypes}
\begin{itemize}
\item \refObject{System}
\end{itemize}

\subsection{Methods}
%\begin{itemize}
%\end{itemize}


%%% Local Variables:
%%% mode: latex
%%% TeX-master: "../urbi-sdk"
%%% End:

\section{Group}
A transparent means to send messages to several objects as if they
were one.

\begin{itemize}
\item \lstinline|asString|\\
  Report the members.
\item \lstinline|add(\var{member}, ...)|\\
  Add members to \lstinline|this| group, and return \lstinline|this|.
\item \lstinline|fallback|\\
  This function is called when a method call on \lstinline|this|
  failed.  It bounces the call to the members of the group, collects
  the results returned as a group.  This allows to chain grouped
  operation in a row.  If the dispatched calls return
  \lstinline|void|, returns a single \lstinline|void|, not a ``group
  of \lstinline|void|''.
\item \lstinline|getProperty(\var{slot}, \var{prop})|\\
  Bounced to the members so that
  \lstinline|this.var{slot}->\var{prop}| actually collects the values
  of the property \var{prop} of the slots \var{slot} of the group
  members.
\item \lstinline|hasProperty(\var{name})|\\
  Bounced to the members.
\item \lstinline|remove(\var{member}, ...)|\\
  Remove members from \lstinline|this| group, and return
  \lstinline|this|.
\item \lstinline|setProperty(\var{slot}, \var{prop}, \var{value})|\\
  Bounced to the members so that
  \lstinline|this.var{slot}->\var{prop} = \var{value}|
  actually updates the value of the property \var{prop}
  in the slots \var{slot} of the group members.
\item \lstinline|updateSlot(\var{name}, \var{value})|\\
  Bounced to the members so that
  \lstinline|this.\var{name} = \var{value}|
  actually updates the value of the slot \var{name} in
  the group members.
\item \lstinline|<< \var{member}|\\
  Syntactic sugar for \lstinline|add|.
\end{itemize}

\begin{urbiscript}
class Sample
{
  var value = 0;
  function init(v) { value = v; };
  function asString() { "<" + value.asString + ">"; };
  function timesTen() { new(value * 10); };
  function plusTwo()  { new(value + 2); };
};
[00000000] <0>

var group = Group.new(Sample.new(1), Sample.new(2));
[00000000] Group [<1>, <2>]
group << Sample.new(3);
[00000000] Group [<1>, <2>, <3>]
group.timesTen.plusTwo;
[00000000] Group [<12>, <22>, <32>]

// Bouncing getSlot and updateSlot.
group.value;
[00000000] Group [1, 2, 3]
group.value = 10;
[00000000] Group [10, 10, 10]
\end{urbiscript}

%%% Local Variables:
%%% mode: latex
%%% TeX-master: "../urbi-sdk"
%%% End:

%% Copyright (C) 2009-2010, Gostai S.A.S.
%%
%% This software is provided "as is" without warranty of any kind,
%% either expressed or implied, including but not limited to the
%% implied warranties of fitness for a particular purpose.
%%
%% See the LICENSE file for more information.

\section{InputStream}

InputStreams are used to read (possibly binary) files by hand.
\refObject{File} provides means to swallow a whole file either as a
single large string, or a list of lines.  \lstinline|InputStream|
provides a more fine-grained interface to read files.

\subsection{Prototypes}
\begin{refObjects}
\item[Object]
\end{refObjects}

\begin{windows}
  Beware that because of limitations in the current implementation,
  one cannot safely read from two different files at the same time
  under Windows.
\end{windows}

\subsection{Construction}

An InputStream is a reading-interface to a file, so its constructor
requires a \refObject{File}.

\begin{urbiscript}[firstnumber=1]
File.save("file.txt", "1\n2\n");
var is = InputStream.new(File.new("file.txt"));
[00000001] InputStream_0x827000
\end{urbiscript}

Bear in mind that open streams should be closed
(\autoref{sec:specs:output-stream:ctor}).

\begin{urbiscript}
is.close;
\end{urbiscript}

\subsection{Slots}

\begin{urbiscriptapi}
\item[close] Close the stream, return void.
\begin{urbiassert}
InputStream.new(File.create("file.txt")).close.isVoid;
\end{urbiassert}

\item[get]
  Get the next available byte as a \refObject{Float}, or
  \lstinline|void| if the end of file was reached.
\begin{urbiscript}
{
  File.save("file.txt", "1\n2\n");
  var i = InputStream.new(File.new("file.txt"));
  var x;
  while (!(x = i.get.acceptVoid).isVoid)
    cout << x;
  i.close;
};
[00000001:output] 49
[00000002:output] 10
[00000003:output] 50
[00000004:output] 10
\end{urbiscript}

\item[getChar]
  Get the next available byte as a \refObject{String}, or
  \lstinline|void| if the end of file was reached.
\begin{urbiscript}
{
  File.save("file.txt", "1\n2\n");
  var i = InputStream.new(File.new("file.txt"));
  var x;
  while (!(x = i.getChar.acceptVoid).isVoid)
    cout << x;
  i.close;
};
[00000001:output] "1"
[00000002:output] "\n"
[00000003:output] "2"
[00000004:output] "\n"
\end{urbiscript}

\item[getLine]
  Get the next available line as a \refObject{String}, or
  \lstinline|void| if the end of file was reached.  The end-of-line
  characters are trimmed.
\begin{urbiscript}
{
  File.save("file.txt", "1\n2\n");
  var i = InputStream.new(File.new("file.txt"));
  var x;
  while (!(x = i.getLine.acceptVoid).isVoid)
    cout << x;
  i.close;
};
[00000001:output] "1"
[00000002:output] "2"
\end{urbiscript}
\end{urbiscriptapi}


%%% Local Variables:
%%% mode: latex
%%% TeX-master: "../urbi-sdk"
%%% ispell-dictionary: "american"
%%% ispell-personal-dictionary: "../urbi.dict"
%%% fill-column: 76
%%% End:

%% Copyright (C) 2010, Gostai S.A.S.
%%
%% This software is provided "as is" without warranty of any kind,
%% either expressed or implied, including but not limited to the
%% implied warranties of fitness for a particular purpose.
%%
%% See the LICENSE file for more information.

\section{IoService}

A \dfn{IoService} is used to manage the various operations of a set of
\refObject{Socket}.

All \refObject{Socket} and \refObject{Server} are by default using the
default \refObject{IoService} which is polled regularly by the system.

\subsection{Example}

Using a different \refObject{IoService} is required if you need to perform
synchronous read operations.

The \refObject{Socket} must be created by the \refObject{IoService} that will
handle it using its \lstinline|makeSocket| function.

\begin{urbiscript}
var io = IoService.new|;
var s = io.makeSocket|;
\end{urbiscript}

You can then use this socket like any other.

\begin{urbiscript}
// Make a simple hello server.
var serverPort = 0|
do(Server.new)
{
  listen("127.0.0.1", "0");
  lobby.serverPort = port;
  at(connection?(var s))
  {
    s.write("hello");
  }
}|;
// Connect to it using our socket.
s.connect("0.0.0.0", serverPort);
at(s.received?(var data))
  echo("received something");
s.write("1;");
\end{urbiscript}

... except that nothing will be read from the socket unless you call one of the
\lstinline|poll| functions of \lstinline|io|.

\begin{urbiscript}
sleep(200ms);
s.isConnected(); // Nothing was received yet
[00000001] true
io.poll();
[00000002] *** received something
sleep(200ms);
\end{urbiscript}

\subsection{Prototypes}
\begin{refObjects}
\item[Object]
\end{refObjects}

\subsection{Construction}

A \lstinline|IoService| is constructed with no argument.

\subsection{Slots}

\begin{urbiscriptapi}
\item[makeServer]
  Create and return a new \refObject{Server} using this \refObject{IoService}.

\item[makeSocket]
  Create and return a new \refObject{Socket} using this \refObject{IoService}.

\item[poll]
  Handle all pending socket operations(read, write, accept) that can be
  performed without waiting.

\item[pollFor](<duration>)%
  Will block for \var{duration} seconds, and handle all ready socket operations
  during this period.

\item[pollOneFor](<duration>)%
  Will block for at most \var{duration}, and handle the first ready socket
  operation and immediately return.

\end{urbiscriptapi}

%%% Local Variables:
%%% coding: utf-8
%%% mode: latex
%%% TeX-master: "../urbi-sdk"
%%% ispell-dictionary: "american"
%%% ispell-personal-dictionary: "../urbi.dict"
%%% fill-column: 76
%%% End:

%% Copyright (C) 2010, 2011, Gostai S.A.S.
%%
%% This software is provided "as is" without warranty of any kind,
%% either expressed or implied, including but not limited to the
%% implied warranties of fitness for a particular purpose.
%%
%% See the LICENSE file for more information.

\section{Job}

Jobs are independent threads of executions.  Jobs can run concurrently.
They can also be managed using \refObject[Tag]{Tags}.

\subsection{Prototypes}

\begin{refObjects}
\item[Object]
\item[Traceable]
\end{refObjects}

\subsection{Construction}

A Job is typically constructed via \refSlot[Control]{detach},
\refSlot[Control]{disown}, or \refSlot[Code]{spawn}.

\begin{urbiscript}
detach(sleep(10));
[00202654] Job<shell_4>

disown(sleep(10));
[00204195] Job<shell_5>

function () { sleep(10) }.spawn(false);
[00274160] Job<shell_6>
\end{urbiscript}

\subsection{Slots}

\begin{urbiscriptapi}
\item[asJob]
  Return \lstinline|this|.
\begin{urbiassert}
Job.current.asJob === Job.current;
\end{urbiassert}


\item[asString] The string \lstinline|Job<\var{name}>| where \var{name} is
  the name of the job.
\begin{urbiassert}
Job.current.asString == "Job<shell>";
var a = function () { sleep(1) }.spawn(false);
a.asString == "Job<" + a.name + ">";
\end{urbiassert}


\item[backtrace] The current backtrace of the job as a \refObject{List} of
  \refObject[StackFrame]{StackFrames} starting from innermost, to outermost.
  Uses \refSlot[Traceable]{backtrace}.

\begin{urbiscript}
//#push 100 "foo.u"
function innermost () { Job.current.backtrace }|;
function inner ()     { innermost }|;
function outer ()     { inner }|;
function outermost () { outer }|;
echoEach(outermost);
[00001732] *** foo.u:100.25-45: backtrace
[00001732] *** foo.u:101.25-33: innermost
[00001732] *** foo.u:102.25-29: inner
[00001732] *** foo.u:103.25-29: outer
[00001732] *** foo.u:104.10-18: outermost
//#pop

//#push 1 "file.u"
var s = detach(sleep(1))|;
// Leave some time for s to be started.
sleep(100ms);
assert
{
  s.backtrace[0].asString == "file.u:1.16-23: sleep";
  s.backtrace[1].asString == "file.u:1.9-24: detach";
};
// Leave some time for sleep to return.
sleep(1);
assert
{
  s.backtrace.size == 1;
  s.backtrace[0].asString == "file.u:1.9-24: detach";
};
//#pop
\end{urbiscript}

  In the case of events, the backtrace is composite: the bottom part
  corresponds to the location of the event handler, and the top part is the
  location of the event emission.  The event emission is denoted by
  \refSlot[Event]{emit} rather than \lstinline|!|, and its reception by
  \refSlot[Event]{onEvent} rather than \lstinline|?| or \lstinline|at|.

\begin{urbiscript}
//#push 1 "file.u"
var e = Event.new|;
function showBacktrace(var where)
{
  echo("==================== " + where)|
  echoEach(Job.current.backtrace);
}|;
at (e?)
  showBacktrace("at-enter")
onleave
  showBacktrace("at-leave");
e!;
[00000647] *** ==================== at-enter
[00000647] *** file.u:5.12-32: backtrace
[00000647] *** file.u:8.3-27: showBacktrace
[00000647] *** file.u:11.1: emit
[00000647] *** ---- event handler backtrace:
[00000647] *** file.u:7.1-10.27: onEvent
[00000647] *** ==================== at-leave
[00000647] *** file.u:5.12-32: backtrace
[00000647] *** file.u:10.3-27: showBacktrace
[00000647] *** file.u:11.1: emit
[00000647] *** ---- event handler backtrace:
[00000647] *** file.u:7.1-10.27: onEvent


//#push 1 "file.u"
var f = Event.new|;
function watchEvent()
{
  at (f?)
    showBacktrace("at-enter")
  onleave
    showBacktrace("at-leave");
}|;
function sendEvent()
{
  f!;
}|;
watchEvent;
sendEvent;
[00000654] *** ==================== at-enter
[00000654] *** file.u:5.12-32: backtrace
[00000654] *** file.u:5.5-29: showBacktrace
[00000654] *** file.u:11.3: emit
[00000654] *** file.u:14.1-9: sendEvent
[00000654] *** ---- event handler backtrace:
[00000654] *** file.u:4.3-7.29: onEvent
[00000654] *** file.u:13.1-10: watchEvent
[00000654] *** ==================== at-leave
[00000654] *** file.u:5.12-32: backtrace
[00000654] *** file.u:7.5-29: showBacktrace
[00000654] *** file.u:11.3: emit
[00000654] *** file.u:14.1-9: sendEvent
[00000654] *** ---- event handler backtrace:
[00000654] *** file.u:4.3-7.29: onEvent
[00000654] *** file.u:13.1-10: watchEvent
\end{urbiscript}


\item[clone]
  Cloning a job is forbidden.


\item[current]%
  The \refObject{Job} in charge of executing this ``thread'' of code.
\begin{urbiscript}
var j = Job.current|;
assert { j.isA(Job) };

var j1; var j2;

// Both sequential compositions use the same job: the current one.
{ j1 = Job.current; j2 = Job.current }|;
assert { j1 == j2; j1 == j };

{ j1 = Job.current | j2 = Job.current }|;
assert { j1 == j2; j1 == j };

// Concurrency requires several jobs: j1 and j2 are different.
//
// As an optimization, the current job is used for one of these
// jobs.  This is an implementation detail, do not rely on it.
{ j1 = Job.current & j2 = Job.current }|;
assert { j1 != j2; j1 == j; j2 != j };

{ j1 = Job.current, j2 = Job.current }|;
assert { j1 != j2; j1 != j; j2 == j };
\end{urbiscript}


\item[dumpState]
  Pretty-print the state of the job.

\begin{urbiscript}
//#push 1 "file.u"
var t = detach(sleep(1))|;
// Leave some time for s to be started.
sleep(100ms);
t.dumpState;
[00004295] *** Job: shell_10
[00004295] ***   State: sleeping
[00004295] ***   Tags:
[00004295] ***     Tag<Lobby_1>
[00004297] ***   Backtrace:
[00004297] ***     file.u:1.16-23: sleep
[00004297] ***     file.u:1.9-24: detach
//#pop
\end{urbiscript}


\item[jobs]%
  All the existing jobs as a \refObject{List}.
\begin{urbiassert}
Job.jobs.isA(List);
Job.current in Job.jobs;
\end{urbiassert}


\item[name] The name of the job as a \refObject{String}.
\begin{urbiscript}
Job.current.name;
[00004293] "shell"
detach(sleep(1)).name;
[00004297] "shell_11"
\end{urbiscript}


\item[resetStats]%
  Reinitialize the \refSlot{stats} computation.

\item[tags] The list of \refObject[Tag]{Tags} that manage this job.

\item[stats]%
  Return a \refObject{Dictionary} containing information about the execution
  cycles of \urbi.  This is an internal feature made for developers, it
  might be changed without notice.  See also \refSlot{resetStats}.
\begin{urbicomment}
removeSlots("j");
\end{urbicomment}
\begin{urbiscript}
var j = detach({
  // Each ';' increment the Cycles with their execution time.
  var i = 0;
  {
    // Increment the Waiting time.
    waituntil(i == 1);
    // Increment the Sleeping time.
    sleep(100ms);
    i = 2;
  }, // Will fork and join a child job.

  sleep(200ms);
  i = 1;
  waituntil(i != 1);

  // Stop breaks the workflow of each job tagged with the tag.
  var t = Tag.new;
  t: t.stop(21);
})|;
j.waitForTermination;
var stats = j.stats|;

Float.epsilonTilde = 0.01 |;
assert
{
  0 < stats["Cycles"];
  stats["CyclesMin"] <= stats["CyclesMean"] <= stats["CyclesMax"];
  stats["WaitingMin"] <= stats["WaitingMean"] <= stats["WaitingMax"] ~= 100ms;
  stats["SleepingMin"] <= stats["SleepingMean"] <= stats["SleepingMax"] ~= 200ms;
  stats["Fork"] == stats["Join"] == 1;
  stats["WorkflowBreak"] == 1;

  stats["TerminatedChildrenWaitingMean"] ~= 200ms;
  stats["TerminatedChildrenSleepingMean"] ~= 100ms;
};
\end{urbiscript}


\item[status] A \refObject{String} that describes the current status of the
  job (starting, running, \ldots), and its properties (frozen, \ldots).
\begin{urbiassert}
var t = detach(sleep(10));
Job.current.status == "running (current job)";
t.status == "sleeping" ;
\end{urbiassert}


\item[tags] The list of \refObject[Tag]{Tags} that manage this job.


\item[terminate]  Kill this job.
\begin{urbiscript}
var r = detach({ sleep(1s); echo("done") })|;
assert (r in Job.jobs);
r.terminate;
assert (r not in Job.jobs);
sleep(2s);
\end{urbiscript}


\item[timeShift]
  Get the total amount of time during which we were frozen.
\begin{urbiscript}
tag: r = detach({ sleep(3); echo("done") })|;
tag.freeze();
sleep(2);
tag.unfreeze();
Math.round(r.timeShift);
[00000001] 2
\end{urbiscript}


\item[waitForTermination] Wait for the job to terminate before resuming
  execution of the current one.  If the job has already terminated, return
  immediately.
\end{urbiscriptapi}


%%% Local Variables:
%%% coding: utf-8
%%% mode: latex
%%% TeX-master: "../urbi-sdk"
%%% ispell-dictionary: "american"
%%% ispell-personal-dictionary: "../urbi.dict"
%%% fill-column: 76
%%% End:

\section{Kernel1}

This object is actually meant to play the role of a name-space in
which obsolete functions from \us 1.0 are provided for backward
compatibility.  Do not use these functions, scheduled for removal.

\subsection{Prototypes}
\begin{itemize}
\item \refObject{Singleton}
\end{itemize}

\subsection{Construction}

Since it is a \refObject{Singleton}, you are not expected to build
other instances.

\subsection{Slots}

\begin{itemize}
\item \lstinline|commands|\\
  Ignored for backward compatibility.

\item \lstinline|connections|\\
  Ignored for backward compatibility.

\item \lstinline|copy(\var{binary})|\\
  Obsolete syntax for \lstinline|\var{binary}.copy|, see
  \refObject{Binary}.

\item \lstinline|devices|\\
  Ignored for backward compatibility.

\item \lstinline|events|\\
  Ignored for backward compatibility.

\item \lstinline|functions|\\
  Ignored for backward compatibility.

\item \lstinline|isvoid(\var{obj})|\\
  Obsolete syntax for \lstinline|\var{obj}.isVoid|, see
  \refObject{Object}.

\item \lstinline|noop|\\
  Do nothing.  Use \lstinline|{}| instead.

\item \lstinline|ping|\\
  Return \lstinline|time| verbosely, see \refObject{System}.
\begin{urbiscript}
Kernel1.ping;
[00000421] *** pong time=0.12s
\end{urbiscript}

\item \lstinline|reset|\\
  Ignored for backward compatibility.

\item \lstinline|runningcommands|\\
  Ignored for backward compatibility.

\item \lstinline|seq(\var{number})|\\
  Obsolete syntax for \lstinline|\var{number}.seq|, see
  \refObject{Float}.

\item \lstinline|size(\var{list})|\\
  Obsolete syntax for \lstinline|\var{list}.size|, see
  \refObject{List}.

\item \lstinline|strict|\\
  Ignored for backward compatibility.

\item \lstinline|strlen(\var{string})|\\
  Obsolete syntax for \lstinline|\var{string}.length|, see
  \refObject{String}.

\item \lstinline|taglist|\\
  Ignored for backward compatibility.

\item \lstinline|undefall|\\
  Ignored for backward compatibility.

\item \lstinline|unstrict|\\
  Ignored for backward compatibility.

\item \lstinline|uservars|\\
  Ignored for backward compatibility.

\item \lstinline|vars|\\
  Ignored for backward compatibility.
\end{itemize}


%%% Local Variables:
%%% mode: latex
%%% TeX-master: "../urbi-sdk"
%%% End:

\section{Lazy}

\dfn{Lazies} are objects that hold a lazy value, that is, a not yet evaluated
value. They provide facilities to evaluate their content only once
(\dfn{memoization}) or several times. Lazy are essentially used in call
messages, to represent lazy arguments, as described in
\autorefObject{CallMessage}.

\subsection{Construction}

Lazies are seldom instantiated manually. They are mainly created
automatically when a lazy function call is made (see
\autoref{sec:us-fun-callmsg}). One can however create a lazy value with the
standard \lstinline|new| method of \lstinline|Lazy|, giving it an
argument-less function which evaluates to the lazified value.

\begin{urbiscript}
Lazy.new(closure () { /* Value to lazify */ });
[00000000:hide] Lazy_0xADDR
\end{urbiscript}

\subsection{Methods}

\begin{itemize}
\item \lstinline|eval|\\
The \lstinline|eval| method forces evaluation of the held lazy
value. Two calls to \lstinline|eval| will systematically evaluate the
expression twice, which can be useful to duplicates its side effects.

\item \lstinline|value|\\
The \lstinline|value| method returns the held value, potentially
evaluating it before. \lstinline|value| performs memoization, that is,
only the first call will actually evaluate the expression, subsequent
calls will return the cached value. Unless you want to explicitly
trigger side effects from the expression by evaluating it several
time, this should be preferred over \lstinline|eval| to avoid
evaluating the expression several times uselessly.

\end{itemize}

\subsection{Examples}

\subsubsection{Evaluating once}

One usage of lazy values is to avoid evaluating an expression unless
it's actually needed, because it's expensive or has undesired side
effects. The listing below presents a situation where an
expensive-to-compute value (\lstinline|heavy_computation|) might be
needed zero, one or two times. The objective is to save time by:

\begin{itemize}
\item Not evaluating it if it's not needed.
\item Evaluating it only once if it's needed one or two time.
\end{itemize}

We thus make the wanted expression lazy, and use the \lstinline|value|
method to fetch its value when needed.

\begin{urbiscript}
// This function supposedly performs expensive computations.
function heavy_computation()
{
  echo("Heavy computation");
  return 1 + 1;
};
[00000000:hide] function () {
[:]  echo("Heavy computation");
[:]  return 1 . '+'(1);
[:]}

// We want to do the heavy computations only if needed,
// and make it a lazy value to be able to evaluate it "on demand".
var v = Lazy.new(closure () { heavy_computation() });
[00000000] Lazy_0xADDR
/* some code */;
// So far, the value was not needed, and heavy_computation
// was not evaluated.
/* some code */;
// If the value is needed, heavy_computation is evaluated.
v.value;
[00000000] *** Heavy computation
[00000000] 2
// If the value is needed a second time, heavy_computation
// is not reevaluated.
v.value;
[00000000] 2
\end{urbiscript}

\subsubsection{Evaluating several times}

Evaluating a lazy several times only makes sense with lazy arguments
and call messages. See example with call messages in
\autoref{sec:std-callmsg-examples-several}.


\subsection{Caching}

\refObject{Lazy} is meant for functions without argument.  If you need
\dfn{caching} for functions that depend on arguments, it is
straightforward to implement using a \refObject{Dictionary}.  In the
future \us might support dictionaries whose indices are not only
strings, but in the meanwhile, you have to convert the arguments into
strings, as the following sample object demonstrates.

\begin{urbiscript}
class UnaryLazy
{
  function init(f)
  {
    results = Dictionary.new;
    func = f;
  };
  function value(p)
  {
    var sp = p.asString;
    if (results.has(sp))
      return results[sp];
    var res = func(p);
    results[sp] = res |
    res
  };
  var results;
  var func;
} |
// The function to cache.
var inc = function(x) { echo("incing " + x) | x+1 } |
// The function with cache.
// Use "getSlot" to get the unevaluated function.
var p = UnaryLazy.new(getSlot("inc"));
[00062847] UnaryLazy_0x78b750
p.value(1);
[00066758] *** incing 1
[00066759] 2
p.value(1);
[00069058] 2
p.value(2);
[00071558] *** incing 2
[00071559] 3
p.value(2);
[00072762] 3
p.value(1);
[00074562] 2
\end{urbiscript}


%%% Local Variables:
%%% mode: latex
%%% TeX-master: "../urbi-sdk"
%%% End:

\section{List}

\lstinline|List|s implement potentially empty ordered collections of
elements.

\subsection{Prototypes}

\begin{itemize}
\item \refObject{Object}
\end{itemize}

\subsection{Construction}

List can be created with their literal syntax, as shown in
\autoref{sec:us-syn-lit-list}.

%% They can also be created with the \lstinline|new| method of
%% \lstinline|List|, given the initial members (\autoref{lst:new-list}).
%%
%% \begin{urbiscript}[caption=List.new, label=lst:new-list]
%% List.new(nil, "foo", 42);
%% [00000000] [nil, "foo", 42]
%% \end{urbiscript}

\subsection{Methods}

\subsubsection{all}

% FIXME: link to predicate glossary entry
Return whether all the members of the target verify the given
predicate.

\begin{urbiscript}
// Are all elements positive?
[-2, 0, 2, 4].all(function (e) { e > 0 });
[00000000] false
// Are all elements even?
[-2, 0, 2, 4].all(function (e) { e % 2 == 0 });
[00000000] true
\end{urbiscript}

\subsubsection{any}

% FIXME: link to predicate glossary entry
Return whether at least one of the members of the target verify the
given predicate.

\begin{urbiscript}
// Is there any even element?
[-3, 1, -1].all(function (e) { e % 2 == 0 });
[00000000] false
// Is there any positive element?
[-3, 1, -1].any(function (e) { e > 0 });
[00000000] true
\end{urbiscript}

\subsubsection{asList}

Return the target.

\begin{urbiscript}
[0, 1, 2].asList;
[00000000] [0, 1, 2]
\end{urbiscript}

\subsubsection{asString}

Return the target as a string describing the list.

\begin{urbiscript}
[0, 1, 2].asString;
[00000000] "[0, 1, 2]"
\end{urbiscript}

\subsubsection{back}

Return the last element of the target. An error if the target is empty.

\begin{urbiscript}
[0, 1, 2].back;
[00000000] 2
[].back;
[00000000:error] !!! back: cannot be applied onto empty list
\end{urbiscript}

\subsubsection{clear}

Empty the target.

\begin{urbiscript}
var x = [0, 1, 2];
[00000000] [0, 1, 2]
x.clear;
[00000000] []
\end{urbiscript}

\subsubsection{each}

Apply the given functional value on all members sequentially.

\begin{urbiscript}
[0, 1, 2].each(function (v) {echo (v * v); echo (v * v)});
[00000000] *** 0
[00000000] *** 0
[00000000] *** 1
[00000000] *** 1
[00000000] *** 4
[00000000] *** 4
\end{urbiscript}

\subsubsection{each\&}

Apply the given functional value on all members simultaneously.

\begin{urbiscript}
[0, 1, 2].'each&'(function (v) {echo (v * v); echo (v * v)});
[00000000] *** 0
[00000000] *** 1
[00000000] *** 4
[00000000] *** 0
[00000000] *** 1
[00000000] *** 4
\end{urbiscript}

\subsubsection{empty}

Return whether the target is empty.

\begin{urbiscript}
[].empty;
[00000000] true
[1].empty;
[00000000] false
\end{urbiscript}

\subsubsection{filter}

Return the list of all member of the target that verify the given
predicate.

\begin{urbiscript}
// Keep only odd numbers
[0, 1, 2, 3, 4, 5].filter(function (v) {v % 2 == 1});
[00000000] [1, 3, 5]
\end{urbiscript}

\subsubsection{foldl}

\fixme{This is too damn hard to explain, yet so simple.}

\begin{urbiscript}

\end{urbiscript}

\subsubsection{front}
\label{sec:std-list-front}

Return the first element of the target. An error if the target is
empty.

\begin{urbiscript}
[0, 1, 2].front;
[00000000] 0
[].front;
[00000000:error] !!! front: cannot be applied onto empty list
\end{urbiscript}

\subsubsection{has}

Return whether one of the member of the target equals the argument.

\begin{urbiscript}
[0, 1, 2].has(1);
[00000000] true
[0, 1, 2].has(5);
[00000000] false
\end{urbiscript}

\subsubsection{hasSame}

Return whether one of the member of the target is physically equal to
the argument.

\begin{urbiscript}
var x = 1;
[00000000:hide] 1
[0, x, 2].hasSame(1);
[00000000] false
[0, x, 2].hasSame(x);
[00000000] true
\end{urbiscript}

\subsubsection{head}

Synonym for \lstinline|front| (\autoref{sec:std-list-front}).

\subsubsection{join}

Concatenate all members of the target to form a string, separating
them with the given separator.

\begin{urbiscript}
["b", "ob", "b"].join("a");
[00000000] "baobab"
\end{urbiscript}

\subsubsection{map}

Apply the given functional value on every member, and return the list
of results.

\begin{urbiscript}
[0, 1, 2, 3].map(function (v) { v % 2 == 0});
[00000000] [true, false, true, false]
\end{urbiscript}

\subsubsection{[]}
\label{sec:std-list-nth}

Return the nth member of the target (indexing is zero-based). An error
if out of bounds.

%\begin{urbiscript}[caption={\lstinline|List.'\[\]'|},
%label=lst:list-nth]
\begin{urbiscript}
var l = [0, 1, 2];
[00000000:hide] [0, 1, 2]
l[1];
[00000000] 1
l[3];
[00007061:error] !!! []: invalid index: 3
\end{urbiscript}

\subsubsection{removeBack}

Remove and return the last element of the target. An error if the
target is empty.

\begin{urbiscript}
var x = [0, 1, 2];
[00000000] [0, 1, 2]
x.removeBack;
[00000000] 2
x;
[00000000] [0, 1]
[].removeBack;
[00000000:error] !!! removeBack: cannot be applied onto empty list
\end{urbiscript}

\subsubsection{removeFront}

Remove and return the first element from the target. An error if the
target is empty.

\begin{urbiscript}
var x = [0, 1, 2];
[00000000] [0, 1, 2]
x.removeFront;
[00000000] 0
x;
[00000000] [1, 2]
[].removeFront;
[00000000:error] !!! removeFront: cannot be applied onto empty list
\end{urbiscript}

\subsubsection{insertBack}
\label{sec:std-list-pushback}

Insert the given element at the end of the target.

\begin{urbiscript}
var x = [0, 1];
[00000000] [0, 1]
x.insertBack(2);
[00000000] [0, 1, 2]
x;
[00000000] [0, 1, 2]
\end{urbiscript}

\subsubsection{insertFront}

Insert the given element at the beginning of the target.

\begin{urbiscript}
var x = [1, 2];
[00000000] [1, 2]
x.insertFront(0);
[00000000] [0, 1, 2]
x;
[00000000] [0, 1, 2]
\end{urbiscript}

\subsubsection{range}

Return a sub-range of the string, from the first index included to the
second index excluded. An error if out of bounds.

\begin{urbiscript}
[0, 1, 2, 3].range(1, 3);
[00000000] [1, 2]
[].range(1, 3);
[00428697:error] !!! urbi/list.u: []: invalid index: 1
\end{urbiscript}


\subsubsection{remove}

Remove all elements from the target that equals the argument.

\begin{urbiscript}
var x = [0, 1, 0, 2, 0, 3];
[00000000] [0, 1, 0, 2, 0, 3]
x.remove(0);
[00000000] [1, 2, 3]
x;
[00000000] [1, 2, 3]
\end{urbiscript}

\subsubsection{removeById}

Remove all elements from the target that physically equals the
argument.

\begin{urbiscript}
var x = 1;
[00000000] 1
var l = [0, 1, x, 1, 2];
[00000000] [0, 1, 1, 1, 2]
l.removeById(x);
[00000000] [0, 1, 1, 2]
l;
[00000000] [0, 1, 1, 2]
\end{urbiscript}

\subsubsection{reverse}

Return the target with the order of elements inverted.

\begin{urbiscript}
[0, 1, 2].reverse;
[00000000] [2, 1, 0]
\end{urbiscript}

\subsubsection{==}

Check whether all elements in the target and the first argument, are
equal two by two.

%\begin{urbiscript}[caption={\lstinline|List.==|}, label=lst:list-sameAs]
\begin{urbiscript}
[0, 1, 2] == [0, 1, 2];
[00000000] true
[0, 1, 2] == [0, 0, 2];
[00000000] false
\end{urbiscript}

\subsubsection{[]=}
\label{sec:std-list-setnth}

Assign a value to the element of the target at the given index.

%\begin{urbiscript}[caption={\lstinline|List.'\[\]='|}, label=lst:list-setNth]
\begin{urbiscript}
var x = [0, 1, 2];
[00000000] [0, 1, 2]
x[1] = 42;
[00000000] 42
x;
[00000000] [0, 42, 2]
\end{urbiscript}

\subsubsection{size}

Return the number of elements in the target.

\begin{urbiscript}
[1, 2, 3].size;
[00000000] 3
[].size;
[00000000] 0
\end{urbiscript}

\subsubsection{sort}

Return the target, sorted with respect to the \lstinline|<| criteria.

\begin{urbiscript}
[1, 0, 3, 2].sort;
[00000000] [0, 1, 2, 3]
\end{urbiscript}

\subsubsection{tail}

Return the target, minus the first element. An error if the target is
empty.

\begin{urbiscript}
[0, 1, 2].tail;
[00000000] [1, 2]
[].tail;
[00000000:error] !!! tail: cannot be applied onto empty list
\end{urbiscript}

\subsubsection{*}

Return the target, concatenated n times to itself, n being the
argument.

%\begin{urbiscript}[caption={\lstinline|List.'*'|}, label=lst:list-times]
\begin{urbiscript}
[0, 1] * 3;
[00000000] [0, 1, 0, 1, 0, 1]
\end{urbiscript}

\subsubsection{+}

Return the concatenation of the target and the given list.

%\begin{urbiscript}[caption={\lstinline|List.'+'|}, label=lst:list-plus,
\begin{urbiscript}
  float=\floatpos]
[0, 1] + [2, 3];
[00000000] [0, 1, 2, 3]
\end{urbiscript}

\subsubsection{-}

Return the target without all element that equals any element in the
given list.

%\begin{urbiscript}[caption={\lstinline|List.'-'|}, label=lst:list-minus,
\begin{urbiscript}
[0, 1, 0, 2, 3] - [1, 2];
[00000000] [0, 0, 3]
\end{urbiscript}

\subsubsection{\textless\textless}

A synonym for \lstinline|insertBack| (\autoref{sec:std-list-pushback}).

\subsubsection{\textless}

Return whether the target is inferior to the given list. A list is
inferior to another if at least one of its element differs from the
other, and the first differing element is inferior to the other.

%\begin{urbiscript}[caption={\lstinline|List.'<'|}, label=lst:list-inf,
\begin{urbiscript}
[0, 1, 2] < [0, 1, 2];
[00000000] false
[0, 1, 2] < [0, 0, 2];
[00000000] false
[0, 1, 2] < [0, 2, 2];
[00000000] true
\end{urbiscript}

%%% Local Variables:
%%% mode: latex
%%% TeX-master: "../urbi-sdk"
%%% End:

% LocalWords:  lst asList asString foldl hasSame removeBack popback removeFront
% LocalWords:  popfront insertBack pushback insertFront pushfront urbi sameAs
% LocalWords:  removeById setNth

%% Copyright (C) 2010, Gostai S.A.S.
%%
%% This software is provided "as is" without warranty of any kind,
%% either expressed or implied, including but not limited to the
%% implied warranties of fitness for a particular purpose.
%%
%% See the LICENSE file for more information.

\section{Loadable}

Loadable objects can be switched on and off --- typically physical
devices.

\subsection{Prototypes}

\begin{refObjects}
\item[Object]
\end{refObjects}

\subsection{Example}

The intended use is rather as follows:

\begin{urbiscript}
class Motor: Loadable
{
  var val = 0;
  function go(var d)
  {
    if (load)
      val += d
    else
      echo("cannot advance, the motor is off")|;
  };
};
[00000002] Motor

var m = Motor.new;
[00000003] Motor_0xADDR

m.load;
[00000004] false

m.go(1);
[00000006] *** cannot advance, the motor is off

m.on;
[00000007] Motor_0xADDR

m.go(123);
m.val;
[00000009] 123
\end{urbiscript}

\subsection{Construction}

\lstinline|Loadable| can be constructed, but it hardly makes sense.
This object should serve as a prototype.

\subsection{Slots}

\begin{urbiscriptapi}
\item[load] The current status.

\item[off](<val>)%
  Set \refSlot{load} to \lstinline|false| and return \this.
\begin{urbiassert}
do (Loadable.new)
{
  assert
  {
    !load;
    off === this;
    !load;
    on === this;
    load;
    off === this;
    !load;
  };
};
\end{urbiassert}

\item[on](<val>)%
  Set \refSlot{load} to \lstinline|true| and return \this.
\begin{urbiassert}
do (Loadable.new)
{
  assert
  {
    !load;
    on === this;
    load;
    on === this;
    load;
  };
};
\end{urbiassert}

\item[toggle]%
  Set \refSlot{load} from \lstinline|true| to \lstinline|false|, and
  vice-versa.  Return \var{val}.
\begin{urbiassert}
do (Loadable.new)
{
  assert
  {
    !load;
    toggle === this;
    load;
    toggle === this;
    !load;
  };
};
\end{urbiassert}
\end{urbiscriptapi}

%%% Local Variables:
%%% coding: utf-8
%%% mode: latex
%%% TeX-master: "../urbi-sdk"
%%% ispell-dictionary: "american"
%%% ispell-personal-dictionary: "../urbi.dict"
%%% fill-column: 76
%%% End:

%% Copyright (C) 2009-2010, Gostai S.A.S.
%%
%% This software is provided "as is" without warranty of any kind,
%% either expressed or implied, including but not limited to the
%% implied warranties of fitness for a particular purpose.
%%
%% See the LICENSE file for more information.

\section{Lobby}

A \dfn{lobby} is the local environment for each (remote or local)
connection to an \urbi server.

\subsection{Prototypes}
\begin{itemize}
\item \refSlot{Channel}{topLevel}, an instance of \refObject{Channel}
  with an empty Channel name.
\end{itemize}

\subsection{Construction}

A lobby is implicitly created at each connection. At the top level,
\lstinline|this| is a \dfn{Lobby}.

\begin{urbiscript}
this.protos;
[00000001] [Lobby]
this.protos[0].protos;
[00000003] [Channel_0xADDR]
\end{urbiscript}

Lobbies cannot be cloned, they must be created using
\lstinline|create|.

\begin{urbiscript}
Lobby.new;
[00000177:error] !!! new: `Lobby' objects cannot be cloned
Lobby.create;
[00000174] Lobby_0x126450
\end{urbiscript}


\subsection{Examples}

Since every lobby is-a \refObject{Channel}, one can use the methods of
Channel.

\begin{urbiscript}
lobby << 123;
[00478679] 123
lobby << "foo";
[00478679] "foo"
\end{urbiscript}

\subsection{Slots}
\begin{urbiscriptapi}
\item[banner] Internal.  Display \usdk banner.
\begin{urbiscript}
lobby.banner;
[00000118] *** ********************************************************
[00000118] *** Urbi SDK version 2.0/rc-4 rev. fb573a2
[00000118] *** Copyright (C) 2005-2010 Gostai S.A.S.
[00000118] ***
[00000118] *** This program comes with ABSOLUTELY NO WARRANTY.
[00000118] *** It can be used under certain conditions.
[00000118] *** Type `license;' or `copyright;' for more information.
[00000118] ***
[00000118] *** Check our community site: http://www.urbiforge.org.
[00000118] *** ********************************************************
\end{urbiscript}

\item[connected]
  Whether \lstinline|this| is connected.
\begin{urbiassert}
connected;
\end{urbiassert}

\item[connectionTag]
  The tag of all code executed in the context of \lstinline|this|.

\item[copyright](var <deep> = true)%~\\
  Display the copyright of \usdk.  Include copyright information
  about sub-components if \var{deep}.
\begin{urbiscript}
lobby.copyright(false);
[00000022] *** Urbi SDK version 2.0/rc-4 rev. e594893
[00000023] *** Copyright (C) 2005-2010 Gostai S.A.S.

lobby.copyright(true);
[00000024] *** Urbi SDK version 2.0/rc-4 rev. e594893
[00000025] *** Copyright (C) 2005-2010 Gostai S.A.S.
[00000027] ***
[00000028] *** Urbi SDK Remote version preview/1.6/rc-1 rev. cab1787
[00000029] *** Copyright (C) 2005-2010 Gostai S.A.S.
[00000031] ***
[00000032] *** Libport version releases/1.0 rev. 88f9f61
[00000033] *** Copyright (C) 2005-2010 Gostai S.A.S.
\end{urbiscript}

\item[create]
  Instantiate a new Lobby.
\begin{urbiassert}
Lobby.create;
\end{urbiassert}

\item[echo](<value>, <channel> = "")%
  Send \lstinline|\var{value}.asString| to \lstinline|this|, prefixed
  by the \refObject{String} \var{channel} name if specified.  This is
  the preferred way to send informative messages (prefixed with
  \samp{***}).
\begin{urbiscript}
lobby.echo("111", "foo");
[00015895:foo] *** 111
lobby.echo(222, "");
[00051909] *** 222
lobby.echo(333);
[00055205] *** 333
\end{urbiscript}

\item[echoEach](<list>, <channel> = "")%
  Apply \lstinline|echo(\var{m}, \var{channel})| for each member \var{m} of
  \var{list}.
\begin{urbiscript}
lobby.echo([1, "2"], "foo");
[00015895:foo] *** [1, "2"]

lobby.echoEach([1, "2"], "foo");
[00015895:foo] *** 1
[00015895:foo] *** 2

lobby.echoEach([], "foo");
\end{urbiscript}

%\item \lstinline|help|\experimental\\
%  Launch the tutorial.

\item[license]
  Display the end user license agreement of the \usdk.
\begin{urbiunchecked}
lobby.license;
[00000000] *** END USER LICENSE AGREEMENT (1.2)
[00000000] ***
[00000000] *** PLEASE READ THIS AGREEMENT CAREFULLY.  BY USING ALL OR ANY PORTION OF
[00000000] *** THE SOFTWARE YOU ("YOU" AND "LICENSEE") ACCEPT THE FOLLOWING TERMS
[00000000] *** FROM GOSTAI S.A.S, FRENCH CORPORATION ("GOSTAI"), REGISTERED AT
[00000000] *** 489 244 624 RCS PARIS.  YOU AGREE TO BE BOUND BY ALL THE TERMS AND
[00000000] *** CONDITIONS OF THIS AGREEMENT.  YOU AGREE THAT IT IS ENFORCEABLE AS IF
[00000000] *** IT WERE A WRITTEN NEGOTIATED AGREEMENT SIGNED BY YOU.  IF YOU DO NOT
[00000000] *** AGREE TO THE TERMS OF THIS AGREEMENT YOU MUST NOT USE THE SOFTWARE.
[00000000] ***
[00000000] *** [...]
\end{urbiunchecked}

\item[lobby]
  Return the current lobby, i.e., \lstinline|this|.
\begin{urbiassert}
lobby === this;
\end{urbiassert}

\item[onDisconnect](<lobby>)%
  Event launched when \lstinline|this| has disconnected.

\item[quit] Shut this lobby down, i.e., close the connection.  The
  server is still running, see \refSlot{System}{shutdown} to quit the
  server.

\item[receive](<value>)%
  This is low-level routine.  Pretend the \refObject{String}
  \var{value} was received from the connection.  There is no guarantee
  that \var{value} will be the next program block that will be
  processed: for instance, if you load a file which, in its middle,
  uses \lstinline|lobby.receive("foo")|, then \lstinline|"foo"| will
  be appended after the end of the file.
\begin{urbiscript}
Lobby.create.receive("12;");
[00478679] 12
\end{urbiscript}

\item[send](<value>, <channel> = "")%
  This is low-level routine.  Send the \refObject{String} \var{value}
  to \lstinline|this|, prefixed by the \refObject{String}
  \var{channel} name if specified.
\begin{urbiscript}
lobby.send("111", "foo");
[00015895:foo] 111
lobby.send("222", "");
[00051909] 222
lobby.send("333");
[00055205] 333
\end{urbiscript}

\item[wall](<value>, <channel> = "")%
  Perform \lstinline|echo(\var{value}, \var{channel})| on all the
  existing lobbies (except Lobby itself).
\begin{urbiscript}[firstnumber=1]
Lobby.wall("111", "foo");
[00015895:foo] *** 111
\end{urbiscript}

\item[write](<value>)%
  This is low-level routine.  Send the \refObject{String} \var{value}
  to the connection.  Note that because of buffering, the output might
  not be visible before an end-of-line character is output.
\begin{urbiscript}
lobby.write("[");
lobby.write("999999999:");
lobby.write("myTag] ");
lobby.write("Hello, World!");
lobby.write("\n");
[999999999:myTag] Hello, World!
\end{urbiscript}
\end{urbiscriptapi}

%%% Local Variables:
%%% mode: latex
%%% TeX-master: "../urbi-sdk"
%%% ispell-dictionary: "american"
%%% ispell-personal-dictionary: "../urbi.dict"
%%% fill-column: 76
%%% End:

\section{Location}

This class aggregate two Positions and provide a way to print them as done
in error messages.

\subsection{Prototypes}
\begin{itemize}
\item \refObject{Object}
\end{itemize}

\subsection{Construction}

Without argument, newly constructed locations has its positions initialized
to the first line and the first column.

\begin{urbiscript}
Location.new;
[00000001] 1.1
\end{urbiscript}

With a position argument \var{p}, the location will clone the position into
the begin and end positions.

\begin{urbiscript}
Location.new(Position.new("file.u",14,25));
[00000001] file.u:14.25
\end{urbiscript}

With two positions arguments \var{begin} and \var{end}, the location will
clone both positions into its own fields.

\begin{urbiscript}
Location.new(Position.new("file.u",14,25), Position.new("file.u",14,35));
[00000001] file.u:14.25-34
\end{urbiscript}

\subsection{Slots}

\begin{itemize}

\item \lstinline|asString|\\
  Present locations with less variability as possible as either:
  \begin{itemize}
  \item \samp{\var{file}:\var{ll}.\var{cc}}
  \item \samp{\var{file}:\var{ll}.\var{cc}-\var{cc}}
  \item \samp{\var{file}:\var{ll}.\var{cc}-\var{ll}.\var{cc}}
  \end{itemize}
  or the same without file name when the file name is not defined.
\begin{urbiassert}[firstnumber=last]
Location.new(Position.new("file.u",14,25)).asString == "file.u:14.25";
Location.new(Position.new(14,25)).asString == "14.25";

Location.new(
  Position.new("file.u",14,25),
  Position.new("file.u",14,35)
).asString == "file.u:14.25-34";

Location.new(
  Position.new(14,25),
  Position.new(14,35)
).asString == "14.25-34";

Location.new(
  Position.new("file.u",14,25),
  Position.new("file.u",15,35)
).asString == "file.u:14.25-15.34";

Location.new(
  Position.new(14,25),
  Position.new(15,35)
).asString == "14.25-15.34";
\end{urbiassert}

\item \lstinline|begin|\\
  The begin position used by the location.  Modifying a copy of this field
  does not modify the location.
\begin{urbiassert}[firstnumber=last]
Location.new(
  Position.new("file.u",14,25),
  Position.new("file.u",16,35)
).begin == Position.new("file.u",14,25);
\end{urbiassert}

\item \lstinline|end|\\
  The end position used by the location.  Modifying a copy of this field
  does not modify the location.
\begin{urbiassert}[firstnumber=last]
Location.new(
  Position.new("file.u",14,25),
  Position.new("file.u",16,35)
).end == Position.new("file.u",16,35);
\end{urbiassert}


\end{itemize}

%%% Local Variables:
%%% mode: latex
%%% TeX-master: "../urbi-sdk"
%%% End:

\section{Math}

This object is actually meant to play the role of a name-space in
which the mathematical functions are defined with a more conventional
notation.  Indeed, in an object-oriented language, writing
\lstinline|pi.cos| makes perfect sense, yet \lstinline|cos(pi)| is
more usual.

\subsection{Prototypes}
\begin{itemize}
\item \refObject{Singleton}
\end{itemize}

\subsection{Construction}

Since it is a \refObject{Singleton}, you are not expected to build
other instances.

\subsection{Slots}

\begin{itemize}
\item \lstinline|abs(\var{float})|\\
  Bounce to \lstinline|\var{float}.abs|.
\begin{urbiassert}
Math.abs(1) == 1;
Math.abs(-1) == 1;
Math.abs(0) == 0;
Math.abs(3.5) == 3.5;
\end{urbiassert}
\item \lstinline|acos(\var{float})|\\
  Bounce to \lstinline|\var{float}.acos|.

\item \lstinline|asin(\var{float})|\\
  Bounce to \lstinline|\var{float}.asin|.

\item \lstinline|atan(\var{float})|\\
  Bounce to \lstinline|\var{float}.atan|.
\begin{urbiassert}
Math.atan(1) ~= pi/4;
\end{urbiassert}

\item \lstinline|atan2(\var{x}, \var{y})|\\
  Bounce to \lstinline|\var{x}.atan2(\var{y})|.
\begin{urbiassert}
Math.atan2(2, 2) ~= pi/4;
Math.atan2(-2, 2) ~= -pi/4;
\end{urbiassert}

\item \lstinline|cos(\var{float})|\\
  Bounce to \lstinline|\var{float}.cos|.
\begin{urbiassert}
Math.cos(0) == 1;
Math.cos(pi) ~= -1;
\end{urbiassert}

\item \lstinline|exp(\var{float})|\\
  Bounce to \lstinline|\var{float}.exp|.

\item \lstinline|inf|\\
  Bounce to \lstinline|Float.inf|.

\item \lstinline|log(\var{float})|\\
  Bounce to \lstinline|\var{float}.log|.
\begin{urbiassert}
Math.log(1) == 0;
\end{urbiassert}

\item \lstinline|max(\var{arg1}, ...)|\\
  Bounce to \lstinline|[\var{arg1}, ...].max|, see \refObject{List}.
\begin{urbiassert}
max( 100,   20,   3 ) == 100;
max("100", "20", "3") == "3";
\end{urbiassert}

\item \lstinline|min(\var{arg1}, ...)|\\
  Bounce to \lstinline|[\var{arg1}, ...].min|, see \refObject{List}.
\begin{urbiassert}
min( 100,   20,   3 ) ==     3;
min("100", "20", "3") == "100";
\end{urbiassert}

\item \lstinline|nan|\\
  Bounce to \lstinline|Float.nan|.

\item \lstinline|pi|\\
  Bounce to \lstinline|Float.pi|.

\item \lstinline|random(\var{float})|\\
  Bounce to \lstinline|\var{float}.random|.

\item \lstinline|round(\var{float})|\\
  Bounce to \lstinline|\var{float}.round|.
\begin{urbiassert}
Math.round(1) == 1;
Math.round(1.1) == 1;
Math.round(1.49) == 1;
Math.round(1.5) == 2;
Math.round(1.51) == 2;
\end{urbiassert}

\item \lstinline|sign(\var{float})|\\
  Bounce to \lstinline|\var{float}.sign|.
\begin{urbiassert}
Math.sign(2)  == 1;
Math.sign(-2) == -1;
Math.sign(0)  == 0;
\end{urbiassert}

\item \lstinline|sin(\var{float})|\\
  Bounce to \lstinline|\var{float}.sin|.
\begin{urbiassert}
Math.sin(0) == 0;
Math.sin(pi) ~= 0;
\end{urbiassert}

\item \lstinline|sqr(\var{float})|\\
  Bounce to \lstinline|\var{float}.sqr|.
\begin{urbiassert}
Math.sqr(2.2) ~= 4.84;
\end{urbiassert}

\item \lstinline|sqrt(\var{float})|\\
  Bounce to \lstinline|\var{float}.sqrt|.
\begin{urbiassert}
Math.sqrt(4) == 2;
\end{urbiassert}

\item \lstinline|tan(\var{float})|\\
  Bounce to \lstinline|\var{float}.tan|.
\begin{urbiassert}
Math.tan(pi/4) ~= 1;
\end{urbiassert}

\item \lstinline|trunc(\var{float})|\\
  Bounce to \lstinline|\var{float}.trunc|.
\begin{urbiassert}
Math.trunc(1) == 1;
Math.trunc(1.1) == 1;
Math.trunc(1.49) == 1;
Math.trunc(1.5) == 1;
Math.trunc(1.51) == 1;
\end{urbiassert}
\end{itemize}


%%% Local Variables:
%%% mode: latex
%%% TeX-master: "../urbi-sdk"
%%% ispell-dictionary: "american"
%%% ispell-personal-dictionary: "../urbi.dict"
%%% End:

%% Copyright (C) 2009-2011, Gostai S.A.S.
%%
%% This software is provided "as is" without warranty of any kind,
%% either expressed or implied, including but not limited to the
%% implied warranties of fitness for a particular purpose.
%%
%% See the LICENSE file for more information.

\section{Mutex}

\dfn{Mutex} allow to define critical sections.

\subsection{Prototypes}
\begin{refObjects}
\item[Tag]
\end{refObjects}

\subsection{Construction}
A Mutex can be constructed like any other Tag but without name.

\begin{urbiscript}[firstnumber=1]
var m = Mutex.new();
[00000000] Mutex_0x964ed40
\end{urbiscript}

You can define critical sections by tagging your code using the Mutex.

\begin{urbiscript}[firstnumber=1]
var m = Mutex.new()|;
m: echo("this is critical section");
[00000001] *** this is critical section
\end{urbiscript}

As a critical section, two pieces of code tagged by the same ``Mutex''
will never be executed at the same time.

Mutexes must be used when manipulating data structures in a non atomic way to
avoid inconsistent states.

Consider this apparently simple code:

\begin{urbiscript}[firstnumber=1]
function appendAndTellIfFirst(list, val)
{
  var res = list.empty;
  list << val;
  res
}|;
var l = [];
[00000001] []
appendAndTellIfFirst(l, 1);
[00000002] true
appendAndTellIfFirst(l, 2);
[00000002] false
\end{urbiscript}

Now look what happens if called twice in parallel:

\begin{urbiscript}
l = [];
[00000001] []
var res1; var res2;
res1 = appendAndTellIfFirst(l, 1) & res2 = appendAndTellIfFirst(l, 2)|;
res1;
[00000002] true
res2;
[00000003] true
l.sort(); // order is unspecified
[00000004] [1, 2]
\end{urbiscript}

Both tasks checked if the list was empty at the same time, and then appened the
element.

A mutex will solve this problem:

\begin{urbiscript}
l = [];
[00000001] []
var m = Mutex.new();
[00000000] Mutex_0x964ed40
// redefine the function using the mutex
appendAndTellIfFirst = function (list, val)
{m:{
  var res = list.empty;
  list << val;
  res
}}|;
// check again
res1 = appendAndTellIfFirst(l, 1) & res2 = appendAndTellIfFirst(l, 2)|;
// we do not know which one was first, but only one was
[res1, res2].sort();
[00000001] [false, true]
l.sort();
[00000004] [1, 2]
\end{urbiscript}

Mutex constructor accepts an optional maximum queue size: code blocks
trying to wait when maximum queue size is reached will not be executed:

\begin{urbiscript}[firstnumber=1]
var m = Mutex.new(1)|;
var e = Event.new()|;
at(e?)
  m: { echo("executing at"); sleep(200ms);};
e!;e!;e!;
sleep(600ms);
[00000001] *** executing at
[00000001] *** executing at
\end{urbiscript}

As you can see above the message is only displayed twice: First at got executed
right away, the second was queued and executed when the first one finished, and
the third one got stopped.

\subsection{Slots}

\begin{urbiscriptapi}
\item[asMutex]  Return \this.
\begin{urbiscript}
var m1 = Mutex.new()|;
assert
{
  m1.asMutex() === m1;
};
\end{urbiscript}
\end{urbiscriptapi}


%%% Local Variables:
%%% coding: utf-8
%%% mode: latex
%%% TeX-master: "../urbi-sdk"
%%% ispell-dictionary: "american"
%%% ispell-personal-dictionary: "../urbi.dict"
%%% fill-column: 76
%%% End:

\section{nil}

The special entity \lstinline|nil| is an object used to denote an
empty value.  Contrary to \refObject{void}, it is a regular value
which can be read.

\subsection{Prototypes}

\begin{itemize}
\item \refObject{Singleton}
\end{itemize}

\subsection{Construction}

Being a singleton, \lstinline|nil| is not to be constructed, just used.

\begin{urbiscript}
assert(nil == nil);
\end{urbiscript}

\subsection{Slots}

\begin{itemize}
\item \lstinline|isNil|\\
  Whether \lstinline|this| is \lstinline|nil|.  I.e., true.  See also
  \lstinline|Object.isNil|.
\begin{urbiscript}[firstnumber=last]
assert(nil.isNil);
assert(!Object.isNil);
\end{urbiscript}

\end{itemize}




%%% Local Variables:
%%% mode: latex
%%% TeX-master: "../urbi-sdk"
%%% End:

\section{Object}

All objects in \us must have \refObject{Object} in their
parents. \refObject{Object} is done for this purpose so that it come
with many primitives that are mandatory for all object in \us.

\subsection{Prototypes}

\begin{itemize}
\item \refObject{Orderable}
\item \refObject{Global}
\end{itemize}

\subsection{Methods}

\begin{itemize}
\item \lstinline|acceptVoid|\\

\item \lstinline|addProto(\var{proto})|\\
  Add the prototype \var{proto} as a parent into the list of
  \lstinline|this|.

\begin{urbiscript}
var Dummy = Object.new | Dummy.addProto(Orderable) | {}
assert_eq(Dummy.protos, [Orderable, Object]);
\end{urbiscript}

\item \lstinline|allProto|\\
  Return a list with \lstinline|this|, all parents of
  \lstinline|this|, the parents of the parents of
  \lstinline|this|,\ldots

\begin{urbiscript}[firstnumber=last]
assert_eq(12.allProtos,[12, 0, Orderable, Object, Comparable, Global,
  Tags, Math, DeprecatedCommands, Kernel1Functions, System, Constants]);
\end{urbiscript}

\item \lstinline|allSlotNames|\\
  Return a list with the slot names of \lstinline|this| and of all
  parents in the hierarchy of \lstinline|this|.

\item \lstinline|apply|\\

\item \lstinline|as|\\

\item \lstinline|bounce|\\

\item \lstinline|bounce_named|\\

\item \lstinline|callMessage|\\

\item \lstinline|clone|\\

\item \lstinline|cloneSlot(\var{from}, \var{to})|\\
  Set the new slot \var{to} using a clone of \var{from}. This can only
  be used into the same object.

\begin{urbiscript}[firstnumber=last]
var foo = Object.new | {}
cloneSlot("foo", "bar") | {}
assert(!(foo === bar));
\end{urbiscript}

\item \lstinline|copySlot(\var{from}, \var{to})|\\
  Same as \lstinline|cloneSlot|, but the slot aren't cloned, so the
  two slot are the same.

\begin{urbiscript}[firstnumber=last]
var moo = Object.new | {}
cloneSlot("moo", "loo") | {}
assert(!(moo === loo));
\end{urbiscript}

\item \lstinline|createSlot(\var{name})|\\
  Create an empty slot (which actually means it is bound to
  \lstinline|void|) named \var{name}.  Raise an error if \var{name}
  was already defined.

\begin{urbiscript}[firstnumber=last]
assert(Object.createSlot("foo").isVoid);
assert(Object.hasSlot("foo"));
\end{urbiscript}

\item \lstinline|dump|\\

\item \lstinline|each(\var{fun})|\\
  Call \lstinline|each| with the given functional value on
  \lstinline|this.asList|.

\item \lstinline|eachBg|\\

\item \lstinline$each|$\\

\item \lstinline|getLazyLocalSlot|\\

\item \lstinline|getPeriod|\\

\item \lstinline|getProperty|\\

\item \lstinline|getSlot(\var{name})|\\
  Return the value associated to \var{name}.

\begin{urbiscript}[firstnumber=last]
assert(!Object.getSlot("getSlot").isVoid);
\end{urbiscript}

\item \lstinline|hasProperty|\\

\item \lstinline|hasSlot(\var{slot})|\\
  Return true if \lstinline|this| has the slot \var{slot}.

\begin{urbiscript}[firstnumber=last]
assert(hasSlot("connectionTag"));
assert(!(hasSlot("thisSlotDoesNotExist")));
\end{urbiscript}

\item \lstinline|id|\\

\item \lstinline|isA(\var{obj})|\\
  Return true if \lstinline|this| has \var{obj} in his parents, false
  otherwise.

\begin{urbiscript}[firstnumber=last]
assert(Float.isA(Orderable));
assert(!(String.isA(Float)));
\end{urbiscript}

\item \lstinline|isNil|\\
  Return true if \lstinline|this| is nil, false otherwise.

\begin{urbiscript}[firstnumber=last]
assert(nil.isNil);
assert(!(0.isNil));
\end{urbiscript}

\item \lstinline|isProto|\\
  Return true if \lstinline|this| is a prototype, false otherwise;

\begin{urbiscript}[firstnumber=last]
assert(Float.isProto);
assert(!(42.isProto));
\end{urbiscript}

\item \lstinline|isVoid|\\
  Return true if \lstinline|this| is \lstinline|void|.

\begin{urbiscript}[firstnumber=last]
assert(void.isVoid);
assert(!(42.isVoid));
\end{urbiscript}

\item \lstinline|locateSlot(\var{slot})|\\
  Return \lstinline|nil| if \lstinline|this| don't have the slot
  \lstinline|slot|. Otherwise it returns the first lowest owner of
  \lstinline|slot| of \lstinline|this|.

\begin{urbiscript}[firstnumber=last]
assert(locateSlot("init"), Channel);
assert(locateSlot("doesNotExist").isNil);
\end{urbiscript}

\item \lstinline|ownsSlot(\var{slot})|\\
  Return true if \lstinline|this| owns the slot \var{slot}, false
  otherwise.

\begin{urbiscript}[firstnumber=last]
assert(Object.ownsSlot("ownsSlot"));
assert(!(24.ownsSlot("asString")));
\end{urbiscript}

\item \lstinline|print|\\

\item \lstinline|protos|\\
  Return the list of prototypes of \lstinline|this|.

\begin{urbiscript}[firstnumber=last]
assert(Object.protos, [Comparable, Global]);
\end{urbiscript}

\item \lstinline|removeProperty|\\

\item \lstinline|removeProto(\var{proto})|\\
  Remove \var{proto} from the list of prototypes of
  \lstinline|this|.

\item \lstinline|removeSlot(\var(slot))|\\
  Remove \var{slot} from the list of slot of
  \lstinline|this|.

\item \lstinline|setConstSlot|\\
  Like \lstinline|setSlot| but the created slot is const.

\begin{urbiscript}[firstnumber=last]
assert_eq(setConstSlot("fortyTwo"), 42);
fortyTwo = 51;
[00000000:error] !!! cannot modify const slot
\end{urbiscript}

\item \lstinline|setProperty|\\

\item \lstinline|setProtos|\\

\item \lstinline|setSlot(\var{name}, \var{value})|\\
  Create a slot \var{name} mapping to \var{value}. Raise an error if
  \var{name} was already defined.

\begin{urbiscript}[firstnumber=last]
assert_eq(Object.setSlot("theObject", Object), Object);
assert(Object.theObject === Object);
assert(theObject === Object);
\end{urbiscript}

\item \lstinline|slotNames|\\
  Returns the list of slot owned by \lstinline|this|.

\item \lstinline|tasks|\\
  Returns the list of the current running tasks;

\item \lstinline|type|\\

\item \lstinline|uid|\\
  Returns ths unique id of \lstinline|this|.

\item \lstinline|unacceptvoid|\\

\item \lstinline|uobject_init|\\

\item \lstinline|updateSlot(\var{name}, \var{value})|\\
  Map the existing slot named \var{name} to \var{value}. Raise an
  error if \var{name} was not defined.
\begin{urbiscript}[firstnumber=last]
assert_eq(Object.setSlot("one", 1), 1);
assert_eq(Object.updateSlot("one", 2), 2);
assert_eq(Object.one, 2);
\end{urbiscript}

\item \lstinline|'&&'(\var{that})|\\
  Short-circuiting logical and. If \lstinline|this| evaluates to true
  evaluate and return \var{that}, otherwise return \lstinline|this|
  without evaluating \var{that}.
\begin{urbiscript}[firstnumber=last]
assert_eq(0 && "foo", 0);
assert_eq(2 && "foo", "foo");

assert_eq(""    && "foo", "");
assert_eq("foo" && "bar", "bar");
\end{urbiscript}

\item \lstinline/'||'(\var{that})/\\
  Short-circuiting logical or. If \lstinline|this| evaluates to false
  evaluate and return \var{that}, otherwise return \lstinline|this|
  without evaluating \var{that}.
\begin{urbiscript}[firstnumber=last]
assert_eq(0 || "foo", "foo");
assert_eq(2 ||  1/0,  2);

assert_eq(""    || "foo", "foo");
assert_eq("foo" || 1/0,   "foo");
\end{urbiscript}

\item \lstinline|'!'|\\
  Logical negation. If \lstinline|this| evaluates to false return
  \lstinline|true| and vice-versa.
\begin{urbiscript}[firstnumber=last]
assert_eq(!1, false);
assert_eq(!0, true);

assert_eq(!"foo", false);
assert_eq(!"",    true);
\end{urbiscript}
\end{itemize}

%%% Local Variables:
%%% mode: latex
%%% TeX-master: "../urbi-sdk"
%%% End:

%% Copyright (C) 2009-2011, Gostai S.A.S.
%%
%% This software is provided "as is" without warranty of any kind,
%% either expressed or implied, including but not limited to the
%% implied warranties of fitness for a particular purpose.
%%
%% See the LICENSE file for more information.

\section{Orderable}

Objects that have a concept of ``less than''.  See also
\refObject{Comparable}.

\subsection{Example}
This object, made to serve as prototype, provides a definition of
\lstinline{<} based on \lstinline{>}, and vice versa; and definition of
\lstinline{<=}/\lstinline{>=} based on
\lstinline{<}/\lstinline{>}\lstinline{==}.  You \strong{must} define either
\lstinline{<} or \lstinline{>}, otherwise invoking either method will result
in endless recursions.

\begin{urbiscript}[firstnumber=1]
class Foo : Orderable
{
  var value = 0;
  function init (v)   { value = v; };
  function '<' (that) { value < that.value; };
  function asString() { "<" + value.asString + ">"; };
}|;
var one = Foo.new(1)|;
var two = Foo.new(2)|;

assert
{
   one <= one  ;   one <= two  ; !(two <= one);
 !(one >  one) ; !(one >  two) ;   two >  one;
  (one >= one) ; !(one >= two) ;   two >= one;
};
\end{urbiscript}

\subsection{Prototypes}
\begin{refObjects}
\item[Object]
\end{refObjects}

\subsection{Slots}
\begin{urbiscriptapi}
\item['<'](<that>)%
  Whether \lstinline{this <= that && this != that}.

\item['<='](<that>)%
  Whether \lstinline{that > this || this == that}.

\item['>'](<that>)%
  Whether \lstinline{this >= that && this != that}.

\item['>='](<that>)%
  Whether \lstinline{that < this || this != that}.
\end{urbiscriptapi}

%%% Local Variables:
%%% mode: latex
%%% TeX-master: "../urbi-sdk"
%%% ispell-dictionary: "american"
%%% ispell-personal-dictionary: "../urbi.dict"
%%% fill-column: 76
%%% End:

\section{OutputStream}

OutputStreams are used to write (possibly binary) files by hand.

\subsection{Prototypes}
\begin{itemize}
\item \refObject{Object}
\end{itemize}

\subsection{Construction}

An OutputStream is a reading-interface to a file, so its constructor
requires a \refObject{File}.  If the file already exists, content is
\emph{appended} to it.  Remove the file beforehand if you want to
override its content.

\begin{urbiscript}
System.system("(echo 1; echo 2) >file.txt")|;
OutputStream.new(File.new("file.txt"));
[00000001] OutputStream_0x827000

OutputStream.new(File.create("new.txt"));
[00000001] OutputStream_0x827000
\end{urbiscript}

\subsection{Slots}

\begin{itemize}
\item \lstinline|<<(\var{that})|\\
  Output \lstinline|\var{this}.asString|.  Return \lstinline|this| to
  enable chains of calls.
\begin{urbiscript}[firstnumber=last]
assert(
  {
    {
      var o = OutputStream.new(File.create("fresh.txt"));
      o << 1 << "2" << [3, [4]];
    };
    File.new("fresh.txt").content.data;
  }
  ==
  "12[3, [4]]");
\end{urbiscript}

\item \lstinline|close|\\
  Flush the buffers, and close the file.
\begin{urbiscript}[firstnumber=last]
assert(OutputStream.new(File.create("file.txt")).close.isVoid);
\end{urbiscript}

\item \lstinline|putByte(\var{byte})|\\
  In order to private efficient input/output operations,
  \dfn[buffer]{buffers} are used.  As a consequence, what is put into
  a stream might not be immediately saved on the actual file.  To
  \dfn{flush} a buffer means to dump its content to the file.
\begin{urbiscript}[firstnumber=last]
assert(OutputStream.new(File.create("file.txt")).flush.isVoid);
\end{urbiscript}
\end{itemize}


%%% Local Variables:
%%% mode: latex
%%% TeX-master: "../urbi-sdk"
%%% End:

\section{Pair}

A \dfn{pair} is a container storing two objects, similar in spirit to
\lstinline|std::pair| in \Cxx.

\subsection{Prototype}
\begin{itemize}
\item \refObject{Object}
\end{itemize}

\subsection{Construction}

A \dfn{Pair} is contructed with one or two arguments. In case on one
argument used, the constructor uses it twice.

\begin{urbiscript}
Pair.new(1, 2);
[00000001] (1, 2)
Pair.new(3);
[00000002] (3, 3)
\end{urbiscript}

\subsection{Methods}
\begin{itemize}
\item \lstinline|asString|\\
  Generate the string \samp{(\var{first}, \var{second})} using
  \code{asPrintable} to convert members to strings.

\item \lstinline|first|\\
  Return the first member of the pair.
\begin{urbiscript}[firstnumber=last]
assert(Pair.new(1, 2).first == 1);
\end{urbiscript}

\item \lstinline|second|\\
  Return the second member of the pair.
\begin{urbiscript}[firstnumber=last]
assert(Pair.new(1, 2).second == 2);
\end{urbiscript}

\item \lstinline|'[]'(\var{index})|\\
  Return the \var{index}-th element.  \var{index} must be 0 or 1.
\begin{urbiscript}[firstnumber=last]
assert(Pair[0] === Pair.first);
assert(Pair[1] === Pair.second);
\end{urbiscript}

\item \lstinline|'[]='(\var{index}, \var{value})|\\
  Set (and return) the \var{index}-th element to \var{value}.
  \var{index} must be 0 or 1.

\item \lstinline|'<'(\var{other})|\\
  Lexicographic comparison between two pairs.
\begin{urbiscript}[firstnumber=last]
assert(Pair.new(0, 0) < Pair.new(0, 1));
assert(Pair.new(0, 0) < Pair.new(1, 0));
assert(Pair.new(0, 1) < Pair.new(1, 0));
\end{urbiscript}

\item \lstinline|'=='(\var{other})|\\
  Whether \lstinline|this| and \lstinline|other| have the same
  contents (equality-wise).
\begin{urbiscript}[firstnumber=last]
assert(Pair.new(1, 2) == Pair.new(1, 2));
assert(!(Pair.new(1, 1) == Pair.new(2, 2)));
\end{urbiscript}
\end{itemize}



%%% Local Variables:
%%% mode: latex
%%% TeX-master: "../urbi-sdk"
%%% End:

%% Copyright (C) 2009-2011, Gostai S.A.S.
%%
%% This software is provided "as is" without warranty of any kind,
%% either expressed or implied, including but not limited to the
%% implied warranties of fitness for a particular purpose.
%%
%% See the LICENSE file for more information.

\section{Path}

A \dfn{Path} points to a file system entity (directory, file and so
forth).

\subsection{Prototypes}
\begin{refObjects}
\item[Comparable]
\item[Orderable]
\end{refObjects}

\subsection{Construction}

Path itself is the root of the file system: \lstinline|/| on Unix, and
\lstinline|C:\| on Windows.

\begin{urbiscript}[firstnumber=1]
Path;
[00000001] Path("/")
\end{urbiscript}
\begin{urbicomment}
skipIfWindows();
\end{urbicomment}

A \lstinline|Path| is constructed with the string that points to the file
system entity.  This path can be relative or absolute.

\begin{urbiscript}[firstnumber=1]
Path.new("foo");
[00000002] Path("foo")

Path.new("/path/file.u");
[00000001] Path("/path/file.u")
\end{urbiscript}

Some minor simplifications are made, such as stripping useless
\file{./} occurrences.

\begin{urbiscript}
Path.new("././///.//foo/");
[00000002] Path("foo")
\end{urbiscript}

\subsection{Slots}
\begin{urbiscriptapi}
\item['/'](<rhs>)%
  Create a new \dfn{Path} that is the concatenation of
  \this and \lstinline|\var{rhs}|. \lstinline|\var{rhs}|
  can be a \dfn{Path} or a \dfn{String} and cannot be absolute.
\begin{urbiscript}
assert(Path.new("/foo/bar") / Path.new("baz/qux/quux")
       == Path.new("/foo/bar/baz/qux/quux"));
Path.cwd / Path.new("/tmp/foo");
[00000003:error] !!! /: Rhs of concatenation is absolute: /tmp/foo
\end{urbiscript}


\item['<'](<that>)%
  Same as comparing the string versions of \this and
  \var{that}.
\begin{urbiassert}
  Path.new("/a")   < Path.new("/a/b");
!(Path.new("/a/b") < Path.new("/a")  );
\end{urbiassert}


\item['=='](<that>)%
  Same as comparing the string versions of \this and
  \var{that}.  Beware that two paths may be different and point to the
  very same location.
\begin{urbiassert}
  Path.new("/a")  == Path.new("/a");
!(Path.new("/a")  == Path.new("a")  );
\end{urbiassert}


\item[absolute]
  Whether \this is absolute.
\begin{urbiassert}
Path.new("/abs/path").absolute;
!Path.new("rel/path").absolute;
\end{urbiassert}


\item[asList]
  List of names used in path (directories and possibly file), from
  bottom up. There is no difference between relative path and absolute
  path.
\begin{urbiassert}
Path.new("/path/to/file.u").asList() == ["path", "to", "file.u"];
Path.new("/path").asList()           == Path.new("path").asList();
\end{urbiassert}


\item[asPrintable]
\begin{urbiassert}
Path.new("file.txt").asPrintable() == "Path(\"file.txt\")";
\end{urbiassert}


\item[asString]
  The name of the file.
\begin{urbiassert}
Path.new("file.txt").asString() == "file.txt";
\end{urbiassert}


\item[basename]%
  Base name of the path.  See also \refSlot{dirname}.
\begin{urbiassert}
Path.new("/absolute/path/file.u").basename == "file.u";
Path.new("relative/path/file.u").basename  == "file.u";
\end{urbiassert}


\item[cd]%
  Change the current working directory to \this. Return the new current
  working directory as a \lstinline|Path|.
\begin{urbiassert}
var a = Directory.create("a").asPath();
var b = Directory.create("a/b").asPath();
var cwd = Path.cwd;     // Current location.
cwd.isA(Path);
// cd returns the new current working directory.
b.cd() == cwd / b == cwd / "a" / "b";
Path.cwd == cwd / b;
// Go back to the original location.
Path.new("../..").cd() == cwd;
Path.cwd == cwd;
\end{urbiassert}
\begin{urbicomment}
removeFs("a");
\end{urbicomment}

Exceptions are thrown on cases of error.
\begin{urbiscript}
Path.new("does/not/exist").cd();
[00003991:error] !!! cd: no such file or directory: does/not/exist

var f = File.create("file.txt")|;
f.asPath().cd();
[00099415:error] !!! cd: not a directory: file.txt
\end{urbiscript}
\begin{urbicomment}
removeFs(f);
removeSlots("f");
\end{urbicomment}

Permissions are not properly handled on Windows, so the following example
would actually fail.
\begin{urbiscript}[firstnumber=1]
var d = Directory.create("forbidden")|;
System.system("chmod 444 %s" % d)|;
d.asPath().cd();
[00140753:error] !!! cd: Permission denied: forbidden
\end{urbiscript}
\begin{urbicomment}
skipIfWindows();
removeFs(d);
removeSlots("d");
\end{urbicomment}

\item[cwd]%
  The current working directory.
% We used to write
% assert(Path.new("/").cd == Path.new("/"));
% assert(Path.cwd         == Path.new("/"));
% which is wrong on Windows, because cwd (like cd) returns Z:/ instead
% of /.
\begin{urbiassert}[firstnumber=1]
// Save current directory.
var pwd = Path.cwd;
// Go into "/".
var root = Path.new("/").cd();
// Current working directory is "/".
Path.cwd == root;
// Go back to the directory we were in.
pwd.cd() == pwd;
\end{urbiassert}

\item[dirname]%
  Directory name of the path.  See also \refSlot{basename}.
\begin{urbiassert}
Path.new("/abs/path/file.u").dirname == Path.new("/abs/path");
Path.new("rel/path/file.u").dirname  == Path.new("rel/path");
\end{urbiassert}


\item[exists]%
  Whether something (a \refSlot{File}, a \refSlot{Directory}, \ldots) exists
  where \this points to.
\begin{urbiassert}
Path.cwd.exists;
Path.new("/").exists;
var p = Path.new("file.txt");
!p.exists;
File.create(p);
p.exists;
\end{urbiassert}
\begin{urbicomment}
removeFs("file.txt");
\end{urbicomment}


\item[isDir]%
  Whether \this is a directory.
\begin{urbiassert}
Path.cwd.isDir;
var f = File.create("file.txt");
!f.asPath().isDir;
!Path.new("does/not/exist").isDir;
\end{urbiassert}
\begin{urbicomment}
removeFs("file.txt");
\end{urbicomment}


\item[isReg]%
  Whether \this is a regular file.
\begin{urbiassert}
var f = File.create("file.txt");
 f.asPath().isReg;
!Path.cwd.isReg;
!Path.new("does/not/exist").isReg;
\end{urbiassert}
\begin{urbicomment}
removeFs("file.txt");
\end{urbicomment}


\item[lastModifiedDate]%
  Last modified date of the path.
\begin{urbiassert}
var p = Path.new("test");
File.create(p);
0 <= Date.now - p.lastModifiedDate <= 5s;
\end{urbiassert}


\item[open] Open \this. Return either a \dfn{Directory} or a \dfn{File}
  according the type of \this. See \refObject{File} and
  \refObject{Directory}.
\begin{urbiassert}
Path.new("/").open().isA(Directory);
\end{urbiassert}


\item[readable]
  Whether \this is readable.  Throw if does not even exist.  On Windows,
  always returns true.
\begin{urbiassert}[firstnumber=1]
Path.new(".").readable;
var p = Path.new("file.txt");
File.create(p);
p.readable;
System.system("chmod a-r %s" % p) == 0;
!p.readable;
System.system("chmod a+r %s" % p) == 0;
p.readable;
\end{urbiassert}
\begin{urbicomment}
skipIfWindows();
removeFs("file.txt");
\end{urbicomment}


\item[rename](<name>)%
  Rename the file-system object (directory, file, etc.) pointed to by \this,
  as \var{name}.  Return \refObject{void}.
\begin{urbiassert}[firstnumber=1]
var dir1 = Directory.create("dir1");
var p = dir1.asPath();
p.rename("dir2").isVoid;
p.basename == "dir2";
\end{urbiassert}
\begin{urbicomment}
removeFs("dir2");
\end{urbicomment}


\item[writable]%
  Whether \this is writable.  Throw if does not even exist.  On Windows,
  always returns true.
\begin{urbiassert}[firstnumber=1]
Path.new(".").writable;
var p = Path.new("file.txt");
File.create(p);
p.writable;
System.system("chmod a-w %s" % p) == 0;
!p.writable;
System.system("chmod a+w %s" % p) == 0;
p.writable;
\end{urbiassert}
\begin{urbicomment}
skipIfWindows();
removeFs("file.txt");
\end{urbicomment}
\end{urbiscriptapi}


%%% Local Variables:
%%% coding: utf-8
%%% mode: latex
%%% TeX-master: "../urbi-sdk"
%%% ispell-dictionary: "american"
%%% ispell-personal-dictionary: "../urbi.dict"
%%% fill-column: 76
%%% End:

\section{Pattern}
Describing shape of data to match.

%%% Local Variables:
%%% mode: latex
%%% TeX-master: "../urbi-sdk"
%%% End:

\section{Position}

This class is used to handle file locations with a line, column and file
name.

\subsection{Prototypes}
\begin{itemize}
\item \refObject{Object}
\end{itemize}

\subsection{Construction}

Without argument, newly constructed position has its fields initialized to
the first line and the first column.

\begin{urbiscript}
Position.new;
[00000001] 1.1
\end{urbiscript}

With a position argument \var{p}, the newly constructed position is a clone
of \var{p}.

\begin{urbiscript}
Position.new(Position.new(2, 3));
[00000001] 2.3
\end{urbiscript}

With two float arguments \var{l} and \var{c}, the newly constructed position
has its line and column defined and an empty file name.

\begin{urbiscript}
Position.new(2, 3);
[00000001] 2.3
\end{urbiscript}

With three arguments \var{f}, \var{l} and \var{c}, the newly constructed position
has its file name, line and column defined.

\begin{urbiscript}
Position.new("file.u", 2, 3);
[00000001] file.u:2.3
\end{urbiscript}

\subsection{Slots}

\begin{itemize}
\item \lstinline|'+'(\var{n})|\\
  Return a new position which is shifted from \var{n} columns to the right.  The minimal
  value of the new position column is 1.
\begin{urbiassert}[firstnumber=last]
Position.new(2, 3) + 2 == Position.new(2, 5);
Position.new(2, 3) + -4 == Position.new(2, 1);
\end{urbiassert}

\item \lstinline|'-'(\var{n})|\\
  Return a new position which is shifted from \var{n} columns to the left.  The minimal
  value of the new position column is 1.
\begin{urbiassert}[firstnumber=last]
Position.new(2, 3) - 1 == Position.new(2, 2);
Position.new(2, 3) - -4 == Position.new(2, 7);
\end{urbiassert}

\item \lstinline|'=='(\var{other})|\\
  Compare the line and column of two positions.
\begin{urbiassert}[firstnumber=last]
Position.new(2, 3) == Position.new(2, 3);
Position.new("a.u", 2, 3) == Position.new("b.u", 2, 3);
Position.new(2, 3) != Position.new(2, 2);
\end{urbiassert}

\item \lstinline|'<'(\var{other})|\\
  Order comparison of lines and columns.
\begin{urbiassert}[firstnumber=last]
Position.new(2, 3) < Position.new(2, 4);
Position.new(2, 3) < Position.new(3, 1);
\end{urbiassert}

\item \lstinline|asString|\\
  Present as \samp{\var{file}:\var{line}.\var{column}}, the file name is
  omitted if it is not defined.
\begin{urbiscript}
Position.new("file.u", 2, 3);
[00000001] file.u:2.3
\end{urbiscript}

\item \lstinline|column|\\
  Field which give access to the column number of the position.
\begin{urbiassert}[firstnumber=last]
Position.new(2, 3).column == 3;
\end{urbiassert}

\item \lstinline|columns(\var{n})|\\
  Identical to \lstinline|'+'(\var{n})|.
\begin{urbiassert}[firstnumber=last]
Position.new(2, 3).columns(2) == Position.new(2, 5);
Position.new(2, 3).columns(-4) == Position.new(2, 1);
\end{urbiassert}

\item \lstinline|file|\\
  The \lstinline|Path| of the position file.
\begin{urbiassert}[firstnumber=last]
Position.new("file.u", 2, 3).file == Path.new("file.u");
Position.new(2, 3).file == nil;
\end{urbiassert}

\item \lstinline|line|\\
  Field which give access to the line number of the position.
\begin{urbiassert}[firstnumber=last]
Position.new(2, 3).line == 2;
\end{urbiassert}

\item \lstinline|lines(\var{n})|\\
  Add \var{n} lines and reset the column number to 1.
\begin{urbiassert}[firstnumber=last]
Position.new(2, 3).lines(2) == Position.new(4, 1);
Position.new(2, 3).lines(-1) == Position.new(1, 1);
\end{urbiassert}


\item \lstinline|data|\\
  The data carried by the Binary.
\begin{urbiassert}[firstnumber=last]
Binary.new("head", "content").data == "content";
\end{urbiassert}

\item \lstinline|empty|\\
  Whether the data is empty.
\begin{urbiassert}[firstnumber=last]
Binary.new("head", "").empty;
!Binary.new("head", "content").empty;
\end{urbiassert}

\item \lstinline|keywords|\\
  The headers carried by the Binary.
\begin{urbiassert}[firstnumber=last]
Binary.new("head", "content").keywords == "head";
\end{urbiassert}
\end{itemize}


%%% Local Variables:
%%% mode: latex
%%% TeX-master: "../urbi-sdk"
%%% End:

%% Copyright (C) 2009-2010, Gostai S.A.S.
%%
%% This software is provided "as is" without warranty of any kind,
%% either expressed or implied, including but not limited to the
%% implied warranties of fitness for a particular purpose.
%%
%% See the LICENSE file for more information.

\section{Primitive}
\Cxx routine callable from \us.

\subsection{Prototypes}
\begin{refObjects}
\item[Executable]
\end{refObjects}

\subsection{Construction}

It is not possible to construct a Primitive.

\subsection{Slots}

\begin{urbiscriptapi}
\item[apply](<args>)%
  Invoke a primitive.  The argument list, \var{args}, must start with
  the target.
\begin{urbiassert}
Float.getSlotValue("+").isA(Global.getSlotValue("Primitive"));
Float.getSlotValue("+").apply([1, 2]) == 3;

String.getSlotValue("+").isA(Global.getSlotValue("Primitive"));
String.getSlotValue("+").apply(["1", "2"]);
\end{urbiassert}


\item[asPrimitive] Return \this.
\begin{urbiassert}
Float.getSlotValue("+").asPrimitive === Float.getSlotValue("+");
\end{urbiassert}
\end{urbiscriptapi}


%%% Local Variables:
%%% coding: utf-8
%%% mode: latex
%%% TeX-master: "../urbi-sdk"
%%% ispell-dictionary: "american"
%%% ispell-personal-dictionary: "../urbi.dict"
%%% fill-column: 76
%%% End:

%% Copyright (C) 2010-2012, Gostai S.A.S.
%%
%% This software is provided "as is" without warranty of any kind,
%% either expressed or implied, including but not limited to the
%% implied warranties of fitness for a particular purpose.
%%
%% See the LICENSE file for more information.

\section{Process}

A Process is a separated task handled by the underneath operating
system.

\begin{windows}
  Process is not yet supported under Windows.
\end{windows}

\subsection{Prototypes}
\begin{itemize}
\item \refObject{Object}
\end{itemize}

\subsection{Example}

The following examples runs the \command{cat} program, a Unix standard
command that simply copies on its (standard) output its (standard)
input.

\begin{urbiscript}
var p = Process.new("cat", []);
[00000004] Process cat
\end{urbiscript}

\noindent
Just created, this process is not running yet.  Use \lstinline|run| to
launch it.

\begin{urbiscript}
p.status;
[00000005] not started

p.run;
p.status;
[00000006] running
\end{urbiscript}

\noindent
Then we feed its input, named \lstinline|stdin| in the Unix
tradition, and close its input.

\begin{urbiscript}
p.stdin << "1\n" |
p.stdin << "2\n" |
p.stdin << "3\n" |;

p.status;
[00000007] running

p.stdin.close;
\end{urbiscript}

\noindent
At this stage, the status of the process is unknown, as it is running
asynchronously.  If it has had enough time to ``see'' that its input
is closed, then it will have finished, otherwise we might have to wait
for awhile.  The method \lstinline|join| means ``wait for the process
to finish''.

\begin{urbiscript}
p.join;

p.status;
[00000008] exited with status 0
\end{urbiscript}

\noindent
Finally we can check its output.

\begin{urbiscript}
p.stdout.asList;
[00000009] ["1", "2", "3"]
\end{urbiscript}

\subsection{Construction}

A Process needs a program name to run and a possibly-empty list of
command line arguments.  Calling \lstinline|run| is required to
execute the process.

\begin{urbiscript}
Process.new("cat", []);
[00000004] Process cat

Process.new("cat", ["--version"]);
[00000004] Process cat
\end{urbiscript}

\subsection{Slots}

\begin{urbiscriptapi}
\item[asProcess] Return \this.
\begin{urbiscript}
do (Process.new("cat", []))
{
  assert (asProcess === this);
}|;
\end{urbiscript}

\item[asString] \lstinline|Process| and the name of the program.
\begin{urbiassert}
Process.new("cat", ["--version"]).asString
  == "Process cat";
\end{urbiassert}

\item[done] Whether the process has completed its execution.
\begin{urbiscript}
do (Process.new("sleep", ["1"]))
{
  assert (!done);
  run;
  assert (!done);
  join;
  assert (done);
}|;
\end{urbiscript}


\item[join] Wait for the process to finish.  Changes its status.
\begin{urbiscript}
do (Process.new("sleep", ["2"]))
{
  var t0 = System.time;
  assert (status.asString == "not started");
  run;
  assert (status.asString == "running");
  join;
  assert (t0 + 2s <= System.time);
  assert (status.asString == "exited with status 0");
}|;
\end{urbiscript}

\item[kill] If the process is not \lstinline|done|, interrupt it (with
  a \lstinline|SIGKILL| in Unix parlance).  You still have to wait for
  its termination with \lstinline|join|.
\begin{urbiscript}
do (Process.new("sleep", ["1"]))
{
  run;
  kill;
  join;
  assert (done);
  assert (status.asString == "killed by signal 9");
}|;
\end{urbiscript}


\item[name] The (base) name of the program the process runs.
\begin{urbiassert}
Process.new("cat", ["--version"]).name == "cat";
\end{urbiassert}

\item[run] Launch the process.  Changes it status.  A process can only
  be run once.
\begin{urbiscript}
do (Process.new("sleep", ["1"]))
{
  assert (status.asString == "not started");
  run;
  assert (status.asString == "running");
  join;
  assert (status.asString == "exited with status 0");
  run;
}|;
[00021972:error] !!! run: Process was already run
\end{urbiscript}

\item[runTo]
  %%% FIXME:
\item[status] An object whose slots describe the status of the
  process.
  %%% FIXME:
\item[stderr] An \refObject{InputStream} (the output of the Process is
  an input for \urbi) to the standard error stream of the process.
\begin{urbiscript}
do (Process.new("urbi-send", ["--no-such-option"]))
{
  run;
  join;
  assert
  {
    stderr.asList ==
    ["urbi-send: invalid option: --no-such-option",
     "Try `urbi-send --help' for more information."];
  };
}|;
\end{urbiscript}

\item[stdin] An \refObject{OutputStream} (the input of the Process is
  an output for \urbi) to the standard input stream of the process.
\begin{urbiscript}
do (Process.new(System.programName, ["--version"]))
{
  run;
  join;
  assert
  {
    stdout.asList[1] == "Copyright (C) 2004-2012 Gostai S.A.S..";
  };
}|;
\end{urbiscript}

\item[stdout] An \refObject{InputStream} (the output of the Process is
  an input for \urbi) to the standard output stream of the process.
\begin{urbiscript}
do (Process.new("cat", []))
{
  run;
  stdin << "Hello, World!\n";
  stdin.close;
  join;
  assert (stdout.asList == ["Hello, World!"]);
}|;
\end{urbiscript}
\end{urbiscriptapi}


%%% Local Variables:
%%% coding: utf-8
%%% mode: latex
%%% TeX-master: "../urbi-sdk"
%%% ispell-dictionary: "american"
%%% ispell-personal-dictionary: "../urbi.dict"
%%% fill-column: 76
%%% End:

%% Copyright (C) 2010, Gostai S.A.S.
%%
%% This software is provided "as is" without warranty of any kind,
%% either expressed or implied, including but not limited to the
%% implied warranties of fitness for a particular purpose.
%%
%% See the LICENSE file for more information.

\section{Profiling}

\lstinline|Profiling| is useful to get an idea of the efficiency of some
small pieces of code.

\subsection{Prototypes}

\begin{refObjects}
\item[Object]
\end{refObjects}

\subsection{Construction}

A \lstinline|Profiling| can be created with two arguments.  The first
argument is the expression which has to be profiled and the second is the
number of iteration it should be run.

Creating a \lstinline|Profiling| session prints the result of the profiled
expression, the number of iterations, the number of cycles and the time of
the evaluation.  The number of cycles corresponds to the number of time the
job is scheduled.

\begin{urbiunchecked}[firstnumber=1]
Profiling.new({1| 2| 3| 4}, 10000);
[00000000] Profiling information
  Expression:       1 | 2 | 3 | 4
  Iterations:       10000
  Cycles:           10000
  Total time:       1.00098 s
  Single iteration: 0.000100098 s
                    1 cycles


Profiling.new({1; 2; 3; 4}, 10000);
[00000000] Profiling information
  Expression:       1;
2;
3;
4
  Iterations:       10000
  Cycles:           40000
  Total time:       1.45856 s
  Single iteration: 0.000145856 s
                    4 cycles
\end{urbiunchecked}

%% \subsection{Slots}

%% \begin{urbiscriptapi}
%% \item[timen](<expr>, <niter>)%
%%   Profile the evaluation of the expression.  The reported message is bugged
%%   when it reports the expression.
%% \end{urbiscriptapi}
%%% Local Variables:
%%% mode: latex
%%% TeX-master: "../urbi-sdk"
%%% ispell-dictionary: "american"
%%% ispell-personal-dictionary: "../urbi.dict"
%%% fill-column: 76
%%% End:

\input{specs/pseudo-lazy}
%% Copyright (C) 2010, Gostai S.A.S.
%%
%% This software is provided "as is" without warranty of any kind,
%% either expressed or implied, including but not limited to the
%% implied warranties of fitness for a particular purpose.
%%
%% See the LICENSE file for more information.

\section{PubSub}

\lstinline|PubSub| provides an abstraction over \lstinline|Barrier|
\refObject{Barrier} to queue signals for each subscriber.

\subsection{Prototypes}

\begin{refObjects}
\item[Object]
\end{refObjects}

\subsection{Construction}

A \lstinline|PubSub| can be created with no arguments.  Values can be
published and read by each subscriber.

\begin{urbiscript}[firstnumber=1]
var ps = PubSub.new;
[00000000] PubSub_0x28c1bc0
\end{urbiscript}

\subsection{Slots}

\begin{urbiscriptapi}

\item[publish](<ev>)%
  Queue the value \var{ev} to the queue of each subscriber.  This method
  returns the value \var{ev}.

\begin{urbiscript}
{
  var sub = ps.subscribe;
  assert
  {
    ps.publish(2) == 2;
    sub.getOne == 2;
  };
  ps.unsubscribe(sub)
}|;
\end{urbiscript}

\item[subscribe] Create a \refSlot{Subscriber} and insert it inside the list
  of subscribers.

\begin{urbiscript}
var sub = ps.subscribe |
ps.subscribers == [sub];
[00000000] true
\end{urbiscript}

\item[Subscriber] See \refObject{PubSub.Subscriber}.

\item[subscribers] Field containing the list of \refSlot{Subscriber} which
  are watching published values.  This field only exists in instances of
  \lstinline|PubSub|.

\item[unsubscribe](<sub>)%
  Remove a subscriber from the list of subscriber watching the published
  values.

\begin{urbiscript}
ps.unsubscribe(sub) |
ps.subscribers;
[00000000] []
\end{urbiscript}


\end{urbiscriptapi}

%%% Local Variables:
%%% coding: utf-8
%%% mode: latex
%%% TeX-master: "../urbi-sdk"
%%% ispell-dictionary: "american"
%%% ispell-personal-dictionary: "../urbi.dict"
%%% fill-column: 76
%%% End:
\section{PubSub.Subscriber}

\lstinline|Subscriber| is a class returned by the \lstinline|subscribe|
method of a \lstinline|PubSub| instance.  It provides some methods to access
the list of value published with \lstinline|PubSub| instances.


\subsection{Prototypes}

\begin{refObjects}
\item[Object]
\end{refObjects}

\subsection{Construction}

A \lstinline|PubSub.Subscriber| can be created with a call to method
\lstinline|subscribe| of a \lstinline|PubSub| instance.  This way of
creating a \lstinline|Subscriber| add the subscriber as a watcher of values
published on the instance of \lstinline|PubSub|.

\begin{urbiscript}[firstnumber=1]
var ps = PubSub.new | {};
var sub = ps.subscribe;
[00000000] Subscriber_0x28607c0
\end{urbiscript}

%% PubSub.Subscriber.new is not documented because there is no method to add
%% a subscriber into the list of subscribers of a PubSub instance except by
%% doing it in a dirty way.  This may allow to have a subscriber watching
%% multiple published values.

\subsection{Slots}

\begin{urbiscriptapi}

%% \lstinline|enqueue(\var{ev})| is not documented because I assume it is an
%% implementation detail of PubSub.

\item[getOne]
  Block until a value is accessible and return it.  If a value is already
  queued, then the method returns it without blocking.

\begin{urbiscript}
echo(sub.getOne) &
ps.publish(3);
[00000000] *** 3
\end{urbiscript}


\item[getAll]
  Block until a value is accessible.  This method returns the list of value
  queued.  If values are already queued, then the method returns them
  without blocking.

\begin{urbiscript}
ps.publish(4) |
ps.publish(5) |
echo(sub.getAll);
[00000000] *** [4, 5]
\end{urbiscript}


\end{urbiscriptapi}

%%% Local Variables:
%%% mode: latex
%%% TeX-master: "../../urbi-sdk"
%%% ispell-personal-dictionary: "../../urbi.dict"
%%% End:

%% Copyright (C) 2009-2011, Gostai S.A.S.
%%
%% This software is provided "as is" without warranty of any kind,
%% either expressed or implied, including but not limited to the
%% implied warranties of fitness for a particular purpose.
%%
%% See the LICENSE file for more information.

\section{RangeIterable}

This object is meant to be used as a prototype for objects that support an
\lstinline|asList| method, to use range-based \lstinline|for| loops
(\autoref{sec:lang:for:each}).

\subsection{Prototypes}

\begin{refObjects}
\item[Object]
\end{refObjects}

\subsection{Slots}

\begin{urbiscriptapi}
\item[all](<fun>)
  % FIXME: link to predicate glossary entry
  Return whether all the members of the target verify the predicate
  \var{fun}.

\begin{urbiassert}
// Are all elements positive?
! [-2, 0, 2, 4].all(function (e) { e > 0 });
// Are all elements even?
[-2, 0, 2, 4].all(function (e) { e % 2 == 0 });
\end{urbiassert}

\item[any](<fun>)
  % FIXME: link to predicate glossary entry
  Whether at least one of the members of the target verifies the
  predicate \var{fun}.

\begin{urbiassert}
// Is there any even element?
! [-3, 1, -1].any(function (e) { e % 2 == 0 });
// Is there any positive element?
[-3, 1, -1].any(function (e) { e > 0 });
\end{urbiassert}

\item[each](<fun>)%
  Apply the given functional value \var{fun} on all ``members'',
  sequentially.  Corresponds to range-\lstinline|for| loops.
\begin{urbiscript}[firstnumber=1]
class range : RangeIterable
{
  var asList = [10, 20, 30];
}|;
for (var i : range)
  echo (i);
[00000000] *** 10
[00000000] *** 20
[00000000] *** 30
\end{urbiscript}

\item['each&'](<fun>)%
  Apply the given functional value \var{fun} on all ``members'', in
  parallel, starting all the computations simultaneously.  Corresponds
  to range-\lstinline|for&| loops.
\begin{urbiscript}
{
  var res = [];
  for& (var i : range)
    res << i;
  assert(res.sort == [10, 20, 30]);
};
\end{urbiscript}

\item \lstinline+'each|'(\var{fun})+%  Better done in master than 2.7.
  Apply the given functional value \var{fun} on all ``members'', with
  tight sequentially.  Corresponds to range-\lstinline'for|' loops.
\begin{urbiscript}
{
  var res = [];
  for| (var i : range)
    res << i;
  assert(res == [10, 20, 30]);
};
\end{urbiscript}
\end{urbiscriptapi}

%%% Local Variables:
%%% coding: utf-8
%%% mode: latex
%%% TeX-master: "../urbi-sdk"
%%% ispell-dictionary: "american"
%%% ispell-personal-dictionary: "../urbi.dict"
%%% fill-column: 76
%%% End:

%% Copyright (C) 2009-2010, Gostai S.A.S.
%%
%% This software is provided "as is" without warranty of any kind,
%% either expressed or implied, including but not limited to the
%% implied warranties of fitness for a particular purpose.
%%
%% See the LICENSE file for more information.

\section{Regexp}

A Regexp is an object which allow you to match strings with a regular
expression.

\subsection{Prototypes}
\begin{itemize}
\item \refObject{Container}
\item \refObject{Object}
\end{itemize}

\subsection{Construction}
\label{stdlib:regexp:ctor}

A \lstinline{Regexp} is created with the regular expression once and
for all, and it can be used many times to match with other strings.

\begin{urbiscript}[firstnumber=1]
Regexp.new(".");
[00000001] Regexp(".")
\end{urbiscript}

\us supports Perl regular expressions, see
\href{http://perldoc.perl.org/perlre.html}{the perlre man page}.
Expressions cannot be empty.

\subsection{Slots}
\begin{urbiscriptapi}
\item[asPrintable] A string that shows that \this is a Regexp, and its
  value.
\begin{urbiassert}
           Regexp.new("abc").asPrintable == "Regexp(\"abc\")";
Regexp.new("\\d+(\\.\\d+)?").asPrintable == "Regexp(\"\\\\d+(\\\\.\\\\d+)?\")";
\end{urbiassert}

\item[asString] The regular expression that was compiled.
\begin{urbiassert}
           Regexp.new("abc").asString == "abc";
Regexp.new("\\d+(\\.\\d+)?").asString == "\\d+(\\.\\d+)?";
\end{urbiassert}

\item[has](<str>)%
  An experimental alias to \refSlot{match}, so that the infix operators
  \lstinline|in| and \lstinline|not in| can be used (see
  \autoref{sec:lang:operators:containers}).
\begin{urbiassert}
"23.03"     in Regexp.new("^\\d+\\.\\d+$");
"-3.14" not in Regexp.new("^\\d+\\.\\d+$");
\end{urbiassert}

\item[match](<str>)%
  Whether \this matches \var{str}.
\begin{urbiscript}
// Ordinary characters
var r = Regexp.new("oo")|
assert
{
  r.match("oo");
  r.match("foobar");
  !r.match("bazquux");
};

// ^, anchoring at the beginning of line.
r = Regexp.new("^oo")|
assert
{
  r.match("oops");
  !r.match("woot");
};

// $, anchoring at the end of line.
r = Regexp.new("oo$")|
assert
{
  r.match("foo");
  !r.match("mooh");
};

// *, greedy repetition, 0 or more.
r = Regexp.new("fo*bar")|
assert
{
  r.match("fbar");
  r.match("fooooobar");
  !r.match("far");
};

// (), grouping.
r = Regexp.new("f(oo)*bar")|
assert
{
  r.match("foooobar");
  !r.match("fooobar");
};
\end{urbiscript}

\item[matchs](<matchs>)%
  The matched groups, as delimited by parentheses in the regular
  expression. The first element is the whole match.

\begin{urbiscript}
var email_regex = Regexp.new("([a-zA-Z0-9._]+)@([a-zA-Z0-9._]+)");
[00000001] Regexp("([a-zA-Z0-9._]+)@([a-zA-Z0-9._]+)")
email_regex.match("Someone <someone@somewhere.com>");
[00000002] true
email_regex.matchs;
[00000003] ["someone@somewhere.com", "someone", "somewhere.com"]
\end{urbiscript}

\end{urbiscriptapi}

%%% Local Variables:
%%% mode: latex
%%% TeX-master: "../urbi-sdk"
%%% ispell-dictionary: "american"
%%% ispell-personal-dictionary: "../urbi.dict"
%%% fill-column: 76
%%% End:

%% Copyright (C) 2009-2010, Gostai S.A.S.
%%
%% This software is provided "as is" without warranty of any kind,
%% either expressed or implied, including but not limited to the
%% implied warranties of fitness for a particular purpose.
%%
%% See the LICENSE file for more information.

\section{Date}

This class is meant to record backtrace (see \refSlot{Exception}{backtrace})
information.
\experimental{}

%% \subsection{Prototypes}
%% \begin{refObjects}
%% \item[Orderable]
%% \item[Comparable]
%% \end{refObjects}

\subsection{Construction}

\lstinline|StackFrame| are not made to be manually constructed.  The
initialization function expect 2 arguments, which are the name of the
called function and the \refObject{Location} which has called it.

\begin{urbiscript}[firstline=1]
//#line 1 "foo.u"
function inner () { throw Exception.new("test") }|;
function outer () { inner }|;
try
{
  outer
}
catch(var e)
{
  e.backtrace
};
[00002329] [foo.u:2.21-25: inner, foo.u:5.3-7: outer]
\end{urbiscript}

\subsection{Slots}

\begin{urbiscriptapi}
\item[name] \refObject{String}, representing the name of the called
  function.
\begin{urbiscript}[firstline=1]
//#line 1 "foo.u"
function inner () { throw Exception.new("test") }|;
function outer () { inner }|;
try
{
  outer
}
catch(var e)
{
  e.backtrace[0].name
};
[00002329] "inner"
\end{urbiscript}

\item[location] \refObject{Location} of the function call.
\begin{urbiscript}[firstline=1]
//#line 1 "foo.u"
function inner () { throw Exception.new("test") }|;
function outer () { inner }|;
try
{
  outer
}
catch(var e)
{
  e.backtrace[0].location
};
[00002329] foo.u:2.21-25
\end{urbiscript}

\item[asString] Clean display of the call location.
\begin{urbiscript}[firstline=1]
//#line 1 "foo.u"
function inner () { throw Exception.new("test") }|;
function outer () { inner }|;
try
{
  outer
}
catch(var e)
{
  e.backtrace[0]
};
[00002329] foo.u:2.21-25: inner
\end{urbiscript}

\end{urbiscriptapi}


%%% Local Variables:
%%% mode: latex
%%% TeX-master: "../urbi-sdk"
%%% ispell-dictionary: "american"
%%% ispell-personal-dictionary: "../urbi.dict"
%%% fill-column: 76
%%% End:

%% Copyright (C) 2010, Gostai S.A.S.
%%
%% This software is provided "as is" without warranty of any kind,
%% either expressed or implied, including but not limited to the
%% implied warranties of fitness for a particular purpose.
%%
%% See the LICENSE file for more information.

\section{Semaphore}

\lstinline|Semaphore| are useful to limit the number of access to a limited
number of resources.

\subsection{Prototypes}

\begin{refObjects}
\item[Object]
\end{refObjects}

\subsection{Construction}

A \lstinline|Semaphore| can be created with as argument the number of
processes allowed to enter critical sections at the same time.

\begin{urbiscript}[firstnumber=1]
Semaphore.new(1);
[00000000] Semaphore_0x8c1e80
\end{urbiscript}

\subsection{Slots}

\begin{urbiscriptapi}

\item \labelSlot{criticalSection}\lstinline|(function() { \var{code} })|\\%
  Put the piece of \var{code} inside a critical section which can be
  executed simultaneously at most the number of time given at the creation
  of the \lstinline|Semaphore|.  This method is similar to a call to
  \refSlot{acquire} and a call to \refSlot{release} when the code ends by
  any means.

\begin{urbiscript}
{
  var s = Semaphore.new(1);
  for& (var i : [0, 1, 2, 3])
  {
    s.criticalSection(function () {
      echo("start " + i);
      echo("end " + i);
    })
  }
};
[00000000] *** start 0
[00000000] *** end 0
[00000000] *** start 1
[00000000] *** end 1
[00000000] *** start 2
[00000000] *** end 2
[00000000] *** start 3
[00000000] *** end 3


{
  var s = Semaphore.new(2);
  for& (var i : [0, 1, 2, 3])
  {
    s.criticalSection(function () {
      echo("start " + i);

      // Illustrate that processes can be intertwined
      sleep(i * 100ms);

      echo("end " + i);
    })
  }
};
[00000000] *** start 0
[00000000] *** start 1
[00000000] *** end 0
[00000000] *** start 2
[00000000] *** end 1
[00000000] *** start 3
[00000000] *** end 2
[00000000] *** end 3
\end{urbiscript}


\item[acquire] Wait to enter a critical section delimited by the execution
  of \refSlot{acquire} and \refSlot{release}.  Enter the critical section
  when the number of processes inside it goes below the maximum allowed.

\item[p] Historical synonym for \refSlot{acquire}.

\item[release] Leave a critical section delimited by the execution of
  \refSlot{acquire} and \refSlot{release}.

\begin{urbiscript}
{
  var s = Semaphore.new(1);
  for& (var i : [0, 1, 2, 3])
  {
    s.acquire;
    echo("start " + i);
    echo("end " + i);
    s.release;
  }
};
[00000000] *** start 0
[00000000] *** end 0
[00000000] *** start 1
[00000000] *** end 1
[00000000] *** start 2
[00000000] *** end 2
[00000000] *** start 3
[00000000] *** end 3
\end{urbiscript}

\item[v] Historical synonym for \refSlot{release}.
\end{urbiscriptapi}

%%% Local Variables:
%%% coding: utf-8
%%% mode: latex
%%% TeX-master: "../urbi-sdk"
%%% ispell-dictionary: "american"
%%% ispell-personal-dictionary: "../urbi.dict"
%%% fill-column: 76
%%% End:

%% Copyright (C) 2009-2010, Gostai S.A.S.
%%
%% This software is provided "as is" without warranty of any kind,
%% either expressed or implied, including but not limited to the
%% implied warranties of fitness for a particular purpose.
%%
%% See the LICENSE file for more information.

\section{Server}

A \dfn{Server} can listen to incoming connections.  See
\refObject{Socket} for an example.

\subsection{Prototypes}
\begin{refObjects}
\item[Object]
\end{refObjects}

\subsection{Construction}

A \lstinline|Server| is constructed with no argument. At creation, a
new \lstinline|Server| has its own slot \lstinline|connection|. This
slot is an event that is launched when a connection establishes.

\begin{urbiscript}
var s = Server.new|
s.localSlotNames;
[00000001] ["connection"]
\end{urbiscript}

\subsection{Slots}
\begin{urbiscriptapi}
\item[connection]
  The event launched at each incoming connection. This event is
  launched with one argument: the socket of the established connection. This
  connection uses the same \refObject{IoService} as the server.
\begin{urbiscript}
at (s.connection?(var socket))
{
  // This code is run at each connection. 'socket' is the incoming
  // connection.
};
\end{urbiscript}

\item[getIoService]
  Return the \refObject{IoService} used by this socket. Only the default
  \lstinline|IoService| is automatically polled.

\item[host]
  The host on which \this is listening. Raise an error if
  \this is not listening.
\begin{urbiscript}
Server.host;
[00000003:error] !!! host: server not listening
\end{urbiscript}

\item[listen](<host>, <port>)%
  Listen incoming connections with \var{host} and \var{port}.

\item[port]
  The port on which \this is listening. Raise an error if
  \this is not listening.
\begin{urbiscript}
Server.port;
[00000004:error] !!! port: server not listening
\end{urbiscript}

\item[sockets]
  The list of the sockets created at each incoming connection.
\end{urbiscriptapi}

%%% Local Variables:
%%% coding: utf-8
%%% mode: latex
%%% TeX-master: "../urbi-sdk"
%%% ispell-dictionary: "american"
%%% ispell-personal-dictionary: "../urbi.dict"
%%% fill-column: 76
%%% End:

\section{Singleton}

A \lstinline{Singleton} is a prototype that cannot be cloned. All
prototype derived of \lstinline{Singleton} are also singleton.

\subsection{Prototypes}
\begin{itemize}
\item \refObject{Object}
\end{itemize}

\subsection{Construction}

To be a singleton, the object must have \lstinline{Singleton} in his
list of prototypes. The common way to do this is \lstinline{var s = Singleton.new},
but this doesn't work : \lstinline|s| isn't a new singleton, it is the
\lstinline|Singleton| itself since it cannot be cloned. There are two
other ways :


\begin{urbiscript}
// Defining a new class and specifying Singleton as a parent.
class NewSingleton1: Singleton
{
  var asString = "NewSingleton1";
}|
var s1 = NewSingleton1.new;
[00000001] NewSingleton1
assert(s1 === NewSingleton1);
assert(NewSingleton1 !== Singleton);

// Create a new Object and set his prototype by hand.
var NewSingleton2 = Object.New|
var NewSingleton2.asString = "NewSingleton2"|
NewSingleton2.protos = [Singleton]|
var s2 = NewSingleton2.new;
[00000001] NewSingleton2
assert(s2 === NewSingleton1);
assert(NewSingleton2 !== Singleton);
\end{urbiscript}

\subsection{Methods}
\begin{itemize}
\item \lstinline|clone|\\
  Return \lstinline|this|.

\item \lstinline|new'|\\
  Return \lstinline|this|.
\end{itemize}

%%% Local Variables:
%%% mode: latex
%%% TeX-master: "../urbi-sdk"
%%% End:

\section{Socket}

A \dfn{Socket} can manage asynchronous input output connections.

\subsection{Prototypes}
\begin{itemize}
\item \refObject{Object}
\end{itemize}

\subsection{Construction}

A \lstinline|Socket| is constructed with no argument. At creation, a
new \lstinline|Socket| has four own slots: \lstinline|connected|,
\lstinline|disconnected|, \lstinline|error| and \lstinline|received|.

\begin{urbiscript}
var s = Socket.new|
s.localSlotNames;
[00000001] ["received", "error", "disconnected", "connected"]
\end{urbiscript}

\subsection{Methods}
\begin{itemize}

\item \lstinline|connect(\var{host}, \var{port})|\\
  Connects \lstinline|this| on \var{host} and \var{port}.

\item \lstinline|connected|\\
  Event launched when the connection is established.

\item \lstinline|connectSerial(\var{device}, \var{baudrate})|\\
  Connect \lstinline|this| on the serial port \var{device}, with given
  \var{baudrate}.

\item \lstinline|disconnect|\\
  Closes the connection.

\item \lstinline|disconnected|\\
  Event launched when a disconnection happens.

\item \lstinline|error|\\
  Event launched when an error happens. This event is launched with
  the error message in argument. The event \lstinline|disconnected| is
  also always launched.

\item \lstinline|host|\\
  The host of the connection.

\item \lstinline|isConnected|\\
  Whether \lstinline|this| is connected.

\item \lstinline|poll|\\
  Runs the event processing loop to execute ready handlers.

\item \lstinline|port|\\
  The port of the connection.

\item \lstinline|received|\\
  Event launched when \lstinline|this| has received data. The data is
  given by argument to the event.

\item \lstinline|write(\var{data})|\\
  Sends \var{data} trough the connection.

\end{itemize}

\section{String}

A \dfn{string} is a sequence of characters.

\subsection{Prototypes}
\begin{itemize}
\item \refObject{Comparable}
\item \refObject{Orderable}
\end{itemize}

\subsection{Construction}
The simplest way to create fresh String is by using the literal
syntax. A null String can also be obtained with \lstinline|String|'s
\lstinline|new| method.

\begin{urbiscript}
String.new;
[00000000] ""
String;
[00000000] ""
"123";
[00000000] "123"
\end{urbiscript}

\subsection{Methods}
\begin{itemize}
\item \lstinline|asFloat|\\
  If the whole content of \lstinline|this| is an integer, return its
  value, otherwise return an error.
\begin{urbiscript}[firstnumber=last]
"23.03".asFloat;
[00000000] 23.03

"123abc".asFloat;
[00000001:error] !!! asFloat: unable to convert to float: "123abc"
\end{urbiscript}

\item \lstinline|asList|\\
  Return a List of one-letter Strings that, concataneted, equal
  \lstinline|this|.  This allows to use \lstinline|for| to iterate
  over the string.
\begin{urbiscript}[firstnumber=last]
assert("123".asList == ["1", "2", "3"]);
for (var v : "123")
  echo(v);
[00000001] *** 1
[00000001] *** 2
[00000001] *** 3
\end{urbiscript}

\item \lstinline|asPrintable|\\
  Return \lstinline|this| as a literal (escaped) string.
\begin{urbiscript}[firstnumber=last]
assert("foo".asPrintable == "\"foo\"");
assert("foo".asPrintable.asPrintable == "\"\\\"foo\\\"\"");
\end{urbiscript}

\item \lstinline|asString|\\
  Return \lstinline|this|.
\begin{urbiscript}[firstnumber=last]
assert("\"foo\"".asString == "\"foo\"");
\end{urbiscript}

\item \lstinline|distance(\var{other})|\\
  Return the
  \href{http://en.wikipedia.org/wiki/Damerau-Levenshtein_distance}
  {Damerau-Levenshtein distance} between \lstinline|this| and
  \var{other}.  The more alike the strings are, the smaller the
  distance is.
\begin{urbiscript}[firstnumber=last]
assert("foo".distance("foo") == 0);
assert("bar".distance("baz") == 1);
assert("foo".distance("bar") == 3);
\end{urbiscript}

\item \lstinline|fresh|\\
  Return a String that has never been used as an identifier, prefixed
  by \lstinline|this|.  It can safely be used with
  \lstinline|Object.setSlot| and so forth.
\begin{urbiscript}[firstnumber=last]
assert(String.fresh == "_5");
assert("foo".fresh == "foo_6");
\end{urbiscript}

\item \lstinline|isLower|\\
  Return true if all character of \lstinline|this| is lower case.
\begin{urbiscript}[firstnumber=last]
assert("".isLower);
assert("lower".isLower);
assert(! "Not Lower".isLower);
\end{urbiscript}

\item \lstinline|isUpper|\\
  Return true if all character of \lstinline|this| is upper case.
\begin{urbiscript}[firstnumber=last]
assert("".isUpper);
assert("UPPER".isUpper);
assert(! "Not Upper".isUpper);
\end{urbiscript}

\item \lstinline|replace(\var{from}, \var{to})|\\
  Replace every occurrence of the string \var{from} in
  \lstinline|this| by \var{to}, and return the result.
  \lstinline|this| is not modified.
\begin{urbiscript}[firstnumber=last]
assert("Hello == World!".replace("Hello", "Bonjour")
                      .replace("World!", "Monde !"),
       "Bonjour, Monde !");
\end{urbiscript}

\item \lstinline|size|\\
  Return the size of the string.
\begin{urbiscript}[firstnumber=last]
assert("foo".size == 3);
assert("".size == 0);
\end{urbiscript}

\item \lstinline|toLower|\\
  Make lower case every upper case character in \lstinline|this| and
  return the result.  \lstinline|this| is not modified.
\begin{urbiscript}[firstnumber=last]
assert("Hello == World!".toLower, "hello, world!");
\end{urbiscript}

\item \lstinline|toUpper|\\
  Make upper case every lower case character in \lstinline|this| and
  return the result.  \lstinline|this| is not modified.
\begin{urbiscript}[firstnumber=last]
assert("Hello == World!".toUpper, "HELLO, WORLD!");
\end{urbiscript}

\item \lstinline|'%'(\var{args})|\\
  Use \lstinline|this| as format string, and convert occurrences of
  \lstinline|%s| in \lstinline|this| by the \var{args} converted as
  strings via \lstinline|asString|.
%  This construct is actually more
%  powerful than this, since it relies on
%  \href{http://www.boost.org/doc/libs/1_39_0/libs/format/doc/format.html,
%    Boost.Format}.  For instance:
\begin{urbiscript}[firstnumber=last]
assert("%s + %s = %s" % [1, 2, 3] == "1 + 2 = 3");
\end{urbiscript}

\item \lstinline|'*'(\var{n})|\\
  Return the concatentation of \lstinline|this| \var{n} times.
\begin{urbiscript}[firstnumber=last]
assert("foo" * 0 == "");
assert("foo" * 1 == "foo");
assert("foo" * 3 == "foofoofoo");
\end{urbiscript}

\item \lstinline|'+'(\var{other})|\\
  Return the concatenation of \lstinline|this| and
  \lstinline|\var{other}.asString|.
\begin{urbiscript}[firstnumber=last]
assert("foo" + "bar" == "foobar");
assert("foo" + "" == "foo");
assert("foo" + 3 == "foo3");
assert("foo" + [1 == 2, 3], "foo[1, 2, 3]");
\end{urbiscript}

\item \lstinline|'<'(\var{other})|\\
  Whether \lstinline|this| is lexicographically before \var{other},
  which must be a String.
\begin{urbiscript}[firstnumber=last]
assert("" < "a");
assert(!("a" < ""));
assert("a" < "b");
assert(!("a" < "a"));
\end{urbiscript}

\item \lstinline|'[]'(\var{from})|\\
  \lstinline|'[]'(\var{from}, \var{to})|\\
  Return the substring starting at \var{from}, up to and not including
  \var{to} (which defaults to \var{to} + 1).
\begin{urbiscript}[firstnumber=last]
assert("foobar"[0 == 3], "foo");
assert("foobar"[0] == "f");
\end{urbiscript}

\item \lstinline|'[]='(\var{from}, \var{other})|\\
  \lstinline|'[]='(\var{from}, \var{to}, \var{other})|\\
  Replace the substring starting at \var{from}, up to and not including
  \var{to} (which defaults to \var{to} + 1), by \var{other}.  Return
  \var{other}.

  Beware that this routine is imperative: it changes the value of
  \lstinline|this|.
\begin{urbiscript}[firstnumber=last]
var s1 = "foobar" | var s2 = s1 |
assert(s1[0 == 3] = "quux", "quux");
assert(s1 == "quuxbar");
assert(s2 == "quuxbar");
assert(s1[4 == 7] = "", "");
assert(s2 == "quux");
\end{urbiscript}
\end{itemize}

%%% Local Variables:
%%% mode: latex
%%% TeX-master: "../urbi-sdk"
%%% End:

\section{System}
Details on the architecture the \urbi server runs on.

\begin{itemize}
\item \lstinline'Platform'\\
  See \refObject{System.Platform}

\item \lstinline'reboot'\\
  Restart the \urbi server.  Architecture dependent.

\item \lstinline'shutdown'\\
  Have the \urbi server shut down.  All the connections are closed,
  the resources are released.  Architecture dependent.
\end{itemize}

%%% Local Variables:
%%% mode: latex
%%% TeX-master: "../urbi-sdk"
%%% End:
%% Copyright (C) 2009-2011, Gostai S.A.S.
%%
%% This software is provided "as is" without warranty of any kind,
%% either expressed or implied, including but not limited to the
%% implied warranties of fitness for a particular purpose.
%%
%% See the LICENSE file for more information.

\section{System.PackageInfo}
Information about \usdk and its components.

\subsection{Prototypes}
\begin{itemize}
\item \refObject{Object}
\end{itemize}

\subsection{Slots}
\begin{urbiscriptapi}
\item[bugReport] The address where to send bug reports.
\begin{urbiassert}
System.PackageInfo.components["Urbi SDK"].bugReport
  == "kernel-bugs@lists.gostai.com";
\end{urbiassert}

\item[components] A Dictionary of the components loaded in the package.
\begin{urbiassert}
System.PackageInfo.components.keys == ["Libport", "Urbi SDK"];
\end{urbiassert}

\item[copyrightHolder] Who owns the copyright of the package.
\begin{urbiassert}
System.PackageInfo.components["Urbi SDK"].copyrightHolder
  == "Gostai S.A.S.";
\end{urbiassert}

\item[copyrightYears] The years that the copyright covers.
\begin{urbiassert}
System.PackageInfo.components["Urbi SDK"].copyrightYears
  == "2004-2011";
\end{urbiassert}

\item[date] A string corresponding to the source date (day and time).
\begin{urbiassert}
System.PackageInfo.components["Urbi SDK"].date.isA(String);
\end{urbiassert}

\item[day] The day part of the date string.
\begin{urbiassert}
System.PackageInfo.components["Urbi SDK"].day.isA(String);
\end{urbiassert}

\item[description] The complete description string, with slashes.
\begin{urbiassert}
System.PackageInfo.components["Urbi SDK"].description.isA(String);
\end{urbiassert}

\item[id] The identification string.
\begin{urbiassert}
System.PackageInfo.components["Urbi SDK"].id.isA(String);
\end{urbiassert}

\item[major] The major version component (see \refSlot{version}), for
  instance \lstinline|2|.
\begin{urbiassert}
System.PackageInfo.components["Urbi SDK"].major.isA(Float);
\end{urbiassert}

\item[minor] The minor version component (see \refSlot{version}), for
  instance \lstinline|7|.
\begin{urbiassert}
System.PackageInfo.components["Urbi SDK"].minor.isA(Float);
\end{urbiassert}

\item[name] The name of the package as a String.
\begin{urbiassert}
System.PackageInfo.components["Urbi SDK"].name
  == "Urbi SDK";
\end{urbiassert}

\item[patch] The number of changes since the \refSlot{version}.
\begin{urbiassert}
System.PackageInfo.components["Urbi SDK"].patch.isA(Float);
\end{urbiassert}

\item[revision] The revision string.
\begin{urbiassert}
System.PackageInfo.components["Urbi SDK"].revision.isA(String);
\end{urbiassert}

\item[string] Name and version concatenated, for instance
  \lstinline|Urbi SDK 2.7.1|.
\begin{urbiassert}
System.PackageInfo.components["Urbi SDK"].string.isA(String);
\end{urbiassert}

\item[subMinor] The sub-minor version component (see \refSlot{version}).
\begin{urbiassert}
System.PackageInfo.components["Urbi SDK"].subMinor
  == 1;
\end{urbiassert}

\item[tarballVersion] The complete version string, with dashes.
\begin{urbiassert}
System.PackageInfo.components["Urbi SDK"].tarballVersion.isA(String);
\end{urbiassert}

\item[tarname] Name of the tarball.
\begin{urbiassert}
System.PackageInfo.components["Urbi SDK"].tarname
  == "urbi-sdk";
\end{urbiassert}

\item[version] The version string, such as \lstinline|"2.7"| or
  \lstinline|"2.7.1"|.
\begin{urbiscript}
do (System.PackageInfo.components["Urbi SDK"])
{
  assert
  {
    version ==
      {
        if (subMinor)
          "%s.%s.%s" % [major, minor, subMinor]
        else
          "%s.%s" % [major, minor]
      };
  };
}|;
\end{urbiscript}

\item[versionRev] Version and revision together.
\begin{urbiscript}
do (System.PackageInfo.components["Urbi SDK"])
{
  assert
  {
    versionRev
    == "version %s patch %s revision %s"
       % [version, patch, revision];
  };
}|;
\end{urbiscript}

\item[versionValue] The version as an integer, for instance 2007001.
\begin{urbiscript}
do (System.PackageInfo.components["Urbi SDK"])
{
  assert
  {
    versionValue == (major * 1e6 + minor * 1e3 + subMinor);
  };
}|;
\end{urbiscript}
\end{urbiscriptapi}

%%% Local Variables:
%%% mode: latex
%%% TeX-master: "../../urbi-sdk"
%%% ispell-dictionary: "american"
%%% ispell-personal-dictionary: "../../urbi.dict"
%%% fill-column: 76
%%% End:
%% Copyright (C) 2009-2010, Gostai S.A.S.
%%
%% This software is provided "as is" without warranty of any kind,
%% either expressed or implied, including but not limited to the
%% implied warranties of fitness for a particular purpose.
%%
%% See the LICENSE file for more information.

\section{System.Platform}
A description of the platform (the computer) the server is running on.

\subsection{Prototypes}
\begin{itemize}
\item \refObject{Object}
\end{itemize}

\subsection{Slots}
\begin{urbiscriptapi}
\item[host] The type of system \usdk runs on.  Composed of the CPU, the
  vendor, and the OS.
\begin{urbiassert}
System.Platform.host ==
  "%s-%s-%s" % [System.Platform.hostCpu,
                System.Platform.hostVendor,
                System.Platform.hostOs];
\end{urbiassert}

\item[hostAlias] The name of the system \usdk runs on as the person who
  compiled it decided to name it.  Typically empty, it is fragile to depend
  on it.
\begin{urbiassert}
System.Platform.hostAlias.isA(String);
\end{urbiassert}

\item[hostCpu] The CPU type of system \usdk runs on.  The following values
  are those for which Gostai provides binary builds.
\begin{urbiassert}
System.Platform.hostCpu in ["i386", "i686", "x86_64"];
\end{urbiassert}

\item[hostOs] The OS type of system \usdk runs on.  For instance
  \lstinline|darwin9.8.0| or \lstinline|linux-gnu| or \lstinline|mingw32|.

\item[hostVendor] The vendor type of system \usdk runs on.  The following
  values are those for which Gostai provides binary builds.
\begin{urbiassert}
System.Platform.hostVendor in ["apple", "pc", "unknown"];
\end{urbiassert}

\item[isWindows] Whether running under Windows.
\begin{urbiassert}
System.Platform.isWindows in [true, false];
\end{urbiassert}

\item[kind] Either \code{"POSIX"} or \code{"WIN32"}.
\begin{urbiassert}
System.Platform.kind in ["POSIX", "WIN32"];
\end{urbiassert}
\end{urbiscriptapi}

%%% Local Variables:
%%% coding: utf-8
%%% mode: latex
%%% TeX-master: "../../urbi-sdk"
%%% ispell-dictionary: "american"
%%% ispell-personal-dictionary: "../../urbi.dict"
%%% fill-column: 76
%%% End:
 %% Beware of sorting lines
\section{Tag}

A Tag is an object which you can use to label blocks of code in order
to control them externally.  Tagged code can be freezed, resumed,
stopped\ldots See also \autoref{sec:tut:tags}.

\subsection{Construction}
\label{stdlib:tag:ctor}

Tags are objects, and must be created as any object by using
\lstinline{new} to create derivatives of the \lstinline{Tag} object.
The name is optional, it makes easier to display a tag and remember
what it is.

\begin{urbiscript}
// Anonymous tag.
var t1 = Tag.new;
[00000001] Tag<tag_8>

// Named tag.
var t2 = Tag.new("cool name");
[00000001] Tag<cool name>
\end{urbiscript}

\subsection{Examples}

\subsubsection{Stop}
\label{sec:specs:tag:stop}

To \dfn{stop} a tag means to kill all the code currently running that
it labels.  It does not affect ``newcomers''.

\begin{urbiscript}
var t = Tag.new|;
var t0 = time|;
t: every(1s) echo("foo"),
sleep(2.2s);
[00000158] *** foo
[00001159] *** foo
[00002159] *** foo

t.stop;
// Nothing runs.
sleep(2.2s);

t: every(1s) echo("bar"),
sleep(2.2s);
[00000158] *** bar
[00001159] *** bar
[00002159] *** bar

t.stop;
\end{urbiscript}

\lstinline|System.stop| can be used to inject a return value to a
tagged expression.

\begin{urbiscript}
var t = Tag.new|;
var res;
detach(res = { t: every(1s) echo("computing") });
sleep(2.2s);
[00000001] *** computing
[00000002] *** computing
[00000003] *** computing

t.stop("result");
assert(res == "result");
\end{urbiscript}

Be extremely cautious, as of today the precedence rules are
misleading: \lstinline|\var{var} = \var{tag}: \var{exp}| is read as
\lstinline|(\var{var} = \var{tag}): \var{exp}| (i.e., defining
\var{var} as an alias to \var{tag} and using it to tag \var{exp}), not as
\lstinline|\var{var} = { \var{tag}: \var{exp} }|.  Contrast the
following example, which is most probably an error from the user, with
the previous, correct, one.

\begin{urbiscript}
var t = Tag.new("t")|;
var res;
res = t: every(1s) echo("computing"),
sleep(2.2s);
[00000001] *** computing
[00000002] *** computing
[00000003] *** computing

t.stop("result");
assert(res == "result");
[00000004:error] !!! failed assertion: res == "result" (Tag<t> != "result")
\end{urbiscript}


\subsubsection{Block/unblock}
\label{sec:specs:tag:block}

To \dfn{block} a tag means:
\begin{itemize}
\item Stop running pieces of code it labels (as with
  \lstinline|Tag.stop|).
\item Ignore new pieces of code it labels (this differs from
  \lstinline|Tag.stop|).
\end{itemize}

One can \dfn{unblock} the tag.  Contrary to
\lstinline|freeze|/\lstinline|unfreeze|, tagged code does not resume
the execution.

\begin{urbiscript}
var ping = Tag.new("ping")|;
ping:
  every (1s)
    echo("ping"),
assert(!ping.blocked);
sleep(2.1s);
[00000000] *** ping
[00002000] *** ping
[00002000] *** ping

ping.block;
assert(ping.blocked);

ping:
  every (1s)
    echo("pong"),

// Neither new nor old code runs.
ping.unblock;
assert(!ping.blocked);
sleep(2.1s);

// But we can use the tag again.
ping:
  every (1s)
    echo("ping again"),
sleep(2.1s);
[00004000] *** ping again
[00005000] *** ping again
[00006000] *** ping again
\end{urbiscript}

As with \lstinline|stop|, one can force the value of stopped
expressions.

\begin{urbiscript}
assert(
  ["foo", "foo", "foo"]
  ==
  {
    var t = Tag.new;
    var res = [];
    for (3)
      detach(res << {t: sleep(inf)});
    t.block("foo");
    res;
  });
\end{urbiscript}

\subsubsection{Freeze/unfreeze}
\label{sec:specs:tag:freeze}

To \dfn{freeze} a tag means holding the execution of code it labels.
This applies to code already being run, and ``arriving'' pieces of code.

\begin{urbiscript}
var t = Tag.new|;
var t0 = time|;
t: every(1s) echo("time   : %.0f" % (time - t0)),
sleep(2.2s);
[00000158] *** time   : 0
[00001159] *** time   : 1
[00002159] *** time   : 2

t.freeze;
assert(t.frozen);
t: every(1s) echo("shifted: %.0f" % (shiftedTime - t0)),
sleep(2.2s);
// The tag is frozen, nothing is run.

// Unfreeze the tag: suspended code is resumed.
// Note the difference between "time" and "shiftedTime".
t.unfreeze;
assert(!t.frozen);
sleep(2.2s);
[00004559] *** shifted: 2
[00005361] *** time   : 5
[00005560] *** shifted: 3
[00006362] *** time   : 6
[00006562] *** shifted: 4
\end{urbiscript}


\subsubsection{Scope tags}
\label{sec:specs:tag:scope}

Scopes feature a \lstindex{scopeTag}, i.e., a tag which will be stop
when the execution reaches the end of the current scope.  This is
handy to implement cleanups, how ever the scope was exited from.

\begin{urbiscript}
{
  var t = scopeTag;
  t: every(1s)
      echo("foo"),
  sleep(2.2s);
};
[00006562] *** foo
[00006562] *** foo
[00006562] *** foo

{
  var t = scopeTag;
  t: every(1s)
      echo("bar"),
  sleep(2.2s);
  throw 42;
};
[00006562] *** bar
[00006562] *** bar
[00006562] *** bar
[00006562:error] !!! 42
sleep(2s);
\end{urbiscript}

\subsubsection{Enter/leave events}
\label{sec:specs:tag:enter-leave}

Tags provide two events, \lstinline|enter| and \lstinline|leave|, that
trigger whenever flow control enters or leaves statements they tag.

\begin{urbiscript}
var t = Tag.new("t");
[00000000] Tag<t>

at (t.enter?)
  echo("enter");
at (t.leave?)
  echo("leave");

t: {echo("inside"); 42};
[00000000] *** enter
[00000000] *** inside
[00000000] *** leave
[00000000] 42
\end{urbiscript}

This feature is fundamental, since it is a concise and safe way to
ensure code will be executed upon exiting a chunk of code (like
\acronym{raii} in \Cxx or \lstinline|finally| in Java). The exit code
will be run no matter what the reason for leaving the block was:
natural exit, exceptions, flow control instructions like
\lstinline|return| or \lstinline|break|, \ldots

For instance, suppose we want to make sure we turn the gas off when
we're done cooking. Here is the \emph{bad} way to do it:

\begin{urbiscript}
function cook()
{
  turn_gas_on();

  // Cooking code ...

  turn_gas_off();
}|

enter_the_kitchen();
cook();
leave_the_kitchen();
\end{urbiscript}

This is bad because there are several situation where we could leave
the kitchen with gas still turned on. Consider the following cooking
code:

\begin{urbiscript}
function cook()
{
  turn_gas_on();

  if (meal_ready)
  {
    echo("The meal is already there, nothing to do!");
    // Oops ...
    return
  };

  for (var ingredient in recipe)
    if (ingredient not in kitchen)
      // Oops ...
      throw Exception("missing ingredient: %s" % ingredient)
    else
      put_ingredient();

  // ...

  turn_gas_off();
}|

enter_the_kitchen();
cook();
leave_the_kitchen();
\end{urbiscript}

Here you can see that if the meal was already prepared, or that if an
ingredient is missing, we will leave the \lstinline|cook| function
without executing the \lstinline|turn_gas_off| statement, through the
\lstinline|return| statement or the exception. The right way to ensure
gas is necessarily turned off is:

\begin{urbiscript}
function cook()
{
  var with_gas = Tag.new("with_gas");

  at (with_gas.enter?)
    turn_gas_on();
  at (with_gas.leave?)
    turn_gas_off();

  with_gas: {
    // Cooking code. Even if exception are thrown here or return is called,
    // the gas will be turned off.
  }
}|

enter_the_kitchen();
cook();
leave_the_kitchen();
\end{urbiscript}

\subsubsection{Begin/end}
\label{sec:specs:tag:begin-end}

The \lstinline|begin| and \lstinline|end| methods enable you to
monitor when your code is executed. This is a good example of a good
use of enter and leave events (\autoref{sec:specs:tag:enter-leave}),
which are use in backend to implement this feature.

\begin{urbiscript}
var mytag = Tag.new("mytag");
[00000000] Tag<mytag>

mytag.begin: echo(1);
[00000000] *** mytag: begin
[00000000] *** 1

mytag.end: echo(2);
[00000000] *** 2
[00000000] *** mytag: end

mytag.begin.end: echo(3);
[00000000] *** mytag: begin
[00000000] *** 3
[00000000] *** mytag: end
\end{urbiscript}

\subsection{Slots}

\begin{itemize}
\item \lstinline|block(\var{result} = void)|~\\
  Block any code tagged by \lstinline|this|.  Blocked tags can be
  unblocked using \lstinline|Tag.unblock|.  If some \var{result} was
  specified, let stopped code return \var{result} as value.  See
  \autoref{sec:specs:tag:block}.

\item \lstinline|begin|~\\
  A subtag that prints out "tag\_name: begin" everytime flow control
  enters the tagged code. See \autoref{sec:specs:tag:begin-end}.

\item \lstinline|blocked|~\\
  Whether code tagged by \lstinline|this| is blocked.  See
  \autoref{sec:specs:tag:block}.

\item \lstinline|end|~\\
  A subtag that prints out "tag\_name: end" everytime flow control
  leaves the tagged code. See \autoref{sec:specs:tag:begin-end}.

\item \lstinline|enter|~\\
  An event that trigger everytime flow control enter the tagged code.
  See \autoref{sec:specs:tag:enter-leave}.

\item \lstinline|freeze|~\\
  Suspend code tagged by \lstinline|this|, already running or
  forthcoming.  Frozen code can be later unfrozen using
  \lstinline|Tag.unfreeze|.  See \autoref{sec:specs:tag:freeze}.

\item \lstinline|frozen|~\\
  Whether the tag is frozen. See  \autoref{sec:specs:tag:freeze}.

\item \lstinline|leave|~\\
  An event that trigger everytime flow control leaves the tagged code.
  See \autoref{sec:specs:tag:enter-leave}.

\item \lstinline|stop(\var{result} = void)|~\\
  Stop any code tagged by \lstinline|this|.  If some \var{result} was
  specified, let stopped code return \var{result} as value.
  See \autoref{sec:specs:tag:stop}.

\item \lstinline|unblock|~\\
  Unblock \lstinline|this|.  See \autoref{sec:specs:tag:block}.

\item \lstinline|unfreeze|~\\
  Unfreeze code tagged by \lstinline|this|.  See
  \autoref{sec:specs:tag:freeze}.
\end{itemize}

%%% Local Variables:
%%% mode: latex
%%% TeX-master: "../urbi-sdk"
%%% End:

\section{Timeout}

\dfn{Timeout} allow to tag some pieces of code which should be executed in limited time.

\subsection{Prototypes}
\begin{itemize}
\item \refObject{Tag}
\end{itemize}

\subsection{Construction}
A Timeout can be constructed like any other Tag but without name and
with time and optional if you want object Timeout to throw exception
or not in case of timeout. By default, Timeout throws exception.

\begin{urbiscript}
var t = Timeout.new(300ms, false);
[00000000] Timeout_0x953c1e0
\end{urbiscript}

Use it as a tag :

\begin{urbiscript}
var t = Timeout.new(300ms, false);
[00000000] Timeout_0x953c1e0
t:{
  echo("This code will execute.");
  sleep(350ms);
  echo("But not this one.");
};
[00000000] *** This code will execute.
\end{urbiscript}

Even if exception has been disabled in constructor, you can know if your piece of code has timed out or not using the member ``timedOut''.

\begin{urbiscript}
t:sleep(350ms);
if (t.timedOut)
  echo("Your code has just timed out !");
[00000000] *** Your code has just timed out !
\end{urbiscript}

%%% Local Variables:
%%% mode: latex
%%% TeX-master: "../urbi-sdk"
%%% End:

%% Copyright (C) 2010, Gostai S.A.S.
%%
%% This software is provided "as is" without warranty of any kind,
%% either expressed or implied, including but not limited to the
%% implied warranties of fitness for a particular purpose.
%%
%% See the LICENSE file for more information.

\section{Traceable}
Objects that have a concept of backtrace.

This object, made to serve as prototype, provides a definition of backtrace
which can be filtered based on the desired level of verbosity.

This prototype is not made to be constructed.

\subsection{Slots}

\begin{urbiscriptapi}
\item[backtrace] A call stack as a \refObject{List} of
  \refObject[StackFrame]{StackFrames}.  Used by \refSlot[Exception]{backtrace} and
  \refSlot[Job]{backtrace}.
\begin{urbiscript}
try
{
  [1].map(closure (v) { throw Exception.new("Ouch") })
}
catch (var e)
{
  for| (var sf: e.backtrace)
    echo(sf.name)
};
[00000001] *** map
\end{urbiscript}

\item[hideSystemFiles] Remove system files from the backtrace if this value
  equals \lstinline|true|.  Defaults to \lstinline|true|.
\begin{urbiscript}
Traceable.hideSystemFiles = false |

try
{
  [1].map(closure (v) { throw Exception.new("Ouch") })
}
catch (var e)
{
  for| (var sf: e.backtrace)
    echo(sf.name)
};
[00000002] *** f
[00000003] *** each|
[00000004] *** map
\end{urbiscript}

\end{urbiscriptapi}


%%% Local Variables:
%%% coding: utf-8
%%% mode: latex
%%% TeX-master: "../urbi-sdk"
%%% ispell-dictionary: "american"
%%% ispell-personal-dictionary: "../urbi.dict"
%%% fill-column: 76
%%% End:

%% Copyright (C) 2010, 2011, Gostai S.A.S.
%%
%% This software is provided "as is" without warranty of any kind,
%% either expressed or implied, including but not limited to the
%% implied warranties of fitness for a particular purpose.
%%
%% See the LICENSE file for more information.

\section{TrajectoryGenerator}

The trajectory generators change the value of a given variable from an
\dfn{initial value} to a \dfn{target value}.  They can be
\dfn{open-loop}, i.e., the intermediate values depend only on the
initial and/or target value of the variable; or \dfn{closed-loop},
i.e., the intermediate values also depend on the current value value
of the variable.

Open-loop trajectories are insensitive to changes made elsewhere to
the variable.  Closed-loop trajectories \emph{are} sensitive to
changes made elsewhere to the variable --- for instance when the human
physically changes the position of a robot's motor.

Trajectory generators are not made to be used directly, rather use the
``continuous assignment'' syntax (\autoref{sec:lang:traj}).


\subsection{Prototypes}
\begin{itemize}
\item \refObject{Object}
\end{itemize}

\subsection{Examples}
\label{sec:traj:examples}

%% \subsubsection{Trajectory Name}
%% -------------------------------
\let\subsubsectionOrig\subsubsection
\renewcommand{\subsubsection}[1]
{%
  \subsubsectionOrig{\label{sec:traj:#1}#1}%
  %% It is on purpose that we pass [] to lstinline, because we do in many
  %% other places, and that would result in index entries that makeindex is
  %% unable to merge.
  \index{#1@\lstinline[]{#1}}%
}

\subsubsection{Accel}

The \lstinline{Accel} trajectory reaches a target value at a fixed
acceleration (\lstinline{accel} attribute).

\urbitrajectory{accel}

\subsubsection{Cos}

The \lstinline{Cos} trajectory implements a cosine around the target
value, given an amplitude (\lstinline{ampli} attribute) and period
(\lstinline{cos} attribute).

This trajectory is not ``smooth'': the initial value of the variable
is not taken into account.

\urbitrajectory{cos}

\subsubsection{Sin}

The \lstinline{Sin} trajectory implements a sine around the target
value, given an amplitude (\lstinline{ampli} attribute) and period
(\lstinline{sin} attribute).

This trajectory is not ``smooth'': the initial value of the variable
is not taken into account.

\urbitrajectory{sin}

\subsubsection{Smooth}

The \lstinline{Smooth} trajectory implements a sigmoid.  It changes
the variable from its current value to the target value ``smoothly''
in a given amount of time (\lstinline{smooth} attribute).

\urbitrajectory{smooth}

\subsubsection{Speed}

The \lstinline{Speed} trajectory changes the value of the variable
from its current value to the target value at a fixed speed (the
\lstinline{speed} attribute).

\urbitrajectory{speed}

If the \lstinline{adaptive} attribute is set to true, then the
duration of the trajectory is constantly reevaluated.

\urbitrajectory{speed-adaptive}

\subsubsection{Time}

The \lstinline{Time} trajectory changes the value of the variable from
its current value to the target value within a given duration (the
\lstinline{time} attribute).

\urbitrajectory{time}

If the \lstinline{adaptive} attribute is set to true, then the
duration of the trajectory is constantly reevaluated.

\urbitrajectory{time-adaptive}

%% Restore the definition of \subsubsection.
\let\subsubsection\subsubsectionOrig

\subsubsection{Trajectories and Tags}

Trajectories can be managed using \refObject[Tag]{Tags}.  Stopping or blocking
a tag that manages a trajectory kill the trajectory.

\urbitrajectory{cos-stop}
\urbitrajectory{cos-block}

When a trajectory is frozen, its local time is frozen too, the movement
proceeds from where it was rather than from where it would have been had it
been not frozen.

\urbitrajectory{cos-freeze}


\subsection{Construction}

You are not expected to construct trajectory generators by hand, using
modifiers is the recommended way to construct trajectories.  See
\autoref{sec:lang:traj} for details about trajectories, and see
\autoref{sec:traj:examples} for an extensive set of examples.

\subsection{Slots}

\begin{urbiscriptapi}
\item[Accel] This class implements the \lstinline|Accel| trajectory
  (\autoref{sec:traj:Accel}).  It derives from
  \refSlot{OpenLoop}.

\item[ClosedLoop] This class factors the implementation of the
  \dfn{closed-loop} trajectories.  It derives from
  \lstinline|TrajectoryGenerator|.

\item[OpenLoop] This class factors the implementation of the \dfn{open-loop}
  trajectories.  It derives from \lstinline|TrajectoryGenerator|.

\item[Sin] This class implements the \lstinline|Cos| and \lstinline|Sin|
  trajectories (\autoref{sec:traj:Cos}, \autoref{sec:traj:Sin}).  It derives
  from \refSlot{OpenLoop}.

\item[Smooth] This class implements the \lstinline|Smooth| trajectory
  (\autoref{sec:traj:Smooth}).  It derives from
  \refSlot{OpenLoop}.

\item[SpeedAdaptive] This class implements the \lstinline|Speed| trajectory
  when the \lstinline|adaptive| attribute is given
  (\autoref{sec:traj:Speed}).  It derives from
  \refSlot{ClosedLoop}.

\item[Time] This class implements the non-adaptive \lstinline|Speed| and
  \lstinline|Time| trajectories (\autoref{sec:traj:Speed},
  \autoref{sec:traj:Time}).  It derives from
  \refSlot{OpenLoop}.

\item[TimeAdaptive] This class implements the \lstinline|Time| trajectory
  when the \lstinline|adaptive| attribute is given
  (\autoref{sec:traj:Time}).  It derives from \refSlot{ClosedLoop}.
\end{urbiscriptapi}

%%% Local Variables:
%%% coding: utf-8
%%% mode: latex
%%% TeX-master: "../urbi-sdk"
%%% ispell-dictionary: "american"
%%% ispell-personal-dictionary: "../urbi.dict"
%%% fill-column: 76
%%% End:

%% Copyright (C) 2009-2010, Gostai S.A.S.
%%
%% This software is provided "as is" without warranty of any kind,
%% either expressed or implied, including but not limited to the
%% implied warranties of fitness for a particular purpose.
%%
%% See the LICENSE file for more information.

\section{Triplet}

A \dfn{triplet} (or \dfn{triple}) is a container storing three
objects.


\subsection{Prototype}
\begin{refObjects}
\item[Tuple]
\end{refObjects}

\subsection{Construction}

A \lstinline|Triplet| is constructed with three arguments.

\begin{urbiscript}[firstnumber=1]
Triplet.new(1, 2, 3);
[00000001] (1, 2, 3)

Triplet.new(1, 2);
[00000003:error] !!! new: expected 3 arguments, given 2

Triplet.new(1, 2, 3, 4);
[00000003:error] !!! new: expected 3 arguments, given 4
\end{urbiscript}

\subsection{Slots}
\begin{urbiscriptapi}
\item[first]
  Return the first member of the pair.
\begin{urbiassert}
Triplet.new(1, 2, 3).first == 1;
Triplet[0] === Triplet.first;
\end{urbiassert}

\item[second]
  Return the second member of the triplet.
\begin{urbiassert}
Triplet.new(1, 2, 3).second == 2;
Triplet[1] === Triplet.second;
\end{urbiassert}

\item[third]
  Return the third member of the triplet.
\begin{urbiassert}
Triplet.new(1, 2, 3).third == 3;
Triplet[2] === Triplet.third;
\end{urbiassert}
\end{urbiscriptapi}



%%% Local Variables:
%%% coding: utf-8
%%% mode: latex
%%% TeX-master: "../urbi-sdk"
%%% ispell-dictionary: "american"
%%% ispell-personal-dictionary: "../urbi.dict"
%%% fill-column: 76
%%% End:

\section{Tuple}

A \dfn{tuple} is a container storing a fixed number of objects.
Examples include \refObject{Pair} and \refObject{Triplet}.

\subsection{Prototype}
\begin{itemize}
\item \refObject{Object}
\end{itemize}

\subsection{Construction}

The \lstinline|Tuple| object is not meant to be instantiated, its main
purpose is to share code for its descendants, such as \refObject{Pair}.
Yet it accepts its members as a list.

\begin{urbiscript}[firstnumber=1]
var t = Tuple.new([1, 2, 3]);
[00000000] (1, 2, 3)
\end{urbiscript}

The output generated for a \lstinline|Tuple| can also be used to create a
\lstinline|Tuple|.  Expressions are put inside parenthesis and separated by
commas.  One extra comma is allowed after the last element.  To avoid
confusion between a 1 member \lstinline|Tuple| and a parenthesized
expression, the extra comma must be added.  \lstinline|Tuple| with no
expressions are also accepted.

\begin{urbiscript}
// not a Tuple
(1);
[00000000] 1

// Tuples
();
[00000000] ()
(1,);
[00000000] (1,)
(1, 2);
[00000000] (1, 2)
(1, 2, 3, 4,);
[00000000] (1, 2, 3, 4)
\end{urbiscript}


\subsection{Slots}
\begin{urbiscriptapi}
\item[asString]
  Generate the string \samp{(\var{first}, \var{second}, ...)} using
  \code{asPrintable} to convert members to strings.

\item \lstinline|'[]'(\var{index})|\\
  Return the \var{index}-th element.  \var{index} must be in bounds.
\begin{urbiassert}
(1, 2, 3)[0] == 1;
(1, 2, 3)[1] == 2;
\end{urbiassert}

\item \lstinline|'[]='(\var{index}, \var{value})|\\
  Set (and return) the \var{index}-th element to \var{value}.
  \var{index} must be in bounds.
\begin{urbiscript}
{
  var t = (1, 2, 3);
  assert
  {
    (t[0] = 2) == 2;
    t == (2, 2, 3);
  };
};
\end{urbiscript}

\item \lstinline|'<'(\var{other})|\\
  Lexicographic comparison between two tuples.
\begin{urbiassert}
(0, 0) < (0, 1);
(0, 0) < (1, 0);
(0, 1) < (1, 0);
\end{urbiassert}

\item \lstinline|'=='(\var{other})|\\
  Whether \lstinline|this| and \lstinline|other| have the same
  contents (equality-wise).
\begin{urbiassert}
  (1, 2) == (1, 2);
!((1, 1) == (2, 2));
\end{urbiassert}
\end{urbiscriptapi}



%%% Local Variables:
%%% mode: latex
%%% TeX-master: "../urbi-sdk"
%%% ispell-dictionary: "american"
%%% ispell-personal-dictionary: "../urbi.dict"
%%% End:

%% Copyright (C) 2010, Gostai S.A.S.
%%
%% This software is provided "as is" without warranty of any kind,
%% either expressed or implied, including but not limited to the
%% implied warranties of fitness for a particular purpose.
%%
%% See the LICENSE file for more information.

\section{UObject}

UObject is used by the \lstinline|UObject| API (see
\autoref{part:uobject}) to represent a bound \Cxx instance.

All the UObjects are copied under a unique name as slots of
\refSlot[Global]{uobjects}.

\subsection{Prototypes}
\begin{refObjects}
\item[Object]
\end{refObjects}

\subsection{Slots}

\begin{urbiscriptapi}
\item[__uobjectName]%
  Unique name assigned to this object. This is also the slot name of
  \refSlot[Global]{uobjects} containing this \lstinline|UObject|.

\item[searchPath] The search-path for UObjects files (see
  \autoref{sec:tools:envvars}) as a \refObject{List} of \refObject[Path]{Paths}.
  See also \refSlot[System]{loadLibrary} and \refSlot[System]{loadModule}.
\begin{urbiassert}
UObject.searchPath.isA(List);
UObject.searchPath[0].isA(Path);
\end{urbiassert}
\end{urbiscriptapi}

%%% Local Variables:
%%% coding: utf-8
%%% mode: latex
%%% TeX-master: "../urbi-sdk"
%%% ispell-dictionary: "american"
%%% ispell-personal-dictionary: "../urbi.dict"
%%% fill-column: 76
%%% End:

%% Copyright (C) 2010, 2011, Gostai S.A.S.
%%
%% This software is provided "as is" without warranty of any kind,
%% either expressed or implied, including but not limited to the
%% implied warranties of fitness for a particular purpose.
%%
%% See the LICENSE file for more information.

\section{UValue}


The UValue object is used internally by the UObject API and is mostly
hidden from the user.  Do not depend on it.

%% FIXME:

\subsection{Prototypes}

\begin{refObjects}
\item[Object]
\end{refObjects}

\subsection{Slots}

\begin{urbiscriptapi}
\item[asPrintable]
\item[asString]
\item[asTopLevelPrintable]
\item[asUValue]
\item[extract]
\item[extractAsToplevelPrintable]
\item[invalidate]
\item[put]
\item[transparent]
\end{urbiscriptapi}

%%% Local Variables:
%%% mode: latex
%%% TeX-master: "../urbi-sdk"
%%% ispell-dictionary: "american"
%%% ispell-personal-dictionary: "../urbi.dict"
%%% fill-column: 76
%%% End:

\section{UVar}

%% FIXME:
%%
%% \subsection{Prototypes}
%%
%% \begin{refObjects}
%% \item[Object]
%% \end{refObjects}
%%
%% \subsection{Slots}
%%
%% \begin{urbiscriptapi}
%%
%% \end{urbiscriptapi}

%%% Local Variables:
%%% mode: latex
%%% TeX-master: "../urbi-sdk"
%%% ispell-dictionary: "american"
%%% ispell-personal-dictionary: "../urbi.dict"
%%% End:

\section{void}

\dfn{void} is an entity which meaning is "no value" so that it has no
prototype and cannot be used as a value.

%%% Local Variables:
%%% mode: latex
%%% TeX-master: "../urbi-sdk"
%%% End:


%% Restore the definition of \section.
\let\section\sectionOrig

%%% Local Variables:
%%% mode: latex
%%% TeX-master: "../urbi-sdk"
%%% ispell-dictionary: "american"
%%% ispell-personal-dictionary: "../urbi.dict"
%%% fill-column: 76
%%% End:

\FloatBarrier

\chapter{\urbi SDK specifications}
\label{sec:sdk}
