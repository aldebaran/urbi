\chapter{Programs}
\label{sec:tools}

\section{\command{urbi}}
\label{sec:tools:urbi}

The \command{urbi} program launches an \urbi server, for either batch,
interactive, or network-based executions.

\subsection{Options}

\paragraph{Tuning}
\begin{options}
\item[-s, --stack-size=\var{size}] set the coroutine \dfn{stack size}.
  The unit of \var{size} is KB; it defaults to 128.

  This option should not be needed unless you have ``stack exhausted''
  messages from \command{urbi} in which case you should try
  \option{--stack-size=512} or more.

  Alternatively you can define the environment variable
  \code{URBI\_STACK\_SIZE}.  The option \option{--stack-size} has
  precedence over the \code{URBI\_STACK\_SIZE}.
\end{options}

\paragraph{Networking}
\begin{options}
\item[-H, --host=\var{address}] Set the \var{address} on which network
  connections are listened to.  Typical values of \var{address}
  include:
  \begin{description}
  \item[localhost] only local connections are allowed (no other
    computer can reach this server).
  \item[127.0.0.1] same as \code{localhost}.
  \item[0.0.0.0] any IP v4 connection is allowed, including from
    remote computers.
  \end{description}
  Defaults to \code{0.0.0.0}.
\item[-P, --port=\var{port}] Set the port to listen incoming
  connections to.  If \var{port} is \code{-1}, no networking.  If
  \var{port} is \code{0}, then the system will chose any available
  port (see \option{--port-file}).  Defaults to \code{-1}.
\item[-w, --port-file=\var{file}] When the system is up and running,
  and when it is ready for network connections, create the file named
  \var{file} which contains the number of the port the server listens
  to.
\end{options}


\paragraph{Execution}
\begin{options}
\item[-e, --expression=\var{exp}] send the \us expression \var{exp}.
  No separator is added, you have to pass yours.
\item[-f, --file=\var{file}] send the contents of the file \var{file}.
  No separator is added, you have to pass yours.
\item[-i, --interactive] start an interactive session.
\end{options}

The options \option{-e}, \option{-f} accumulate, and are run in the
same \refObject{Lobby} as \option{-i} if used.  In other words, the
following session is valid:

\begin{shell}
# Create a file "two.u".
$ echo "var two = 2;" >two.u
# urbi -e 'var one = 1;' -f two.u -i
[00000000] 1
[00000000] 2
one + two;
[00000000] 3
\end{shell}%$

%%% Local Variables:
%%% mode: latex
%%% TeX-master: "urbi-specs"
%%% End:
