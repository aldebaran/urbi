\subsection{Rationale}

\lstinline|List|s in \us represent potentially empty collections of
elements.

\subsection{Prototypes}

\begin{itemize}
\item Object (\autoref{sec:std-object})
\end{itemize}

\subsection{Construction}

List can be created with their literal syntax, as shown in
\autoref{sec:us-syn-lit-list}. They can also be created with the
\lstinline|new| method of \lstinline|List|, given the size and an
initialization value (\autoref{lst:new-list}).

\begin{lstlisting}[caption=List.new, label=lst:new-list,
  float=\floatposh]
  List.new(5, "foo");
  [00000000] ["foo", "foo", "foo", "foo", "foo"]
\end{lstlisting}

\subsection{Methods}

\subsubsection{all}

% FIXME: link to predicate glossary entry
Return whether all the members of the target verify the given
predicate (\autoref{lst:list-all}).

\begin{lstlisting}[caption=List.all, label=lst:list-all, float=\floatposh]
  // Are all elements positive?
  [-2, 0, 2, 4].all(function (e) { e > 0 });
  [00000000] false
  // Are all elements even?
  [-2, 0, 2, 4].all(function (e) { e % 2 == 0 });
  [00000000] true
\end{lstlisting}

\subsubsection{any}

% FIXME: link to predicate glossary entry
Return whether at least one of the members of the target verify the
given predicate (\autoref{lst:list-any}).

\begin{lstlisting}[caption=List.any, label=lst:list-any, float=\floatposh]
  // Is there any even element?
  [-3, 1, -1].all(function (e) { e % 2 == 0 });
  [00000000] false
  // Is there any positive element?
  [-3, 1, -1].all(function (e) { e > 0 });
  [00000000] true
\end{lstlisting}

\subsubsection{asList}

Return the target (\autoref{lst:list-asList}).

\begin{lstlisting}[caption=List.asList, label=lst:list-asList, float=\floatposh]
  [0, 1, 2].asList;
  [00000000] [0, 1, 2]
\end{lstlisting}

\subsubsection{asString}

Return the target as a string describing the list
(\autoref{lst:list-asString}).

\begin{lstlisting}[caption=List.asString, label=lst:list-asString, float=\floatposh]
  [0, 1, 2].asString;
  [00000000] "[0, 1, 2]"
\end{lstlisting}

\subsubsection{back}

Return the last element of the target. An error if the target is empty
(\autoref{lst:list-back}).

\begin{lstlisting}[caption=List.back, label=lst:list-back, float=\floatposh]
  [0, 1, 2].back;
  [00000000] 2
  [].back;
  [00000000:error] !!! back: cannot be applied onto empty list
\end{lstlisting}

\subsubsection{clear}

Empty the target (\autoref{lst:list-clear}).

\begin{lstlisting}[caption=List.clear, label=lst:list-clear, float=\floatposh]
  var x = [0, 1, 2];
  [00000000] [0, 1, 2]
  x.clear;
  [00000000] []
\end{lstlisting}

\subsubsection{each}

Apply the given functional value on all members (\autoref{lst:list-each})
sequentially.

\begin{lstlisting}[caption=List.each, label=lst:list-each, float=\floatposh]
  [0, 1, 2].each(function (v) {echo (v * v); echo (v * v)});
  [00000000] *** 0
  [00000000] *** 0
  [00000000] *** 2
  [00000000] *** 2
  [00000000] *** 4
  [00000000] *** 4
\end{lstlisting}

\subsubsection{each\&}

Apply the given functional value on all members simultaneously
(\autoref{lst:list-each-and}).

\begin{lstlisting}[caption=List.each, label=lst:list-each-and, float=\floatposh]
  [0, 1, 2].'each&'(function (v) {echo (v * v); echo (v * v)});
  [00000000] *** 0
  [00000000] *** 2
  [00000000] *** 4
  [00000000] *** 0
  [00000000] *** 2
  [00000000] *** 4
\end{lstlisting}

\subsubsection{empty}

Return whether the target is empty (\autoref{lst:list-empty}).

\begin{lstlisting}[caption=List.empty, label=lst:list-empty, float=\floatposh]
  [].empty;
  [00000000] true
  [1].empty;
  [00000000] false
\end{lstlisting}

\subsubsection{filter}

Return the list of all member of the target that verify the given
predicate (\autoref{lst:list-filter}).

\begin{lstlisting}[caption=List.filter, label=lst:list-filter, float=\floatposh]
  // Keep only odd numbers
  [0, 1, 2, 3, 4, 5].filter(function (v) {v % 2 == 1});
  [00000000] [1, 3, 5]
\end{lstlisting}

\subsubsection{foldl}

FIXME: this is too damn hard to explain, yet so simple
(\autoref{lst:list-foldl}).

\begin{lstlisting}[caption=List.foldl, label=lst:list-foldl, float=\floatposh]

\end{lstlisting}

\subsubsection{front}
\label{sec:std-list-front}

Return the first element of the target. An error if the target is
empty (\autoref{lst:list-front}).

\begin{lstlisting}[caption=List.front, label=lst:list-front, float=\floatposh]
  [0, 1, 2].front;
  [00000000] 0
  [].front;
  [00000000:error] !!! front: cannot be applied onto empty list
\end{lstlisting}

\subsubsection{has}

Return whether one of the member of the target equals the argument
(\autoref{lst:list-has}).

\begin{lstlisting}[caption=List.has, label=lst:list-has, float=\floatposh]
  [0, 1, 2].has(1);
  [00000000] true
  [0, 1, 2].has(5);
  [00000000] false
\end{lstlisting}

\subsubsection{hasSame}

Return whether one of the member of the target is physically equal to
the argument (\autoref{lst:list-hasSame}).

\begin{lstlisting}[caption=List.hasSame, label=lst:list-hasSame, float=\floatposh]
  var x = 1;
  [0, x, 2].has(1);
  [00000000] false
  [0, x, 2].has(x);
  [00000000] true
\end{lstlisting}

\subsubsection{head}

Synonym for \lstinline|front| (\autoref{sec:std-list-front}).

\subsubsection{join}

Concatenate all members of the target to form a string, separating
them with the given separator (\autoref{lst:list-join}).

\begin{lstlisting}[caption=List.join, label=lst:list-join, float=\floatposh]
  ["b", "ob", "b"].join("a");
  [00000000] "baobab"
\end{lstlisting}

\subsubsection{map}

Apply the given functional value on every member, and return the list
of results (\autoref{lst:list-map}).

\begin{lstlisting}[caption=List.map, label=lst:list-map, float=\floatposh]
  [0, 1, 2, 3].map(function (v) { v % 2 == 0});
  [00000000] [true, false, true, false]
\end{lstlisting}

\subsubsection{nth}
\label{sec:std-list-nth}

Return the nth member of the target (indexing is zero-based). An error
if out of bounds (\autoref{lst:list-nth}).

\begin{lstlisting}[caption=List.nth, label=lst:list-nth, float=\floatposh]
  [0, 1, 2].nth(1);
  [00000000] 1
  [0, 1, 2].nth(3);
  [00007061:error] !!! nth: invalid index: 3
\end{lstlisting}

\subsubsection{pop\_back}

Remove and return the last element of the target. An error if the
target is empty (\autoref{lst:list-popback}).

\begin{lstlisting}[caption=List.pop\_back, label=lst:list-popback, float=\floatposh]
  var x = [0, 1, 2];
  [00000000] [0, 1, 2]
  x.pop_back;
  [00000000] 2
  x;
  [00000000] [0, 1]
  [].pop_back;
  [00000000:error] !!! pop_back: cannot be applied onto empty list
\end{lstlisting}

\subsubsection{pop\_front}

Remove and return the first element from the target. An error if the
target is empty (\autoref{lst:list-popfront}).

\begin{lstlisting}[caption=List.pop\_front, label=lst:list-popfront, float=\floatposh]
  var x = [0, 1, 2];
  [00000000] [0, 1, 2]
  x.pop_front;
  [00000000] 0
  x;
  [00000000] [1, 2]
  [].pop_front;
  [00000000:error] !!! pop_front: cannot be applied onto empty list
\end{lstlisting}

\subsubsection{push\_back}
\label{sec:std-list-pushback}

Insert the given element at the end of the target
(\autoref{lst:list-pushback}).

\begin{lstlisting}[caption=List.push\_back, label=lst:list-pushback, float=\floatposh]
  var x = [0, 1];
  [00000000] [0, 1]
  x.push_back(2);
  [00000000] [0, 1, 2]
  x;
  [00000000] [0, 1, 2]
\end{lstlisting}

\subsubsection{push\_front}

Insert the given element at the beginning of the target
(\autoref{lst:list-pushfront}).

\begin{lstlisting}[caption=List.push\_front, label=lst:list-pushfront, float=\floatposh]
  var x = [1, 2];
  [00000000] [1, 2]
  x.push_front(0);
  [00000000] [0, 1, 2]
  x;
  [00000000] [0, 1, 2]
\end{lstlisting}

\subsubsection{range}

Return a subrange of the string, from the first index included to the
second index excluded. An error if out of bounds (\autoref{lst:list-range}).

\begin{lstlisting}[caption=List.range, label=lst:list-range, float=\floatposh]
  [0, 1, 2, 3].range(1, 3);
  [00000000] [1, 2]
  [].range(1, 3);
  [00428697:error] !!! nth: invalid index: 1
\end{lstlisting}


\subsubsection{remove}

Remove all elements from the target that equals the argument
(\autoref{lst:list-remove}).

\begin{lstlisting}[caption=List.remove, label=lst:list-remove, float=\floatposh]
  var x = [0, 1, 0, 2, 0, 3];
  [00000000] [0, 1, 0, 2, 0, 3];
  x.remove(0);
  [00000000] [1, 2, 3];
  x;
  [00000000] [1, 2, 3];
\end{lstlisting}

\subsubsection{removeById}

Remove all elements from the target that physically equals the
argument (\autoref{lst:list-removeById}).

\begin{lstlisting}[caption=List.removeById, label=lst:list-removeById,
  float=\floatposh]
  var x = 1;
  [00000000] 1
  var l = [0, 1, x, 1, 2];
  [00000000] [0, 1, 1, 1, 2]
  l.removeById(x);
  [00000000] [0, 1, 1, 2]
  l;
  [00000000] [0, 1, 1, 2]
\end{lstlisting}

\subsubsection{reverse}

Return the target with the order of elements inverted
(\autoref{lst:list-reverse}).

\begin{lstlisting}[caption=List.reverse, label=lst:list-reverse, float=\floatposh]
  [0, 1, 2].reverse;
  [00000000] [2, 1, 0]
\end{lstlisting}

\subsubsection{==}

Check whether all elements in the target and the first argument, are
equal two by two (\autoref{lst:list-sameAs}).

\begin{lstlisting}[caption={List.==}, label=lst:list-sameAs, float=\floatposh]
  [0, 1, 2] == [0, 1, 2];
  [00000000] true
  [0, 1, 2] == [0, 0, 2];
  [00000000] false
\end{lstlisting}

\subsubsection{setNth}
\label{sec:std-list-setnth}

Assign a value to the element of the target at the given index
(\autoref{lst:list-setNth}).

\begin{lstlisting}[caption=List.setNth, label=lst:list-setNth, float=\floatposh]
  var x = [0, 1, 2];
  [00000000] [0, 1, 2]
  x.setNth(1, 42);
  [00000000] 42
  x;
  [00000000] [0, 42, 2]
\end{lstlisting}

\subsubsection{size}

Return the number of elements in the target (\autoref{lst:list-size}).

\begin{lstlisting}[caption=List.size, label=lst:list-size, float=\floatposh]
  [1, 2, 3].size;
  [00000000] 3
  [].size;
  [00000000] 0
\end{lstlisting}

\subsubsection{sort}

Return the target, sorted with respect to the \lstinline|<| criteria
(\autoref{lst:list-sort}).

\begin{lstlisting}[caption=List.sort, label=lst:list-sort, float=\floatposh]
  [1, 0, 3, 2].sort;
  [00000000] [0, 1, 2, 3]
\end{lstlisting}

\subsubsection{tail}

Return the target, minus the first element. An error if the target is
empty (\autoref{lst:list-tail}).

\begin{lstlisting}[caption=List.tail, label=lst:list-tail, float=\floatposh]
  [0, 1, 2].tail;
  [00000000] [1, 2]
  [].tail;
  [00000000:error] !!! tail: cannot be applied onto empty list
\end{lstlisting}

\subsubsection{*}

Return the target, concatenated n times to itself, n being the
argument (\autoref{lst:list-times}).

\begin{lstlisting}[caption=List.'*', label=lst:list-times,
  float=\floatposh]
  [0, 1] * 3;
  [00000000] [0, 1, 0, 1, 0, 1]
\end{lstlisting}

\subsubsection{+}

Return the concatenation of the target and the given list
(\autoref{lst:list-plus}).

\begin{lstlisting}[caption=List.'+', label=lst:list-plus,
  float=\floatpos]
  [0, 1] + [2, 3];
  [00000000] [0, 1, 2, 3]
\end{lstlisting}

\subsubsection{-}

Return the target without all element that equals any element in the
given list(\autoref{lst:list-minus}).

\begin{lstlisting}[caption=List.'', label=lst:list-minus,
  float=\floatpos]
  [0, 1, 0, 2, 3] - [1, 2];
  [00000000] [0, 0, 3]
\end{lstlisting}

\subsubsection{$<<$}

A synonym for \lstinline|push_back| (\autoref{sec:std-list-pushback}).

\subsubsection{$<$}

Return whether the target is inferior to the given list. A list is
inferior to another if at least one of its element differs from the
other, and the first differing element is inferior to the other
(\autoref{lst:list-inf}).

\begin{lstlisting}[caption=List.'', label=lst:list-inf,
  float=\floatpos]
  [0, 1, 2] < [0, 1, 2];
  [00000000] false
  [0, 1, 2] < [0, 0, 2];
  [00000000] false
  [0, 1, 2] < [0, 2, 2];
  [00000000] true
\end{lstlisting}

\subsubsection{[]}
A synonym for \lstinline|nth| (\autoref{sec:std-list-nth}).

\subsubsection{[]=}
A synonym for \lstinline|setNth| (\autoref{sec:std-list-setnth}).

