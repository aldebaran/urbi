%% Redefine \section is this chapter so that we don't have to
%% call \labelObject each time.  See the bottom of this file for the
%% restoring of \section.
\let\sectionOrig\section
\renewcommand{\section}[1]{\sectionOrig{\labelObject{#1}\index{#1@\lstinline{#1}}#1}}

\chapter{\us Standard Library}
\label{sec:stdlib}

\section{Boolean}
\section{CallMessage}

\subsection{Examples}
\subsubsection{Evaluating an argument several times}
\label{sec:std-callmsg-examples-several}
\section{Channel}
\section{Code}
\section{Comparable}

\section{Dictionary}

A \dfn{dictionary} is an \dfn{associative array}, also known as a
\dfn{hash} in some programming languages.  They are arrays whose
indexes are strings.

In a way objects are dictionaries: one can use \lstinline|setSlot|,
\lstinline|updateSlot|, and \lstinline|getSlot|.  This is unsafe since
slots also contains value and methods that object depend upon to run
properly.

\subsection{Example}

The following session demonstrates the features of the Dictionary
objects.

\begin{urbiscript}
var d = Dictionary.new("one", 1, "two", 2);
[00000001] Dictionary {"one" => 1, "two" => 2}
for (var p : d)
  echo (p.first + " -> " + p.second);
[00000003] *** one -> 1
[00000002] *** two -> 2
"three" in d;
[00000004] false
d["three"] = d["one"] + d["two"] | {};
"three" in d;
[00000005] true
d.getWithDefault("four", 4);
[00000006] 4
\end{urbiscript}


\subsection{Prototypes}

\begin{itemize}
\item \refObject{Object}
\end{itemize}

\subsection{Construction}

Dictionnary are created like any other object. The constructor can
take couples of argument to store them in the dictionary.

\begin{urbiscript}[firstnumber=last]
Dictionary.new("one", 1, "two", 2);
[00000000] Dictionary {"one" => 1, "two" => 2}
\end{urbiscript}

\subsection{Methods}

\begin{itemize}
\item \lstinline|asBool|\\
  Negation of \lstinline|Dictionary.empty|.
\begin{urbiscript}[firstnumber=last]
assert(Dictionary.new.asBool == false);
assert(Dictionary.new.set("key", "value").asBool == true);
\end{urbiscript}

\item \lstinline|asList|\\
  Return the contents of the dictionary as a \refObject{Pair} list
  (\var{key}, \var{value}).  This makes it easier to iterate over a
  Dictionary using \lstinline|for|.  No particular order is ensured.

\begin{urbiscript}[firstnumber=last]
Dictionary.new("one", 1, "two", 2).asList;
[00000000] [("one", 1), ("two", 2)]
\end{urbiscript}

\item \lstinline|clear|\\
  Empty the dictionary.

\begin{urbiscript}[firstnumber=last]
assert(Dictionary.new("one", 1).clear.empty());
\end{urbiscript}

\item \lstinline|empty|\\
  Whether the dictionary is empty.

\begin{urbiscript}[firstnumber=last]
assert(Dictionary.new.empty);
assert(!Dictionary.new.set("key", "value").empty);
\end{urbiscript}

\item \lstinline|erase(\var{key})|\\
  Remove the mapping for \lstinline|\var{key}|.

\begin{urbiscript}[firstnumber=last]
Dictionary.new("one", 1, "two", 2).erase("two");
[00386250] Dictionary {"one" => 1}
\end{urbiscript}

\item \lstinline|get(\var{key})|\\
  Return the value associated to  \lstinline|\var{key}| if it exists,
  \lstinline|void| otherwise.
\begin{urbiscript}[firstnumber=last]
assert(Dictionary.new("one", 1, "two", 2).get("one") == 1);
assert(Dictionary.new("one", 1, "two", 2).get("three").isVoid);
\end{urbiscript}


\item \lstinline|getWithDefault(\var{key}, \var{default-value})|\\
  Return the value associated to  \lstinline|\var{key}| if it exists,
  \lstinline|\var{default-value}| otherwise.

\begin{urbiscript}[firstnumber=last]
do (Dictionary.new("one", 1, "two", 2))
{
  assert(getWithDefault("one",  -1) == 1);
  assert(getWithDefault("three", 3) == 3);
}|;
\end{urbiscript}


\item \lstinline|has(\var{key})|\\
  Whether the dictionary has a mapping for \lstinline|\var{key}|.

\begin{urbiscript}[firstnumber=last]
do (Dictionary.new("one", 1))
{
  assert(has("one"));
  assert(!has("zero"));
}|;
\end{urbiscript}

\item \lstinline|init(\var{key1}, \var{value1}, ...)|~\\
  Insert the mapping from \lstinline|\var{key1}| to
  \lstinline|\var{value1}| and so forth.

\begin{urbiscript}[firstnumber=last]
Dictionary.clone.init("one", 1, "two", 2);
[00000000] Dictionary {"one" => 1, "two" => 2}
\end{urbiscript}

\item \lstinline|keys|\\
  The list of all the keys.  No particular order is ensured.
\begin{urbiscript}[firstnumber=last]
assert(Dictionary.new("one", 1, "two", 2).keys == ["one", "two"]);
\end{urbiscript}

\item \lstinline|set(\var{key}, \var{value})|\\
  Map \lstinline|\var{key}| to \lstinline|\var{value}| and return
  \lstinline|this| so that invocations to \lstinline|set| can be
  chained.  The possibly existing previous mapping is overriden.

\begin{urbiscript}[firstnumber=last]
Dictionary.new.set("one", 2).set("one", 1);
[00000000] Dictionary {"one" => 1}
\end{urbiscript}

\item \lstinline|[]=(\var{key}, \var{value})|\\
  Syntactic sugar for \lstinline|set(\var{key}, \var{value})|, but
  returns \var{value}.

\begin{urbiscript}[firstnumber=last]
{
  var d = Dictionary.new("one", "2");
  assert((d["one"] = 1) == 1);
  assert(d["one"] == 1);
};
\end{urbiscript}

\item \lstinline|[](\var{key})|\\
  Syntactic sugar for \lstinline|get(\var{key})|.

\begin{urbiscript}[firstnumber=last]
assert(Dictionary.new("one", 1)["one"] == 1);
\end{urbiscript}

\end{itemize}


%%% Local Variables:
%%% mode: latex
%%% TeX-master: "../urbi-sdk"
%%% End:


\section{Event}

\section{Float}

Float is an \us primitive to represent floating point number.

\subsection{Prototypes}

\begin{itemize}
\item \refObject{Comparable}
\item \refObject{Orderable}
\end{itemize}

\subsection{Construction}

The most common way to create fresh floats is by using the literal
syntax presented in \autoref{sec:us-syn-lit-float}. A null float can also be
obtained with \lstinline|Float|'s \lstinline|new| method.

\begin{urbiscript}
Float.new;
[00000000] 0
\end{urbiscript}

\subsection{Methods}

\subsubsection{Float.abs}

Return the absolute value of the target.

\begin{urbiscript}
(-5).abs;
[00000000] 5
(5).abs;
[00000000] 5
\end{urbiscript}

\subsubsection{Float.acos}

Return the arccosine of the target.

\begin{urbiscript}
1.acos;
[00000000] 0
\end{urbiscript}

\subsubsection{Float.asin}

Return the arcsine of the target.

\begin{urbiscript}
0.asin;
[00000000] 0
\end{urbiscript}

\subsubsection{Float.asString}

Return a string representing the target.

\begin{urbiscript}
42.asString;
[00000000] "42"
\end{urbiscript}

\subsubsection{Float.atan}

Return the arctangent of the target.

\begin{urbiscript}
0.atan;
[00000000] 0
\end{urbiscript}

\subsubsection{Float.bitand}

Return the bitwise and between the target and the argument.

\begin{urbiscript}
3 bitand 6;
[00000000] 2
\end{urbiscript}

\subsubsection{Float.bitor}

Return the bitwise or between the target and the argument.

\begin{urbiscript}
3 bitor 6;
[00000000] 7
\end{urbiscript}

\subsubsection{Float.clone}

Return a fresh Float with the same value as the target.

\begin{urbiscript}
var x = 0;
[00000000] 0
var y = x.clone;
[00000000] 0
x === y;
[00000000] false
\end{urbiscript}

\subsubsection{Float.compl}

Return the 1-complement of the target.

\begin{urbiscript}
compl 0;
[00000000] 4294967295
compl 4294967295;
[00000000] 0
\end{urbiscript}

\subsubsection{Float.cos}

Return the cosine of the target.

\begin{urbiscript}
0.cos;
[00000000] 1
\end{urbiscript}

\subsubsection{Float.exp}

Return the exponential of the target.

\begin{urbiscript}
1.exp;
[00000000] 2.71828
\end{urbiscript}

\subsubsection{Float.inf}

Return the infinity.

\begin{urbiscript}
Float.inf;
[00000000] inf
\end{urbiscript}

\subsubsection{Float.log}

Return the logarithm of the target.

\begin{urbiscript}
2.71828.log;
[00000000] 0.999999
\end{urbiscript}

\subsubsection{Float.nan}

Return the ``not a number'' special float value.

\begin{urbiscript}
Float.nan;
[00000000] nan
\end{urbiscript}

\subsubsection{Float.random}

Return a random number between 0 and the target.

\begin{urbiscript}
5.random;
[00000000] 3
5.random;
[00000000] 1
5.random;
[00000000] 2
\end{urbiscript}

\subsubsection{Float.round}

Return the target, rounded.

\begin{urbiscript}
1.6.round;
[00000000] 2
1.4.round;
[00000000] 1
\end{urbiscript}

\subsubsection{Float.seq}

Return the sequence of integer from 0 to target - 1 as a list.

\begin{urbiscript}
3.seq;
[00000000] [0, 1, 2]
\end{urbiscript}

\subsubsection{Float.sin}

Return the sinus of the target.

\begin{urbiscript}
0.sin;
[00000000] 0
\end{urbiscript}

\subsubsection{Float.sqrt}

Return the square root of the target.

\begin{urbiscript}
1024.sqrt;
[00000000] 32
\end{urbiscript}

\subsubsection{Float.tan}

Return the tangent of the target.

\begin{urbiscript}[caption=Float.tan, label=lst:float-tan]
0.tan;
[00000000] 0
\end{urbiscript}

\subsubsection{Float.trunc}

Return the target truncated.

\begin{urbiscript}
1.9.trunc;
[00000000] 1
\end{urbiscript}

\subsubsection{Float.asFloat}

Return the target.

\begin{urbiscript}
51.asFloat;
[00000000] 51
\end{urbiscript}

\subsubsection{Float.sqr}

Return the square of the target.

\begin{urbiscript}
32.sqr;
[00000000] 1024
\end{urbiscript}

\subsubsection{Float.sgn}

Return 1 if the target is positive, -1 otherwise.

\begin{urbiscript}
(-1164).sgn;
[00000000] -1
(1164).sgn;
[00000000] 1
\end{urbiscript}

\subsubsection{Float.times}

Take one functional argument and call it target times.

\begin{urbiscript}
5.times(function () { echo("ping")});
[00000000] *** ping
[00000000] *** ping
[00000000] *** ping
[00000000] *** ping
[00000000] *** ping
\end{urbiscript}

\subsubsection{Float.each}

Take one functional argument and call it on every integer from 0 to
target - 1.

\begin{urbiscript}
5.each(function (i) { echo(i * 2)});
[00000000] *** 0
[00000000] *** 2
[00000000] *** 4
[00000000] *** 6
[00000000] *** 8
\end{urbiscript}

\subsubsection{Float.\^{}}

Return the bitwise exclusive or between the target and the first argument.

\begin{urbiscript}
3 ^ 6;
[00000000] 5
\end{urbiscript}

\subsubsection{Float.\textgreater\textgreater}

\lstinline|a >> b| return the \lstinline|a| shifted by \lstinline|b|
bit towards the right.

\begin{urbiscript}
4 >> 2;
[00000000] 1
\end{urbiscript}

\subsubsection{Float.\textless}

\lstinline|a < b| returns whether \lstinline|a| is inferior to
\lstinline|b|. Note that other comparison operators
(\lstinline|<=|, \lstinline|>|, \ldots) can thus also be applied on
floats since Float inherits \refObject{Orderable}.

\begin{urbiscript}
0 < 1;
[00000000] true
1 < 0;
[00000000] false
\end{urbiscript}

\subsubsection{Float.\textless\textless}

\lstinline|a << b| return the \lstinline|a| shifted by \lstinline|b|
bit towards the left.

\begin{urbiscript}
4 << 2;
[00000000] 16
\end{urbiscript}

\subsubsection{Float.$-$}

\lstinline|a - b| returns \lstinline|a| minus \lstinline|b|.

\begin{urbiscript}
6 - 3;
[00000000] 3
\end{urbiscript}

\subsubsection{Float.+}

\lstinline|a + b| returns \lstinline|a| plus \lstinline|b|.

\begin{urbiscript}
1 + 1;
[00000000] 2
\end{urbiscript}

\subsubsection{Float./}

\lstinline|a / b| returns the quotient of \lstinline|a| divided by
\lstinline|b|.

\begin{urbiscript}
50 / 10;
[00000000] 5
\end{urbiscript}

\subsubsection{Float.\%}

\lstinline|a % b|
returns \lstinline|a| modulo \lstinline|b|.

\begin{urbiscript}
50 % 11;
[00000000] 6
\end{urbiscript}

\subsubsection{Float.*}

\lstinline|a * b| returns the product of \lstinline|a| by
\lstinline|b|.

\begin{urbiscript}
2 * 3;
[00000000] 6
\end{urbiscript}

\subsubsection{Float.**}

\lstinline|a ** b| returns \lstinline|a| to the \lstinline|b| power.

\begin{urbiscript}
2 ** 10;
[00000000] 1024
\end{urbiscript}

\subsubsection{Float.==}

\lstinline|a == b| returns whether \lstinline|a| equals \lstinline|b|.

\begin{urbiscript}
1 == 1;
[00000000] true
1 == 2;
[00000000] false
\end{urbiscript}

%%% Local Variables:
%%% mode: latex
%%% TeX-master: "../urbi-sdk"
%%% End:

% LocalWords:  Orderable lst acos arccosine asin arcsine asString atan bitand
% LocalWords:  arctangent bitwise bitor compl sqrt trunc asFloat sqr sgn rshift
% LocalWords:  lshift eq


\section{Group}

\section{Lazy}

\dfn{Lazies} are objects that hold a lazy value, that is, a not yet evaluated
value. They provide facilities to evaluate their content only once
(\dfn{memoization}) or several times. Lazy are essentially used in call
messages, to represent lazy arguments, as described in
\autorefObject{CallMessage}.

\subsection{Construction}

Lazies are seldom instantiated manually. They are mainly created
automatically when a lazy function call is made (see
\autoref{sec:us-fun-callmsg}). One can however create a lazy value with the
standard \lstinline|new| method of \lstinline|Lazy|, giving it an
argument-less function which evaluates to the lazified value.

\begin{urbiscript}
Lazy.new(closure () { /* Value to lazify */ });
[00000000:hide] Lazy_0xADDR
\end{urbiscript}

\subsection{Methods}

\begin{itemize}
\item \lstinline|eval|\\
The \lstinline|eval| method forces evaluation of the held lazy
value. Two calls to \lstinline|eval| will systematically evaluate the
expression twice, which can be useful to duplicates its side effects.

\item \lstinline|value|\\
The \lstinline|value| method returns the held value, potentially
evaluating it before. \lstinline|value| performs memoization, that is,
only the first call will actually evaluate the expression, subsequent
calls will return the cached value. Unless you want to explicitly
trigger side effects from the expression by evaluating it several
time, this should be preferred over \lstinline|eval| to avoid
evaluating the expression several times uselessly.

\end{itemize}

\subsection{Examples}

\subsubsection{Evaluating once}

One usage of lazy values is to avoid evaluating an expression unless
it's actually needed, because it's expensive or has undesired side
effects. The listing below presents a situation where an
expensive-to-compute value (\lstinline|heavy_computation|) might be
needed zero, one or two times. The objective is to save time by:

\begin{itemize}
\item Not evaluating it if it's not needed.
\item Evaluating it only once if it's needed one or two time.
\end{itemize}

We thus make the wanted expression lazy, and use the \lstinline|value|
method to fetch its value when needed.

\begin{urbiscript}
// This function supposedly performs expensive computations.
function heavy_computation()
{
  echo("Heavy computation");
  return 1 + 1;
};
[00000000:hide] function () {
[:]  echo("Heavy computation");
[:]  return 1 . '+'(1);
[:]}

// We want to do the heavy computations only if needed,
// and make it a lazy value to be able to evaluate it "on demand".
var v = Lazy.new(closure () { heavy_computation() });
[00000000] Lazy_0xADDR
/* some code */;
// So far, the value was not needed, and heavy_computation
// was not evaluated.
/* some code */;
// If the value is needed, heavy_computation is evaluated.
v.value;
[00000000] *** Heavy computation
[00000000] 2
// If the value is needed a second time, heavy_computation
// is not reevaluated.
v.value;
[00000000] 2
\end{urbiscript}

\subsubsection{Evaluating several times}

Evaluating a lazy several times only makes sense with lazy arguments
and call messages. See example with call messages in
\autoref{sec:std-callmsg-examples-several}.


\subsection{Caching}

\refObject{Lazy} is meant for functions without argument.  If you need
\dfn{caching} for functions that depend on arguments, it is
straightforward to implement using a \refObject{Dictionary}.  In the
future \us might support dictionaries whose indices are not only
strings, but in the meanwhile, you have to convert the arguments into
strings, as the following sample object demonstrates.

\begin{urbiscript}
class UnaryLazy
{
  function init(f)
  {
    results = Dictionary.new;
    func = f;
  };
  function value(p)
  {
    var sp = p.asString;
    if (results.has(sp))
      return results[sp];
    var res = func(p);
    results[sp] = res |
    res
  };
  var results;
  var func;
} |
// The function to cache.
var inc = function(x) { echo("incing " + x) | x+1 } |
// The function with cache.
// Use "getSlot" to get the unevaluated function.
var p = UnaryLazy.new(getSlot("inc"));
[00062847] UnaryLazy_0x78b750
p.value(1);
[00066758] *** incing 1
[00066759] 2
p.value(1);
[00069058] 2
p.value(2);
[00071558] *** incing 2
[00071559] 3
p.value(2);
[00072762] 3
p.value(1);
[00074562] 2
\end{urbiscript}


%%% Local Variables:
%%% mode: latex
%%% TeX-master: "../urbi-sdk"
%%% End:

\section{List}

\lstinline|List|s implement potentially empty ordered collections of
elements.

\subsection{Prototypes}

\begin{itemize}
\item \refObject{Object}
\end{itemize}

\subsection{Construction}

List can be created with their literal syntax, as shown in
\autoref{sec:us-syn-lit-list}.

%% They can also be created with the \lstinline|new| method of
%% \lstinline|List|, given the initial members (\autoref{lst:new-list}).
%%
%% \begin{urbiscript}[caption=List.new, label=lst:new-list]
%% List.new(nil, "foo", 42);
%% [00000000] [nil, "foo", 42]
%% \end{urbiscript}

\subsection{Methods}

\subsubsection{all}

% FIXME: link to predicate glossary entry
Return whether all the members of the target verify the given
predicate.

\begin{urbiscript}
// Are all elements positive?
[-2, 0, 2, 4].all(function (e) { e > 0 });
[00000000] false
// Are all elements even?
[-2, 0, 2, 4].all(function (e) { e % 2 == 0 });
[00000000] true
\end{urbiscript}

\subsubsection{any}

% FIXME: link to predicate glossary entry
Return whether at least one of the members of the target verify the
given predicate.

\begin{urbiscript}
// Is there any even element?
[-3, 1, -1].all(function (e) { e % 2 == 0 });
[00000000] false
// Is there any positive element?
[-3, 1, -1].any(function (e) { e > 0 });
[00000000] true
\end{urbiscript}

\subsubsection{asList}

Return the target.

\begin{urbiscript}
[0, 1, 2].asList;
[00000000] [0, 1, 2]
\end{urbiscript}

\subsubsection{asString}

Return the target as a string describing the list.

\begin{urbiscript}
[0, 1, 2].asString;
[00000000] "[0, 1, 2]"
\end{urbiscript}

\subsubsection{back}

Return the last element of the target. An error if the target is empty.

\begin{urbiscript}
[0, 1, 2].back;
[00000000] 2
[].back;
[00000000:error] !!! back: cannot be applied onto empty list
\end{urbiscript}

\subsubsection{clear}

Empty the target.

\begin{urbiscript}
var x = [0, 1, 2];
[00000000] [0, 1, 2]
x.clear;
[00000000] []
\end{urbiscript}

\subsubsection{each}

Apply the given functional value on all members sequentially.

\begin{urbiscript}
[0, 1, 2].each(function (v) {echo (v * v); echo (v * v)});
[00000000] *** 0
[00000000] *** 0
[00000000] *** 1
[00000000] *** 1
[00000000] *** 4
[00000000] *** 4
\end{urbiscript}

\subsubsection{each\&}

Apply the given functional value on all members simultaneously.

\begin{urbiscript}
[0, 1, 2].'each&'(function (v) {echo (v * v); echo (v * v)});
[00000000] *** 0
[00000000] *** 1
[00000000] *** 4
[00000000] *** 0
[00000000] *** 1
[00000000] *** 4
\end{urbiscript}

\subsubsection{empty}

Return whether the target is empty.

\begin{urbiscript}
[].empty;
[00000000] true
[1].empty;
[00000000] false
\end{urbiscript}

\subsubsection{filter}

Return the list of all member of the target that verify the given
predicate.

\begin{urbiscript}
// Keep only odd numbers
[0, 1, 2, 3, 4, 5].filter(function (v) {v % 2 == 1});
[00000000] [1, 3, 5]
\end{urbiscript}

\subsubsection{foldl}

\fixme{This is too damn hard to explain, yet so simple.}

\begin{urbiscript}

\end{urbiscript}

\subsubsection{front}
\label{sec:std-list-front}

Return the first element of the target. An error if the target is
empty.

\begin{urbiscript}
[0, 1, 2].front;
[00000000] 0
[].front;
[00000000:error] !!! front: cannot be applied onto empty list
\end{urbiscript}

\subsubsection{has}

Return whether one of the member of the target equals the argument.

\begin{urbiscript}
[0, 1, 2].has(1);
[00000000] true
[0, 1, 2].has(5);
[00000000] false
\end{urbiscript}

\subsubsection{hasSame}

Return whether one of the member of the target is physically equal to
the argument.

\begin{urbiscript}
var x = 1;
[00000000:hide] 1
[0, x, 2].hasSame(1);
[00000000] false
[0, x, 2].hasSame(x);
[00000000] true
\end{urbiscript}

\subsubsection{head}

Synonym for \lstinline|front| (\autoref{sec:std-list-front}).

\subsubsection{join}

Concatenate all members of the target to form a string, separating
them with the given separator.

\begin{urbiscript}
["b", "ob", "b"].join("a");
[00000000] "baobab"
\end{urbiscript}

\subsubsection{map}

Apply the given functional value on every member, and return the list
of results.

\begin{urbiscript}
[0, 1, 2, 3].map(function (v) { v % 2 == 0});
[00000000] [true, false, true, false]
\end{urbiscript}

\subsubsection{[]}
\label{sec:std-list-nth}

Return the nth member of the target (indexing is zero-based). An error
if out of bounds.

%\begin{urbiscript}[caption={\lstinline|List.'\[\]'|},
%label=lst:list-nth]
\begin{urbiscript}
var l = [0, 1, 2];
[00000000:hide] [0, 1, 2]
l[1];
[00000000] 1
l[3];
[00007061:error] !!! []: invalid index: 3
\end{urbiscript}

\subsubsection{removeBack}

Remove and return the last element of the target. An error if the
target is empty.

\begin{urbiscript}
var x = [0, 1, 2];
[00000000] [0, 1, 2]
x.removeBack;
[00000000] 2
x;
[00000000] [0, 1]
[].removeBack;
[00000000:error] !!! removeBack: cannot be applied onto empty list
\end{urbiscript}

\subsubsection{removeFront}

Remove and return the first element from the target. An error if the
target is empty.

\begin{urbiscript}
var x = [0, 1, 2];
[00000000] [0, 1, 2]
x.removeFront;
[00000000] 0
x;
[00000000] [1, 2]
[].removeFront;
[00000000:error] !!! removeFront: cannot be applied onto empty list
\end{urbiscript}

\subsubsection{insertBack}
\label{sec:std-list-pushback}

Insert the given element at the end of the target.

\begin{urbiscript}
var x = [0, 1];
[00000000] [0, 1]
x.insertBack(2);
[00000000] [0, 1, 2]
x;
[00000000] [0, 1, 2]
\end{urbiscript}

\subsubsection{insertFront}

Insert the given element at the beginning of the target.

\begin{urbiscript}
var x = [1, 2];
[00000000] [1, 2]
x.insertFront(0);
[00000000] [0, 1, 2]
x;
[00000000] [0, 1, 2]
\end{urbiscript}

\subsubsection{range}

Return a sub-range of the string, from the first index included to the
second index excluded. An error if out of bounds.

\begin{urbiscript}
[0, 1, 2, 3].range(1, 3);
[00000000] [1, 2]
[].range(1, 3);
[00428697:error] !!! urbi/list.u: []: invalid index: 1
\end{urbiscript}


\subsubsection{remove}

Remove all elements from the target that equals the argument.

\begin{urbiscript}
var x = [0, 1, 0, 2, 0, 3];
[00000000] [0, 1, 0, 2, 0, 3]
x.remove(0);
[00000000] [1, 2, 3]
x;
[00000000] [1, 2, 3]
\end{urbiscript}

\subsubsection{removeById}

Remove all elements from the target that physically equals the
argument.

\begin{urbiscript}
var x = 1;
[00000000] 1
var l = [0, 1, x, 1, 2];
[00000000] [0, 1, 1, 1, 2]
l.removeById(x);
[00000000] [0, 1, 1, 2]
l;
[00000000] [0, 1, 1, 2]
\end{urbiscript}

\subsubsection{reverse}

Return the target with the order of elements inverted.

\begin{urbiscript}
[0, 1, 2].reverse;
[00000000] [2, 1, 0]
\end{urbiscript}

\subsubsection{==}

Check whether all elements in the target and the first argument, are
equal two by two.

%\begin{urbiscript}[caption={\lstinline|List.==|}, label=lst:list-sameAs]
\begin{urbiscript}
[0, 1, 2] == [0, 1, 2];
[00000000] true
[0, 1, 2] == [0, 0, 2];
[00000000] false
\end{urbiscript}

\subsubsection{[]=}
\label{sec:std-list-setnth}

Assign a value to the element of the target at the given index.

%\begin{urbiscript}[caption={\lstinline|List.'\[\]='|}, label=lst:list-setNth]
\begin{urbiscript}
var x = [0, 1, 2];
[00000000] [0, 1, 2]
x[1] = 42;
[00000000] 42
x;
[00000000] [0, 42, 2]
\end{urbiscript}

\subsubsection{size}

Return the number of elements in the target.

\begin{urbiscript}
[1, 2, 3].size;
[00000000] 3
[].size;
[00000000] 0
\end{urbiscript}

\subsubsection{sort}

Return the target, sorted with respect to the \lstinline|<| criteria.

\begin{urbiscript}
[1, 0, 3, 2].sort;
[00000000] [0, 1, 2, 3]
\end{urbiscript}

\subsubsection{tail}

Return the target, minus the first element. An error if the target is
empty.

\begin{urbiscript}
[0, 1, 2].tail;
[00000000] [1, 2]
[].tail;
[00000000:error] !!! tail: cannot be applied onto empty list
\end{urbiscript}

\subsubsection{*}

Return the target, concatenated n times to itself, n being the
argument.

%\begin{urbiscript}[caption={\lstinline|List.'*'|}, label=lst:list-times]
\begin{urbiscript}
[0, 1] * 3;
[00000000] [0, 1, 0, 1, 0, 1]
\end{urbiscript}

\subsubsection{+}

Return the concatenation of the target and the given list.

%\begin{urbiscript}[caption={\lstinline|List.'+'|}, label=lst:list-plus,
\begin{urbiscript}
  float=\floatpos]
[0, 1] + [2, 3];
[00000000] [0, 1, 2, 3]
\end{urbiscript}

\subsubsection{-}

Return the target without all element that equals any element in the
given list.

%\begin{urbiscript}[caption={\lstinline|List.'-'|}, label=lst:list-minus,
\begin{urbiscript}
[0, 1, 0, 2, 3] - [1, 2];
[00000000] [0, 0, 3]
\end{urbiscript}

\subsubsection{\textless\textless}

A synonym for \lstinline|insertBack| (\autoref{sec:std-list-pushback}).

\subsubsection{\textless}

Return whether the target is inferior to the given list. A list is
inferior to another if at least one of its element differs from the
other, and the first differing element is inferior to the other.

%\begin{urbiscript}[caption={\lstinline|List.'<'|}, label=lst:list-inf,
\begin{urbiscript}
[0, 1, 2] < [0, 1, 2];
[00000000] false
[0, 1, 2] < [0, 0, 2];
[00000000] false
[0, 1, 2] < [0, 2, 2];
[00000000] true
\end{urbiscript}

%%% Local Variables:
%%% mode: latex
%%% TeX-master: "../urbi-sdk"
%%% End:

% LocalWords:  lst asList asString foldl hasSame removeBack popback removeFront
% LocalWords:  popfront insertBack pushback insertFront pushfront urbi sameAs
% LocalWords:  removeById setNth


\section{Lobby}
\section{Object}

All objects in \us must have \refObject{Object} in their
parents. \refObject{Object} is done for this purpose so that it come
with many primitives that are mandatory for all object in \us.

\subsection{Prototypes}

\begin{itemize}
\item \refObject{Orderable}
\item \refObject{Global}
\end{itemize}

\subsection{Methods}

\begin{itemize}
\item \lstinline|acceptVoid|\\

\item \lstinline|addProto(\var{proto})|\\
  Add the prototype \var{proto} as a parent into the list of
  \lstinline|this|.

\begin{urbiscript}
var Dummy = Object.new | Dummy.addProto(Orderable) | {}
assert_eq(Dummy.protos, [Orderable, Object]);
\end{urbiscript}

\item \lstinline|allProto|\\
  Return a list with \lstinline|this|, all parents of
  \lstinline|this|, the parents of the parents of
  \lstinline|this|,\ldots

\begin{urbiscript}[firstnumber=last]
assert_eq(12.allProtos,[12, 0, Orderable, Object, Comparable, Global,
  Tags, Math, DeprecatedCommands, Kernel1Functions, System, Constants]);
\end{urbiscript}

\item \lstinline|allSlotNames|\\
  Return a list with the slot names of \lstinline|this| and of all
  parents in the hierarchy of \lstinline|this|.

\item \lstinline|apply|\\

\item \lstinline|as|\\

\item \lstinline|bounce|\\

\item \lstinline|bounce_named|\\

\item \lstinline|callMessage|\\

\item \lstinline|clone|\\

\item \lstinline|cloneSlot(\var{from}, \var{to})|\\
  Set the new slot \var{to} using a clone of \var{from}. This can only
  be used into the same object.

\begin{urbiscript}[firstnumber=last]
var foo = Object.new | {}
cloneSlot("foo", "bar") | {}
assert(!(foo === bar));
\end{urbiscript}

\item \lstinline|copySlot(\var{from}, \var{to})|\\
  Same as \lstinline|cloneSlot|, but the slot aren't cloned, so the
  two slot are the same.

\begin{urbiscript}[firstnumber=last]
var moo = Object.new | {}
cloneSlot("moo", "loo") | {}
assert(!(moo === loo));
\end{urbiscript}

\item \lstinline|createSlot(\var{name})|\\
  Create an empty slot (which actually means it is bound to
  \lstinline|void|) named \var{name}.  Raise an error if \var{name}
  was already defined.

\begin{urbiscript}[firstnumber=last]
assert(Object.createSlot("foo").isVoid);
assert(Object.hasSlot("foo"));
\end{urbiscript}

\item \lstinline|dump|\\

\item \lstinline|each(\var{fun})|\\
  Call \lstinline|each| with the given functional value on
  \lstinline|this.asList|.

\item \lstinline|eachBg|\\

\item \lstinline$each|$\\

\item \lstinline|getLazyLocalSlot|\\

\item \lstinline|getPeriod|\\

\item \lstinline|getProperty|\\

\item \lstinline|getSlot(\var{name})|\\
  Return the value associated to \var{name}.

\begin{urbiscript}[firstnumber=last]
assert(!Object.getSlot("getSlot").isVoid);
\end{urbiscript}

\item \lstinline|hasProperty|\\

\item \lstinline|hasSlot(\var{slot})|\\
  Return true if \lstinline|this| has the slot \var{slot}.

\begin{urbiscript}[firstnumber=last]
assert(hasSlot("connectionTag"));
assert(!(hasSlot("thisSlotDoesNotExist")));
\end{urbiscript}

\item \lstinline|id|\\

\item \lstinline|isA(\var{obj})|\\
  Return true if \lstinline|this| has \var{obj} in his parents, false
  otherwise.

\begin{urbiscript}[firstnumber=last]
assert(Float.isA(Orderable));
assert(!(String.isA(Float)));
\end{urbiscript}

\item \lstinline|isNil|\\
  Return true if \lstinline|this| is nil, false otherwise.

\begin{urbiscript}[firstnumber=last]
assert(nil.isNil);
assert(!(0.isNil));
\end{urbiscript}

\item \lstinline|isProto|\\
  Return true if \lstinline|this| is a prototype, false otherwise;

\begin{urbiscript}[firstnumber=last]
assert(Float.isProto);
assert(!(42.isProto));
\end{urbiscript}

\item \lstinline|isVoid|\\
  Return true if \lstinline|this| is \lstinline|void|.

\begin{urbiscript}[firstnumber=last]
assert(void.isVoid);
assert(!(42.isVoid));
\end{urbiscript}

\item \lstinline|locateSlot(\var{slot})|\\
  Return \lstinline|nil| if \lstinline|this| don't have the slot
  \lstinline|slot|. Otherwise it returns the first lowest owner of
  \lstinline|slot| of \lstinline|this|.

\begin{urbiscript}[firstnumber=last]
assert(locateSlot("init"), Channel);
assert(locateSlot("doesNotExist").isNil);
\end{urbiscript}

\item \lstinline|ownsSlot(\var{slot})|\\
  Return true if \lstinline|this| owns the slot \var{slot}, false
  otherwise.

\begin{urbiscript}[firstnumber=last]
assert(Object.ownsSlot("ownsSlot"));
assert(!(24.ownsSlot("asString")));
\end{urbiscript}

\item \lstinline|print|\\

\item \lstinline|protos|\\
  Return the list of prototypes of \lstinline|this|.

\begin{urbiscript}[firstnumber=last]
assert(Object.protos, [Comparable, Global]);
\end{urbiscript}

\item \lstinline|removeProperty|\\

\item \lstinline|removeProto(\var{proto})|\\
  Remove \var{proto} from the list of prototypes of
  \lstinline|this|.

\item \lstinline|removeSlot(\var(slot))|\\
  Remove \var{slot} from the list of slot of
  \lstinline|this|.

\item \lstinline|setConstSlot|\\
  Like \lstinline|setSlot| but the created slot is const.

\begin{urbiscript}[firstnumber=last]
assert_eq(setConstSlot("fortyTwo"), 42);
fortyTwo = 51;
[00000000:error] !!! cannot modify const slot
\end{urbiscript}

\item \lstinline|setProperty|\\

\item \lstinline|setProtos|\\

\item \lstinline|setSlot(\var{name}, \var{value})|\\
  Create a slot \var{name} mapping to \var{value}. Raise an error if
  \var{name} was already defined.

\begin{urbiscript}[firstnumber=last]
assert_eq(Object.setSlot("theObject", Object), Object);
assert(Object.theObject === Object);
assert(theObject === Object);
\end{urbiscript}

\item \lstinline|slotNames|\\
  Returns the list of slot owned by \lstinline|this|.

\item \lstinline|tasks|\\
  Returns the list of the current running tasks;

\item \lstinline|type|\\

\item \lstinline|uid|\\
  Returns ths unique id of \lstinline|this|.

\item \lstinline|unacceptvoid|\\

\item \lstinline|uobject_init|\\

\item \lstinline|updateSlot(\var{name}, \var{value})|\\
  Map the existing slot named \var{name} to \var{value}. Raise an
  error if \var{name} was not defined.
\begin{urbiscript}[firstnumber=last]
assert_eq(Object.setSlot("one", 1), 1);
assert_eq(Object.updateSlot("one", 2), 2);
assert_eq(Object.one, 2);
\end{urbiscript}

\item \lstinline|'&&'(\var{that})|\\
  Short-circuiting logical and. If \lstinline|this| evaluates to true
  evaluate and return \var{that}, otherwise return \lstinline|this|
  without evaluating \var{that}.
\begin{urbiscript}[firstnumber=last]
assert_eq(0 && "foo", 0);
assert_eq(2 && "foo", "foo");

assert_eq(""    && "foo", "");
assert_eq("foo" && "bar", "bar");
\end{urbiscript}

\item \lstinline/'||'(\var{that})/\\
  Short-circuiting logical or. If \lstinline|this| evaluates to false
  evaluate and return \var{that}, otherwise return \lstinline|this|
  without evaluating \var{that}.
\begin{urbiscript}[firstnumber=last]
assert_eq(0 || "foo", "foo");
assert_eq(2 ||  1/0,  2);

assert_eq(""    || "foo", "foo");
assert_eq("foo" || 1/0,   "foo");
\end{urbiscript}

\item \lstinline|'!'|\\
  Logical negation. If \lstinline|this| evaluates to false return
  \lstinline|true| and vice-versa.
\begin{urbiscript}[firstnumber=last]
assert_eq(!1, false);
assert_eq(!0, true);

assert_eq(!"foo", false);
assert_eq(!"",    true);
\end{urbiscript}
\end{itemize}

%%% Local Variables:
%%% mode: latex
%%% TeX-master: "../urbi-sdk"
%%% End:

\section{Orderable}
\section{Pair}
\section{Pattern}
\section{Primitive}
\section{String}

\section{System}

\section{System.Platform}

A description of the platform (the computer) the server is running on.
\subsection{Slots}

\subsubsection{kind}
Either \code{"POSIX"} or \code{"WIN32"}.

\section{Tag}
\section{Tuple}
\section{void}


%% Restore the definition of \section.
\let\section\sectionOrig

%%% Local Variables:
%%% mode: latex
%%% TeX-master: "urbi-specs"
%%% End:

% LocalWords:  CallMessage memoization callmsg lst eval xADDR createSlot POSIX
% LocalWords:  getSlot setSlot updateSlot Orderable Tuple
