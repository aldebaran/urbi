\section{Formater}

A \dfn{formater} is an object whose job is to store format
informations of a format string like used in printf in libc or
boost::format.

\subsection{Prototypes}

\begin{itemize}
\item \refObject{Object}
\end{itemize}

\subsection{Construction}

Formaters are created with the format string. It cut the string to
separate normal parts of string and formatting pattern and store them
in list keeping the same order.

\begin{urbiscript}
var f = Formater.new("Name:%s, Surname:%s;");
[00000001] Formater ["Name:", %s, ", Surname:", %s, ";"]
\end{urbiscript}

Actually, formatting pattern are translated into
\refObject{FormatInfo}.

\subsection{Methods}

\begin{itemize}
\item \lstinline|asList|\\
  Return the content of the \dfn{formater} as a list of strings and
  \refObject{FormatInfo}.
\begin{urbiscript}[firstnumber=last]
assert_eq(f.asList.asString, "[\"Name:\", %s, \", Surname:\", %s, \";\"]");
\end{urbiscript}

\item \lstinline|'%'(\var{args})|\\
  Use \lstinline|this| as format string and \var{args} as the list of
  argument. For this purpose, this operator concatenate normal string
  and the string that are result of asString called on elements of
  \var{args} with the suiting \refObject{FormatInfo}.
\begin{urbiscript}[firstnumber=last]
assert_eq(f % ["Foo", "Bar"], "Name:Foo, Surname:Bar;");
\end{urbiscript}

\end{itemize}
