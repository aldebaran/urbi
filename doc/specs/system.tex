\section{System}
Details on the architecture the \urbi server runs on.

\subsection{Prototypes}
\begin{itemize}
\item \refObject{Object}
\end{itemize}

\subsection{Slots}
\begin{itemize}
%% \item \lstinline|aliveJobs|
%% \item \lstinline|arguments|
%% \item \lstinline|backtrace|
%% \item \lstinline|breakpoint|
%% \item \lstinline|currentRunner|
%% \item \lstinline|cycle|
%% \item \lstinline|eval|
%% \item \lstinline|fresh|
%% \item \lstinline|jobs|
%% \item \lstinline|loadFile|
%% \item \lstinline|loadModule|
%% \item \lstinline|loadLibrary|
%% \item \lstinline|lobbies|
%% \item \lstinline|lobby|
%% \item \lstinline|nonInterruptible|
%% \item \lstinline|programName|
%% \item \lstinline|quit|
%% \item \lstinline|registerAtJob|
%% \item \lstinline|resetStats|
%% \item \lstinline|searchFile|
%% \item \lstinline|searchPath|
%% \item \lstinline|shutdown|
%% \item \lstinline|spawn|
%% \item \lstinline|stats|
%% \item \lstinline|stopall|

\item \lstinline|_exit(\var{status})|\\
  Shut the server down brutally: the connections are not closed, and
  the resources are not explicitly released (the operating system
  reclaims most of them: memory, file descriptors and so forth).
  Architecture dependent.

\item \lstinline|assert_(\var{assertion}, \var{message})|\\
  If \var{assertion} does not evaluate to true, throw the failure
  \var{message}.
\begin{urbiscript}
assert_(true,       "true failed");
assert_(42,         "42 failed");
assert_(1 == 1 + 1, "one is not two");
[00000001:error] !!! failed assertion: one is not two
\end{urbiscript}

\item \lstinline|assert(\var{assertion})|\\
  Unless \lstinline|System.ndebug| is true, throw an error if
  \var{assertion} is not verified.
\begin{urbiscript}[firstnumber=last]
assert(true);
assert(42);
assert(1 == 1 + 1);
[00000002:error] !!! failed assertion: 1 == 1 . '+'(1) (1 != 2)
\end{urbiscript}

\item \lstinline|assert_op(\var{operator}, \var{lhs}, \var{rhs})|\\
  Unless \lstinline|System.ndebug| is true, throw an error if
  \lstinline|\var{lhs} \var{operator} \var{rhs}| is not verified.
\begin{urbiscript}[firstnumber=last]
assert_op("<",  1, 1 + 1);
assert_op("<=", 1, 1 + 1);
assert_op(">",  1, 1 + 1);
[00000003:error] !!! failed assertion: 1 > 1 . '+'(1) (1 <= 2)
\end{urbiscript}

\item \lstinline|getenv(\var{name})|\\
  Return the value of the environment variable \var{name} as a
  \refObject{String} if set, \lstinline|nil| otherwise.  See also
  \lstinline|System.setenv| and \lstinline|System.unsetenv|.
\begin{urbiscript}[firstnumber=last]
assert(getenv("UndefinedEnvironmentVariable").isNil);
assert(!getenv("PATH").isNil);
\end{urbiscript}

\item \lstinline|load(\var{file})|\\
  Look for \var{file} in the \urbi path (\autoref{sec:tools:envvars}),
  and load it.  Throw a \lstinline|FileNotFound| error if the file
  cannot be found.  Return the last value of the file.
\begin{urbiscript}[firstnumber=last]
// Create the file ``123.u'' that contains exactly ``123;''.
assert(System.system("echo '123;' >123.u") == 0);
assert(load("123.u") == 123);
\end{urbiscript}

\item \lstinline|loadLibrary(\var{library})|\\
  Load the library \var{library}, to be found in the
  \env{URBI\_UOBJECT\_PATH} search-path (see
  \autoref{sec:tools:envvars}).  The \var{library} may be a
  \refObject{String} or a \refObject{Path}.  The \Cxx symbols are made
  available to the other \Cxx components.  See also
  \lstinline|loadModule|.

\item \lstinline|loadModule(\var{module})|\\
  Load the UObject \var{module}.  Same as \lstinline|loadLibrary|,
  except that the low-level \Cxx symbols are not made ``global'' (in
  the sense of the shared library loader).

\item \lstinline|maybeLoad(\var{file}, \var{channelName} = "")|\\
  Look for \var{file} in the \urbi path
  (\autoref{sec:tools:envvars}).  If the file is found and
  \var{channelName} is non-empty, announce on it that \var{file} is
  about to be loaded, and load it.

\begin{urbiscript}[firstnumber=last]
// Create the file ``123.u'' that contains exactly ``123;''.
assert(System.system("echo '123;' >123.u") == 0);
assert(maybeLoad("123.u") == 123);
assert(maybeLoad("u.123").isVoid);
\end{urbiscript}

\item \lstinline|Platform|\\
  See \refObject{System.Platform}

\item \lstinline|reboot|\\
  Restart the \urbi server.  Architecture dependent.

\item \lstinline|setenv(\var{name}, \var{value})|\\
  Set the environment variable \var{name} to
  \lstinline|\var{value}.asString|, and return this value.  See also
  \lstinline|System.getenv| and \lstinline|System.unsetenv|.

\begin{urbiscript}[firstnumber=last]
assert(setenv("MyVar", 12) == "12");
assert(getenv("MyVar") == "12");

// A child process that uses the environment variable.
assert(System.system("exit $MyVar") >> 8 ==
       {if (Platform.isWindows) 0 else 12});
assert(setenv("MyVar", 23) == "23");
assert(System.system("exit $MyVar") >> 8 ==
       {if (Platform.isWindows) 0 else 23});

// Defining to empty is not undefining.
assert(setenv("MyVar", "") == "");
assert(!getenv("MyVar").isNil);
\end{urbiscript}

\item \lstinline|scopeTag|\\
  Return a fresh Tag whose \lstinline|stop| will be invoked a the end
  of the current scope.  This function is likely to be removed, or
  maybe just moved into \refObject{Tag}.  See
  \autoref{sec:specs:tag:scope}.

\item \lstinline|sleep(\var{duration})|\\
  Suspend the execution for \var{duration} seconds.  No CPU cycle is
  wasted during this wait.

\begin{urbiscript}[firstnumber=last]
assert((time - {sleep(1s); time}).round == -1);
\end{urbiscript}

\item \lstinline|shiftedTime|\\
  Return the number of seconds elapsed since the \urbi server was
  launched.  Contrary to \lstinline|System.time|, time spent in frozen
  code is not counted.
\begin{urbiscript}[firstnumber=last]
assert({ var t0 = shiftedTime | sleep(1s) | shiftedTime - t0 }.round ~= 1);
assert (
  1 ==
  {
    var t = Tag.new|;
    var t0 = time|;
    var res;
    t: { sleep(1s) | res = shiftedTime - t0 },
    t.freeze;
    sleep(1s);
    t.unfreeze;
    sleep(1s);
    res.round;
  });
\end{urbiscript}

\item \lstinline|shutdown|\\
  Have the \urbi server shut down.  All the connections are closed,
  the resources are released.  Architecture dependent.

\item \lstinline|system(\var{command})|\\
  Ask the operating system to run the \var{command}.  This is
  typically used to start new processes.  The exact syntax of
  \var{command} depends on your system.  On Unix systems, this is
  typically \file{/bin/sh}, while Windows uses \file{command.exe}.

  Return the exit status.  Under Windows, this is always 0.

\begin{urbiscript}[firstnumber=last]
assert(System.system("exit 0") == 0);
assert(System.system("echo '[00000002] hello world'") == 0);
[00000002] hello world

assert(System.system("exit 23") >> 8
       == { if (System.Platform.isWindows) 0 else 23 });
\end{urbiscript}


\item \lstinline|time|\\
  Return the number of seconds elapsed since the \urbi server was
  launched.  In presence of a frozen \refObject{Tag}, see also
  \lstinline|System.shiftedTime|.
\begin{urbiscript}[firstnumber=last]
assert({ var t0 = time | sleep(1s) | time - t0 }.round ~= 1);
assert (
  2 ==
  {
    var t = Tag.new|;
    var t0 = time|;
    var res;
    t: { sleep(1s) | res = time - t0 },
    t.freeze;
    sleep(1s);
    t.unfreeze;
    sleep(1s);
    res.round;
  });
\end{urbiscript}

\item \lstinline|unsetenv(\var{name})|\\
  Undefine the environment variable \var{name}, return its previous
  value.  See also \lstinline|System.getenv| and
  \lstinline|System.setenv|.

\begin{urbiscript}[firstnumber=last]
assert(setenv("MyVar", 12) == "12");
assert(!getenv("MyVar").isNil);
assert(unsetenv("MyVar") == "12");
assert(getenv("MyVar").isNil);
\end{urbiscript}


\end{itemize}

%%% Local Variables:
%%% mode: latex
%%% TeX-master: "../urbi-sdk"
%%% End:
