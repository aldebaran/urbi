%% Copyright (C) 2011, Gostai S.A.S.
%%
%% This software is provided "as is" without warranty of any kind,
%% either expressed or implied, including but not limited to the
%% implied warranties of fitness for a particular purpose.
%%
%% See the LICENSE file for more information.

\section{Matrix}

\subsection{Prototypes}
\begin{refObjects}
\item[Object]
\end{refObjects}

\subsection{Construction}

\subsection{Slots}

\begin{urbiscriptapi}
\item[appendRow](<arg>)%
\begin{urbiassert}

\end{urbiassert}

\item[asMatrix]%
  \this.
\begin{urbiassert}
Matrix.asMatrix === Matrix;
\end{urbiassert}

\item[asPrintable]%
  A String that denotes \this.
\begin{urbiassert}
Matrix                      .asPrintable == "Matrix([])";
Matrix.new([[1, 2], [3, 4]]).asPrintable == "Matrix([[1, 2], [3, 4]])";
\end{urbiassert}

\item[asString]%
  A String that denotes \this.
\begin{urbiassert}
Matrix                      .asString == "<>";
Matrix.new([[1, 2], [3, 4]]).asString == "<<1, 2>, <3, 4>>";
\end{urbiassert}

\item[asTopLevelPrintable]%
\begin{urbiassert}

\end{urbiassert}

\item[clone](<arg>)%
\begin{urbiassert}

\end{urbiassert}

\item[column](<arg>)%
\begin{urbiassert}

\end{urbiassert}

\item[createIdentity](<size>)%
  The unit Matrix of dimensions \var{size}|.
\begin{urbiassert}
Matrix.createIdentity(0) == Matrix;
Matrix.createIdentity(3) == Matrix.new([[1, 0, 0], [0, 1, 0], [0, 0, 1]]);
\end{urbiassert}

\item[createOnes](<row>, <col>)%
  Same as \lstinline|createScalar(\var{row}, \var{col}, 1)|.
\begin{urbiassert}
Matrix.createOnes(0, 0) == Matrix;
Matrix.createOnes(2, 3) == Matrix.new([[1, 1, 1], [1, 1, 1]]);
\end{urbiassert}

\item[createScalars](<row>, <col>, <scalar>)%
  A Matrix of size \lstinline|\var{row} * \var{col}| filled with
  \var{scalar}.
\begin{urbiassert}
Matrix.createScalars(0, 0, 99) == Matrix;
Matrix.createScalars(2, 3, 99) == Matrix.new([[99, 99, 99], [99, 99, 99]]);
\end{urbiassert}

\item[createZeros](<arg>)%
  Same as \lstinline|createScalar(\var{row}, \var{col}, 0)|.
\begin{urbiassert}
Matrix.createZeros(0, 0) == Matrix;
Matrix.createZeros(2, 3) == Matrix.new([[0, 0, 0], [0, 0, 0]]);
\end{urbiassert}

\item[distanceMatrix](<arg>)%
\begin{urbiassert}

\end{urbiassert}

\item[distanceToMatrix](<arg>)%
\begin{urbiassert}

\end{urbiassert}

\item[get](<row>, <col>)%
  The element at \lstinline|\var{row}, \var{col}|.  The index \var{row} must
  verify $0 \le \textrm{\var{row}} < \texttt{\this.size.first}$, or
  $-\texttt{\this.size.first} \le \texttt{\var{row}} < 0$, in which case it
  is equivalent to using index $\textrm{\var{row}} +
  \texttt{\this.size.first}$: it counts ``backward''.  Similarly for
  \var{col}.
\begin{urbiscript}
var m = Matrix.new([[1, 2, 3], [10, 20, 30]])|;
assert
{
  m.get(0, 0) == 1;   m.get(0, -3) == 1;
  m.get(0, 1) == 2;   m.get(0, -2) == 2;
  m.get(0, 2) == 3;   m.get(0, -1) == 3;

  m.get(1, 2) == 30;  m.get(-1, -1) == 30;
};

m.get(2, 0);
[00127812:error] !!! get: invalid row: 2

m.get(-3, 0);
[00127824:error] !!! get: invalid row: -3

m.get(0, 3);
[00127836:error] !!! get: invalid column: 3

m.get(0, -4);
[00127850:error] !!! get: invalid column: -4
\end{urbiscript}

\item[init](<arg>)%
\begin{urbiassert}

\end{urbiassert}

\item[invert](<arg>)%
\begin{urbiassert}

\end{urbiassert}

\item[resize](<arg>)%
\begin{urbiassert}

\end{urbiassert}

\item[row](<arg>)%
\begin{urbiassert}

\end{urbiassert}

\item[rowAdd](<arg>)%
\begin{urbiassert}

\end{urbiassert}

\item[rowDiv](<arg>)%
\begin{urbiassert}

\end{urbiassert}

\item[rowMul](<arg>)%
\begin{urbiassert}

\end{urbiassert}

\item[rowNorm](<arg>)%
\begin{urbiassert}

\end{urbiassert}

\item[rowSub](<arg>)%
\begin{urbiassert}

\end{urbiassert}

\item[set](<arg>)%
\begin{urbiassert}

\end{urbiassert}

\item[setRow](<arg>)%
\begin{urbiassert}

\end{urbiassert}

\item[size](<arg>)%
  The dimensions of the \this, as a Pair of Floats.
\begin{urbiassert}
Matrix.size == Pair.new(0, 0);
Matrix.new([[1, 2], [3, 4], [5, 6]]).size == Pair.new(3, 2);
Matrix.new([[1, 2, 3], [4, 5, 6]]).size == Pair.new(2, 3);
\end{urbiassert}

\item[transpose](<arg>)%
\begin{urbiassert}

\end{urbiassert}

\item[type]%
\begin{urbiassert}

\end{urbiassert}

\item['=='](<that>)%
  Whether \this and \that have the same dimensions and members.
\begin{urbiscript}
do (Matrix)
{
  assert
  {
      new([[1], [2], [3]]) == new([[1], [2], [3]]);
    !(new([[1], [2], [3]]) == new([[1], [2]]));
    !(new([[1], [2], [3]]) == new([[3], [2], [1]]));
  };
}|;
\end{urbiscript}

\item['*'](<that>)%
  Matrix product between \this and \that.  The sizes must be compatibles
  ($\texttt{\this.size.second} = \texttt{\that.size.first}$).
\begin{urbiassert}
Matrix.new([[1, 2]]) * Matrix.new([[10], [20]])
  == Matrix.new([[50]]);
Matrix.new([[3, 4]]) * Matrix.new([[10], [20]])
  == Matrix.new([[110]]);
Matrix.new([[1, 2], [3, 4]]) * Matrix.new([[10], [20]])
  == Matrix.new([[50], [110]]);
\end{urbiassert}

\item['*='](<that>)%
  In place product (\refSlot{'*'}).  The same constraints apply.
\begin{urbiscript}
var lhs = Matrix.new([[1, 2], [3, 4]])|;
var rhs = Matrix.new([[10], [20]])|;
var res = lhs * rhs|;
assert
{
  (lhs *= rhs) === lhs;
  lhs == res;
};

// FIXME: rhs *= rhs;
\end{urbiscript}
\begin{urbicomment}
  removeSlots("lhs", "rhs", "res");
\end{urbicomment}


\item['+'](<that>)%
  The sum of \this and \that.  Their sizes must be equal.
\begin{urbiassert}
Matrix.new([[1, 2]]) + Matrix.new([[10, 20]])
  == Matrix.new ([[11, 22]]);

Matrix.new([[1, 2], [3, 4]]) + Matrix.new([[10, 20], [30, 40]])
  == Matrix.new ([[11, 22], [33, 44]]);

// FIXME: Matrix.new([[1, 2]]) + Matrix.new([[10, 20], [30, 40]]);
\end{urbiassert}

\item['+='](<that>)%
  In place sum (\refSlot{'+'}).   The same constraints apply.
\begin{urbiscript}
var lhs = Matrix.new([[ 1 , 2], [ 3,  4]])|;
var rhs = Matrix.new([[10, 20], [30, 40]])|;
var res = lhs + rhs|;
assert
{
  (lhs += rhs) === lhs;
  lhs == res;
};

// FIXME: rhs += rhs;
\end{urbiscript}
\begin{urbicomment}
  removeSlots("lhs", "rhs", "res");
\end{urbicomment}

\item['-'](<that>)%
  The difference of \this and \that.  Their sizes must be equal.
\begin{urbiassert}
Matrix.new([[1, 2]]) - Matrix.new([[10, 20]])
  == Matrix.new ([[-9, -18]]);

Matrix.new([[1, 2], [3, 4]]) - Matrix.new([[10, 20], [30, 40]])
  == Matrix.new ([[-9, -18], [-27, -36]]);

// FIXME: Matrix.new([[1, 2]]) - Matrix.new([[10, 20], [30, 40]]);
\end{urbiassert}


\item['-='](<that>)%
  In place difference (\refSlot{'-'}).  The same constraints apply.
\begin{urbiscript}
var lhs = Matrix.new([[ 1 , 2], [ 3,  4]])|;
var rhs = Matrix.new([[10, 20], [30, 40]])|;
var res = lhs - rhs|;
assert
{
  (lhs -= rhs) === lhs;
  lhs == res;
};

// FIXME: rhs -= rhs;
\end{urbiscript}
\begin{urbicomment}
  removeSlots("lhs", "rhs", "res");
\end{urbicomment}

\item['/'](<that>)%
  Same as \lstinline|this * that.invert|.  \that must be invertible.
\begin{urbiscript}
var lhs = Matrix.new([[20, 0], [0, 200]])|;
var rhs = Matrix.new([[10, 0], [0, 100]])|;
var res = Matrix.new([[ 2, 0], [0,   2]])|;
assert
{
  lhs / rhs == res;
  (lhs / rhs) * rhs == lhs;
  rhs * (lhs / rhs) == lhs;
};

// FIXME: Failures.
\end{urbiscript}
\begin{urbicomment}
  removeSlots("lhs", "rhs", "res");
\end{urbicomment}

\item['/='](<that>)%
  In place division (\refSlot{'/'}).  The same constraints apply.
\begin{urbiscript}
var lhs = Matrix.new([[20, 0], [0, 200]])|;
var rhs = Matrix.new([[10, 0], [0, 100]])|;
var res = Matrix.new([[ 2, 0], [0,   2]])|;
assert
{
  (lhs /= rhs) === lhs;
  lhs == res;
};

// FIXME: rhs -= rhs;
\end{urbiscript}
\begin{urbicomment}
  removeSlots("lhs", "rhs", "res");
\end{urbicomment}


\item|'[]'|(<that>)%
\begin{urbiassert}

\end{urbiassert}

\item|'[]='|(<that>)%
\begin{urbiassert}

\end{urbiassert}
\end{urbiscriptapi}

%%% Local Variables:
%%% mode: latex
%%% TeX-master: "../urbi-sdk"
%%% ispell-dictionary: "american"
%%% ispell-personal-dictionary: "../urbi.dict"
%%% fill-column: 76
%%% End:
