%% Copyright (C) 2009-2011, Gostai S.A.S.
%%
%% This software is provided "as is" without warranty of any kind,
%% either expressed or implied, including but not limited to the
%% implied warranties of fitness for a particular purpose.
%%
%% See the LICENSE file for more information.

\chapter{Installation}
\label{sec:installation}

Read this chapter is you plan to install a pre-compiled package from Gostai.
If \urbi is already installed, see \autoref{sec:tut:started},
``\nameref{sec:tut:started}''.  If you plan to compile \urbi yourself, be
sure to read the \autoref{sec:build}, ``\nameref{sec:build}''.

\section{Download}

If you already have an \urbi installed on your machine, proceed to
\autoref{sec:install:install}.

\subsection{Download \urbi \packageVersion}
\label{sec:dl:latest}

The following table gives a quick access to the packages we provide.  You
may also directly browse \url{\downloadUrl/urbi/\packageVersion/}.  For
other versions of \usdk, see \autoref{sec:dl:all}.

\begin{center}
  % \dl{PACKAGE}{EXT}
  % -----------------
  % PACKAGE is sdk, runtime, or doc. EXT is .tar.bz2 for the source, and for
  % instance linux-x86-gcc4.tar.bz2 for GNU/Linux Etch.
  \newcommand{\dl}[2]{%
    \href{\downloadUrl/urbi/\packageVersion/urbi-#1-\packageVersion-#2}{Download}%
%%    (\href{\downloadUrl/urbi/\packageVersion/urbi-#1-\packageVersion-#2.md5}{MD5})%
  }

  % \dlsr{KIND}{EXT}
  % ----------------
  % Name stands for: Down-Load Sdk Runtime.
  % KIND is "Release", "Debug", etc. only used for the user (not url).
  \newcommand{\dlsr}[2]
  {
    \multicolumn{1}{c|}{#1}
    & \dl{sdk}{#2}
    & \dl{runtime}{#2}
  }

  % \dlarch{ARCH}{KIND}{EXT}
  % ------------------------
  % For ARCH provide both the SDK and runtime of KIND, with EXT.
  \newcommand{\dlarch}[3]
  {
    \multicolumn{1}{|c|}{#1}
    & \dlsr{#2}{#3}
    \\
  }

  \begin{tabular}{ll|c|c|}
    \cline{3-4}
    & & \multicolumn{2}{c|}{Package Type}\\
    \hline
    \multicolumn{1}{|c|}{Architecture} & \multicolumn{1}{c|}{Kind} & SDK & Runtime
    \\
    \hline
    \dlarch{GNU/Linux Debian Etch x86}{Release}{linux-x86-gcc4.tar.bz2}
    \dlarch{GNU/Linux Debian Etch x86\_64}{Release}{linux-x86\_86-gcc4.tar.bz2}
    \hline
    \dlarch{\multirow{2}{*}{GNU/Linux Ubuntu Lucid x86}}{Release}{linux\_lucid-x86-gcc4.tar.bz2}
    \dlarch{}{Debug}{linux\_lucid-x86-gcc4-debug.tar.bz2}
    \hline
    \dlarch{\multirow{3}{*}{Microsoft Visual \Cxx 2005}}{Release}{windows-x86-vcxx2005.zip}
    \dlarch{}{Debug}{windows-x86-vcxx2005-debug\_dynamic.zip}
    \dlarch{}{Combo}{windows-x86-vcxx2005.exe}
    \hline
    \dlarch{\multirow{3}{*}{Microsoft Visual \Cxx 2008}}{Release}{windows-x86-vcxx2008.zip}
    \dlarch{}{Debug}{windows-x86-vcxx2008-debug\_dynamic.zip}
    \dlarch{}{Combo}{windows-x86-vcxx2008.exe}
    \hline
    \multicolumn{2}{|c|}{Documentation}
    & \multicolumn{2}{c|}{\dl{doc}{tar.bz2}}
    \\
    \hline
    \multicolumn{2}{|c|}{Sources}
    & \multicolumn{2}{c|}{\dl{sdk}{tar.bz2}}
    \\
    \hline
  \end{tabular}
\end{center}

%% The MD5 files may be useful if you want to check the integrity of the
%% file you downloaded.

\subsection{Download a Specific Version of \urbi}
\label{sec:dl:all}

All the versions of \urbi are available at \url{\downloadUrl/urbi/}.

Our packages are named as follows:
\begin{center}
\file{urbi-\var{kind}-\var{version}-\var{arch}-\var{os}-\var{compiler}.\var{ext}}
\end{center}
where
\begin{description}
\item[\var{kind}] defines what the package is about:
  \begin{description}
  \item[sdk] \usdk, i.e., the full package, including libraries, headers,
    programs, documentation, etc.  This is meant for the computers of the
    developers.
  \item[runtime] Only what is needed to \emph{run} (as opposed to
    \emph{compile}) \urbi programs.  This is meant for robots.
  \item[doc] The whole documentation.
  \end{description}
\item[\var{version}] specifies the exact revision of Urbi that you are
  using.  It can be simple, \code{2.0}, or more complex,
  \code{2.0-beta3-137-g28f8880}.  In that case,
  \begin{description}
  \item[2.0] is the version of the \urbi Kernel,
  \item[beta3] designates the third pre-release,
  \item[137] is the number of changes since beta3 (not counting
    changes in sub-packages),
  \item[g28f8880] is a version control identifier, used internally to
    track the exact version that is being tested by our users.
  \end{description}
\item[\var{arch}] describes the architecture, the \acro{cpu}:
  \code{ARM}, \code{ppc}, or \code{x86}.
\item[\var{os}] is the operating system: \code{linux} for GNU/Linux,
  \code{osx} for Mac OS X, or \code{windows} for Microsoft Windows.
\item[\var{compiler}] is the tool chain used to compile the programs:
  \code{gcc4} for the GNU Compiler Collection 4.x, \code{vcxx2005} for
  Microsoft Visual \Cxx 2005, \code{vcxx2008} for Microsoft Visual
  \Cxx 2008.
\item[\var{ext}] is the package format extension.  For Unix architectures,
  \code{tar.bz2}; uncompress them with \command{tar xf \var{tarfile}}.  For
  Windows hosts, we provide zip files (\file{*.zip}) for both ``release''
  and ``debug'' flavors, and installers (for instance \file{*.exe}).  You
  are encouraged to use the installers, since in addition to installing
  headers and libraries, they also install Visual \Cxx Wizards to create
  UObjects, they take care of installing the Visual Runtime if needed, and
  they install Gostai Console and Gostai Editor.
\end{description}

\section{Install \&{} Check}
\label{sec:install:install}

The package is \dfn{relocatable}, i.e., it does not need to be put at
a specific location, nor does it need special environment variables to
be set.  It is not necessary to be a super-user to install it.  The
\dfn{root} of the package, denoted by \var{urbi-root} hereafter, is
the absolute name of the directory which contains the package.

After the install, the quickest way to test your installation is to run
the various programs.

\subsection{GNU/Linux and Mac OS X}

Decompress the package where you want to install it.  If \var{urbi-sdk-2.x}
denotes the version of Urbi SDK you downloaded (say, \var{urbi-sdk-2.x} is
\file{urbi-sdk-2.3-linux-x86-gcc4}), run something like:

\begin{shell}
$ rm -rf \var{urbi-root}
$ cd /tmp
$ tar xf \var{path-to}/\var{urbi-sdk-2.x}.tar.bz2
$ mv \var{urbi-sdk-2.x} \var{urbi-root}
\end{shell}

This directory, \var{urbi-root}, should contain \file{bin}, \file{FAQ.txt}
and so forth.  Do not move things around inside this directory.  In order to
have an easy access to the \urbi programs, set up your \env{PATH}:

\begin{shell}[style=varInString]
$ export PATH="\var{urbi-root}/bin:$PATH"
\end{shell}%$

\begin{shell}
# Check that urbi is properly set up.
$ urbi --version

# Check that urbi-launch is properly installed.
$ urbi-launch --version
# Check that urbi-launch can find its dependencies.
$ urbi-launch -- --version

# Check that Urbi can compute.
$ urbi -e '1+2*3; shutdown;'
[00000175] 7
\end{shell}%$

\subsection{Windows}

Decompress the zip file wherever you want or execute the installer.

Execute the script \file{urbi.bat}, located at the root of the
uncompressed package. It should open a terminal with an interactive
Urbi session.

\begin{cygwin}
Inputs and outputs of windows native application are buffered under Cygwin.
Thus, either running the interactive mode of Urbi or watching the output of the
server under Cygwin is not recommended.
\end{cygwin}

% FIXME: add a listing with the urbi banner under windows?

%%% Local Variables:
%%% coding: utf-8
%%% mode: latex
%%% TeX-master: "../urbi-sdk"
%%% ispell-dictionary: "american"
%%% ispell-personal-dictionary: "../urbi.dict"
%%% fill-column: 76
%%% End:
