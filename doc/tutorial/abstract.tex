%% Copyright (C) 2010-2012, Gostai S.A.S.
%%
%% This software is provided "as is" without warranty of any kind,
%% either expressed or implied, including but not limited to the
%% implied warranties of fitness for a particular purpose.
%%
%% See the LICENSE file for more information.

\begin{partDescription}{part:tut}
  {%
    This part, also known as the ``\us tutorial'', teaches the reader how to
    program in \us.  It goes from the basis to concurrent and event-based
    programming.  No specific knowledge is expected.  There is no need for a
    \Cxx compiler, as \UObject will not be covered here (see
    \autoref{part:uobject}).  The reference manual contains a terse and
    complete definition of the \urbi environment (\autoref{part:specs}).
    %
  }
\item[sec:tut:first]
  First contacts with \us.
\item[sec:tut:value]
  A quick introduction to objects and values.
\item[sec:tut:flow]
  Basic control flow: \lstinline{if}, \lstinline{for} and the like.
\item[sec:tut:function]
  Details about functions, scopes, and lexical closures.
\item[sec:tut:object]
  A more in-depth introduction to object-oriented programming in \us.
\item[sec:tut:functional]
  Functions are first-class citizens.
\item[sec:tut:concurrent]
  The \us operators for concurrency, tags.
\item[sec:tut:event-prog]
  Support for event-driven concurrency in \us.
\item[sec:tut:ros] How to use ROS from \urbi, and vice-versa.
\end{partDescription}


%%% Local Variables:
%%% coding: utf-8
%%% mode: latex
%%% TeX-master: "../urbi-sdk"
%%% ispell-dictionary: "american"
%%% ispell-personal-dictionary: "../urbi.dict"
%%% fill-column: 76
%%% End:
