\begin{partDescription}{part:uobject}
  {
    %
    This part guides you through the various steps of writing an \urbi
    \Cxx component using the UObject API. This API can be used to add
    new objects written in \Cxx to the \us language, and to interact
    from \Cxx with the objects that are already defined. We cover the
    use cases of controlling a physical device (servomotor, speaker,
    camera\ldots), and interfacing higher-lever components (voice
    recognition, object detection\ldots) with \urbi.

    The API defines the UObject class. Each instance of a derived
    class in your \Cxx code will correspond to an \us object sharing
    some of its methods and attributes. The API provides methods to
    declare which elements of your object are to be shared. To share a
    variable with \urbi, you have to give it the type UVar. This type
    is a container that provides cast and assignment operators for all
    types known to \urbi: double, string and char*, and the
    binary-holding structures UBinary, USound and UImage. This type
    can also read from and write to the liburbi UValue class. The API
    provides methods to set up callbacks functions that will be
    notified when a variable is modified or read from \urbi
    code. Instance methods of any prototype can be rendered accessible
    from \us, providing all the parameters types and the return type
    can be converted to/from UValue.
  }
%\item[sec:uob:quick] This chapter, self-contained, shows the potential
%  of \urbi used as a middleware.
\item[sec:uob:api]
  Building simple UObjects.
\item[sec:uob:uses]
  Interfacing a servomotor device as an example on how to use the
  UObject architecture as a middleware.
\end{partDescription}

%%% Local Variables:
%%% mode: latex
%%% TeX-master: "../urbi-sdk"
%%% ispell-dictionary: "american"
%%% ispell-personal-dictionary: "../urbi.dict"
%%% End:
