\chapter*{Introduction}
% \phantomsection % otherwise hyperlinks to previous chapter.
\addcontentsline{toc}{chapter}{Introduction}

\us is a programming language primarily designed for robotics. It's a
dynamic, prototype-based, object-oriented scripting language. It
supports and emphasizes parallel and event-based programming, which
are very popular paradigms in robotics, by providing core primitives
and language constructs.

Its main features are:
\begin{itemize}
\item \us is designed to be syntactically close to \Cxx. If you know
  \C or \Cxx, you can easily write \us programs.
\item \us is fully integrated with \Cxx. You can bind \Cxx classes
  in \us seamlessly. \us is also integrated with many other
  languages such as \java, \matlab or \python.
\item \us is object oriented. It supports encapsulation, inheritance
  and inclusion polymorphism. Dynamic dispatching is available through
  monomethods --- just as \Cxx, \Cs or \java.
\item \us includes parallelism at the core of its semantics. It
  provides you with natural constructs to run and control high numbers
  of interacting concurrent tasks.
\item \us supports event-based programming. Triggering events and
  reacting to them is absolutely straightforward.
\item \us supports functional programming, inspired from languages
  such as \lisp or \caml. This includes first class functions and
  pattern matching.
\item \us uses a client/server architecture: the interpreter accepts
  multiple connections from different sources (human users, robots,
  other servers \ldots) and enables them to interact.
\item \us supports distributed architectures: you can run objects in
  different processes, potentially on another computer across the
  network.
\end{itemize}

This documentation is far from complete.  It is updated regularly on
the Gostai Web site, in \href{\docurl}{\docurl}, both as a PDF
document, \href{\docurl/urbi-sdk.pdf}{urbi-sdk.pdf}, and as HTML
pages, \href{\docurl/urbi-sdk.html}{urbi-sdk.html}.


\section*{Outline}

\newenvironment{partDescription}[1]
{%
  \subsection*{\autoref{#1} --- \nameref{#1}}%
  \begin{description}%
    \let\itemOrig\item%
    \renewcommand{\item}[1][]{\itemOrig[\autoref{##1} --- \nameref{##1}]~\\}%
  }{%
  \end{description}%
}

\begin{partDescription}{part:tut}
\item[sec:tut:started]
  Quick-install guide.
\item[sec:tut:first]
  First contact with \us.
\item[sec:tut:value]
  A quick introduction to objects and values.
\item[sec:tut:flow]
  Basic control flow: \lstinline{if}, \lstinline{for} and the like.
\item[sec:tut:function]
  Details about functions, scoped, and lexical closures.
\item[sec:tut:object]
  A more in-depth introduction to object-oriented programming in \us.
\item[sec:tut:functional]
  Functions are first-class citizens.
\item[sec:tut:concurrent]
  The \us operators for concurrency, tags.
\item[sec:tut:event-prog]
  Support for event-driven concurrency in \us.
\end{partDescription}

\begin{partDescription}{part:uobject}
\item[sec:uob:api]
  Buildding simple UObjects.
\item[sec:uob:uses]
  Interfacing a servomotor device as an example on how to use the
  UObject architecture as a middleware.
\end{partDescription}

\begin{partDescription}{part:install}
\item[sec:installation]
  Complete instructions on how to install \urbi SDK.
\end{partDescription}

\begin{partDescription}{part:guide}
\item[sec:faq]
  Some answers to common questions.
\end{partDescription}

\begin{partDescription}{part:specs}
\item[sec:tools]
  Presentation and usage of the different tools available with the
  \urbi framework related to \us, such as the \urbi server, the
  command line client, \umake, \ldots

\item[sec:lang]
  Core constructs of the language and their behavior.

\item[sec:stdlib]
  Listing of all classes and methods provided in the standard library.

\item[sec:naming]
  Also known as ``The \urbi Naming Standard'': naming conventions in
  for standard hardware/software devices and components implemented as
  UObject and the corresponding slots/events to access them.

%\item[sec:sdk]
%  The \urbi software development kit that enable to
%  interact with \urbi from \Cxx.
\end{partDescription}

\begin{partDescription}{part:platforms}
\item[sec:nao]
  Programming Aldebaran's Nao using \urbi.

\item[sec:webots] Using Cyberbotics' Webots simulation environment
  with \urbi.
\end{partDescription}

%\begin{partDescrption}{part:ref}
%\end{partDescrption}


%% \paragraph{Thanks}
%% The \us language what first designed and implemented by
%% Jean-Christophe Baillie, together with Matthieu Nottale.  Because
%% \urbi was widely acclaimed by its users, Jean-Christophe founded
%% Gostai, a France-based Company that develops software for robotics
%% with a strong emphasis on personal robotics.
%%
%% \us SDK 1 was further developped by Akim Demaille, Guillaume
%% Deslandes, Quentin Hocquet, Thomas Moulard, and Benoît Sigoure.
%%
%% The \us SDK 2 project was started and developped by Akim Demaille,
%% Quentin Hocquet, Matthieu Nottale, and Benoît Sigoure.  Samuel Tardieu
%% provided an immense help during the year 2008, in particular for the
%% concurrency and event support.
%%
%% The maintenance is currently carried out by Akim Demaille, Quentin
%% Hocquet, and Matthieu Nottale.  Jean-Christophe Baillie is still
%% deeply involved in the development of \us, he regularly submits ideas,
%% and occasionnally even code!

%%% Local Variables:
%%% mode: latex
%%% TeX-master: "urbi-sdk"
%%% End:
